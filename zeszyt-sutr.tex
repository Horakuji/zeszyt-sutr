%% -------------------------------------------------------------------------- %%
%                                                                              %
%                              http://mahajana.net                             %
%                                                                              %
%% -------------------------------------------------------------------------- %%

% (!!!) TO DO dla bieżącej wersji:
% - wstawić Daichi Zenji, Niemieckie testy od Daikana i Shushogi sprawdzić/zrobione
% - inna wersja namu kie butsu (od Murakami Roshiego)
% - sprawdzić nowy Kanromon i~ofiarowania, czy są zgodne z~orginałem

% TO DO dla wersji kolejnej:
% - przejrzeć jeszcze ``Soto school scriptures''
% - zrobić podział na sutry, dharani, itd.
% - zrobić PDF Linki

%%-- pakiety --%%
\documentclass[12pt]{article}
\usepackage{polski}
\usepackage[paperwidth=148.5mm,paperheight=210mm,top=1.5cm,bottom=1.5cm,left=1.925cm,right=1.925cm,footskip=0.7cm,includefoot]{geometry}
\usepackage{multicol}
\usepackage{amssymb}
\usepackage{latexsym}
\usepackage{longtable}
\usepackage{hyperref}

\def\wersja{6.0}
\def\data{2025}

{
\newlinechar=`@
\message{@=====================================}
\message{@Zeszyt Sutr wersja \wersja}
\message{@Wroclaw, \data}
\message{@=====================================@@@}
}

% słowniczek
% po japońsku np. cztery ślubowania (kopia z~Bukkokuji) czy hannya shin gyo
% czy nie ma kanzen / janusz jura dai hi shin dharani po polsku?
% część tekstów jap. gdzie linie się nie mieszczą - zmienić drukowane literki na małe

%%%%%%%%%%%%%%%%%%%%%%%%%%%%%%%%%%%%%%%%%%%%%%%%%%%%%%%%%%%%%%%%%%%%%%%%%%
%%%% Ustawienia LaTeX-u

\brokenpenalty 10000	% zakaz łamania strony po przeniesieniu wyrazu
\widowpenalty 10000	% zakaz łamania strony przed ostatnim wierszem akapitu
\clubpenalty 10000	% zakaz łamania strony po pierwszym wierszu akapitu
\sloppy			% zakaz wydłużania linii

\makeatletter	% @=litera
\renewcommand{\section}{\@startsection%
 {section}			% nazwa zmienianego elementu
 {1}				% poziom (w strukturze dokumentu)
 {0mm}				% wcięcie (lewy margines)
 {18pt plus 3pt minus 7pt}	% pionowy odstęp przed elementem
 { 2pt}				% pionowy odstęp za elementem
 {\bfseries}}			% styl/krój pisma

\renewcommand*\l@section{\@dottedtocline{1}{1.5em}{2.3em}}
\makeatother	% @=nie-litera

%%%%%%%%%%%%%%%%%%%%%%%%%%%%%%%%%%%%%%%%%%%%%%%%%%%%%%%%%%%%%%%%%%%%%%%%%%
%%%% Polecenia

\newcommand{\keisu}		{\mbox{\ $\bigodot$\ \ }}
\newcommand{\shokei}		{\mbox{\ $\bullet$\ \ }}
\newcommand{\keisuzgaszenie}	{\mbox{\ $\varnothing$\ \ }}

%%%% \samepage -- objęcie kawałka tekstu klamrą:
%
% \samepage{
%	tekst
% }
% 
% powoduje, że całość tego tekstu znajdzie się na jednej stronie
\let\samepage\vbox

%%%% \stanza -- zwrotka gathy, zawsze znajdzie się na jednej stronie
%
% \stanza{
%	wers1\\
%	wers2
% }
%
\def\stanza{%
	\bigskip
	\vbox}

%%%%%%%%%%%%%%%%%%%%%%%%%%%%%%%%%%%%%%%%%%%%%%%%%%%%%%%%%%%%%%%%%%%%%%%%%%
%%%% Środowiska

\newcommand{\novspace}{\setlength{\itemsep}{0pt}\setlength{\parskip}{0pt}}
% itemize i enumerate bez odstępów pionowych między elementami listy
\newenvironment{Itemize}%
	{\begin{itemize}\novspace}%
	{\end{itemize}}
\newenvironment{Enumerate}%
	{\begin{enumerate}\novspace}%
	{\end{enumerate}}

\newcounter{gathalabelnum}
\newcounter{textlabelnum}
\newcounter{labelnum}

\newenvironment{Prayer}[4]{%
	\section*{#2}
	\ifx -#1
		\stepcounter{labelnum}\label{label.\arabic{labelnum}}\nopagebreak
	\else
		\label{#1}\nopagebreak
	\fi
	\ifx -#3
		\addcontentsline{toc}{section}{#2}
	\else
		\addcontentsline{toc}{section}{#3}
	\fi
	\ifx -#4
		{}
	\else
		\noindent\emph{#4}\par\nopagebreak
	\fi
	\begingroup
}
{\par\endgroup}

\newcommand{\wersaliki}{% -- z art. ,,TeXnologia a typografia''
	\newbox\Abox
	\newdimen\Aspacja
	\setbox\Abox=\hbox{A}
	\Aspacja=\wd\Abox
	\spaceskip \Aspacja plus .25em minus .1em
}

\newenvironment{Verse}{%
	\raggedright
	\leftskip   1.5cm
	\parindent -1.0cm
	\language   255
	\par
}{\par}


%%%% Japanese -- ,,długie'' (kilka akapitów) transkrypcje japońskie 
%%%%             i chińskie składane w ,,kwadrat'': wcięcie w pierwszej 
%%%%             linii równe 0pt, w ostatniej linii tekst dosunięty do 
%%%%             prawej krawędzi; bez dzielenia wyrazów
\newenvironment{JAPANESE}
{
	\parindent	0pt
	\parfillskip	0pt
	\language 255
	\begingroup
	\wersaliki
	\par
}
{\par\endgroup}

\newenvironment{japanese}
{
	\parindent	0pt
	\parfillskip	0pt
	\language 255
	\begingroup
	\par
}
{\par\endgroup}

%%-- dokument --%%
\begin{document}

\setcounter{page}{-100}
%%-- strona tytułowa--%%
\pagestyle{empty}

\begin{center}
\ \\
\vspace{2.5cm}
\ \\
{\LARGE
SUTRY I~DHARANI\\
\smallskip
DO CODZIENNYCH OFIAROWAŃ\\
\bigskip
I PRAKTYKI}

\vspace{9cm}
\rule{\textwidth}{1pt}
\vspace{1pt}
{\small Wrocław, \data}
\end{center}

%% -------------------------------------------------------------------------- %%
\newpage

\null
\vspace{12cm}

\begin{center}
\it
Każdą sutrę należy traktować z~szacunkiem, chronić, czcić i~nie kłaść
w~niewłaściwych miejscach\\
(np. bezpośrednio na podłodze).
\end{center}
%% -------------------------------------------------------------------------- %%
\newpage


%%-- treść dokumentu --%%
\section*{O śpiewaniu sutr}

{
\parindent 0pt
\bigskip
Jedna z~podstawowych zasad zen to ta, że jaźń i~Buddha są jednym, Umysł
Buddhy jest naszym umysłem. Śpimy, budzimy się, jemy, pracujemy z~tym umysłem
Buddhy. Po prostu śpiewajcie sutry całym sobą, śpiewajcie sutry dla samych
sutr. Dokładniej: uszy, usta, głos i~umysł stają się jednym w~głębokim
samadhi poprzez śpiewanie sutr i~w ten sposób zostaje urzeczywistniony
głęboki związek ze wszystkim. Umysł rozciągnie się, wewnętrzne i~zewnętrzne
światy staną się jednym.

\begin{description}
	\item[Dharani ---]
	dosłownie ,,to co utrzymuje''. Rodzaj mistycznego wersetu lub
	długiej mantry, niekoniecznie posiadającej intelektualny sens, która
	zawiera głębokie znaczenie i~obdarza mocą tych, którzy praktykują
	jej śpiew.

	\item[Sutra ---]
	w~dosłownym znaczeniu~-- słowa Buddhy spisane po jego śmierci przez
	jego najbliższych uczniów. W~tradycji buddyzmu mahajany sutry
	zawierają nie tylko istotne nauki Buddhy Siakjamuniego, ale również
	powiedzenia wielkich patriarchów, gathy oraz dharani.

	\item[Ek\=o ---]
	po śpiewie jednej lub kilku sutr lub dharani często śpiewane jest
	ek\=o.  Ek\=o dosłownie oznacza ,,zwrot w~kierunku'' i~jest
	ofiarowaniem zasług zyskanych poprzez samadhi śpiewu sutr i~dharani
	wszystkim Buddhom, Bodhisattwom, Patriarchom i~wszystkim istotom.
\end{description}
}


%%~-- spis treści ----------------------------------------------------------- %%
\newpage
{
	\tableofcontents
}

%% -------------------------------------------------------------------------- %%
\newpage
\setcounter{page}{1}	% numerowanie stron rozpoczyna się odtąd
\pagestyle{plain}

%%%%%%%%%%%%%%%%%%%%%%%%%%%%%%%%%%%%%%%%%%%%%%%%%%%%%%%%%%%%%%%%%%%%%%%%%%%%%%%
%%%% SHICHI BUTSU TS\=UKAIGE
\begin{Prayer}{shichi_butsu_tsukaige}
	{SHICHI BUTSU TS\=UKAIGE}{-}
	{Gatha Siedmiu Buddhów}

	\begin{Verse}
	\stanza{
		Sho Aku makusa		\\
		Shu Zen Bukky\=o	\\
		Jij\=o Go I		\\
		Ze sho Bukky\=o
	}
	\end{Verse}
\end{Prayer}

%%%%%%%%%%%%%%%%%%%%%%%%%%%%%%%%%%%%%%%%%%%%%%%%%%%%%%%%%%%%%%%%%%%%%%%%%%%%%%%
%%%% GATHA SIEDMIU BUDDÓW !
\begin{Prayer}{gatha_siedmiu_buddow}
	{GATHA SIEDMIU BUDDHÓW}{-}
	{Shichi Butsu Ts\=ukaige}

	\begin{Verse}
	\stanza{
		Nie czyń żadnego zła,	\\
		Praktykuj tylko dobro,	\\
		Oczyść swój umysł,	\\
		To jest nauka wszystkich Buddhów.
	}
	\end{Verse}
\end{Prayer}

%%%%%%%%%%%%%%%%%%%%%%%%%%%%%%%%%%%%%%%%%%%%%%%%%%%%%%%%%%%%%%%%%%%%%%%%%%%%%%%
%%%% TAKKESAGE
\begin{Prayer}{tekkesage}
	{TAKKESAGE}{-}
	{Wiersz Kasiaja}

	\begin{Verse}\wersaliki
	\stanza{
		DAI SAI GE DAP-PUKU.	\\
		MU S\=O FUKU DEN E.	\\
		HI BU NYO RAI KY\=O.	\\
		K\=O DO SHO SHU J\=O.
	}
	\end{Verse}
\end{Prayer}

%%%%%%%%%%%%%%%%%%%%%%%%%%%%%%%%%%%%%%%%%%%%%%%%%%%%%%%%%%%%%%%%%%%%%%%%%%%%%%%
%%%% WIERSZ KASIAJA !
\begin{Prayer}{wiersz_kesy}
	{WIERSZ KASIAJA}{-}
	{Takkesage}

	\begin{Verse}
	\stanza{
		Bezmierna jest szata wyzwolenia,	\\
		Szata bezforemnego pola dobrodziejstwa.	\\
		Głoszę z czcią naukę Tathagaty,		\\
		Szeroko wyzwalając wszystkie odczuwające istoty.
	}
	\end{Verse}
\end{Prayer}

%%%%%%%%%%%%%%%%%%%%%%%%%%%%%%%%%%%%%%%%%%%%%%%%%%%%%%%%%%%%%%%%%%%%%%%%%%%%%%%
%%%% TI-SARANA
\begin{Prayer}{ti-sarana}
	{TI-SARA\d NA}{-}
	{Trzy Schronienia}

	\begin{Verse}
	\stanza{\wersaliki
		BUDDHA\.M SARA\d NA\.M GACCH\=AMI.\\
		DHAMMA\.M SARA\d NA\.M GACCH\=AMI.\\
		SA\.NGHA\.M SARA\d NA\.M GACCH\=AMI.
	}

	\stanza{\wersaliki
		NAMU KIE BUTSU.	\\
		NAMU KIE H\=O.	\\
		NAMU KIE S\=O.
	}

	\stanza{
		Znajduje schronienie w~Buddzie.\\
		Znajduje schronienie w~Dharmie.\\
		Znajduje schronienie w~Sandze.
	}
	\end{Verse}
\end{Prayer}

%%%%%%%%%%%%%%%%%%%%%%%%%%%%%%%%%%%%%%%%%%%%%%%%%%%%%%%%%%%%%%%%%%%%%%%%%%%%%%%
%%%% ATTA DIPA !
\begin{Prayer}{atta_dipa}
	{ATTA DIPA}{-}
	{Ostatnie Upomnienie Buddhy}

	\begin{Verse}
	\stanza{\wersaliki
		ATTA DIPA	\\
		VIHARATHA	\\
		ATTA SARANA	\\
		ANANNA SARANA	\\
		DHAMMA DIPA	\\
		DHAMMA SARANA	\\
		ANANNA SARANA
	}

	\stanza{
		Rozważ!
		Sam jesteś Światłem	\\
		Polegaj na sobie	\\
		Nie polegaj na innych	\\
		Dharma jest Światłem	\\
		Polegaj na Dharmie	\\
		Nie polegaj na niczym innym niż Dharmie.
	}
	\end{Verse}
\end{Prayer}

\newpage
%%%%%%%%%%%%%%%%%%%%%%%%%%%%%%%%%%%%%%%%%%%%%%%%%%%%%%%%%%%%%%%%%%%%%%%%%%%%%%%
%%%% VANDANA
\begin{Prayer}{vandana}
	{VANDANA}{-}
	{Oddanie Hołdu}

	\begin{Verse}
	\stanza{\wersaliki
		NAMO TASSA		\\
		BHAGAVATO ARAHATO	\\
		SAMMA SAMBUDDHASSA.
	}

	\stanza{
		Chwała Buddzie,	\\
		Czcigodnemu,	\\
		Oświeconemu,	\\
		Doskonale Przebudzonemu!
	}
	\end{Verse}
\end{Prayer}

%%%%%%%%%%%%%%%%%%%%%%%%%%%%%%%%%%%%%%%%%%%%%%%%%%%%%%%%%%%%%%%%%%%%%%%%%%%%%%%
%%%% DZIESIĘĆ BUDDYJSKICH WSKAZAŃ
\begin{Prayer}{dziesiec_buddyjskich_wskazan}
	{DZIESIĘĆ BUDDYJSKICH WSKAZAŃ}{-}
	{-}

\samepage{
\begin{Enumerate}
	\item Nie zabijać
	\item Nie kraść
	\item Nie pragnąć seksu
	\item Nie kłamać
	\item Nie pić alkoholu (narkotyki też są wykluczone)
	\item Nie mówić o~błędach innych
	\item Nie chwalić się
	\item Nie być skąpym
	\item Nie wpadać w~gniew
	\item Nie oczerniać Trzech Klejnotów
\end{Enumerate}
}


Powyższe wskazania pochodzą z~Brahmadżali, sutry z~kanonu mahajany.
Istnieje również Brahmadżala hinajany, różniąca się od wcześniej wspomnianej.


Te wskazania oraz ich przekaz istnieje m.in. w~Japonii, jest tam
szczególnie praktykowany i~przyjmowany zarówno przez osoby świeckie jak
i~mnichów, ale w~różnej formie.


Wskazania te są wskazaniami Bodhisattwy, co odróżnia je od wskazań
hinajany i~innych, oznacza to że właściwe ich przyjęcie od osoby, która
otrzymała ich prawidłowy przekaz czyni z~upasaki Bodhisattwę świeckiego
w~oparciu o~przyjęte wskazania.
\end{Prayer}

%%%%%%%%%%%%%%%%%%%%%%%%%%%%%%%%%%%%%%%%%%%%%%%%%%%%%%%%%%%%%%%%%%%%%%%%%%%%%%%
%%%% TRZY OGÓLNE POSTANOWIENIA
\begin{Prayer}{trzy_ogolne_postanowienia}
	{TRZY OGÓLNE POSTANOWIENIA}{-}
	{-}

	\begin{Verse}
	\stanza{
		Postanawiam unikać zła.		\\
		Postanawiam czynić dobro.	\\
		Postanawiam wyzwalać wszystkie odczuwające istoty.
	}
	\end{Verse}
\end{Prayer}

%%%%%%%%%%%%%%%%%%%%%%%%%%%%%%%%%%%%%%%%%%%%%%%%%%%%%%%%%%%%%%%%%%%%%%%%%%%%%%%
%%%% SANKIRAIMO
\begin{Prayer}{sankiraimon}
	{SANKIRAIMON}{-}
	{Modlitwa Pokłonów przed Trzema Klejnotami}

	\begin{Verse}\wersaliki
	\stanza{
		J\=I KI E BUTSU		\\
		T\=O GAN SHU J\=O	\\
		TAI GE DAI D\=O		\\
		HOTSU MU J\=O I.
	}

	\stanza{
		J\=I KI E H\=O		\\
		T\=O GAN SHU J\=O	\\
		JIN NY\=U KY\=O Z\=O	\\
		CHI E NY\=O KAI.
	}

	\stanza{
		J\=I KI E S\=O		\\
		T\=O GAN SHU J\=O	\\
		T\=O RI DAI SH\=U	\\
		IS-SAI M\=U G\=E.
	}
	\end{Verse}
\end{Prayer}

%%%%%%%%%%%%%%%%%%%%%%%%%%%%%%%%%%%%%%%%%%%%%%%%%%%%%%%%%%%%%%%%%%%%%%%%%%%%%%%
%%%% MODLITWA POKŁONÓW PRZED TRZEMA KLEJNOTAMI !
%%%% (!!!) tu ewentualnie można jeszcze wstawić wersję japońską (poniżej
%%%%       chińskiej) z soto shu scriptures
\begin{Prayer}{trzy_skarby}
	{MODLITWA POKŁONÓW\\ PRZED TRZEMA KLEJNOTAMI}{-}
	{Sankiraimon}

	\begin{Verse}
	\stanza{
		Znajduję schronienie w Buddzie		\\
		i ślubuję razem ze wszystkimi istotami	\\
		zrozumieć Wielką Drogę i rozwinąć nieprześcigniony umysł.\\
		Znajduję schronienie w Dharmie		\\
		i ślubuję razem ze wszystkimi istotami	\\
		głęboko wejść w Skarbnicę Sutr i osiągnąć mądrość\\
		podobną wielkiemu oceanowi.		\\
		Znajduję schronienie w Sandze		\\
		i ślubuję razem ze wszystkimi istotami	\\
		pozostać w związku z Wielkim Zgromadzeniem,	\\
		nie powodując żadnych przeszkód i niepokoju.
	}
	\end{Verse}
\end{Prayer}

%%%%%%%%%%%%%%%%%%%%%%%%%%%%%%%%%%%%%%%%%%%%%%%%%%%%%%%%%%%%%%%%%%%%%%%%%%%%%%%
%%%% SANZON RAIMON
\begin{Prayer}{sanzonraimon}
	{SANZON RAIMON}{-}
	{Pokłon przed Trzema Szlachetnymi}

	\begin{Verse}\wersaliki
	\stanza{
		NAMU DAI ON KY\=OSHU HONSHI SHAKA MUNI BUTSU NAMU\\
		K\=OSO JO YO DAISHI NAMU TAISO JO SAI DAISHI NAMU DAIZU DAIHI AI MIN
		SH\=OJU\\
		SHO SHO SESE CHIG\=U CHO DAI.
	}
	\end{Verse}
\end{Prayer}

%%%%%%%%%%%%%%%%%%%%%%%%%%%%%%%%%%%%%%%%%%%%%%%%%%%%%%%%%%%%%%%%%%%%%%%%%%%%%%%
%%%% STROFA SZACUNKU DLA TRZECH CZCIGODNYCH !
\begin{Prayer}{strofa_szacunku}
	%{STROFA SZACUNKU\\ DLA TRZECH CZCIGODNYCH}{-}
	{POKŁON PRZED TRZEMA SZLACHETNYMI}{-}
	{Sanzon Raimon}

	\begin{Verse}
	\stanza{
		Namah Wielkiemu Błogosławionemu Mistrzowi Nauk Pierwotnemu Mistrzowi Siakiamuniemu Buddzie,\\
		Namah Wysokiemu Patriarsze Wielkiemu Mistrzowi Dziojo,\\
		Namah Wielkiemu Patriarsze Wielkiemu Mistrzowi Dziosai.\\
		Gromadzimy i otrzymujemy Wielkie Współczucie i Wielką Litość,\\
		Obyśmy przez wszystkie żywoty i we wszystkich światach spotykali je i osiągali.
	}
	\end{Verse}
\end{Prayer}

%%%%%%%%%%%%%%%%%%%%%%%%%%%%%%%%%%%%%%%%%%%%%%%%%%%%%%%%%%%%%%%%%%%%%%%%%%%%%%%
%%%% RAIHAI GE
\begin{Prayer}{raihaige}
	{RAIHAI GE}{-}
	{Strofa Pokłonów}

	\begin{Verse}\wersaliki
	\stanza{
		NO RAI SHORAI SHO KU JAKU	\\
		JISHIN TASHIN TAI MU NI		\\
		GANGU SHU JO TOKU GEDATSU	\\
		HOTSU MUJ\=OI KI SHIN SAI.
	}
	\end{Verse}
\end{Prayer}

%%%%%%%%%%%%%%%%%%%%%%%%%%%%%%%%%%%%%%%%%%%%%%%%%%%%%%%%%%%%%%%%%%%%%%%%%%%%%%%
%%%% STROFA POKŁONÓW !
\begin{Prayer}{strofa_poklonow}
	{MODLITWA POKŁONÓW}{-}
	{RAIHAI GE}

	\begin{Verse}
	\stanza{
		Natura pokłonów i miejsce pokłonów są puste i wyzwolone.\\
		Własne ciało i ciała innych nie są dwoma.\\
		Ślubuję razem z odczuwającymi istotami osiągnąć wyzwolenie\\
		I rozwinąć najwyższy umysł i schronienie w ostatecznej prawdzie.
	}
	\end{Verse}
\end{Prayer}

%%%%%%%%%%%%%%%%%%%%%%%%%%%%%%%%%%%%%%%%%%%%%%%%%%%%%%%%%%%%%%%%%%%%%%%%%%%%%%%
%%%% KAIKYOGE
\begin{Prayer}{kaikyoge}
	{KAIKYOGE}{-}
	{Gatha Otwarcia Sutr}

	\begin{Verse}\wersaliki
	\stanza{
		M\=U J\=O JIN JIN M\=I MY\=O H\=O	\\
		HYAKU SEN MAN G\=O NAN S\=O G\=U	\\
		G\=A KON KEN MON TOKU J\=U J\=I		\\
		GAN G\=E NY\=O-RAI SHIN JITSU G\=I.
	}
	\end{Verse}
\end{Prayer}

%%%%%%%%%%%%%%%%%%%%%%%%%%%%%%%%%%%%%%%%%%%%%%%%%%%%%%%%%%%%%%%%%%%%%%%%%%%%%%%
%%%% GATHA OTWARCIA SUTR !
\begin{Prayer}{gatha_otwarcia_sutr}
	{GATHA OTWARCIA SUTR}{-}
	{Kaikyoge}

	\begin{Verse}
	\stanza{
		Nieprześcigniona, niezwykle głęboka i całkowicie wspaniała Dharma,\\
		Jest trudna do napotkania nawet w setkach tysięcy miliardów kalp.\\
		Teraz ją widzimy, słyszymy, przyjmujemy i podtrzymujemy,\\
		Ślubujemy zrozumieć prawdziwe znaczenie Tathagaty.
	}
	\end{Verse}
\end{Prayer}

%%%%%%%%%%%%%%%%%%%%%%%%%%%%%%%%%%%%%%%%%%%%%%%%%%%%%%%%%%%%%%%%%%%%%%%%%%%%%%%
%%%% SANGEMON
\begin{Prayer}{sangemon}
	{SANGEMON}{-}
	{Gatha Skruchy}

	\begin{Verse}\wersaliki
	\stanza{
		\textbf{GA SHAKU SHO Z\=O SHO AKU G\=O.}	\\
		KAI Y\=U MU SHI TON JIN CHI.	\\
		J\=U SHIN KU I~SHI SHO SH\=O.	\\
		IS-SAI GO KON KAI SAN-GE.
	}
	\end{Verse}
\end{Prayer}

%%%%%%%%%%%%%%%%%%%%%%%%%%%%%%%%%%%%%%%%%%%%%%%%%%%%%%%%%%%%%%%%%%%%%%%%%%%%%%%
%%%% MODLITWA SKRUCHY
\begin{Prayer}{gatha_skruchy}
	{MODLITWA SKRUCHY}{-}
	{Sangemon}

	\begin{Verse}
	\stanza{
		Wszystkie złe czyny popełnione przeze mnie w przeszłości,\\
		wszystkie mające przyczynę w nie mającej początku\\
		chciwości, gniewie i głupocie,\\
		narodzone z ciała, mowy i umysłu,\\
		wszystkie teraz całkowicie wyznaję ze skruchą.
	}
	\end{Verse}
\end{Prayer}

%%%%%%%%%%%%%%%%%%%%%%%%%%%%%%%%%%%%%%%%%%%%%%%%%%%%%%%%%%%%%%%%%%%%%%%%%%%%%%%
%%%% SANKIE MON
\begin{Prayer}{sankiemon}
	{SANKIE MON / SANKIKAI}{-}
	{Trzy Schronienia}

	\begin{Verse}\wersaliki
	\stanza{
		NAMU KIE BUTSU\\
		NAMU KIE H\=O\\
		NAMU KIE S\=O.
	}
% !!! skąd ta wersja jest? sprawdzić w soto shu scriptures
	\stanza{
		KIE BUTSU MU J\=O SON\\
		KIE H\=ORIJIN SON (lub: KIE HO RI YOKU SON)\\
		KIE S\=OWA G\=O SON.
	}

	\stanza{
		KIE BUKKY\=O\\
		KIE H\=OKY\=O\\
		KIE S\=OKY\=O.
	}
	\end{Verse}
\end{Prayer}

%%%%%%%%%%%%%%%%%%%%%%%%%%%%%%%%%%%%%%%%%%%%%%%%%%%%%%%%%%%%%%%%%%%%%%%%%%%%%%%
%%%% TRZY SCHRONIENIA
\begin{Prayer}{trzy_schronienia}
	{TRZY SCHRONIENIA}{-}
	{Sankie mon}

	\begin{Verse}
	\stanza{
		Z czcią przyjmuję schronienie w~Buddzie\\
		Z czcią przyjmuję schronienie w~Dharmie\\
		Z czcią przyjmuję schronienie w~Sandze
	}

	\stanza{
		Znajduję schronienie w Szlachetnym Buddzie, Najwyższym Nauczycielu.\\
		Znajduję schronienie w Szlachetnej Dharmie, oddzielającej nieczystości.\\
		Znajduję schronienie w Szlachetnej Sandze pokoju i~harmonii.
	}

	\stanza{
		Znalazłem schronienie w~Buddzie\\
		Znalazłem schronienie w~Dharmie\\
		Znalazłem schronienie w~Sandze
	}
	\end{Verse}
\end{Prayer}

%%%%%%%%%%%%%%%%%%%%%%%%%%%%%%%%%%%%%%%%%%%%%%%%%%%%%%%%%%%%%%%%%%%%%%%%%%%%%%%
%%%% SHARIRAIMON
% !!! SHARI RAI MON
\begin{Prayer}{shariraimon}
	{SHARIRAIMON}{-}
	{Modlitwa przed Relikwiami Buddhy}

	\begin{Verse}\wersaliki
	\stanza{
		\textbf{IS-SHIN} CH\=O RAI		\\
		MAN TOKU EN MAN			\\
		\keisu SH\=A-K\=A NY\=O RAI	\\
		SHIN JIN SH\=A-R\=I		\\
		HON J\=I HOS-SHIN		\\
		HOK-KAI T\=O B\=A		\\
		G\=A T\=O RAI KY\=O		\\
		\=I G\=A GEN SHIN		\\
		NY\=U G\=A G\=A NY\=U		\\
		BUTSU G\=A J\=I K\=O		\\
		G\=A SH\=O B\=O DAI		\\
		\=I BUTSU JIN RIKI		\\
		R\=I YAKU SH\=U J\=O		\\
		\shokei HOTSU B\=O DAI SHIN	\\
		SH\=U B\=O-SATSU GY\=O		\\
		D\=O NY\=U EN JAKU		\\
		\shokei BY\=O D\=O DAI CH\=I	\\
		KON J\=O CH\=O RAI
% !!! dodatkowo:
% HOJA SHIN KIN
% HANNYA SHIN GO
	}
	\end{Verse}
\end{Prayer}

\newpage
%%%%%%%%%%%%%%%%%%%%%%%%%%%%%%%%%%%%%%%%%%%%%%%%%%%%%%%%%%%%%%%%%%%%%%%%%%%%%%%
%%%% MODLITWA PRZED RELIKWIAMI BUDDHY, SIARIRA
\begin{Prayer}{przed_relikwiami}
	{MODLITWA PRZED RELIKWIAMI BUDDHY, SIARIRA}{-}
	{Shariraimon}

\medskip
Całym sercem-umysłem, czynimy pokłon przed \keisu\footnote{Uderzenie w~keisu
przy pierwszym i~trzecim razie} Siakiamunim Tathgatą, mającym pełne
i~niepoliczalne zasługi, przed prawdziwymi relikwiami siarira, pierwotną
Dharmakają, Ciałem Prawdy Buddhy i~stupą wszechświata~-- dharmadatu. Z~czcią
i~szacunkiem czynimy pokłon, by w~naszym obecnym ciele wejść w~świat Prawdy
i~by świat Prawdy wstąpił w~nas.


Łącząc się z~Buddhą osiągniemy Oświecenie Bodhi, dzięki duchowej mocy
Buddhy, my i~żyjące istoty otrzymujemy pomoc, budzi się w~nas umysł
Oświecenia, praktykujemy czyny \shokei Bodhisattwów i~razem wchodzimy
w~Nirwanę, \shokei w~świat spokoju i~Wielkiej Mądrości.


Teraz i~w przyszłości czynimy pokłon z~czcią i~szacunkiem.
\end{Prayer}

% !!! Kanzen - jak dokładnie mają w~tym powyżej lecieć shokei!!!

%%%%%%%%%%%%%%%%%%%%%%%%%%%%%%%%%%%%%%%%%%%%%%%%%%%%%%%%%%%%%%%%%%%%%%%%%%%%%%%
%%%% PIEŚŃ KLEJNOTU BUDDHY
\begin{Prayer}{piesn_klejnotu_buddy}
	{PIEŚŃ KLEJNOTU BUDDHY}{-}
	{-}

	\begin{Verse}
	\stanza{
		Ponad niebiosami i pod niebiosami nie ma niczego, co można by porównać z Buddhą.\\
		W dziesięciu kierunkach świata jest nieprześcigniony.\\
		Żadna z istot na świecie nie jest do niego podobna.\\
		Chwała Nauczycielowi trzech światów w świecie saha,\\
		Współczującemu Ojcu czterech rodzajów istot,\\
		Mistrzowi ludzi i bogów, posiadającemu trzy ciała przemienienia,\\
		Pierwotnemu Mistrzowi Siakiamuniemu Buddzie.\\
		Namu Honsi Siakiamuni Butsu.
	}
	\end{Verse}
\end{Prayer}

%%%%%%%%%%%%%%%%%%%%%%%%%%%%%%%%%%%%%%%%%%%%%%%%%%%%%%%%%%%%%%%%%%%%%%%%%%%%%%%
%%%% PIEŚŃ OBMYWANIA BUDDHY
\begin{Prayer}{piesn_obmywania_buddy}
	{PIEŚŃ OBMYWANIA BUDDHY}{-}
	{-}
	
	\begin{Verse}
	\stanza{
		Teraz obmywam wszystkich Tathagatów\\
		ozdobionych czystą Mądrością, którzy zgromadzili Cnotę Zasługi,\\
		niech odczuwające istoty pięciu skalań\\
		oddzielą się od nieczystości i urzeczywistnią tak samo Dharmakaję --
		Czyste Ciało Dharmy Tathagaty.\\
		Z czcią czynimy głęboki pokłon przed\\
		Wielkim Świętym Bhagawatem,\\
		Najwyższym ponad Niebiosami i pod Niebiosami.\\
		Teraz wodą Cnoty Zasługi\\
		Obmywamy Czyste Ciało Dharmy Tathagaty.
	}
	\end{Verse}
\end{Prayer}

%%%%%%%%%%%%%%%%%%%%%%%%%%%%%%%%%%%%%%%%%%%%%%%%%%%%%%%%%%%%%%%%%%%%%%%%%%%%%%%
%%%% JYU BUTSU MYO
% !!! JU BUTSU MYO
\begin{Prayer}{jyu_butsu_myo}
	{JYU BUTSU MYO}{-}
	{Dziesięć Imion Buddhy}

% !!! w niem.:
% 1. MAKA HANNYA HARAMITA SHIN GYO
% 2. JU BUTSU MYO

	\begin{Verse}\wersaliki
	\stanza{
		SHIN JIN PA SHIN BIR\=USH\=A N\=O F\=U	\\
		ENMON H\=OSHIN RUSH\=A N\=O F\=U	\\
		SEN PA I~KASHIN SHIKY\=AM\=UNI F\=U	\\
		T\=ORAI A~SAN MIRUSON B\=U		\\
		J\=IH\=O SAN SH\=I ISH\=I SH\=I F\=U	\\
		DAI JIN MY\=OHA RING\=AKIN		\\
		DAI SHIN BUNJUSUR\=I B\=US\=A		\\
		DAI JIN FUEN B\=US\=A			\\
		DAI HI KANSHIIN B\=US\=A		\\
		SH\=I SON B\=US\=A M\=OK\=OS\=A		\\
		M\=OK\=O H\=OJA H\=OR\=OM\=I
	}
	\end{Verse}
\end{Prayer}

%%%%%%%%%%%%%%%%%%%%%%%%%%%%%%%%%%%%%%%%%%%%%%%%%%%%%%%%%%%%%%%%%%%%%%%%%%%%%%%
%%%% DZIESIĘĆ IMION BUDDÓW
\begin{Prayer}{dziesiec_imion_buddy}
	{DZIESIĘĆ IMION BUDDHÓW}{-}
	{Jyu Butsu Myo}

	\begin{Verse}
	\stanza{
		Całkowicie Czysta Dharmakaya Wairoczana Buddha, \\
		Doskonale Pełne Ciało Odpowiedzi Sambhogakaya Wairoczana Buddha,\\
		Dziesięć Tysięcy Bilionów Ciał Przemienienia Nirmanakaya Siakiamuni Buddha,\\
		Mający narodzić się w przyszłości Maitreja Buddha,\\
		Wszyscy Buddhowie w dziesięciu kierunkach trzech światach,\\
		Mahajana Saddharma Pundarika Sutra, \\
		Wielki Święty Mańdziuśri Bodhisattwa, \\
		Wielki Pojazd Mahajana Samantabhadra Bodhisattwa,\\
		Wielki Współczujący Mahakaruna Awalokiteśwara Bodhisattwa,\\
		Wszyscy Szlachetni Bodhisattwowie Mahasattwowie,\\
		Wielka Doskonała Mądrość.
	}
	\end{Verse}
\end{Prayer}

%%%%%%%%%%%%%%%%%%%%%%%%%%%%%%%%%%%%%%%%%%%%%%%%%%%%%%%%%%%%%%%%%%%%%%%%%%%%%%%
%%%% SANKIRAI (Koonji, 68-69)
\begin{Prayer}{sankirai}
	{SANKIRAI}{-}
	{}

	\begin{Verse}\wersaliki
	\stanza{
		MIZUKARA HOTOKE NI KI E SHI TATEMATSURU. \\
		MASA NI NEGAWAKUWA SHIYUJY\=O.\\
		DAI D\=O O DAI GE SHITE. \\
		MU JY\=O SHIN O HATSU SEN KOTO??? O.\\

		MIZUKARA H\=O NI KI E SHI TATEMATSURU.\\
		MASA NI NEGAWAKUWA SHIYUJY\=O.\\

		FUKAKU KY\=O Z\=O NI ITSUTE.\\
		CHI E UMI NO GOTOKU NARANKOTO O\\
		MIZUKARA S\=O NI KI E SHI TATEMATSURU.\\
		MASA NI NEGAWAKUWA SHIYUJY\=O.\\
		DAI SHIYU O T\=O RI SHITE.\\
		ITSU SAI MU GE NARANKOTO O.
	}
	\end{Verse}
\end{Prayer}


%%%%%%%%%%%%%%%%%%%%%%%%%%%%%%%%%%%%%%%%%%%%%%%%%%%%%%%%%%%%%%%%%%%%%%%%%%%%%%%
%%%% SHIGUSEIGANMON
\begin{Prayer}{shiguseiganmon}
	{SHIGUSEIGANMON}{-}
	{Cztery Ślubowania}

	\begin{Verse}\wersaliki
	\stanza{
		\textbf{SHU-J\=O MU-HEN SEI-GAN DO}	\\
		BON-N\=O MU-JIN SEI-GAN DAN	\\
		H\=O-MON MU-RY\=O SEI-GAN GAKU	\\
		BUTSU-D\=O MU-J\=O SEI-GAN J\=O
	}
	\end{Verse}
\end{Prayer}

%%%%%%%%%%%%%%%%%%%%%%%%%%%%%%%%%%%%%%%%%%%%%%%%%%%%%%%%%%%%%%%%%%%%%%%%%%%%%%%
%%%% CZTERY ŚLUBOWANIA !
\begin{Prayer}{cztery_slubowania}
	{CZTERY ŚLUBOWANIA}{-}
	{Shiguseiganmon}

	\begin{Verse}
	\stanza{
		Niezliczone istoty ślubuję wyzwolić,\\
		Nieskończone klesia ślubuję odciąć,\\
		Niepoliczalne bramy Dharmy ślubuję zrozumieć,\\
		Nieprześcignioną Drogę Buddhy ślubuję osiągnąć.
	}
	\end{Verse}
\end{Prayer}

%%%%%%%%%%%%%%%%%%%%%%%%%%%%%%%%%%%%%%%%%%%%%%%%%%%%%%%%%%%%%%%%%%%%%%%%%%%%%%%
%%%% KAN NIN FU MON PIN KIN
%%%% Daikan
\begin{Prayer}{kai_nin_fu_mon_pin_kin}
	{KAN NIN FU MON PIN KIN\\KANNON GYO}{KAN NIN FU MON PIN KIN}
	{Sutra Lotosu Wspaniałego Prawa; Rozdział: Powszechna Brama Awalokiteśwary, Obserwującego Dźwięki Świata Bodhisattwy}

\medskip
\begin{JAPANESE}
{\bf MYO HO REN GE KYO}\\
KAN ZE ON BO SA FU MON BON DAI NI JU GO NI JI MU JIN NI  BO SA SOKU JU ZA KI
HEN DAN U KEN GA SHO KO BUTSU NI SA ZE GON SE SON KAN ZE ON BO SA I GA IN NEN
MYO KAN ZE ON BUTSU GO MU JIN NI BO SA ZEN NAN SHI NYAKU U MU RYO HYAKU SEN MAN
NOKU SHU JO JU SHO KU NO MON ZE KAN ZE ON BO SA IS SHIN SHO MYO KAN ZE ON BO SA
SOKU JI KAN GO ON JO KAI TOKU GE DATSU NYAKU U JI ZE KAN ZE ON BO SA MYO SHA
SETSU NYU DAI KA KA FU NO SHO YU ZE BO SA I JIN RIKI KO NYAKU I DAI SUI SHO HYO
SHO GO MYO GO SOKU TOKU SEN SHO NYAKU U HYAKU SEN MAN NOKU SHU JO I GU KON GON
RU RI SHA KO ME NO SAN GO KO HAKU SHIN JU TO HO NYU O DAI KAI KE SHI KOKU FU
SUI GO SEN PO HYO DA RA SETSU KI KOKU GO CHU NYAKU U NAI SHI ICHI NIN SHO KAN
ZE ON BO SA MYO SHA ZE SHO NIN TO KAI TOKU GE DATSU RA SETSU SHI NAN I ZE IN
NEN MYO KAN ZE ON NYAKU BU U NI RIN TO HI GAI SHO KAN ZE ON BO SA MYO SHA HI
SHO SHU TO JO JIN DAN DAN E NI TOKU GE DATSU NYAKU SAN ZEN DAI SEN KOKU DO MAN
CHU YA SHA RA SETSU YOKU RAI NO NIN MON GO SHO KAN ZE ON BO SA MYO SHA ZE SHO
AK KI SHO FU NO I AKU GEN JI SHI KYO BU KA GAI SETSU BU U NI NYAKU U ZAI NYAKU
MU ZAI CHU KAI KA SA KEN GE GO SHIN SHO KAN ZE ON BO SA MYO SHA KAI SHITSU DAN
E SOKU TOKU GE DATSU NYAKU SAN ZEN DAI SEN KOKU DO MAN CHU ON ZOKU U ICHI SHO
SHU SHO SHO SHO NIN SAI JI JU HO KYO KA KEN RO GO CHU ICHI NIN SA ZE SHO GON
SHO ZEN NAN SHI MOT TOKU KU FU NYO TO O TO IS SHIN SHO KAN ZE ON BO SA MYO GO
ZE BO SA NO I MU I SE O SHU JO NYO TO NYAKU SHO MYO SHA O SHI ON ZOKU TO TOKU
GE DATSU SHU SHO NIN MON GU HOS SHO GON NA MU KAN ZE ON BO SA SHO GO MYO GO
SOKU TOKU GE DATSU MU JIN NI KAN ZE ON BO SA MA KA SA I JIN SHI RIKI GI GI NYO
ZE NYAKU U SHU JO TA O IN YOKU JO NEN KU GYO KAN ZE ON BO SA BEN TOKU RI YOKU
NYAKU TA SHIN NI JO NEN KU GYO KAN ZE ON BO SA BEN TOKU RI SHIN NYAKU TA GU CHI
JO NEN KU GYO KAN ZE ON BO SA BEN TOKU RI CHI MU JIN NI KAN ZE ON BO SA U NYO
ZE TO DAI I JIN RIKI TA SHO NYO YAKU ZE KO SHU JO JO O SHIN NEN NYAKU U NYO NIN
SE CHOKU GU NAN RAI HAI KU YO KAN ZE ON BO SA BEN SHO FUKU TOKU CHI E SHI NAN
SEC CHOKU GU NYO BEN SHO TAN SO U SO SHI NYO SHUKU JIKI TOKU HON SHU NIN AI KYO
MU JIN NI KAN ZE ON BO SA U NYO ZE RIKI NYAKU U SHU JO KU GYO RAI HAI KAN ZE ON
BO SA FUKU FU TO EN ZE KO SHU JO KAI O JU JI KAN ZE ON BO SA MYO GO MU JIN NI
NYAKU U NIN JU JI ROKU JU NI OKU GO GA SHA BO SA MYO JI BU JIN GYO KU YO ON
JIKI E FUKU GA GU I YAKU O NYO I UN GA ZE ZEN NAN SHI ZEN NYO NIN KU DOKU TA FU
MU JIN NI GON JIN TA SE SON BUTSU GON NYAKU BU U NIN JU JI KAN ZE ON BO SA MYO
GO NAI SHI ICHI JI RAI HAI KU YO ZE NI NIN FUKU SHO TO MU I O HYAKU SEN MAN
NOKU GO FU KA GU JIN MU JIN NI JU JI KAN ZE ON BO SA MYO GO TOKU NYO ZE MU RYO
MU HEN FUKU TOKU SHI RI MU JIN NI BO SA BYAKU BUTSU GON SE SON KAN ZE ON BO SA
UN GA YU SHI SHA BA SEI KAI UN GA NI I SHU JO SEP PO HO BEN SHI RIKI GO JI UN
GA BUTSU GO MU JIN NI BO SA ZEN NAN SHI NYAKU U KOKU DO SHU JO O I BUS SHIN
TOKU DO SHA KAN ZE ON BO SA SOKU GEN BUS SHIN NI I SEP PO O I BYAKU SHI BUS
SHIN TOKU DO SHA SOKU GEN BYAKU SHI BUS SHIN NI I SEP PO O I SHO MON SHIN TOKU
DO SHA SOKU GEN SHO MON SHIN NI I SEP PO O I BON NO SHIN TOKU DO SHA SOKU GEN
BON NO SHIN NI I SEP PO O I TAI SHAKU SHIN TOKU DO SHA SOKU GEN TAI SHAKU SHIN
NI I SEP PO O I JI ZAI TEN SHIN TOKU DO SHA SOKU GEN JI ZAI TEN SHIN NI I SEP
PO O I DAI JI ZAI TEN SHIN TOKU DO SHA SOKU GEN DAI JI ZAI TEN SHIN NI I SEP PO
O I TEN DAI SHO GUN SHIN TOKU DO SHA SOKU GEN TEN DAI SHO GUN SHIN NI I SEP PO
O I BI SHA MON SHIN TOKU DO SHA SOKU GEN BI SHA MON SHIN NI I SEP PO O I SHO O
SHIN TOKU DO SHA SOKU GEN SHO O SHIN NI I SEP PO O I CHO JA SHIN TOKU DO SHA
SOKU GEN CHO JA SHIN NI I SEP PO O I KO JI SHIN TOKU DO SHA SOKU GEN KO JI SHIN
NI I SEP PO O I SAI KAN SHIN TOKU DO SHA SOKU GEN SAI KAN SHIN NI I SEP PO O I
BA RA MON SHIN TOKU DO SHA SOKU GEN BA RA MON SHIN NI I SEP PO O I BI KU BI KU
NI U BA SOKU U BA I SHIN TOKU DO SHA SOKU GEN BI KU BI KU NI U BA SOKU U BA I
SHIN NI I SEP PO O I CHO JA KO JI SAI KAN BA RA MON BU NYO SHIN TOKU DO SHA
SOKU GEN BU NYO SHIN NI I SEP PO O I DO NAN DO NYO SHIN TOKU DO SHA SOKU GEN DO
NAN DO NYO SHIN NI I SEP PO O I TEN RYU YA SHA KEN DATSU BA A SHU RA KA RU RA
KIN NA RA MA GO RA KA NIN PI NIN TO SHIN TOKU DO SHA SOKU KAI GEN SHIN NI I SEP
PO O I SHU KON GO SHIN TOKU DO SHA SOKU GEN SHU KON GO SHIN NI I SEP PO MU JIN
NI ZE KAN ZE ON BO SA JO JU NYO ZE KU DOKU I SHU JU GYO YU SHO KOKU DO DO DATSU
SHU JO ZE KO NYO TO O TO IS SHIN KU YO KAN ZE ON BO SA ZE KAN ZE ON BO SA MA KA
SA O FU I KYU NAN SHI CHU NO SE MU I ZE KO SHI SHA BA SE KAI KAI GO SHI I SE MU
I SHA MU JIN NI BO SA BYAKU BUTSU GON SE SON GA KON TO KU YO KAN ZE ON BO SA
SOKU GE KYO SHU HO SHU YO RAKU KE JIKI HYAKU SEN RYO GON NI I YO SHI SA ZE GON
NIN SHA JU SHI HOS SE CHIN HO YO RAKU JI KAN ZE ON BO SA FU KO JU SHI MU JIN NI
BU BYAKU KAN ZE ON BO SA GON NIN SHA MIN GA TO KO JU SHI YO RAKU NI JI BUTSU GO
KAN ZE ON BO SA TO MIN SHI MU JIN NI BO SA GYU SHI SHU TEN RYU YA SHA KEN DATSU
BA A SHU RA KA RU RA KIN NA RA MA GO RA KA NIN PI NIN TO KO JU ZE YO RAKU SOKU
JI KAN ZE ON BO SA MIN SHO SHI SHU GYU O TEN RYU NIN PI NIN TO JU GO YO RAKU
BUN SA NI BUN ICHI BUN BU SHA KA MUNI BUTSU ICHI BUN BU TA HO BUT TO MU JIN NI
KAN ZE ON BO SA U NYO ZE JI ZAI JIN RIKI YU O SHA BA SE KAI NI JI MU JIN NI BO
SA I GE MON WATSU. {\bf SE SON MYO SO GU}. GA KON JU MON PI BUS SHI GA IN NEN
MYO I KAN ZE ON GU SOKU MYO SO SON GE TO MU JIN NI NYO CHO KAN NON GYO ZEN O
SHO HO SHO GU ZEI JIN NYO KAI RYAKU GO FU SHI GI JI TA SEN NOKU BUTSU HOTSU DAI
SHO JO GAN GA I NYO RAKU SETSU MON MYO GYU KEN SHIN SHIN NEN FU KU KA NO METSU
SHO U KU KE SHI KO GAI I SUI RAKU DAI KA KYO NEN PI KAN NON RIKI KA KYO HEN JO
CHI WAKU HYO RU KO KAI RYU GYO SHO KI NAN NEN PI KAN NON RIKI HA RO FU NO MOTSU
WAKU ZAI SHU MI BU I NIN SHO SUI DA NEN PI KAN NON RIKI NYO NICHI KO KU JU WAKU
HI AKU NIN CHIKU DA RAKU KON GO SEN NEN PI KAN NON RIKI FU NO SON ICHI MO WAKU
CHI ON ZOKU NYO KAKU SHU TO KA GAI NEN PI KAN NON RIKI GEN SOKU KI JI SHIN WAKU
SO O NAN KU RIN KYO YOKU JU JU NEN PI KAN NON RIKI TO JIN DAN DAN E WAKU SHU
KIN KA SA SHU SOKU HI CHU KAI NEN PI KAN NON RIKI SHAKU NEN TOKU GE DATSU SHU
SO SHO DOKU YAKU SHO YOKU GAI SHIN SHA NEN PI KAN NON RIKI GEN JAKU O HON NIN
WAKU GU AKU RA SETSU DOKU RYU SHO KI TO NEN PI KAN NON RIKI JI SHIP PU KAN GAI
NYAKU AKU JU I NYO RI GE SO KA FU NEN PI KAN NON RIKI SHITSU SO MU HEN HO GAN
JA GYU FUKU KATSU GE DOKU EN KA NEN NEN PI KAN NON RIKI JIN JO JI E KO UN RAI
GU SEI DEN GO BAKU JU DAI U NEN PI KAN NON RIKI O JI TOKU SHO SAN SHU JO HI KON
YAKU MU RYO KU HIS SHIN KAN NON MYO CHI RIKI NO GU SE KEN KU GU SOKU JIN TSU
RIKI KO SHU CHI HO BEN JIP PO SHO KOKU DO MU SETSU FU GEN SHIN SHU JU SHO AKU
SHU JI GOKU KI CHIKU SHO SHO RO BYO SHI KU I ZEN SHITSU RYO METSU SHIN KAN SHO
JO KAN KO DAI CHI E KAN HI KAN GYU JI KAN JO GAN JO SEN GO MU KU SHO JO KO E
NICHI HA SHO AN NO BUKU SAI FU KA FU MYO SHO SE KEN HI TAI KAI RAI SHIN JI I
MYO DAI UN JU KAN RO HO U METSU JO BON NO EN JO SHO KYO KAN SHO FU I GUN JIN
CHU NEN PI KAN NON RIKI SHU ON SHIT TAI SAN MYO ON KAN ZE ON BON NON KAI CHO ON
SHO HI SEI KEN NON ZE KO SHU JO NEN NEN NEN MOS SHO GI KAN ZE ON JO SHO O KU NO
SHI YAKU NO I SA E KO GU IS SAI KU DOKU JI GEN JI SHU JO FUKU JU KAI MU RYO ZE
KO O CHO RAI NI JI JI SHI BO SA SOKU JU ZA KI ZEN BYAKU BUTSU GON SE SON NYAKU
U SHU JO MON ZE KAN ZE ON BO SA BON JI ZAI SHI GO FU MON JI GEN JIN TSU RIKI
SHA TO CHI ZE NIN KU DOKU FU SHO BU SETSU ZE FU MON BON JI SHU CHU HACHI MAN
SHI SEN SHU JO KAI HOTSU MU TO DO A NOKU TA RA SAN MYAKU SAN BO DAI SHIN.
\end{JAPANESE}
\end{Prayer}

%%%%%%%%%%%%%%%%%%%%%%%%%%%%%%%%%%%%%%%%%%%%%%%%%%%%%%%%%%%%%%%%%%%%%%%%%%%%%%%
%%%% MY\=OH\=ORENGEKY\=O KANZEON BOSATSU FUMONBONGE
\begin{Prayer}{myohorengekyo_kanzeonbosatsu_fumonbonge}
	{MY\=OH\=ORENGEKY\=O KANZEON\\ BOSATSU FUMONBONGE}{-}
	{Sutra Lotosu Wspaniałego Prawa; Rozdział: Powszechna Brama Awalokiteśwary, Obserwującego Dźwięki Świata Bodhisattwy}

\bigskip

\begin{center}
\samepage{
	\small
	Skrócony tytuł: \textit{Fumonbon ge}\\
	(język chiński)
}
\end{center}

\begin{Verse}
\stanza{
	\keisu S\=e son my\=o s\=o g\=u	\\
	g\=a kon j\=u mon p\=\i		\\
	bus-sh\=\i g\=a in nen		\\
	my\=o \=\i kan z\=e on
}

\stanza{
	g\=u soku my\=o s\=o son	\\
	g\=e to m\=u jin n\=\i		\\
	ny\=o ch\=o kan non gy\=o	\\
	zen n\=o sh\=o ho sho
}

\stanza{
	g\=u zei jin ny\=o kai		\\
	ryak-k\=o f\=u sh\=\i g\=\i	\\
	j\=\i t\=a sen noku butsu	\\
	\keisu hotsu dai sh\=o j\=o gan
}

\stanza{
	g\=a \=\i ny\=o ryaku setsu	\\
	mon my\=o gy\=u ken shin	\\
	shin nen f\=u k\=u k\=a		\\
	n\=o mes sh\=o \=u k\=u
}

\stanza{
	k\=e sh\=\i k\=o gai \=\i	\\
	sui raku dai k\=a ky\=o		\\
	nen p\=\i kan non riki		\\
	k\=a ky\=o hen j\=o ch\=\i
}

\stanza{
	waku hy\=o r\=u k\=o kai	\\
	ry\=u gy\=o sh\=o k\=\i nan	\\
	nen p\=\i kan non riki		\\
	h\=a r\=o f\=u n\=o motsu
}

\stanza{
	waku zai sh\=u m\=\i b\=u	\\
	\=\i nin sh\=o sui d\=a		\\
	nen p\=\i kan non riki		\\
	nyo nichi k\=o k\=u j\=u
}

\stanza{
	waku h\=\i aku nin chiku	\\
	d\=a raku kon g\=o sen		\\
	nen p\=\i kan non riki		\\
	f\=u n\=o son ichi m\=o
}

\stanza{
	waku ch\=\i on zoku ny\=o	\\
	kaku sh\=u t\=o k\=a gai	\\
	nen p\=\i kan non riki		\\
	gen soku k\=\i j\=\i shin
}

\stanza{
	waku s\=o \=o nan k\=u		\\
	rin gy\=o yoku j\=u sh\=u	\\
	nen p\=\i kan non riki		\\
	t\=o jin dan dan \=e
}

\stanza{
	waku sh\=u kin k\=a s\=a	\\
	sh\=u soku h\=\i ch\=u kai	\\
	nen p\=\i kan non riki		\\
	shaku nen toku g\=e datsu
}

\stanza{
	shu s\=o sh\=o doku yaku	\\
	sh\=o yoku gai shin sha		\\
	nen p\=\i kan non riki		\\
	gen jaku \=o hon nin
}

\stanza{
	waku g\=u aku r\=a setsu	\\
	doku ry\=u sh\=o k\=\i t\=o	\\
	nen p\=\i kan non riki		\\
	j\=\i ship-p\=u kan gai
}

\stanza{
	nyaku aku j\=u \=\i nyo		\\
	r\=\i g\=e s\=o k\=a f\=u	\\
	nen p\=\i kan non riki		\\
	shis-s\=o m\=u hen p\=o
}

\stanza{
	gan ja gy\=u buk-katsu		\\
	k\=e doku en k\=a nen		\\
	nen p\=\i kan non riki		\\
	jin sh\=o j\=\i \=e k\=o
}

\stanza{
	un rai k\=u sei den		\\
	g\=o baku j\=u dai \=u		\\
	nen p\=\i kan non riki		\\
	\=o j\=\i toku sh\=o san
}

\stanza{
	sh\=u j\=o h\=\i kon yaku	\\
	m\=u ry\=o k\=u his-shin	\\
	kan non my\=o ch\=\i riki	\\
	n\=o g\=u s\=e ken k\=u
}

\stanza{
	g\=u soku jin z\=u riki		\\
	k\=o sh\=u ch\=\i h\=o ben	\\
	Jip-p\=o sh\=o koku d\=o	\\
	m\=u setsu f\=u gen shin
}

\stanza{
	shu j\=u sh\=o aku sh\=u	\\
	j\=\i goku k\=\i chiku sh\=o	\\
	sh\=o r\=o by\=o sh\=\i k\=u	\\
	\=\i zen shitsu ry\=o metsu
}

\stanza{
	shin kan sh\=o j\=o kan		\\
	k\=o dai ch\=\i \=e kan		\\
	h\=\i kan gyu j\=\i kan		\\
	j\=o gan j\=o sen g\=o
}

\stanza{
	m\=u k\=u sh\=o j\=o k\=o	\\
	\=e nichi ha sho an		\\
	n\=o buku sai f\=u k\=a		\\
	f\=u my\=o sh\=o s\=e ken
}

\stanza{
	h\=\i tak kai rai shin		\\
	j\=\i \=\i my\=o dai un		\\
	j\=u kan r\=o h\=o \=u		\\
	metsu j\=o bon n\=o en
}

\stanza{
	j\=o sh\=o ky\=o kan sho	\\
	f\=u \=\i gun jin ch\=u		\\
	nen p\=\i kan non riki		\\
	\keisu sh\=u on shit-tai san
}

\stanza{
	my\=o on kan z\=e on		\\
	bon non kai ch\=o on		\\
	sh\=o h\=\i s\=e ken non	\\
	z\=e k\=o sh\=u j\=o nen
}

\stanza{
	nen nen mos-sh\=o g\=\i		\\
	kan z\=e on j\=o sh\=o		\\
	\=o k\=u n\=o sh\=\i yaku 	\\
	n\=o \=\i s\=a \=e k\=o
}

\stanza{
	g\=u is-sai k\=u doku		\\
	j\=\i gen j\=\i sh\=u j\=o	\\
	fuku j\=u kai m\=u ry\=o 	\\
	z\=e k\=o \=o ch\=o rai
}
\end{Verse}

\bigskip
\begin{japanese}
	\keisu N\=\i j\=\i. j\=\i j\=\i b\=o s\=a. soku j\=u z\=a k\=\i. zen
	byaku butsu gon. s\=e son. nyaku \=u sh\=u j\=o. mon z\=e kan z\=e on
	b\=o s\=a hon.  j\=\i zai sh\=\i g\=o. f\=u mon j\=\i gen. jin z\=u
	riki sha. t\=o ch\=\i z\=e nin. k\=u doku f\=u sh\=o. bus-setsu z\=e
	f\=u mon hon j\=\i. \shokei sh\=u ch\=u hachi man sh\=\i sen sh\=u
	j\=o. kai hotsu m\=u t\=o d\=o \shokei \=a noku t\=a r\=a san myaku san
	b\=o dai shin.
\end{japanese}

\end{Prayer}

\newpage
%%%%%%%%%%%%%%%%%%%%%%%%%%%%%%%%%%%%%%%%%%%%%%%%%%%%%%%%%%%%%%%%%%%%%%%%%%%%%%%
%%%% SUTRA LOTOSU WSPANIAŁEGO PRAWA
\begin{Prayer}{sutralotosu_powszechna_brama}
	{SUTRA LOTOSU WSPANIAŁEGO PRAWA\\ROZDZIAŁ: POWSZECHNA BRAMA AWALOKITEŚWARY -- OBSERWUJĄCEGO DŹWIĘKI ŚWIATA -- BODHISATTWY}{-}
	{My\=oh\=orengeky\=o\\Kanzeonbosatsu Fumonbonge}

\bigskip
\keisu W tym czasie Aksiajamati~-- Niezmierzony Umysł~-- Bodhisattwa odkrył swe prawe
ramię i~ze złożonymi dłońmi zwrócił się do Buddhy:


,,Czczony Przez Świat. Z~jakiego powodu Awalokiteśwara~-- Obserwujący Dźwięki
Świata~-- Bodhisattwa jest tak nazywany?''


Buddha odpowiedział Aksiajamatiemu~-- Niezmierzonemu Umysłowi~-- Bodhisattwie:


\keisu ,,Dobry, cnotliwy synu. Jeśli niezmierzone tryliony istot żyjące w~cierpieniu
i~niepokoju, słysząc o~Awalokiteśwarze~-- Obserwującym Dźwięki Świata~--
Bodhisattwie, z~jednym sercem-umysłem wyrecytują: Awalokiteśwaro~-- Obserwujący
Dźwięki Świata~-- Bodhisattwo, wówczas usłyszy on dźwięk tego głosu i~wszyscy
osiągną wyzwolenie. Jeśli ktoś wpadnie w~wielki ogień, to ogień nie będzie mógł
go spalić dzięki nadprzyrodzonej mocy Bodhisattwy. Jeśli ktoś został zmyty
przez wielką wodę i~zawoła imię Bodhisattwy, to natychmiast znajdzie się
w~płytkiej wodzie. Jeśli niezliczone tryliony istot wyruszą na wielki ocean
w~poszukiwaniu złota, srebra, lazurytu, księżycowych kamieni, agatów, korali,
bursztynu, pereł i~innych skarbów i~czarny wiatr zepchnie je do krainy demonów
raksz, a~choćby jedna osoba zawołała imię Awalokiteśwary Bodhisattwy, to
wszyscy zostaną wyzwoleni z~opresji z~rakszami. Taka jest przyczyna imienia
Awalokiteśwary~-- Obserwującego Dźwięki Świata.


Jeśli ktoś może zostać zraniony i~zawoła imię Awalokiteśwary, to miecz
atakującego zostanie złamany, a~ta osoba osiągnie wyzwolenie.


Jeśli trzy tysiące wielkich tysięcy krain są wypełnione jakszami i~rakszami
pragnącymi zaszkodzić ludziom, a~ludzie ci zawołają imię Awalokiteśwary, to
wszystkie te demony nie będą mogły dostrzec ich swymi niedobrymi oczyma, a~cóż
dopiero skrzywdzić.


Jeśli jest ktoś, winny lub niewinny, zakuty w~kajdany, spętany w~łańcuchy
i~zawoła imię Awalokiteśwary Bodhisattwy, to okowy odpadną i~zostanie
uwolniony.


Jeśli trzy tysiące wielkich tysięcy światów pełne są wrogów i~rabusiów i~jest
jakiś przywódca kupców, który poprowadzi wielu z~nich z~ładunkiem drogocennych
kamieni wzdłuż niebezpiecznej drogi, a~znajdzie się pośród nich jeden człowiek,
który powie: ,,Dobrzy ludzie. Nie bójcie się. Z~jednym sercem- umysłem
recytujcie imię Awalokiteśwary Bodhisattwy, ponieważ ten Bodhisattwa może dodać
odwagi wszystkim żyjącym istotom. Jeśli recytujecie jego imię, zostaniecie
uwolnieni od wrogów i~rabusiów''. Jeśli wszyscy kupcy, słysząc to, jednym głosem
wyrecytują: \keisu Namah Awalokiteśwaraya Bodhisattwaya!~-- Chwała Awalokiteśwarze~--
Obserwującemu Dźwięki Świata~-- Bodhisattwie!~-- wówczas dzięki recytacji jego
imienia zostaną uwolnieni od niebezpieczeństwa. Aksiajamati! Taka jest tajemna,
nadprzyrodzona moc Awalokiteśwary~-- Obserwującego Dźwięki Świata~--
Bodhisattwy.


Jeśli ktoś ma wiele pragnień seksualnych, a~zatrzymuje w~umyśle i~czci
Awalokiteśwarę Bodhisattwę, to zostanie uwolniony od tych namiętności.


Jeśli ktoś ma skłonność do drażliwości i~gniewu, to utrzymując w~umyśle
i~czcząc Awalokiteśwarę Bodhisatwę, zostanie uwolniony od takich uczuć.


Jeśli ktoś popadnie w~szaleńczą miłość, a~zatrzyma w~umyśle i~będzie czcił
Awalokiteśwarę Bodhisattwę, zostanie uwolniony od tego szalonego uczucia.


Aksiajamati! Takie są obfite dobrodziejstwa Awalokiteśwary~-- Obserwującego
Dźwięki Świata~-- Bodhisatwy, dzięki jego wielkiej, nadprzyrodzonej duchowej
mocy. Niech wszystkie żyjące istoty utrzymują go w~swoich umysłach.


Jeśli jest jakaś kobieta, która pragnie mieć syna, czci Awalokiteśwarę
Bodhisattwę, modli się do niego i~składa mu ofiary, to urodzi syna, który
będzie szczęśliwy, cnotliwy i~mądry. Jeśli będzie pragnęła córki, to urodzi
córkę, która będzie się dobrze zachowywać i~dobrze wyglądać, która
w~przeszłości kultywowała korzeń cnoty zasługi i~będzie kochana i~szanowana
przez wszystkich.


Aksiajamati! Taka jest duchowa moc Awalokiteśwary~-- Obserwującego Dźwięki
Świata~-- Bodhisattwy. Jeśli ktoś czci, modli się i~czyni pokłony przed
Awalokiteśwarą Bodhisattwą, to jego szczęście nie jest puste. Dlatego niech
wszystkie istoty przyjmą i~zachowają imię Awalokiteśwary~-- Obserwującego
Dźwięki Świata~-- Bodhisattwy.


Aksiajamati! Przypuśćmy, że ktoś przyjmuje i~utrzymuje imiona tak wielu
Bodhisattwów, jak wiele jest ziaren piasku w~sześćdziesięciu dwóch miliardach
rzek Ganges i~przez całe życie składa im w~ofierze jedzenie, picie, ubrania,
pościel i~lekarstwa, to jak myślisz, czy wiele cnót zasługi zbiera taki dobry
mężczyzna lub dobra kobieta?''


Aksiajamati odpowiedział: ,,Bardzo wiele''.


Czczony Przez Świat Buddha powiedział: ,,A jednak, jeśli ktoś przyjmie
i~utrzyma imię Awalokiteśwary Bodhisattwy, albo raz uczci Awalokiteśwarę
Bodhisattwę, pomodli się, pokłoni i~złoży mu ofiarę, to szczęście obu tych
ludzi nie będzie różne i~nie zostanie wyczerpane przez setki tysięcy miliardów
kalp.


Aksiajamati! Takie jest niezmierzone i~nieograniczone szczęście kogoś, kto
przyjmuje i~utrzymuje imię Awalokiteśwary~-- Obserwującego Dźwięki Świata~--
Bodhisattwy.


Aksiajamati~-- Nieskończony Umysł~-- Bodhisattwa powiedział do Buddhy:
,,Czczony Przez Świat. W~jaki sposób Awalokiteśwara Bodhisattwa pojawia się
w~świecie saha? Jak wyjaśnia Dharmę żyjącym istotom? Jakiej mocy upaya używa do
tego?''


Buddha odpowiedział Aksiajamatiemu Bodhisattwie:


,,Dobry synu. Jeśli istoty w~jakiejś krainie mają być zbawione przez Buddhę, to
Awalokiteśwara Bodhisattwa pojawia się w~ciele Buddhy i~objaśnia im Dharmę.


Tym, których może wyzwolić w~postaci pratjekabuddhy, pojawia się jako
pratjekabuddha i~objaśnia im Dharmę.


Tym, których może wyzwolić w~postaci śrawaki, pojawia się jako śrawaka
i~objaśnia im Dharmę.


Tym, których może wyzwolić w~postaci Brahmy, pojawia się jako Brahma i~objaśnia
im Dharmę.


Tym, których może wyzwolić w~postaci Siakry, pojawia się jako Siakra i~objaśnia
im Dharmę.


Tym, których może wyzwolić w~postaci Iśwary, pojawia się jako Iśwara i~objaśnia
im Dharmę.


Tym, których może wyzwolić w~postaci Maheśwary, pojawia się jako Maheśwara
i~objaśnia im Dharmę.


Tym, których może wyzwolić w~postaci wielkiego niebiańskiego generała, pojawia
się jako wielki niebiański generał i~objaśnia im Dharmę.


Tym, których może wyzwolić w~postaci Waiśrawany, pojawia się jako Waiśrawana
i~objaśnia im Dharmę.


Tym, których może wyzwolić w~postaci małego króla, pojawia się jako mały król
i~objaśnia im Dharmę.


Tym, których może wyzwolić w~postaci starszego (rodu), pojawia się jako starszy
rodu i~objaśnia im Dharmę.


Tym, których może wyzwolić w~postaci świeckiego ucznia Buddhy, pojawia się jako
świecki uczeń Buddhy i~objaśnia im Dharmę.


Tym, których może wyzwolić w~postaci ministra stanu, pojawia się jako minister
stanu i~objaśnia im Dharmę.


Tym, których może wyzwolić w~postaci brahmina, pojawia się jako brahmin
i~objaśnia im Dharmę.


Tym, których może wyzwolić w~postaci mnicha, mniszki, upasaki lub upasiki,
pojawia się jako mnich, mniszka, upasaka lub upasika i~objaśnia im Dharmę.


Tym, których może wyzwolić w~postaci żony starszego rodu lub żony świeckiego
ucznia Buddhy, ministra lub brahmina, pojawia się jako taka żona i~objaśnia im
Dharmę.


Tym, których może wyzwolić w~postaci chłopca lub dziewczyny, pojawia się jako
chłopiec lub dziewczyna i~objaśnia im Dharmę.


Tym, których może wyzwolić w~postaci niebiańskiego smoka, jakszy, gandharwy,
asiury, garudy, kinnary, mahoragi, ludzkiej lub nie-ludzkiej, pojawia się
w~każdej z~tych form i~objaśnia im Dharmę.


Tym, których może wyzwolić w~postaci Wadżrapani, pojawia się jako Wadżrapani
i~objaśnia im Dharmę.


Aksiajamati! Taka jest moc zasługi osiągnięta przez Awalokiteśwarę~--
Obserwującego Dźwięki Świata~-- Bodhisattwę i~takie są różne formy, które
przybiera wędrując przez wiele światów, by wyzwolić odczuwające istoty. Dlatego
z~jednym sercem-umysłem powinieneś czynić ofiary Awalokiteśwarze Bodhisattwie.
Awalokiteśwara Bodhisattwa Mahasattwa~-- Obserwująca Dźwięki Świata Oświecona
Wielka Istota~-- może uczynić nieustraszonymi tych, którzy tkwią w~lęku
i~niepokoju. Dlatego w~tym świecie cierpienia saha nazywany jest Obdarzający
Nieustraszonością.''


Aksiajamati Bodhisattwa powiedział do Buddhy: ,,Czczony Przez Świat! Teraz chcę
złożyć ofiarę Awalokiteśwarze Bodhisattwie''. Po czym ściągnął naszyjnik z~pereł
wart setki tysięcy sztuk złota i~ofiarował mu go, mówiąc:


,,Szlachetny Panie! Przyjmij tę ofiarę Dharmy z~naszyjnika pereł''.
Awalokiteśwara Bodhisattwa nie przyjął ofiary. Aksiajamati ponownie powiedział
do Awalokiteśwary Bodhisattwy:


,,Szlachetny Panie! Ze względu na współczucie dla nas przyjmij ten naszyjnik''.
Wówczas Buddha powiedział Awalokiteśwarze Bodhisattwie: ,,Ze względu na
współczucie wobec Aksiajamatiego Bodhisattwy i~czterech grup zgromadzenia oraz
niebiańskich smoków, jaksz, gandharw, asur, garud, kinnar, mahorag, ludzkich
i~nie-ludzkich istot przyjmij ten naszyjnik''. Wtedy Awalokiteśwara Bodhisattwa,
mając współczucie dla wszystkich czterech grup, niebiańskich smoków, ludzi,
nie-ludzi i~innych przyjął tę ofiarę i~podzielił na dwie części. Jedną
ofiarował Siakiamuniemu Buddzie, a~drugą stupie Prabhutaratna Buddhy.


,,Aksiajamati! Z~takimi nadprzyrodzonymi duchowymi mocami Awalokiteśwara
Bodhisattwa wędruje w~tym świecie cierpienia saha.''


Wówczas Aksiajamati Bodhisattwa spytał wierszem:


,,Czczony Przez Świat, który posiadasz wszystkie wspaniałe znamiona. Ponownie
pytam się, z~jakiego powodu syn Buddhy, Awalokiteśwara~-- Obserwujący Dźwięki
Świata jest tak nazywany?''


Czczony Przez Świat, posiadający wszystkie znamiona wspaniałości odpowiedział
Aksiajamati wierszem:


\keisu ,,Posłuchaj o~czynach Awalokiteśwary, który dobrem odpowiada we wszystkich
kierunkach i~miejscach, którego szerokie ślubowania są tak głębokie, jak ocean
i~zdumiewająca jest jego historia.


Służył wielu tysiącom miliardów Buddhów i~złożył wielkie czyste ślubowanie,
które teraz wyjaśnię ci krótko.


Kto słyszy jego imię i~widzi jego ciało, pamięta o~nim nieprzerwanie, będzie
w~stanie położyć kres wszystkim cierpieniom i~niepokojom.


Choćby został złośliwie potraktowany i~wrzucony do olbrzymiego ognia, jeśli
pomyśli o~mocy Awalokiteśwary, ogień zmieni się w~staw.


Albo wciągnięty w~wielki ocean, zagrożony przez smoki, ryby i~wszystkie demony,
jeśli pomyśli o~mocy Awalokiteśwary, fale nie będą mogły go zatopić.


Lub jeśli ktoś zrzuci go ze szczytu góry Sumeru, a~on pomyśli o~mocy
Awalokiteśwary, zawiśnie na niebie podobnie jak słońce.


Lub gdy ścigany przez złych ludzi zostanie zrzucony z~Diamentowej Góry
i~pomyśli o~mocy Awalokiteśwary, nawet jeden włos nie spadnie mu z~głowy.


Gdy otoczony wrogami wznoszącymi miecze, gotowymi do uderzenia, pomyśli o~mocy
Awalokiteśwary, obudzi się w~nich współczucie.


Jeśli ktoś cierpi pod rządami władcy i~otrzyma wyrok śmierci, to gdy pomyśli
o~mocy Awalokiteśwary, miecz pęknie na kawałki.


Lub jeśli ktoś uwięziony, skrępowany i~skuty z~rękami i~nogami w~kajdanach
i~dybach pomyśli o~mocy Awalokiteśwary, nieoczekiwanie zostanie uwolniony.


Gdy ktoś usiłuje was zniszczyć mantrami, przeklęciami i~wszystkimi truciznami,
to pomyślcie o~mocy Awalokiteśwary, a~wszystkie te rzeczy zwrócą się przeciw
sprawcy.


Lub jeśli do kogoś podejdą złe raksze, trujące smoki i~wszystkie demony,
a~pomyśli on o~mocy Awalokiteśwary, żaden z~nich nie ośmieli się go skrzywdzić.


Jeśli ktoś jest otoczony przez dzikie bestie z~ostrymi kłami i~strasznymi
pazurami i~pomyśli o~mocy Awalokiteśwary, to rozpierzchną się one na wszystkie
strony.


Parzony płomieniem trucizn węży, żmii i~skorpionów, gdy pomyśli o~mocy
Awalokiteśwary, to nagle całe palenie ustanie.


Gdy rozlegają się przerażające grzmoty z~chmur i~rozbłyskują błyskawice na
niebie, pada grad lub wielki deszcz, gdy wówczas ktoś pomyśli o~mocy
Awalokiteśwary, w~odpowiedzi wszystko się skończy.


Gdy ludzie cierpią z~powodu nieszczęść i~niezliczonych cierpień fizycznych, to
moc mądrości Awalokiteśwary może uratować ich od cierpień świata.


Posiadając pełne nadprzyrodzone moce, szeroko używa mądrości różnych sposobów
nauczania upaja we wszystkich światach dziesięciu kierunków.


Tam, gdzie się przejawia różnego rodzaju całe zło, w~piekle, w~świecie głodnych
duchów, zwierząt i~cierpienia narodzin, starości, choroby i~śmierci, stopniowo
kończy się ono i~zanika.


Widzenie prawdy, widzenie całkowitej czystości oraz wielkiej nieograniczonej
mądrości, współczucia i~miłości, to są Jego wieczne ślubowania i~wieczne
pragnienie.


Awalokiteśwara to nieskazitelny, czysty promień, światło mądrości rozpraszające
wszelkie ciemności, mogące powstrzymać kataklizmy wiatru i~ognia, uniwersalne
światło rozświetlające cały świat.


Współczujący piorun i~wstrząs wskazań moralnej dyscypliny, wspaniały wielki
obłok współczującego serca, zsyłający deszcz Dharmy słodkiego nektaru amrity,
gaszący płomienie skalań~-- klesia.


Jeśli ktoś w~sądzie lub wylękniony pośród żołnierzy pomyśli o~mocy
Awalokiteśwary, to wszyscy, którzy go nienawidzą rozpierzchną się.


Awalokiteśwara to cudowny dźwięk, czysty dźwięk, dźwięk morskich fal, który
jest zwycięskim dźwiękiem świata, dlatego zawsze powinno się go powtarzać
w~myślach i~nigdy nie powinna zrodzić się w~myślach wątpliwość.


Awalokiteśwara~-- czysty, święty~-- może być oparciem w~cierpieniu, bólu,
śmierci i~niebezpieczeństwie.


Posiadający wszystkie cnoty zasługi, okiem współczucia spogląda na wszystkie
odczuwające istoty, przynosi ocean szczęścia, dlatego powinniśmy uczynić przed
nim pokłon.''


\keisu W tym czasie Dharanindhara~-- Dzierżący Ziemię Bodhisattwa powstał ze swego
miejsca, stanął przed Buddhą i~powiedział: ,,Czczony Przez Świat, jeśli jakaś
istota usłyszy o~Awalokiteśwarze~-- Obserwującym Dźwięki Świata~--
Bodhisattwie, o~jego czynach pełnych wolności, ukazujących powszechną bramę
nadprzyrodzonych duchowych mocy, to powinniśmy wiedzieć, że cnota zasługi
takiej istoty jest niemała''. Gdy Buddha wyjaśnił tę powszechną bramę, \shokei pośród
zgromadzenia osiemdziesięciu czterech tysięcy istot każda bez wyjątku obudziła
w~sobie umysł \shokei Doskonałego Nieprześcignionego Oświecenia~-- Anuttara Samiak
Sambodhi.

\begin{center}* * *\end{center}

Dobrą i~powszechną praktyką jest recytowanie imienia Awalokiteśwary
Bodhisattwy, zgodnie z~naukami XXV rozdziału Sutry Lotosu Wspaniałej Dharmy.
Codzienna recytacja powoduje powstanie głębokiego związku z~Bodhisattwą. Jego
wszystkie duchowe i~nadprzyrodzone moce, moc chronienia, uwalniania od
cierpienia i~moc przewodnictwa na Drodze Buddhy zawarte są w~Jego Imieniu.
Dlatego ludzie szukający Jego przewodnictwa mogą praktykować recytację Imienia
Awalokiteśwary Bodhisattwy zgodnie z~własnym życzeniem. 
\end{Prayer}

% (!!!) brakuje chyba jeszcze jednego myohore
\newpage
%%%%%%%%%%%%%%%%%%%%%%%%%%%%%%%%%%%%%%%%%%%%%%%%%%%%%%%%%%%%%%%%%%%%%%%%%%%%%%%
%%%% MY\=OH\=ORENGEKY\=O NYORAI JURY\=OHONGE
\begin{Prayer}{myohorengekyo_nyorai_juryohonge}
	{MY\=OH\=ORENGEKY\=O NYORAI JURY\=OHONGE}{-}
	{Sutra Lotosu Wspaniałego Prawa, rozdział Wieczne Życie	Tathagathy}

\begin{center}
\samepage{
	\small
	{\it Jury\=ohon ge}\\
	(język chiński)
}
\end{center}

\begin{Verse}
\stanza{
	J\=\i g\=a toku butsu rai	\\
	sho ky\=o sh\=o k\=o shu	\\
	m\=u ry\=o hyaku sen man	\\
	oku sai \=a s\=o g\=\i
}

\stanza{
	j\=o sep-po ky\=o k\=e		\\
	m\=u sh\=u oku sh\=u j\=o	\\
	ry\=o ny\=u \=o butsu d\=o	\\
	\keisu n\=\i rai m\=u ry\=o k\=o
}

\stanza{
	\=i d\=o sh\=u j\=o k\=o	\\
	h\=o ben gen n\=e han		\\
	n\=\i jitsu f\=u metsu d\=o	\\
	j\=o j\=u sh\=\i sep-p\=o
}

\stanza{
	g\=a j\=o j\=u \=o sh\=\i	\\
	\=\i sh\=o jin z\=u riki	\\
	ry\=o ten d\=o sh\=u j\=o	\\
	sui gon n\=\i f\=u ken
}

\stanza{
	sh\=u ken g\=a metsu d\=o	\\
	k\=o k\=u y\=o sh\=a r\=\i	\\
	gen kai \=e ren b\=o		\\
	n\=\i sh\=o katsu g\=o shin
}

\stanza{
	sh\=u j\=o k\=\i shin buku	\\
	shitsu jiki \=\i ny\=u nan	\\
	is-shin yoku ken butsu		\\
	f\=u j\=\i shaku shin my\=o	\\
	j\=\i g\=a gy\=u sh\=u s\=o	\\
	g\=u shutsu ry\=o j\=u sen
}

\stanza{
	g\=a j\=\i g\=o sh\=u j\=o	\\
	j\=o zai sh\=\i f\=u metsu	\\
	\=i h\=o ben riki k\=o		\\
	gen n\=u metsu f\=u metsu
}

\stanza{
	y\=o koku \=u sh\=u j\=o	\\
	k\=u gy\=o shin gy\=o sha	\\
	g\=a b\=u \=o h\=\i ch\=u	\\
	\=\i setsu m\=u j\=o h\=o
}

\stanza{
	ny\=o t\=o f\=u mon sh\=\i	\\
	tan n\=\i g\=a metsu d\=o	\\
	g\=a ken sh\=o sh\=u j\=o	\\
	motsu zai \=o k\=u kai
}

\stanza{
	k\=o f\=u \=\i gen shin		\\
	ry\=o g\=o sh\=o katsu g\=o	\\
	in g\=o shin ren b\=o		\\
	nai shutsu \=\i sep-p\=o
}

\stanza{
	jin z\=u riki ny\=o z\=e	\\
	\=o \=a s\=o g\=\i k\=o		\\
	j\=o zai ry\=o j\=u sen		\\
	gy\=u y\=o sh\=o j\=u sh\=o
}

\stanza{
	sh\=u j\=o ken k\=o jin		\\
	dai k\=a sh\=o sh\=o j\=\i	\\
	g\=a sh\=\i d\=o an non		\\
	ten nin j\=o j\=u man
}

\stanza{
	on rin sh\=o d\=o kaku		\\
	shu ju h\=o sh\=o gon		\\
	h\=o j\=u t\=a k\=e k\=a	\\
	sh\=u j\=o sh\=o y\=u raku
}

\stanza{
	sh\=o ten kyaku ten k\=u	\\
	j\=o s\=a sh\=u g\=\i gaku	\\
	\=u man d\=a r\=a k\=e		\\
	san butsu gy\=u dai sh\=u
}

\stanza{
	g\=a j\=o d\=o f\=u k\=\i	\\
	n\=\i sh\=u ken sh\=o jin	\\
	\=u f\=u sh\=o k\=u n\=o	\\
	nyo z\=e shitsu j\=u man
}

\stanza{
	z\=e sh\=o zai sh\=u j\=o	\\
	\=\i aku g\=o in nen		\\
	k\=a \=a s\=o g\=\i k\=o	\\
	f\=u mon san b\=o my\=o
}

\stanza{
	sh\=o \=u sh\=u k\=u doku	\\
	ny\=u w\=a shitsu jiki sha	\\
	soku kai ken g\=a shin		\\
	zai sh\=\i n\=\i sep-p\=o
}

\stanza{
	waku j\=\i \=\i sh\=\i sh\=u	\\
	setsu butsu j\=u m\=u ry\=o	\\
	k\=u nai ken bus-sha		\\
	\=\i setsu butsu nan ch\=\i
}

\stanza{
	g\=a ch\=\i riki ny\=o z\=e	\\
	\=e k\=o sh\=o m\=u ry\=o	\\
	j\=u my\=o m\=u sh\=u k\=o	\\
	k\=u sh\=u g\=o sh\=o toku
}

\stanza{
	nyo t\=o \=u ch\=\i sha		\\
	mot-t\=o sh\=\i sh\=o g\=\i	\\
	\keisu t\=o dan ry\=o y\=o jin	\\
	butsu g\=o jitsu f\=u k\=o
}

\stanza{
	ny\=o \=\i zen h\=o ben		\\
	\=\i j\=\i \=o sh\=\i k\=o	\\
	jitsu zai n\=\i gon sh\=\i	\\
	m\=u n\=o sek-k\=o m\=o
}

\stanza{
	g\=a yaku \=\i s\=e b\=u	\\
	g\=u sh\=o k\=u gen sha		\\
	\keisu \=i bon b\=u ten d\=o	\\
	jitsu zai n\=\i gon metsu
}

\stanza{
	\=i j\=o ken g\=a k\=o		\\
	n\=\i sh\=o ky\=o sh\=\i shin	\\
	h\=o itsu jaku g\=o yoku	\\
	d\=a \=o aku d\=o ch\=u
}

\stanza{
	g\=a j\=o ch\=\i sh\=u j\=o	\\
	gy\=o d\=o f\=u gy\=o d\=o	\\
	zui \=o sh\=o k\=a d\=o		\\
	\=\i setsu shu ju h\=o
}

\stanza{
	\shokei mai j\=\i s\=a z\=e nen		\\
	\=\i g\=a ry\=o sh\=u j\=o		\\
	\shokei toku ny\=u m\=u j\=o d\=o	\\
	soku j\=o j\=u bus-shin
}
\end{Verse}
\end{Prayer}

%%%%%%%%%%%%%%%%%%%%%%%%%%%%%%%%%%%%%%%%%%%%%%%%%%%%%%%%%%%%%%%%%%%%%%%%%%%%%%%
%%%% SUTRA LOTOSU WSPANIAŁEGO PRAWA ROZDZIAŁ: WIECZNE ŻYCIE TATHAGATY
\begin{Prayer}{sutralotosu_wieczne_zycie}
	{SUTRA LOTOSU WSPANIAŁEGO PRAWA ROZDZIAŁ: WIECZNE ŻYCIE TATHAGATY}{-}
	{My\=oh\=orengeky\=o Nyorai Jury\=ohonge}


\bigskip
\keisu Odkąd osiągnąłem Stan Buddhy, minęły nieskończone miliardy kalp. Zawsze
objaśniałem na różne sposoby Dharmę niezliczonym istotom, które weszły na
Drogę Buddhy. \keisu Od tego czasu minęły niezliczone kalpy.


By wyzwolić wszystkie istoty, używam upaya~-- szczególny sposób nauczania~--
poprzez ukazanie Nirwany. W~rzeczywistości nie przeszedłem jeszcze
w~Nirwanę, przebywam tu wiecznie, objaśniając Dharmę. Ciągle tu pozostaję,
używając wszystkich boskich mocy tak, że zdeprawowane istoty, choć jestem
blisko, nie mogą mnie dostrzec.


Wszyscy uważając mnie za wygasłego, wszędzie będą czcić moje relikwie.
Utrzymując tęskne życzenia, obudzą spragnione serca szacunku i~czci. Gdy
odczuwające istoty rozwiną wiarę i~pokorę, będą miały szczerą, łagodną
i~dobrą naturę umysłu i~całym sercem będą pragnęły ujrzeć Buddhę, nie
zważając na swoje ciała i~życie, wówczas wraz ze zgromadzeniem mnichów
pojawię się na szczycie Góry Sępa.


Wtedy powiem żyjącym istotom, że zawsze istnieję i~nie odszedłem w~Nirwanę
i~tylko używając szczególnych metod nauczania raz objawiam Nirwanę, a~raz
ukazuję stan nie-Nirwany.


Jeśli w~innych krainach są istoty pełne czci, wiary i~radości, pojawiam się
wśród nich i~objaśniam nieprześcignioną najwyższą Dharmę. Wy nie słysząc
tego, będziecie mówić, że wszedłem w~Nirwanę.


Widzę wszystkie odczuwające istoty, które wpadły do oceanu cierpienia,
dlatego nie ukazuję mojego ciała, by obudzić w~nich pragnienie i~szacunek,
i~przyczynić się do powstania tęsknoty w~ich sercach. Dopiero wówczas
pojawiam się i~objaśniam Dharmę.


Dzięki takim nadprzyrodzonym mocom minęło już wiele kalp i~cały czas
przebywam na Duchowej Górze Sępa, jak i~we wszystkich innych miejscach.
Odczuwające istoty, kiedy będą widziały koniec kalpy i~wielki ogień
niszczący wszystko, to w~tym czasie moja kraina pozostanie w~spokoju, zawsze
pełna ludzi i~niebiańskich istot, z~ogrodami, gajami, wysokimi świątyniami,
różnego rodzaju klejnotami i~wspaniałymi ozdobami.


Jej klejnotowe drzewa pełne są kwiatów i~owoców, żyjący w~tym miejscu są
szczęśliwi. Wszystkie niebiańskie istoty uderzają w~niebiańskie bębny,
grając wszelkiego rodzaju wspaniałą muzykę, obsypując deszczem kwiatów
mandarwa Buddhę i~wielkie zgromadzenie. Choć moja kraina jest niezniszczalna,
zwykłe istoty zostaną spalone i~zniszczone.


Widzę przerażenie i~wszystkie cierpienia, które je wypełnią. Choć przeminie
wiele niezliczonych kalp, wszystkie istoty pełne wszelkiego rodzaju
wykroczeń ze względu na karmiczną przyczynę złych czynów nie słyszą i~nie
usłyszą imienia Trzech Klejnotów.


Wszyscy ci, którzy praktykują wszelkie cnoty zasługi, są łagodnego
i~pokojowego usposobienia, są szczerzy, widzą mnie i~słyszą, jak objaśniam
Dharmę.


Albo wyjaśniam zgromadzeniu, że życie Buddhy jest niepoliczalne. Tym, którzy
od dawna nie widzieli Buddhy, głoszę, że spotkanie Buddhy jest bardzo trudne.
Taka jest moc mojej mądrości, nieskończony oświecający promień mądrości
i~życie trwające niezliczone kalpy. Osiągnąłem to dzięki bardzo długiej
praktyce.


Posiadający mądrość, gdy tu jesteście obecni, nie wolno wam dopuścić, by
narodziła się jakakolwiek wątpliwość, powinniście odrzucić od siebie każdą
i~na zawsze je wyczerpać. Słowa Buddhy są prawdą, nie fałszem. Jak lekarz,
który dobrym sposobem, by wyleczyć swych obłąkanych synów, choć naprawdę
żywy, ogłasza swoją śmierć i~nie można tego tłumaczyć jako bycie fałszywym.


\keisu Podobnie ja, będąc ojcem tego świata, który chroni przed wszelkimi
cierpieniami i~chorobami ze względu na upadłe istoty, choć istnieję,
ogłaszam moją Nirwanę. \keisu Inaczej ciągle mnie widząc wzbudziliby aroganckie
serca, popadliby w~pięć żądz i~wpadli w~sam środek złych dróg.


Zawsze znam istoty, które praktykują Drogę i~te, które nie praktykują Drogi,
dlatego by je wyzwolić stosownie do okoliczności objaśniam różne nauki
Dharmy. Ostatecznie sam zastanawiam się: \shokei ,,Jakim sposobem sprawić,
by istoty weszły na najwyższą, nieprześcignioną \shokei Drogę i~osiągnęły
ostatecznie ciało Buddhy?''.
\end{Prayer}

\newpage
%%%%%%%%%%%%%%%%%%%%%%%%%%%%%%%%%%%%%%%%%%%%%%%%%%%%%%%%%%%%%%%%%%%%%%%%%%%%%%%
%%%% MAKA HANNYA HARAMITTA SHINGY\=O
\begin{Prayer}{hannya_shin_gyo}
	{MAKA HANNYA HARAMITTA SHINGY\=O}{-}
	{Sutra Serca Wielkiej Doskonałej Mądrości}

\begin{center}
\samepage{
	\small
        Inny tytuł: \textit{Hoja Shin Kin}\\
	Skrócony tytuł: \textit{Hannya shingy\=o}\\
	(język chiński)
}
\end{center}

\begin{JAPANESE}
KAN-J\=I ZAI B\=O SA. GY\=O JIN HAN-NY\=A H\=A-R\=A-MIT-T\=A J\=I. SH\=O
KEN G\=O \keisu ON KAI K\=U. D\=O IS-SAI K\=U YAKU. SH\=A-R\=I-SH\=I. SHIKI
F\=U \=I K\=U. K\=U F\=U \=I SHIKI. SHIKI SOKU Z\=E K\=U. K\=U SOKU Z\=E
SHIKI. J\=U S\=O GY\=O SHIKI. YAKU B\=U NY\=O Z\=E. SH\=A-R\=I-SH\=I Z\=E
SHO H\=O K\=U S\=O. F\=U-SH\=O F\=U-METSU. F\=U-K\=U F\=U-J\=O. F\=U-Z\=O
F\=U-GEN. Z\=E-K\=O K\=U CH\=U. M\=U-SHIKI M\=U J\=U S\=O GY\=O SHIKI.
M\=U-GEN N\=I B\=I ZES-SHIN N\=I. M\=U-SHIKI SH\=O K\=O M\=I SOKU H\=O.
M\=U-GEN KAI NAI-SH\=I M\=U-\=I-SHIKI-KAI. M\=U M\=U-MY\=O YAKU M\=U
M\=U-MY\=O JIN. NAI-SH\=I M\=U-R\=O-SH\=I. YAKU M\=U-R\=O-SH\=I JIN.
M\=U-K\=U SH\=U METSU D\=O. M\=U-CH\=I YAKU M\=U-TOKU. \=I
M\=U-SH\=O-TOK-K\=O. B\=O-DAI SAT-T\=A. \=E HAN-NY\=A H\=A-R\=A-M\=I-TA
\keisu K\=O. SHIN M\=U KEI-G\=E. M\=U-KEI-G\=E K\=O. M\=U \=U K\=U-F\=U.
ON-R\=I IS-SAI TEN-D\=O M\=U-S\=O. K\=U-GY\=O N\=E-HAN.
SAN-Z\=E-SH\=O-BUTSU. \=E HAN-NY\=A H\=A-R\=A-M\=I-T\=A \keisu K\=O. TOKU
\=A-NOKU T\=A-R\=A-SAN-MYAKU-SAN-B\=O-DAI. K\=O CH\=I HAN-NY\=A
H\=A-R\=A-M\=I-T\=A. Z\=E DAI-JIN-SH\=U. Z\=E DAI-MY\=O-SH\=U. Z\=E M\=U
J\=O-SH\=U. Z\=E M\=U T\=O-D\=O-SH\=U. N\=O-J\=O IS-SAI-K\=U. SHIN-JITSU
F\=U-K\=O. K\=O SETSU HAN-NY\=A H\=A-R\=A-M\=I-T\=A SH\=U. SOKU SETSU SH\=U
WATSU. GY\=A-T\=E GY\=A-T\=E. \shokei H\=A-R\=A GY\=A-TEI. HARA S\=O
GY\=A-T\=E. \shokei B\=O-J\=I SOWA-K\=A. HAN-NY\=A SHIN-GY\=O.
\end{JAPANESE}
\end{Prayer}

%%%%%%%%%%%%%%%%%%%%%%%%%%%%%%%%%%%%%%%%%%%%%%%%%%%%%%%%%%%%%%%%%%%%%%%%%%%%%%%
%%%% SUTRA SERCA WIELKIEJ DOSKONAŁEJ MĄDROŚCI
\begin{Prayer}{serce_doskonalej_madrosci}
	{SUTRA SERCA WIELKIEJ DOSKONAŁEJ MĄDROŚCI}{-}
	{Maka Hannya Haramitta Shingy\=o Maha Pradźńa Paramita Hridaya Sutra}

% ??? ten drugi tytuł ``Maha pradźńa . . .'' to pali chyba?


\bigskip
\keisu Awalokiteśwara~-- Obserwujący Dźwięki Świata~-- Bodhisattwa, gdy głęboko
praktykował Doskonałą Mądrość, w~jednym błysku ujrzał \keisu pustkę pięciu
składników (skandh) i~uwolnił się od całego bólu i~cierpienia.


Siariputro. Forma nie jest różna od pustki. Pustka nie jest różna od
formy. Forma jest dokładnie pustką. Pustka jest dokładnie formą. Wrażenia,
myśli, wola i~świadomość są również takie.


Siariputro. Wszystkie zjawiska (dharmy) są cechą pustki. Nie rodzą się
i nie umierają. Nie są skalane i~nie są czyste. Nie zwiększają się i~nie
zmniejszają. Tak więc w~pustce nie ma formy, wrażeń, myśli, woli i
świadomości. Nie ma oczu, uszu, nosa, języka, ciała, umysłu. Nie ma
koloru, dźwięku, zapachu, smaku, dotyku, zjawisk (dharm). Nie ma widzenia,
ani świadomości. Nie ma złudzeń ani kresu złudzeń. Nie ma starości i
śmierci, ani kresu starości i~śmierci. Nie ma cierpienia, przyczyny
cierpienia, kresu cierpienia i~drogi wyzwalającej z~cierpienia. Nie ma
mądrości, ani osiągnięcia mądrości. Ponieważ nie ma niczego do osiągnięcia
Bodhisattwa w~oparciu o~\keisu~Doskonałą Mądrość ma serce-umysł wolne od
przeszkód. Bez przeszkód, zatem bez lęku, oddzielony od złudnych jak sen
myśli~-- to jest ostateczna Nirwana.


Buddhowie trzech okresów urzeczywistniają \keisu Doskonałą Mądrość i~osiągają
Najwyższe Nieprześcignione Oświecenie Annuttara Samiak Sambodhi.


Wiedz zatem, że Doskonała Mądrość jest mantrą wielkiej mocy, mantrą
wielkiego światła, nieprześcignioną mantrą, mantrą nie mającą sobie
równej, usuwającą całe cierpienie, jest Prawdą nie fałszem.


Powtarzaj więc mantrę Doskonałej Mądrości:

\begin{center}
	GATE GATE	\\
	PARA GATE	\\
	PARASAM GATE	\\
	BODHI SWAHA!
\end{center}
\end{Prayer}

%%%%%%%%%%%%%%%%%%%%%%%%%%%%%%%%%%%%%%%%%%%%%%%%%%%%%%%%%%%%%%%%%%%%%%%%%%%%%%%
%%%% KON GO HAN NYA HA RA MI KYO
\begin{Prayer}{kon_go_han_nya_ha_ra_mi_kyo}
	{KON GO HAN NYA HA RA MI KYO}{-}
	{Sutra Diamentowa}%\footnote{Polski przekad sutry dostpny jest pod adresem \href{https://mahajana.net/biblioteka/teksty/diamentowa-sutra-w-przekladzie-marka-mejora}{https://mahajana.net/biblioteka/teksty/diamentowa-sutra-w-przekladzie-marka-mejora}}

\newcommand{\Stanzanum}[1]{%
	\begin{center}#1\end{center}
}

\Stanzanum{I}
\begin{JAPANESE}
{\bf NYO ZE GA MON} ICHI JI BUTSU ZAI SHA E KOKU GI JU GIKKO DOKU ON YO DAI BI
KU SHU SEN NI HYAKU GO JU NIN GU NI JI SE SON JIKI JI JAKU E JI HATSU NYU SHA E
DAI JO KOTSU JIKI O GO JO CHU SHI DAI KOC CHI GEN SHI HON JO BON JIKI KO SHU E
HATSU SEN ZOKU I FU ZA NI ZA.
\end{JAPANESE}

\Stanzanum{II}
\begin{JAPANESE}
JI CHO RO SHU BO DAI ZAI DAI SHU CHU SOKU JU ZA KI HEN DAN U KEN U SHITSU JAKU
JI GA SSHO KU GYO NI BYAKU BUTSU GON KE U SE SON NYO RAI ZEN GO NEN SHO BO SA
ZEN FU ZOKU SHO BO SA SE SON ZEN NAN SHI ZEN NYO NIN HOTSU A NOKU TA RA SAM
MYAKU SAM BO DAI SHIN UN GA O JU UN GA GO BUKU GO SHIN BUTSU GON ZEN ZAI ZEN
ZAI SHU BO DAI NYO NYO SHO SETSU NYO RAI ZEN GO NEN SHO BO SA ZEN FU ZOKU SHO
BO SA NYO KON TAI CHO TO I NYO SETSU ZEN NAN SHI ZEN NYO NIN HOTSU A NOKU TA RA
SAM MYAKU SAM BO DAI SHIN O NYO ZE JU NYO ZE GO BUKU GO SHIN YUI NEN SE SON GAN
GYO YOKU MON.
\end{JAPANESE}

\Stanzanum{III}
\begin{JAPANESE}
BUTSU GO SHU BO DAI SHO BO SA MA KA SA O NYO ZE GO BUKU GO SHIN SHO U ISSAI SHU
JO SHI RUI NYAKU RAN SHO NYAKU TAI SHO NYAKU SHIS SHO NYAKU KEKE SHO NYAKU U
SHIKI NYAKU MU SHIKI NYAKU U SO NYAKU MU SO NYAKU HI U SO NYAKU HI MU SO GA KAI
RYO NYU MU YO NE HAN NI METSU DO SHI NYO ZE METSU DO MU RYO MU SHU MU HEN SHU
JO JITSU MU SHU JO TOKU METSU DO SHA GA I KO SHU BO DAI NYAKU BO SA U GA SO NIN
SO SHU JO SO JU SHA SO SOKU HI BO SA.
\end{JAPANESE}

\Stanzanum{IV}
\begin{JAPANESE}
BU JI SHU BO DAI BO SA O HO O MU SHO JU GYO O FU SE SHO I FU JU SHIKI FU SE FU
JU SHO KO MI SOKU HO FU SE SHU BO DAI BO SA O NYO ZE FU SE FU JU O SO GA I KO
NYAKU BO SA FU JU SO FU SE GO FUKU TOKU FU KA SHI RYO SHU BO DAI O I UN GA TO
HO KO KU KA SHI RYO FU HOC CHA SE SON SHU BO DAI NAN ZAI HO PPO SHI YUI JO GE
KO KU KA SHI RYO FU HOC CHA SE SON SHU BO DAI BO SA MU JU SO FU SE FUKU TOKU
YAKU BU NYO ZE FU KA SHI RYO SHU BO DAI BO SA TAN NO NYO SHO KYO JU.
\end{JAPANESE}

\newpage
\Stanzanum{V}
\begin{JAPANESE}
SHU BO DAI O I UN GA KA I SHIN SO KEN NYO RAI FU HOC CHA SE SON FU KA I SHIN SO
TO KKEN NYO RAI GA I KO NYO RAI SHO SETSU SHIN SO SOKU HI SHIN SO BUTSU GO SHU
BO DAI BON SHO U SO KAI ZE KO MO NYA KKEN SHO SO HI SO SO KKEN NYO RAI.
\end{JAPANESE}

\Stanzanum{VI}
\begin{JAPANESE}
SHU BO DAI BYAKU BUTSU GON SE SON HA U SHU JO TOKU MON NYO ZE GON SETSU SHO KU
SHO JI SSHIN FU BUTSU GO SHU BO DAI MAKU SA ZE SETSU NYO RAI METSU GO GO GO
HYAKU SAI U JI KAI SHU FUKU SHA O SHI SHO KU NO SHO SHIN JIN I SHI I JI TTO CHI
ZE NIN FU O ICHI BUTSU NI BUTSU SAN SHI GO BUTSU NI SHU ZEN GON I O MU RYO SEM
MAN BU SSHO SHU SHO ZEN GON MON ZE SHO KU NAI SHI ICHI NEN SHO JO SHIN JA SHU
BO DAI NYO RAI SHI CCHI SHI KKEN ZE SHO SHU JO TOKU NYO ZE MU RYO FUKU TOKU GA
I KO ZE SHO SHU JO MU BU GA SO NIN SO SHU JO SO JU SHA SO MU HO SO YAKU MU HI
HO SO GA I KO ZE SHO SHU JO NYAKU SHIN SHU SO SOKU I JAKU GA NIN SHU JO JU SHA
NYAKU SHU HO SO SOKU JAKU GA NIN SHU JO JU SHA GA I KO NYAKU SHU HI HO SO SOKU
JAKU GA NIN SHU JO JU SHA ZE KO FU O SHU HO FU O SHU HI HO I ZE GI KO NYO RAI
JO SETSU NYO TO BI KU CHI GA SE PPO NYO BATSU YU SHA HO SHO O SHA GA KYO HI HO.
\end{JAPANESE}

\Stanzanum{VII}
\begin{JAPANESE}
SHU BO DAI O I UN GA NYO RAI TOKU A NOKU TA RA SAM MYAKU SAM BO DAI YA NYO RAI
U SHO SE PPO YA SHU BO DAI GON SE SON NYO GA GE BU SSHO SETSU GI MU U JO HO MYO
A NOKU TA RA SAM MYAKU SAM BO DAI YAKU MU U JO HO NYO RAI KA SETSU GA I KO NYO
RAI SHO SE PPO KAI FU KA SHU FU KA SETSU HI HO HI HI HO SHO I SHA GA I SSAI KEN
SHO KAI I MU I HO NI U SHA BETSU.
\end{JAPANESE}

\Stanzanum{VIII}
\begin{JAPANESE}
SHU BO DAI O I UN GA NYAKU NIN MAN SAN ZEN DAI SEN SE KAI SHI PPO I YU FU SE ZE
NIN SHO TOKU FUKU TOKU NYO I TA FU SHU BO DAI GON JIN TA SE SON GA I KO ZE FUKU
TOKU SOKU HI FUKU TOKU SHO ZE KO NYO RAI SETSU FUKU TOKU TA NYAKU BU U NIN O
SHI KYO CHU JU JI NAI SHI SHI KU GE TO I TA NIN SETSU GO FUKU SHO HI GA I KO
SHU BO DAI {\bf I SSAI SHO BUTSU GYU SHO BUTSU A NOKU TA RA SAM MYAKU SAM BO
DAI HO KAI JU SHI KYO SHUTSU} SHU BO DAI SHO I BU PPO SHA SOKU HI BU PPO.
\end{JAPANESE}

\Stanzanum{IX}
\begin{JAPANESE}
SHU BO DAI O I UN GA SHU DA ON NO SA ZE NEN GA TOKU SHU DA ON GA FU SHU BO DAI
GON HO CCHA SE SON GA I KO SHU DA ON MYO I NYU RU NI MU SHO NYU FU NYU SHIKI
SHO KO MI SOKU HO ZE MYO SHU DA ON SHU BO DAI O I UN GA SHI DA GON NO SA ZE NEN
GA TOKU SHI DA GON GA FU SHU BO DAI GON HO CCHA SE SON GA I KO SHI DA GON MYO
ICHI O RAI NI JITSU MU O RAI ZE MYO SHI DA GON SHU BO DAI O I UN GA A NA GON NO
SA ZE NEN GA TOKU A NA GON GA FU SHU BO DAI GON HO CCHA SE SON GA I KO A NA GON
MYO I FU RAI NI JITSU MU FU RAI ZE KO MYO A NA GON SHU BO DAI O I UN GA A RA
KAN NO SA ZE NEN GA TOKU A RA KAN DO FU SHU BO DAI GON HO CCHA SE SON GA I KO
JITSU MU U HO MYO A RA KAN SE SON NYAKU A RA KAN SA ZE NEN GA TOKU A RA KAN DO
SOKU I JAKU GA NIN SHU JO JU SHA SE SON BU SETSU GA TOKU MU JO ZAM MAI NIN CHU
SAI I DAI ICHI ZE DAI ICHI RI YOKU A RA KAN SE SON GA FU SA ZE NEN GA ZE RI
YOKU A RA KAN SE SON GA NYAKU SA ZE NEN GA TOKU A RA KAN DO SE SON SOKU FU
SETSU SHU BO DAI ZE GYO A REN NA GYO SHA I SHU BO DAI JITSU MU SHO GYO NI MYO
SHU BO DAI ZE GYO A REN NA GYO.
\end{JAPANESE}

\Stanzanum{X}
\begin{JAPANESE}
BUTSU GO SHU BO DAI O I UN GA NYO RAI SHAKU ZAI NEN TO BU SSHO O HO U SHO TOKU
FU HO CCHA SE SON NYO RAI ZAI NEN TO BU SSHO O HO JITSU MU SHO TOKU SHU BO DAI
O I UN GA BO SA SHO GON BUTSU DO FU HO CCHA SE SON GA I KO SHO GON BUTSU DO SHA
SOKU HI SHO GON ZE MYO SHO GON ZE KO SHU BO DAI SHO BO SA MA KA SA O NYO ZE SHO
SHO JO SHIN FU O JU SHIKI SHO SHIN FU O JU SHO KO MI SOKU HO SHO SHIN {\bf O MU
SHO JU NI SHO GO SHIN} SHU BO DAI HI NYO U NIN SHIN NYO SHU MI SEN NO O I UN GA
ZE SHIN I DAI FU SHU BO DAI GON JIN DAI SE SON GA I KO BU SSETSU HI SHIN ZE MYO
DAI SHIN.
\end{JAPANESE}

\Stanzanum{XI}
\begin{JAPANESE}
SHU BO DAI NYO GO GA CHU SHO U SHA SHU NYO ZE SHA TO GO GA O I UN GA ZE SHO GO
GA SHA NYO I TA FU SHU BO DAI GON JIN TA SE SON TAN SHO GO GA SHO TA MU SHU GA
KYO GO SHA SHU BO DAI GA KON JITSU GON GO NYO NYAKU U ZEN NAN SHI ZEN NYO NIN I
SHI PPO MAN NI SHO GO GA SHA SHU SAN ZEN DAI SEN SE KAI I YU FU SE TOKU FUKU TA
FU SHU BO DAI GON JIN TA SE SON BUTSU GO SHU BO DAI NYAKU ZEN NAN SHI ZEN NYO
NIN O SHI KYO CHU NAI SHI JU JI SHI KU GE TO I TA NIN SETSU NI SHI FUKU TOKU
SHO ZEN FUKU TOKU.
\end{JAPANESE}

\Stanzanum{XII}
\begin{JAPANESE}
BU SHI SHU BO DAI ZUI SETSU ZE KYO NAI SHI SHI KU GE TO TO CHI SHI SHO I SSAI
SE KEN TEN NIN A SHU RA KAI O KU YO NYO BUTSU TO BYO GA KYO U NIN JIN NO JU JI
DOKU JU SHU BO DAI TO CHI ZE NIN JO JU SAI JO DAI ICHI KE U SHI HO NYAKU ZE KYO
TEN SHO ZAI SHI SHO SOKU I U BUTSU NYAKU SON JU DE SHI.
\end{JAPANESE}

\Stanzanum{XIII}
\begin{JAPANESE}
NI JI SHU BO DAI BYAKU BUTSU GON SE SON TO GA MYO SHI KYO GA TO UN GA BU JI
BUTSU GO SHU BO DAI ZE KYO MYO I {\bf KON GO HAN NYA HA RA MI} I ZE MYO JI NYO TO
BU JI SHO I SHA GA SHU BO DAI BU SSETSU HAN NYA HA RA MITSU SOKU HI HAN NYA HA
RA MITSU ZE MYO HAN NYA HA RA MITSU SHU BO DAI O I UN GA NYO RAI U SHO SE PPO
FU SHU BO DAI BYAKU BUTSU GON SE SON NYO RAI MU SHO SETSU SHU BO DAI O I UN GA
SAN ZEN DAI SEN SE KAI SHO U MI JIN ZE I TA FU SHU BO DAI GON JIN TA SE SON SHU
BO DAI SHO MI JIN NYO RAI SE PPI MI JIN ZE MYO MI JIN NYO RAI SETSU SE KAI HI
SE KAI ZE MYO SE KAI SHU BO DAI O I UN GA KA I SAN JU NI SO KEN NYO RAI FU HO
CHA SE SON FU KA I SAN JU NI SO TO KKEN NYO RAI GA I KO NYO RAI SETSU SAN JU NI
SO SOKU ZE HI SO ZE MYO SAN JU NI SO SHU BO DAI NYAKU U ZEN NAN SHI ZEN NYO NIN
I GO GA SHA TO SHIM MYO FU SE NYAKU BU U NIN O SHI KYO CHU NAI SHI JU JI SHI KU
GE TO I TA NIN SETSU GO FUKU JIN TA.
\end{JAPANESE}

\Stanzanum{XIV}
\begin{JAPANESE}
NI JI SHU BO DAI MON SETSU ZE KYO SHIN GE GI SHU TAI RUI HI KYU NI BYAKU BUTSU
GON KE U SE SON BU SETSU NYO ZE JIN JIN KYO TEN GA JU SHAKU RAI SHO TOKU E GEN
MI ZO TOKU MON NYO ZE SHI KYO SE SON NYAKU BU U NIN TOKU MON ZE KYO SHIN JIN
SHO JO SOKU SHO JI SSO TO CHI ZE NIN JO JU DAI ICHI KE U KU DOKU SE SON ZE JI
SSO SHA SOKU ZE HI SO ZE KO NYO RAI SETSU MYO JI SSO SE SON GA KON TOKU MON NYO
ZE KYO TEN SHIN GE JU JI FU SOKU I NAN NYAKU TO RAI SE GO GO HYAKU SAI GO U SHU
JO TOKU MON ZE KYO SHIN GE JU JI ZE NIN SOKU I DAI ICHI KE U GA I KO SHI NIN MU
GA SO MU NIN SO MU SHU JO SO MU JU SHA SO SHO I SHA GA GA SO SOKU ZE HI SO NIN
SO SHU JO SO JU SHA SO SOKU ZE HI SO GA I KO RI I SSAI SHO SO SOKU MYO SHO
BUTSU BUTSU GO SHU BO DAI NYO ZE NYO ZE NYAKU BU U NIN TOKU MON ZE KYO FU KYO
FU FU FU I TO CHI ZE NIN JIN I KE U GA I KO SHU BO DAI NYO RAI SETSU DAI ICHI
HA RA MITSU SOKU HI DAI ICHI HA RA MITSU ZE MYO DAI ICHI HA RA MITSU SHU BO DAI
NIN NIKU HA RA MITSU NYO RAI SE PPI NIN NIKU HA RA MITSU ZE MYO NIN NIKU HA RA
MITSU GA I KO SHU BO DAI NYO GA SHAKU I KA RI O KA SSETSU SHIN TAI GA O NI JI
MU GA SO MU NIN SO MU SHU JO SO MU JU SHA SO GA I KO GA O O SHAKU SETSU SETSU
SHI GE JI NYAKU U GA SO NIN SO SHU JO SO JU SHA SO O SHO SHIN GON SHU BO DAI U
NEN KA KO O GO HYAKU SE SA NIN NIKU SEN NIN O NI SHO SE MU GA SO MU NIN SO MU
SHU JO SO MU JU SHA SO ZE KO SHU BO DAI BO SA O RI I SSAI SO HOTSU A NOKU TA RA
SAM MYAKU SAM BO DAI SHIN FU O JU SHIKI SHO SHIN FU O JU SHO KO MI SOKU HO SHO
SHIN O SHO MU SHO JU SHIN NYAKU SHIN U JU SOKU I HI JU ZE KO BU SSETSU BO SA
SHIN FU O JU SHIKI FU SE SHU BO DAI BO SA I RI YAKU I SSAI SHU JO KO O NYO ZE
FU SE NYO RAI SETSU I SSAI SHO SO SOKU ZE HI SO U SETSU I SSAI SHU JO SOKU HI
SHU JO SHU BO DAI NYO RAI ZE SHIN GO SHA JITSU GO SHA NYO GO SHA FU O GO SHA FU
I GO SHA SHU BO DAI NYO RAI SHO TOKU HO SHI HO MU JITSU MU KO SHU BO DAI NYAKU
BO SA SHIN JU O HO NI GYO FU SE NYO NIN NYU AN SOKU MU SHO KEN NYAKU BO SA SHIN
FU JU HO NI GYO FU SE NYO NIN U MOKU NI KKO MYO SHO KEN SHU JU SHIKI SHU BO DAI
TO RAI SHI SE NYAKU U ZEN NAN SHI ZEN NYO NIN NO O SHI KYO JU JI DOKU JU SOKU I
NYO RAI I BUTSU CHI E SHI CCHI ZE NIN SHI KKEN ZE NIN KAI TOKU JO JU MU RYO MU
HEN KU DOKU.
\end{JAPANESE}

\Stanzanum{XV}
\begin{JAPANESE}
SHU BO DAI NYAKU U ZEN NAN SHI ZEN NYO NIN SHO NICHI BUN I GO GA SHA TO SHIN FU
SE CHU NICHI BUN BU I GO GA SHA TO SHIN FU SE GO NICHI BUN YAKU I GO GA SHA TO
SHIN FU SE NYO ZE MU RYO HYAKU SEN MAN NOKU KO I SHIN FU SE NYAKU BU U NIN MON
SHI KYO TEN SHIN JIN FU GYAKU GO FUKU SHO HI GA KYO SHO SHA JU JI DOKU JU I NIN
GE SETSU SHU BO DAI I YO GON SHI ZE KYO U FU KA SHI GI FU KA SHO RYO MU HEN KU
DOKU NYO RAI I HOTSU DAI JO SHA SETSU I HOTSU SAI JO JO SHA SETSU NYAKU U NIN
NO JU JI DOKU JU KO I NIN SETSU NYO RAI SHI CCHI ZE NIN SHI KKEN ZE NIN KAI
TOKU JO JU FU KA RYO FU KA SHO MU U HEN FU KA SHI GI KU DOKU NYO ZE NIN TO SOKU
I KA TAN NYO RAI A NOKU TA RA SAM MYAKU SAM BO DAI GA I KO SHU BO DAI NYAKU GYO
SHO HO SHA JAKU GA KEN NIN KEN SHU JO KEN JU SHA KEN SOKU O SHI KYO FU NO CHO
JU DOKU JU I NIN GE SETSU SHU BO DAI ZAI ZAI SHO SHO NYAKU U SHI KYO I SSAI SE
KEN TEN NIN A SHU RA SHO O KU YO TO CHI SHI SHO SOKU I ZE TO KAI O KU GYO SA
RAI I NYO I SHO GE KO NI SAN GO SHO.
\end{JAPANESE}

\Stanzanum{XVI}
\begin{JAPANESE}
BU SHI SHU BO DAI {\bf NYAKU ZEN NAN SHI ZEN NYO NIN JU JI DOKU JU SHI KYO
NYAKU I NIN KYO SEN ZE NIN SEN SE ZAI GO O DA AKU DO I KON SE NIN KYO SEN KO
SEN SE ZAI GO SOKU I SHO METSU TO TOKU A NOKU TA RA SAM MYAKU SAM BO DAI} SHU
BO DAI GA NEN KA KO MU RYO A SO GI KO O NEN TO BUTSU ZEN TOKU CHI HA PPYAKU SHI
SEN MAN NOKU NA YU TA SHO BUTSU SHI KKAI KU YO JO JI MU KU KA SHA NYAKU BU U
NIN O GO MA SSE NO JU JI DOKU JU SHI KYO SHO TOKU KU DOKU O GA SHO KU YO SHO
BUTSU KU DOKU HYAKU BUN FU GYU I SSEM MAN NOKU BUN NAI SHI SAN JU HI YU SHO FU
NO GYU SHU BO DAI NYAKU ZEN NAN SHI ZEN NYO NIN O GO MA SSE U JU JI DOKU JU SHI
KYO SHO TOKU KU DOKU GA NYAKU GU SE SSHA WAKU U NIN MON SHIN SOKU O RAN KO GI
FU SHIN SHU BO DAI TO CHI ZE KYO GI FU KA SHI GI KA HO YAKU FU KA SHI GI.
\end{JAPANESE}

\Stanzanum{XVII}
\begin{JAPANESE}
NI JI SHU BO DAI BYAKU BUTSU GON SE SON ZEN NAN SHI ZEN NYO NIN HOTSU A NOKU TA
RA SAN MYAKU SAN BO DAI SHIN UN GA O JU UN GA GO BUKU GO SHIN BUTSU GO SHU BO
DAI ZEN NAN SHI ZEN NYO NIN HOTSU A NOKU TA RA SAN MYAKU SAN BO DAI SHIN SHA TO
SHO NYO ZE SHIN GA O METSU DO I SSAI SHU JO METSU DO I SSAI SHU JO I NI MU U
ICHI SHU JO JITSU METSU DO SHA GA I KO SHU BO DAI NYAKU BO SA U GA SO NIN SO
SHU JO SO JU SHA SO SOKU HI BO SA SHO I SHA GA SHU BO DAI JITSU MU U HO HOTSU A
NOKU TA RA SAN MYAKU SAN BO DAI SHIN SHA SHU BO DAI O I UN GA NYO RAI O NEN TO
BU SSHO U HO TOKU A NOKU TA RA SAN MYAKU SAN BO DAI FU HO CHA SE SON NYO GA GE
BU SSHO SETSU GI BUTSU O NEN TO BU SSHO MU U HO TOKU A NOKU TA RA SAM MYAKU SAM
BO DAI BUTSU GON NYO ZE NYO ZE SHU BO DAI JITSU MU U HO NYO RAI TOKU A NOKU TA
RA SAN MYAKU SAN BO DAI SHU BO DAI NYAKU U HO NYO RAI TOKU A NOKU TA RA SAN
MYAKU SAN BO DAI SHA NEN TO BUTSU SOKU FU YO GA JU KI NYO O RAI SE TO TOKU SA
BUTSU GO SHA KYA MU NI I JITSU MU U HO TOKU A NOKU TA RA SAN MYAKU SAN BO DAI
ZE KO NEN TO BUTSU YO GA JU KI SA ZE GON NYO O RAI SE TO TOKU SA BUTSU GO SHA
KYA MU NI GA I KO NYO RAI SHA SOKU SHO HO NYO GI NYAKU U NIN GON NYO RAI TOKU A
NOKU TA RA SAN MYAKU SAN BO DAI SHU BO DAI JITSU MU U HO BUTSU TOKU A NOKU TA
RA SAN MYAKU SAN BO DAI SHU BO DAI NYO RAI SHO TOKU A NOKU TA RA SAN MYAKU SAN
BO DAI O ZE CHU MU JITSU MU KO ZE KO NYO RAI SETSU I SSAI HO KAI ZE BU PPO SHU
BO DAI SHO GON I SSAI HO SHA SOKU HI I SSAI HO ZE KO MYO I SSAI HO SHU BO DAI
HI NYO NIN SHIN CHO DAI SHU BO DAI GON SE SON NYO RAI SETSU NIN SHIN CHO DAI
SOKU I HI DAI SHIN ZE MYO DAI SHIN SHU BO DAI BO SA YAKU NYO ZE NYAKU SA ZE GON
GA TO METSU DO MU RYO SHU JO SOKU FU MYO BO SA GA I KO SHU BO DAI JITSU MU U HO
MYO I BO SA ZE KO BU SSETSU I SSAI HO MU GA MU NIN MU SHU JO MU JU SHA SHU BO
DAI NYAKU BO SA SA ZE GON GA TO SHO GON BUTSU DO ZE FU MYO BO SA GA I KO NYO
RAI SETSU SHO GON BUTSU DO SHA SOKU HI SHO GON ZE MYO SHO GON SHU BO DAI NYAKU
BO SA TSU DATSU MU GA HO SHA NYO RAI SETSU MYO SHIN ZE BO SA.
\end{JAPANESE}

\Stanzanum{XVIII}
\begin{JAPANESE}
SHU BO DAI O I UN GA NYO RAI U NIKU GEN FU NYO ZE SE SON NYO RAI U NIKU GEN SHU
BO DAI O I UN GA NYO RAI U TEN GEN FU NYO


ZE SE SON NYO RAI U TEN GEN SHU BO DAI O I UN GA NYO RAI U E GEN FU NYO ZE SE
SON NYO RAI U E GEN SHU BO DAI O I UN GA NYO


RAI U HO GEN FU NYO ZE SE SON NYO RAI U HO GEN SHU BO DAI O I UN GA NYO RAI U
BUTSU GEN FU NYO ZE SE SON NYO RAI U BUTSU GEN SHU BO DAI O I UN GA NYO GO GA
CHU SHO U SHA BU SSETSU ZE SHA FU NYO ZE SE SON NYO RAI SETSU ZE SHA SHU BO DAI
O I UN GA NYO ICHI GO GA CHU SHO U SHA U NYO ZE SHA TO GO GA ZE SHO GO GA SHO U
SHA SHU BUTSU SE KAI NYO ZE NYO I TA FU JIN TA SE SON BUTSU GO SHU BO DAI NI
SHO KOKU DO CHU SHO U SHU JO NYA KKAN SHU SHIN NYO RAI SHI CHI GA I KO NYO RAI
SETSU SHO SHIN KAI I HI SHIN ZE MYO I SHIN SHO I SHA GA SHU BO DAI {\bf KA KO
SHIN FU KA TOKU GEN ZAI SHIN FU KA TOKU MI RAI SHIN FU KA TOKU.}
\end{JAPANESE}

\Stanzanum{XIX}
\begin{JAPANESE}
SHU BO DAI O I UN GA NYAKU U NIN MAN SAN ZEN DAI SEN SE KAI SHI PPO I YO FU SE
ZE NIN I ZE IN NEN TOKU FUKU TA FU NYO ZE SE SON SHI NIN I ZE IN NEN TOKU FUKU
JIN TA SHU BO DAI NYAKU FUKU TOKU U JITSU NYO RAI FU SETSU TOKU FUKU TOKU TA I
FUKU TOKU MU KO NYO RAI SE TTOKU FUKU TOKU TA.
\end{JAPANESE}

\newpage
\Stanzanum{XX}
\begin{JAPANESE}
SHU BO DAI O I UN GA BU KKA I GU SOKU SHIKI SHIN KEN FU HO CHA SE SON NYO RAI
FU O I GU SOKU SHIKI SHIN KEN GA I KO NYO RAI SETSU GU SOKU SHIKI SHIN SOKU HI
GU SOKU SHIKI SHIN ZE MYO GU SOKU SHIKI SHIN SHU BO DAI O I UN GA NYO RAI KA I
GU SOKU SHO SO KEN FU HO CHA SE SON NYO RAI FU O I GU SOKU SHO SO KEN GA I KO
NYO RAI SETSU SHO SO GU SOKU SOKU HI GU SOKU ZE MYO SHO SO GU SOKU.
\end{JAPANESE}

\Stanzanum{XXI}
\begin{JAPANESE}
SHU BO DAI NYO MO CCHI NYO RAI SA ZE NEN GA TO U SHO SE PPO MAKU SA ZE NEN GA I
KO NYAKU NIN GON NYO RAI U SHO SE PPO SOKU I BO BUTSU FU NO GE GA SHO SE KKO
SHU BO DAI SE PPO SHA MU HO KA SETSU ZE MYO SE PPO NI JI E MYO SHU BO DAI BYAKU
BUTSU GON SE SON HA U SHU JO O MI RAI SE MON SETSU ZE HO SHO SHIN JIN FU BUTSU
GO SHU BO DAI HI HI SHU JO HI FU SHU JO GA I KO SHU BO DAI SHU JO SHU JO SHA
NYO RAI SE PPI SHU JO ZE MYO SHU JO.
\end{JAPANESE}

\Stanzanum{XXII}
\begin{JAPANESE}
SHU BO DAI BYAKU BUTSU GON SE SON BU TTOKU A NOKU TA RA SAM MYAKU SAM BO DAI I
MU SHO TOKU YA BUTSU GON NYO ZE NYO ZE SHU BO DAI GA O A NOKU TA RA SAM MYAKU
SAM BO DAI NAI SHI MU U SHO HO KA TOKU ZE MYO A NOKU TA RA SAM MYAKU SAM BO
DAI.
\end{JAPANESE}

\Stanzanum{XXIII}
\begin{JAPANESE}
BU SHI SHU BO DAI {\bf ZE HO BYO DO MU U KO GE ZE MYO A NOKU TA RA SAM MYAKU
SAM BO DAI} I MU GA MU NIN MU SHU JO MU JU SHA SHU I SSAI ZEM PO SOKU TOKU A
NOKU TA RA SAM MYAKU SAM BO DAI SHU BO DAI SHO GON ZEN PO SHA NYO RAI SETSU
SOKU HI ZEN PO ZE MYO ZEN PO.
\end{JAPANESE}

\Stanzanum{XXIV}
\begin{JAPANESE}
SHU BO DAI NYAKU SAN ZEN DAI SEN SE KAI CHU SHO U SHO SHU MI SEN NO NYO ZE TO
SHI PPO JU U NIN JI YO FU SE NYAKU NIN I SHI HAN NYA HA RA MI KYO NAI SHI SHI
KU GE TO JU JI DOKU JU I TA NIN SETSU O ZEN FUKU TOKU HYAKU BUN FU GYU I PPYAKU
SEN MAN NOKU BUN NAI SHI SAN JU HI YU SHO FU NO GYU.
\end{JAPANESE}

\Stanzanum{XXV}
\begin{JAPANESE}
SHU BO DAI O I UN GA NYO TO MO CCHI NYO RAI SA ZE NEN GA TO DO SHU JO SHU BO
DAI MAKU SA ZE NEN GA I KO JITSU MU U SHU JO NYO RAI DO SHA NYAKU U SHU JO NYO
RAI DO SHA NYO RAI SOKU U GA NIN SHU JO JU SHA SHU BO DAI NYO RAI SETSU U GA
SHA SOKU HI U GA NI BON PU SHI NIN I I U GA SHU BO DAI BON PU SHA NYO RAI SETSU
SOKU HI BON PU ZE MYO BON PU.
\end{JAPANESE}

\Stanzanum{XXVI}
\begin{JAPANESE}
SHU BO DAI O I UN GA KA I SAN JU NI SO KAN NYO RAI FU SHU BO DAI GON NYO ZE NYO
ZE I SAN JU NI SO KAN NYO RAI BUTSU GO SHU BO DAI NYAKU I SAN JU NI SO KAN NYO
RAI SHA TEN RIN SHO O SOKU ZE NYO RAI SHU BO DAI BYAKU BUTSU GON SE SON NYO GA
GE BU SSHO SETSU GI FU O I SAN JU NI SO KAN NYO RAI NI JI SE SON NI SETSU GE
GON
\end{JAPANESE}

\begin{center}\bf
NYAKU I SHI KKEN GA	\\
I ON JO GU GA		\\
ZE NIN GYO JA DO	\\
FU NO KEN NYO RAI
\end{center}

\Stanzanum{XXVII}
\begin{JAPANESE}
SHU BO DAI NYO NYAKU SA ZE NEN NYO RAI FU I GU SOKU SO KO TOKU A NOKU TA RA SAN
MYAKU SAN BO DAI SHU BO DAI MAKU SA ZE NEN NYO RAI FU I GU SOKU SO KO TOKU A
NOKU TA RA SAN MYAKU SAN BO DAI SHU BO DAI NYO NYAKU SA ZE NEN HOTSU A NOKU TA
RA SAN MYAKU SAN BO DAI SHIN JA SETSU SHO HO DAN ME SSO MAKU SA ZE NEN GA I KO
HOTSU A NOKU TA RA SAN MYAKU SAN BO DAI SHIN SHA O HO FU SETSU DAN ME SSO. 
\end{JAPANESE}

\newpage
\Stanzanum{XXVIII}
\begin{JAPANESE}
SHU BO DAI NYAKU BO SA I MAN GO GA SHA TO SE KAI SHI PPO JI YO FU SE NYAKU BU U
NIN CHI I SSAI HO MU GA TOKU JO O NIN SHI BO SA SHO ZEN BO SA SHO TOKU KU DOKU
GA I KO SHU BO DAI I SHO BO SA FU JU FUKU TOKU KO SHU BO DAI BYAKU BUTSU GON SE
SON UN GA BO SA FU JU FUKU TOKU SHU BO DAI BO SA SHO SA FUKU TOKU FU O TON JAKU
ZE KO SETSU FU JU FUKU TOKU.
\end{JAPANESE}

\Stanzanum{XXIX}
\begin{JAPANESE}
SHU BO DAI NYAKU U NIN GON NYO RAI NYAKU RAI NYA KKO NYAKU ZA NYAKU GA ZE NIN
FU GE GA SHO SETSU GI GA I KO NYO RAI SHA MU SHO JU RAI YAKU MU SHO KO KO MYO
NYO RAI.
\end{JAPANESE}

\Stanzanum{XXX}
\begin{JAPANESE}
SHU BO DAI NYAKU ZEN NAN SHI ZEN NYO NIN I SAN ZEN DAI SEN SE KAI SUI I MI JIN
O I UN GA ZE MI JIN SHU NYO I TA FU SHU BO DAI GON JIN TA SE SON GA I KO NYAKU
ZE MI JIN SHU JITSU U SHA BU SSOKU FU SETSU ZE MI JIN SHU SHO I SHA GA BU SETSU
MI JIN SHU SOKU HI MI JIN SHU ZE MYO MI JIN SHU SE SON NYO RAI SHO SETSU SAN
ZEN DAI SEN SE KAI SOKU HI SE KAI ZE MYO SE KAI GA I KO NYAKU SE KAI JITSU U
SHA SOKU ZE ICHI GO SO NYO RAI SETSU ICHI GO SO SOKU HI ICHI GO SO ZE MYO ICHI
GO SO SHU BO DAI ICHI GO SO SHA SOKU ZE FU KA SETSU TAN BON PU SHI NIN DON JAKU
GO JI.
\end{JAPANESE}

\newpage
\Stanzanum{XXXI}
\begin{JAPANESE}
SHU BO DAI NYAKU NIN GON BU SSETSU GA KEN NIN KEN SHU JO KEN JU SHA KEN SHU BO
DAI O I UN GA ZE NIN GE GA SHO SETSU GI FU HO CCHA SE SON ZE NIN FU GE NYO RAI
SHO SETSU GI GA I KO SE SON SETSU GA KEN NIN KEN SHU JO KEN JU SHA KEN SOKU HI
GA KEN NIN KEN SHU JO KEN JU SHA KEN ZE MYO GA KEN NIN KEN SHU JO KEN JU SHA
KEN SHU BO DAI HOTSU A NOKU TA RA SAN MYAKU SAN BO DAI SHIN JA O I SSAI HO O
NYO ZE CHI NYO ZE KEN NYO ZE SHIN GE FU SHO HO SSO SHU BO DAI SHO GON HO SSO
SHA NYO RAI SETSU SOKU HI HO SSO ZE MYO HO SSO.
\end{JAPANESE}

\Stanzanum{XXXII}
\begin{JAPANESE}
SHU BO DAI NYAKU U NIN I MAN MU RYO A SO GI SE KAI SHI PPO JI YO FU SE NYAKU U
ZEN NAN SHI ZEN NYO NIN HOTSU BO DAI SHIN SHA JI O SHI KYO NAI SHI SHI KU GE TO
JU JI DOKU JU I NIN EN ZETSU GO FUKU SHO HI UN GA I NIN EN ZETSU FU SHU O SO
NYO NYO FU DO GA I KO
\end{JAPANESE}

\begin{center}\bf
I SSAI U I HO		\\
NYO MU GEN HO YO	\\
NYO RO YAKU NYO DEN	\\
O SA NYO ZE KAN
\end{center}

\begin{JAPANESE}
BU SETSU ZE KYO I CHO RO SHU BO DAI GYU SHO BI KU BI KU NI U BA SOKU U BA I I
SSAI SE KEN TEN NIN A SHU RA TO MON BU SSHO SETSU KAI DAI KAN GI SHIN JU BU GYO
{\bf KON GO HAN YA HA RA MI KYO}.
\end{JAPANESE}

\end{Prayer}

%%%%%%%%%%%%%%%%%%%%%%%%%%%%%%%%%%%%%%%%%%%%%%%%%%%%%%%%%%%%%%%%%%%%%%%%%%%%%%%
%%%% DAIHI SHIN DARANI
\begin{Prayer}{dai_hi_sin_dharani}
	{DAIHI SHIN DARANI}{-}
	{Dharani Wielce Współczującego}

\begin{center}
\samepage{
        Inny tytuł: \textit{Dai Hi En Man Mu Ge Jin Sh\=u}\\
	Skrócony tytuł: \textit{Daihi shu}\\
	(język chiński)
}
\end{center}

\begin{JAPANESE}
\textbf{NAMU KARA TAN N\=O}, TORA Y\=A Y\=A,
NAMU ORI Y\=A, BORYO K\=I CH\=I SHIFU R\=A
Y\=A, FUJI SATO B\=O Y\=A, MOKO SATO B\=O Y\=A,
M\=O K\=O KY\=A RUNI KY\=A Y\=A, \keisu EN, S\=A
HARA H\=A EI SH\=U TAN N\=O TON SH\=A,
NAMU SHIKI R\=I TOI M\=O, ORI Y\=A,
BORYO K\=I CH\=I, SHIFU R\=A
RIN T\=O B\=O, N\=A M\=U N\=O R\=A, KIN J\=I,
K\=I R\=I, M\=O K\=O H\=O D\=O,
SH\=A M\=I S\=A B\=O, \=O T\=O J\=O SH\=U BEN,
\=O SH\=U IN, S\=A B\=O S\=A T\=O, N\=O M\=O
B\=O G\=YA, M\=O H\=A T\=E CH\=O,
T\=O J\=I T\=O, EN, \=O B\=O RY\=O K\=I,
R\=U GY\=A CH\=I, KY\=A R\=A CH\=I, \=I
KIRI M\=O K\=O, FUJI S\=A T\=O, S\=A B\=O S\=A B\=O
M\=O R\=A M\=O R\=A, M\=O K\=I M\=O K\=I,
R\=I T\=O IN K\=U RY\=O K\=U RY\=O,
K\=E M\=O T\=O RY\=O T\=O RY\=O,
H\=O J\=A Y\=A CH\=I, M\=O K\=O H\=O J\=A Y\=A CH\=I,
T\=O R\=A T\=O R\=A, CHIRI N\=I, SHIFU R\=A Y\=A,
SH\=A R\=O SH\=A R\=O, M\=O M\=O H\=A M\=O R\=A,
H\=O CH\=I R\=I, YU (\=I) K\=I YU (\=I) K\=I,
SH\=I N\=O SH\=I N\=O, ORA SAN FURA SH\=A R\=I,
H\=A Z\=A H\=A ZA, FURA SH\=A Y\=A,
K\=U RY\=O K\=U RY\=O, M\=O R\=A K\=U RY\=O K\=U RY\=O,
K\=I R\=I SH\=A R\=O SH\=A R\=O, SH\=I R\=I SH\=I R\=I,
S\=U RY\=O S\=U RY\=O, FUJI Y\=A, FUJI Y\=A,
FUDO Y\=A, FUDO Y\=A, M\=I CHIRI Y\=A, \keisu NORA KIN J\=I,
CHIRI SHUNI N\=O, HOYA MONO, SOMO K\=O,
SHIDO Y\=A, SOMO K\=O, MOKO SHIDO Y\=A, SOMO K\=O,
SHIDO Y\=U K\=I, SHIFU R\=A Y\=A, SOMO K\=O \keisu\ \
NORA KIN J\=I, SOMO K\=O, M\=O R\=A N\=O R\=A
SOMO K\=O, SHIRA S\=U OMO GY\=A Y\=A,
SOMO K\=O, SOBO MOKO SHIDO Y\=A, SOMO K\=O,
SHAKI R\=A OSHI D\=O Y\=A, SOMO K\=O, HODO MOGYA
SHIDO Y\=A, SOMO K\=O, NORA KIN J\=I H\=A GYARA Y\=A,
SOMO K\=O, M\=O HORI SHIN GYARA Y\=A, SOMO K\=O,
NAMU KARA TAN N\=O TORA Y\=A Y\=A, \shokei
NAMU ORI Y\=A, BORYO K\=I CH\=I, SHIFU R\=A Y\=A,
SOMO K\=O, \shokei SHITE D\=O MODO RA, HODO Y\=A S\=OM\=O K\=O.
\end{JAPANESE}
\end{Prayer}

%%%%%%%%%%%%%%%%%%%%%%%%%%%%%%%%%%%%%%%%%%%%%%%%%%%%%%%%%%%%%%%%%%%%%%%%%%%%%%%
%%%% DHARANI WIELCE WSPÓŁCZUJĄCEGO
\begin{Prayer}{dharani_wielce_wspolczujacego}
	{DHARANI WIELCE WSPÓŁCZUJĄCEGO}{-}
	{Daihi Shin Dharani}

\medskip
\begin{Verse}
	Chwała Trzem Klejnotom\\
	Buddzie, Dharmie i~Sandze!\\
	Chwała Bodhisattwie Mahasattwie, Awalokiteśwarze,\\
	Bodhisattwie współczucia.\\
	Chwała temu, który usuwa wszelki strach i~cierpienia!\\
	Czcząc Bodhisattwę Awalokiteśwarę, śpiewajmy tę cudowną Dharani, która oczyszcza wszystkie odczuwające istoty, która spełnia pragnienia wszystkich istot.\\
	Chwała Bodhisattwie Mahasattwie, który jest ucieleśnieniem Trikayi, który jest obdarzony transcendentalną mądrością.\\
	Chwała Bodhisattwie Mahasattwie, który nieustająco wyzwala wszelkie odczuwające istoty z~nieskalanym umysłem.\\
	Chwała Bodhisattwie Mahasattwie, który nieprzerwanie podtrzymuje Najwyższą i~Całkowitą mądrość i~który wolny jest od wszelkich przeszkód.\\
	Chwała Bodhisattwie Mahasattwie, którego czyny ukazują fundamentalną czystość wszystkich istot.\\
	Chwała Bodhisattwie Mahasattwie, który usuwa trzy złe ułudy~-- chciwość, gniew i~szaleństwo.\\
	Szybko! Szybko! Przyjdź! Przyjdź! Tu! Tu!\\
	Radość tryska z~nas. Pomóż nam wejść w~Krainę Wielkiego Urzeczywistnienia.\\
	Bodhisattwo Awalokiteśwaro, Bodhisattwo Współczucia, prowadź nas do duchowej radości.\\
	Spełnienie! Spełnienie!\\
	Dając świadectwo wolności i~współczuciu umysłu Awalokiteśwary, oczyszczając swe własne ciało i~umysł,\\
	Stając się nieustraszonym jak lew,\\
	Manifestując się we wszystkich istotach,\\
	Doprowadzając do perfekcji Koło Dharmy i~Kwiat Lotosu, możemy bez przeszkód wyzwolić wszystkie odczuwające istoty.\\
	Oby Rozumienie tajemnej natury Awalokiteśwary rozpowszechniło się na zawsze i~wszędzie.\\
	Chwała Trzem Klejnotom~-- Buddzie Dharmie i~Sandze!\\
	Chwała Bodhisattwie Mahasattwie Awalokiteśwarze,\\
	Bodhisattwie współczucia!\\
	Oby ta Dharani była skuteczna.\\
	Chwała!\\
\end{Verse}
\end{Prayer}

%%%%%%%%%%%%%%%%%%%%%%%%%%%%%%%%%%%%%%%%%%%%%%%%%%%%%%%%%%%%%%%%%%%%%%%%%%%%%%%
%%%% SHO SAI MYO KI J\=O JIN SH\=U
%%%% Daikan
\begin{Prayer}{sho_sai_myo_ki_jo}
	{SHO SAI MYO KI J\=O JIN SH\=U (rinzai)\\SHO SAI SHU}{SHO SAI MYO KI J\=O JIN SH\=U}
	{Dharani Usuwająca Nieszczęścia}

\bigskip
\begin{JAPANESE}
\textbf{NA MU SA MAN DA}\\
MO TO NAN O HA RA CHI KO TO SHA SO NO NAN TO JI TO EN GYA GYA GYA KI GYA KI UN
NUN SHI FU RA SHI FU RA HA RA SHI FU RA HA RA SHI FU RA CHI SHU SA CHI SHU SA
SHU SHI RI SHU SHI RI SO HA JA SO HA JA SE CHI GYA SHI RI EI SO MO KO
\end{JAPANESE}
\end{Prayer}

%%%%%%%%%%%%%%%%%%%%%%%%%%%%%%%%%%%%%%%%%%%%%%%%%%%%%%%%%%%%%%%%%%%%%%%%%%%%%%%
%%%% SH\=OSAI MY\=OKICHIJ\=O DARANI
\begin{Prayer}{sio_saj_mio}
	{SH\=OSAI MY\=OKICHIJ\=O DARANI (soto)}{-}
	{Dharani Usuwająca Nieszczęścia}

\begin{center}
	Skrócony tytuł: \textit{SH\=OSAI SHU}\\
	(język chiński)
\end{center}

% !!! czy tutaj układ tekstu ma być jak w wierszu; formatowanie źródła na to
% wskazuje, ale nie ma znaczników /w.
\begin{JAPANESE}
	N\=O M\=O SAM MAN D\=A, MOTO NAN, OHA R\=A CH\=I KOTO SH\=A,
	SONO NAN \keisu\footnotemark{} T\=O J\=I T\=O, EN, GY\=A GY\=A, GY\=A K\=I GY\=A
	K\=I, UN NUN, SHIHU R\=A SHIHU R\=A, HARA SHIHU R\=A HARA SHIHU R\=A,
	CHISHU S\=A CHISHU S\=A, CHI \shokei\footnotemark{} SHU R\=I CHI SHU R\=I,
	SOWA J\=A SOWA J\=A, \shokei\footnotemark[\value{footnote}] SEN CH\=I GY\=A, SHIRI EI S\=O M\=O K\=O.
\end{JAPANESE}

\footnotetext{Uderzenie w~keisu tylko za pierwszym i~ostatnim razem.}
\footnotetext{Uderzenie w~shokei tylko za ostatnim razem.}

\bigskip
\begin{Verse}
	Chwała wszystkim Buddhom w~Trzech Światach i~Dziesięciu Kierunkach.\\
	Cześć Niezrównanemu, który potrafi usunąć wszelkie nieszczęścia.\\
	Cześć Niezrównanemu, który przenika cały wszechświat, odsłaniając
	Trikaya odpowiednio do okoliczności.\\
	On jest Światłem, wielkim Światłem samym w~sobie. To wspaniałe
	Światło przyciąga wszystkie istoty do sfery Buddhy i~ocala je. Tak
	właśnie znikają nieszczęścia.\\
	Jasne urzeczywistnienie pojawia się i~moc tej Dharani została
	wypełniona.\\
\end{Verse}
\end{Prayer}

\newpage
%%%%%%%%%%%%%%%%%%%%%%%%%%%%%%%%%%%%%%%%%%%%%%%%%%%%%%%%%%%%%%%%%%%%%%%%%%%%%%%
%%%% BU CHIN SON SHIN DARANI
\begin{Prayer}{bu_chin_son_shin_dharani}
	{BU CHIN SON SHIN DARANI}{-}
	{dla wszystkich bóstw chroniących świątynię}

\begin{center}
	\small
	Inna wersja tytułu: \textit{Butch\=o sonsh\=o darani (Sonsh\=o darani)}\\
	(język chiński)
\end{center}

\begin{JAPANESE}
N\=O BO BA GYA BA TE TA RE RO KI Y\=A HA RA CH\=I BI SHI SHU DA YA HO DA YA BA
GYA BA TE TA NI YA TA \keisu OM BI SHU DA YA BI SHU DA YA SA MA SA MA SAN MAN
DA HA BA SHA SO HA RA DA GYA CHI GYA GYA N\=O SO BA HAN BA BI SHU TEI A BI SHIN
SH\=A TO MAN SO GYA T\=A HA RA HA SHA NO A MI RI T\=A BI SEI KEI MA KA MAN DA
RA HA DA I A KA R\=A A KA R\=A A YU SAN DA RA N\=I SHU DA YA SHU DA YA GYA GYA
N\=O BI SHU TEI U SHU NI SHA BI SHA YA BI SHU TEI SA KA SA R\=A A RA SHIN MEI
SAN SO NI TEI SA RA B\=A TA TA GYA T\=A BA RO GYA N\=I SA TA HA RA MI T\=A HA
RI HO RA N\=I SA RA B\=A TA TA GYA T\=A KI RI DA YA CHI SHU TA N\=O CHI SHU CHI
T\=A MA KA BO DA REI BA SA RA GYA Y\=A SO GYA TA NO BI SHU TEI SA RA B\=A HA RA
D\=A HA YA TO RI GYA CHI HA RI BI SHU TEI HA RA CHI N\=I HA RA DA Y\=A A YU KU
SHU TEI SAN MA Y\=A CHI SHU CHI TEI MA NI MA N\=I MA KA MA N\=I TA TA TA BO
D\=A KU CHI HA RI SHU TEI BI SO BO D\=A BOCHI SHU TEI \keisu SHA YA SHA YA B\=I
SHA YA BI SHA Y\=A SA MO R\=A SA MO R\=A SA RA BA BO DA CHI SHU CHI TA SHU TEI
BA SHI RI BA SA RAN GYA RA BEI BA SA RAN HA BA \keisu TO BA MAN SHA RI RAN SA
RA BA SA TO BA BNAN SHA KYA Y\=A HA RI BI SHU TEI SA RA BA GYA CHI HA RI SHU
TEI SA RA B\=A TA TA GYA T\=A SHI SHA MEI SAN BA JIN BA SO EN TO SA RA B\=A TA
TA GYA T\=A SAN BA JIN BA SO JI SHU CHI TEI BO JI YA BO JI YA BI BO JI YA BI BO
JI YA BO DA YA BO DA YA BI BO DA YA BI BO DA YA SAN MAN D\=A HA RI SHU TEI SA
RA B\=A TA TA GYA T\=A \shokei KI RI DA Y\=A CHI SHU TA N\=O CHI SHU CHI T\=A
\shokei MA KA BO DA REI SO WA KA
\end{JAPANESE}
\end{Prayer}

%%%%%%%%%%%%%%%%%%%%%%%%%%%%%%%%%%%%%%%%%%%%%%%%%%%%%%%%%%%%%%%%%%%%%%%%%%%%%%%
%%%% YAKUSHI SON SHO DHARANI
%%%% Tekst z zeszytu Daikana
\begin{Prayer}{yakushi_son_sho_dharani}
	{YAKUSHI SON SHO DHARANI}{-}
	{-}

\medskip
\begin{JAPANESE}
NO BO BA GYA BA TE BI SE JA KUROBE RUBIYA HARABA HARASHA YA TA TA GYA TA YA O
RA KO CHI SAN MYAKU SAN BO DA YA TA GYA TA ON BI SE ZE BI SE ZE BI SE JA SAN
BORI GYA TE SO WA KA
\end{JAPANESE}
\end{Prayer}

%%%%%%%%%%%%%%%%%%%%%%%%%%%%%%%%%%%%%%%%%%%%%%%%%%%%%%%%%%%%%%%%%%%%%%%%%%%%%%%
%%%% GYAKU ON JIN SHU
%%%% Tekst z zeszytu Daikana
\begin{Prayer}{gyaku_on_jin_shu}
	{GYAKU ON JIN SHU}{-}
	{Dharani oczyszczenia}

\begin{JAPANESE}
NAMU FU DO Y\=A. NAMU DABO Y\=A. NAMU SUN GYA Y\=A. NAMU JI H\=O SH\=I B\=U.
NAMU SH\=I BU S\=A MOKO S\=A. NAMU SH\=I SHIN SUN. NAMU SHU SH\=I. SARA GY\=A
SARA GY\=A SARA GY\=A. MUTO NAN K\=I. AGYA N\=I K\=I. NIGYA SH\=I K\=I. AGYA
N\=A K\=I. BARA N\=I K\=I. ABI RA K\=II. HA DAI R\=I K\=I. SHIK K\=O SHIK K\=O.
MAKU TOKU KU JU.
\end{JAPANESE}
\end{Prayer}

\bigskip

\begin{Verse}
	Znajduję schronienie w~Buddzie,\\
	Znajduję schronienie w~Dharmie,\\
	Znajduję schronienie w~Sandze,\\
	Znajduję schronienie w~Buddhach Dzięsięciu Kierunków,\\
	Znajduję schronienie w~wielkich Bodhisattwach,\\
	Znajduję schronienie w~Arhatach,\\
	Znajduję schronienie w~oświeconych mistrzach.\\
	Cała zła karma\\
	Precz! Precz! Precz!\\
	Siedem demonów\\
	Szybko! Szybko!\\
	Wynoście się! Wynoście się!
\end{Verse}

%%%%%%%%%%%%%%%%%%%%%%%%%%%%%%%%%%%%%%%%%%%%%%%%%%%%%%%%%%%%%%%%%%%%%%%%%%%%%%%
%%%% HAKU SHIN DHARANI
%%%% Tekst z zeszytu Daikana
\begin{Prayer}{haku_shin_dharani}
	{HAKU SHIN DHARANI}{-}
	{-}

\begin{JAPANESE}
\textbf{ON O NORI}\\
BI SHA CHI BI RA HOJA RATORI HO DO HO DO NI HO JA RA HO NI HAN KU KI TSU RO YO HAN SO MO KO
\end{JAPANESE}
\end{Prayer}

%%%%%%%%%%%%%%%%%%%%%%%%%%%%%%%%%%%%%%%%%%%%%%%%%%%%%%%%%%%%%%%%%%%%%%%%%%%%%%%
%%%% EMMEI JIKKU KANNON GY\=O
\begin{Prayer}{emmei_jikku}
	{EMMEI JIKKU KANNON GY\=O}{-}
	{Przedłużająca Życie Sutra Kannon w~Dziesięciu Wersach}

\begin{center}
	\scriptsize
	Inny tytuł: \textit{Enmei Jukku Kannon Gyo}
\end{center}

\begin{Verse}
\stanza{
	KAN ZE ON		\\
	N\=A M\=U BUTSU		\\ % !!! u Daikana N\=AM\=U
	Y\=O BUTSU \=U~IN	\\
	Y\=O BUTSU \=U~EN	\\
	BUP\=O P\=O S\=O EN	\\ % !!! u Daikana BUPPO S\=O EN
	J\=O RAKU GA J\=O	\\ % !!! u Daikana JO RAKU G\=A JO
	CH\=O NEN KAN Z\=E ON	\\
	B\=O NEN KAN Z\=E ON	\\
	NEN NEN J\=U SHIN K\=I	\\
	NEN NEN F\=U R\=I SHIN
}
\end{Verse}
\end{Prayer}

%%%%%%%%%%%%%%%%%%%%%%%%%%%%%%%%%%%%%%%%%%%%%%%%%%%%%%%%%%%%%%%%%%%%%%%%%%%%%%%
%%%% PRZEDŁUŻAJĄCA ŻYCIE SUTRA KANNON W~DZIESIĘCIU WERSACH
\begin{Prayer}{sutra_kannon}
	{PRZEDŁUŻAJĄCA ŻYCIE SUTRA KANNON W~DZIESIĘCIU WERSACH}{-}
	{Emmei Jikku Kannon Gy\=o}

\begin{Verse}
\stanza{
	Kandzeon\\
	Chwała Buddzie\\
	Jesteśmy w przyczynie z Buddhą\\
	Jesteśmy w związku z Buddhą\\
	W przyczynowym związku z Buddhą, Dharmą, Sanghą\\
	Wieczność, Radość, Jaźń i Czystość\\
	Poranna myśl Kandzeon\\
	Wieczorna myśl Kandzeon\\
	Każda myśl powstaje z Umysłu\\
	Każda myśl nie jest oddzielona od Umysłu.
}
\end{Verse}
\end{Prayer}

\newpage
%%%%%%%%%%%%%%%%%%%%%%%%%%%%%%%%%%%%%%%%%%%%%%%%%%%%%%%%%%%%%%%%%%%%%%%%%%%%%%%
%%%% RYO GON SHU (z zeszytu Dai Bosatsu)
\begin{Prayer}{ryo_gon_shu}
	{RYO GON SHU}{-}
	{}
% !!! sprawdzić jak ten tekst wyjdzie po kompilacji

\begin{center}
\it	Prolog
\end{center}

\begin{JAPANESE}
\textbf{NAO REN NEN UI JO} JI HO ZO \\

REN NEN UI JO JI HO ZO
REN NEN UI JO JI HO ZO
REN NEN UI JO JI HO ZO
\end{JAPANESE}

\begin{center}
\it Część I
\end{center}

\begin{JAPANESE}
\textbf{NAMU SATAN DO}\\
SUGYATO YA ORAKO CHI SAMYASA FUDOSHA SATADO
FUDOKYU SHI SHUNISAN NAMUSA BO FUDOFU CHI
SATOBI BYA NAMUSATONAN SAMYASAFUDO KYUSHINAN
SOJARABOGYA SUGYANAN NAMURYO KI ORAKATONAN
NAMUSU RYO TOBONONAN NAMUSOGERITO GYAMINAN
NAMURYO KI SAMYAGYATONAN SAMYAGYAHORA CHIBOTONONAN
NAMUCHI BO RISHUNAN NAMUSHIDOYA BICHIYA
TORARISHUNAN SHA HO NO KERAKO SOKO SORAMOTONAN
NAMUHORAKOMONI NAMUIN TO RA YA NAM BOGYABO CHI
RYO TO RA YA UMO HO CHI SO KI YA YA NAM BOGYABO
CHI NORAYANOYA HOJAMO KO SAMOTO RA NAMUSHIGERITOYA
NAM BOGYABO CHI MOKOKYARAYA CHIRIHORANO
KYE RA BIDORA HONO GYARAYA OCHIMO CHI
SHIMOSHANONI HOSHINI MOTORIKYANO
NAMUSHIGERITOYA NAM BOGYABO CHI TOTOGYATO
KYURAYA NAMUHOCHIMO KYURAYA NAMUHOJARA KYURAYA
NAMUMO NI KYURAYA NAMUKYAJAKYURAYA NAM
BOGYABO CHI CHIRISA SHURASHINO HORAKORA 
NORASHAYA TOTOGYATOYA NAM BO GYA BO CHI
NAMUOMITOBOYA TOTOGYATOYA NAM BO GYA BO CHI
NAMUOMITOBOYA TOTOGYATOYA ORAKO CHI SAMYASA FUDOYA 
NAM BO GYA BO CHI O SU BIYA TOTOGYATOYA ORAKO CHI
SAMYASA FUDOYA NAM BO GYA BO CHI BI SHA JAYA
KYURYO BISHURIYA HORA HORASHAYA TOTOGYATOYA NAM
BO GYA BO CHI SAN BUSU BITO SAREN NORARASHAYA
TOTOGYATOYA ORAKO CHI SAMYASA FUDOYA NAM BO
GYA BO CHI SHAKIYA MONOEI TOTOGYATOYA ORAKO
CHI SAMYASA FUDOYA NAM BO GYA BO CHI RATANO
KITSURASHAYA TOTOGYATOYA ORAKO CHI SAMYASA FUDOYA
CHIBYANA MU SOGERITO EI TAN BOGYABO TO  SATADO
GYATSUSHUNISAN SATADO HODORA NAMUOHORA SHITAN
HORA CHIYO KIRA SARABO FUDO KERAKO NIKERAKO
KEGYARAKONI HORABICHIYA SHIDONI OKYARA MIRISHU
HORITORAYA NIKERI SARABO HODONO MOSHANI SARABO
TOSHUSA TOSHIHAN HONONI HORANI SHA TSU RA SHICHINAN
KERAKO SOKO SORASHAJA BIDOBEN SANOKE RI OSHUSA
BI SHACHINAN NOSHAJA TORASHAJA HORA SATONOKERI
OSHUSA NAN MOKO KERAKOSHAJA BIDOBEN SANOKE
RI SABOSHA TSURYONI HORASHAJA KORATOSHIHAN
NOSHANOSHANI BISHAJA SHIDORA OKINI UTO
KYARASHAJA OHORA SHIDO KYURA MOKOHORASEN SHI
MOKOTECHO MOKOCHISHA MOKOSUI TO SHAHORA
MOKOHORA HODORA HOSHINI ORIYATORA BIRIKYU SHI
SHIBOBISHAYA HOJARA MORICHI BISHARYO TO HODOMOKYA
HOJARA SHI KANO-OSHA MORASHIBO HORASHIDO HOJARA
SEN SHI BISHARASHA SETOSHA BICHIBO FUSHIDO SUMORYO
BO MOKOSUI TO ORIYATORA MOKOHORA OHORA HOJARA
SHAKERASHIBO HOJARA KYUMORI KYURATO RI HOJARA
KASATOSHA BICHIYA KESHANO MORIGYA KUSOMOBO
KERATONO BIRUSHANO KYURIYA YARATO SHUNISAN BISHARYO
BO MONISHA HOJARA KYANO KYAHORABO RYOSHANO HOJARA
TOSHISHA SUTOSHA KYAMORA SASHASHI HORABO EI CHI CHI
MOTORAKENO SOBIRASAN KIHAN TSUIN TSUNOMO MO SHA
\end{JAPANESE}

\begin{center}
\it Część II
\end{center}

\begin{JAPANESE}
\textbf{UKIRI SHU KE NO}
HORASHA SHI DO SATADO KYATSUSHUNISAN KUKITSURYOYO
SEBONO KUKITSURYOYO SHITAHONO KUKITSURYOYO
HORASHUCHIYA SABOSHA NOKERA KUKITSURYOYO
SABOYASHA KARASASO KERAKOSHAJA BIDOBEN SANOKE
RA KUKITSURYOYO SHATSURA SHICHINAN KERAKO SOKO
SORANAN BIDOBEN SANORA KUKITSURYOYO RASHABOGYABAN
SATADO KYATSUSHUNISAN HORATEN SHAKIRI MOKO
SOKOSARA FUJUSOKO SARASHIRISA KYUSHISOKO SANICHIRI OBICHI
SHIBORITO SA SA AGYA MOKOHOJARYOTORA
CHIRIFUBONO MAN SARA-UKIN SOSHICHI HOBOTSU MOMO IN 
TSUNOMO MO SHA
\end{JAPANESE}

\begin{center}
\it Część III
\end{center}

\begin{JAPANESE}
\textbf{RASHA BO YA}
SHURABO YA OKINIBOYA UTOKYABOYA BISHABOYA
SHASATORABOYA HORA SHAKERABOYA TOSHISHABOYA
OSHANIBOYA OKYARA MIRISHUBOYA TORANI FUMIKEN
BOGYABOTOBOYA URAKYABOTOBOYA RASHA TAN
SHABOYA NOKYA BOYA BISHUTABOYA SUBORANOBOYA
YASHA KERAKO RASHASU KERAKO BIRIDO KERAKO
BISHAJA KERAKO FUDOKERAKO KYUHAZA KERAKO FUTANO
KERAKO KYASHAFUTANO KERAKO SHIGEDO KERAKO OHASHIMORA
KERAKO UTAMOTOKERAKO SHAYAKERAKO KIRIHOCHIKERAKO
SHATOKORINAN KEBOKORINAN RYOCHIRA KORINAN MOSO
KORINAN METO KORINAN MOSHA KORINAN SHATO KORINI
SHIBIDO KORINAN BIDO KORINAN HODO KORINAN
OSHUSA KORINI SHIDO KORINI CHISASABISAN SABO KERAKONAN
BIDOYASHA SHIDOYAMI KIRAYAMI HORIHORA SHAGYA
KIRITAN BIDOYASHA SHIDOYAMI KIRAYAMI SAENI KIRITAN
BIDOYASHA SHIDOYAMI KIRAYAMI MOKOHOJUHODOYA
RYOTORA KIRITAN BIDOYASHA SHIDOYAMI KIRAYAMI
NORAYANO KIRITAN BIDOYASHA SHIDOYAMI KIRAYAMI TOTOGYA
RYOSASHI KIRITAN BIDOYASHA SHIDOYAMI KIRAYAMI
MOKOKYARA MOTORIKYANO KIRITAN BIDOYASHA SHIDOYAMI
KIRAYAMI KYAHORIGYA KIRITAN BIDOYASHA SHIDOYAMI
KIRAYAMI SHAYAKERA MOTOKERA SABORATO SOTONO
KIRITAN BIDOYASHA SHIDOYAMI KIRAYAMI SHATSURA HOKINI
KIRITAN BIDOYASHA SHIDOYAMI KIRAYAMI BIRIYO KIRISHI NATO
KISARA KYANOHOCHI SOKIYA KIRITAN BIDOYASHA SHIDOYAMI
KIRAYAMI NOKENO SHARAHONO KIRITAN BIDOYASHA SHIDOYAMI
KIRAYAMI ORAKAN KIRITAN BIDOYASHA SHIDOYAMI
KIRAYAMI BIDORAGYA KIRITAN BIDOYASHA SHIDOYAMI
KIRAYAMI HOJARAHONI KYUKIYA KYUKIYA KYACHIHOCHIKIRITAN
BIDOYASHA SHIDOYAMI KIRAYAMI RASHABO BOGYABAN
INTSUNO MOMO SHA
\end{JAPANESE}

\begin{center}
\it Część IV
\end{center}

\begin{JAPANESE}
\textbf{BOGYA BASA TAN DO}
HODORA NAMUSUI TO CHI OSHIDO NORARAGYA HORABO
SHIFUSA BIGYASATADO HOCHIRI SHIFURA SHIFURA TORA TORA
BIDORA BIDORA SHIDOSHIDO KUKIKUKI HAZAHAZA HAZAHAZA
HAZASOKO KI KI HAN OMOGYAYAHAN OHORACHI
KOTOHAN HORA HORATOHAN OSURA BIDORA BOGYABAN
SABOCHI BIBIHAN SABO NOKYABIHAN SABOYASHABIHAN
SABOKETOBOBIHAN SABOFUTANOBIHAN KYASHAFUTANOBIHAN
SABOTORYO KICHIBIHAN SABOTOSHUBIRI KISHUCHIBIHAN
SABO SHIBORIBIHAN SABO OHASHIMORIBIHAN
SABOSHARA HONOBIHAN SABOCHICHIKIBIHAN SABOTAMO
TOKIBIHAN SABOBIDOYARA SHISHARIBIHAN SHAYAKERA
MOTOKERA SABORATO SOTOKIBIHAN BICHIYA
SHARIBIHAN SHATSURA HOKINIBIHAN HOJARA KYUMORI
BIDOYARASHIBIHAN MOKO HORA CHIYO SHAKIRIBIHAN
HOJARA SHO KERAYA HORA SHAKIRASHA EHAN MOKO
KYARAYA MOKOMOTORIKYANO NAMUSOGERITO YAHAN
BISHUNO BIEHAN HORAKO MONIEHAN OKINIEHAN 
MOKOKERIEHAN KERATOJIEHAN METORIEHAN RO TORIEHAN
SHAN BUSO EHAN KERARATORIEHAN KYAHORIEHAN
OCHIMOSHIDO KYASHIMOSHANO HOSUNIEHAN EN KISHISA
TOBOSHA MOMOIN TSUNOMO MO SHA
\end{JAPANESE}

\begin{center}
\it Część V
\end{center}

\begin{JAPANESE}
\textbf{TO SHU SA SHI DO}
OMOTORISHI DO USHA KORA KYABOKORA RYOCHIRAKORA
HOSOKORA MOSHAKORA SHADOKORA SHIBIDOKORA
HORAYA KORA KEN TOKORA FUSUBOKORA HORAKORA
HIJAKORA HOBOSHIDO TOSHUSASHIDO RO TORASHIDO
YASHA KERAKO RASHASU KERAKO BIRIDO KERAKO
BISHAJA KERAKO FUDOKERAKO KYUHAZA KERAKO SHIGEDO
KERAKO UTAMOTOKERAKO SHAYAKERAKO OHASAMORA
KERAKO SAKIGASA KINIKERAKO RIFUJI KERAKO SHAMIGYA
KERAKO SHAKINI KERAKO MOTORA MACHIGYA
KERAKO ORABO KERAKO KETOHONIKERAKO SHIFURA IGYA
KIGYA SU-I-CHIYAGYA TORICHIYAGYA SHATO TAGYA
NICHISHIFURA BISAMO SHIFURA HOCHIGYA BI CHIGYA
SHIRISHUMIGYA SONIHOCHIGYA SABO SHIFURA SHIRYO KICHI
MOTOBI TARYOSHIKEN OKIRYOKEN MOKIRYOKEN
KERITO RYOKEN KERAKOKERAN KENOSHURAN TAN TOSHURAN
KIRIYA SHURAN MOMOSHURAN HORISHIBOSHURAN
BIRISHUSASHURAN UTORASHURAN KESHISHURAN HOSHICHISHURAN
URYO SHURAN SHA GYASHURAN KASHIDOSHURAN HODOSHURAN
SOBOAGYA HORASHAGYASHURAN FUDOBIDOSA SAKINI SHIFURA
TOTORYOGYA KETORYOKISHI HORUTO BI SAHORYO KORIGYA
SHUSATORA SONOKERA BISAYU GYA OKINI UTOGYA MORABIRA
KENTORA OGYARA MIRISHU TAREBOGYA CHIRIRASA
BIRISHUSHIGYA SABONO KYURA SU-IGYA BI KERARIYASHA
TORASU MORASU BICHISAN SOBISAN SHITEDO HODORA
MOKOHOJARYO SHUNISAN MOKOHORA SHAKIRAN
YAHOTODO SHAYUSHANO METORINO BIDOYA
HODOKYARUMI CHISHUHODOKYARUMI HORABIDO
HODOKYARUMI TOJITO EN ONORI BISHACHI 
BIRAHOJARATORI HODOHODONI HOJARA HONIHAN
KUKITSURYOYO HAN SO MO KO\\

MO KO HO JA
HO RO MI\\
MO KO HO JA HO RO MI \\
MO KO HO JA HO RO MI\\
MO KO HO JA HO RO MI\\


\textbf{RYOGON SHU}
\textbf{FU E KO}\\

JO RAI EN ZEN BI KYU SHU \\
FUN ZU REN NEN HI MI SHU\\
UI KYO U HA SHU RUN TEN\\
TSU CHI GYA RAN SHI SHIN ZO\\
SAN ZU HA NAN KYU RI KU\\
SU IN SAN NYU JIN SEN IN\\
KU KAI AN NIN HIN KA SHO\\
FUN CHO I JUN MIN KO RA\\
I SHU HIN SHU KI SHIN JIN\\
JI JI ZUN JO BU NAN ZU\\
SAN MON SHIN JIN ZE HI NI\\
DAN SHIN KI SUN ZUN PU I\\
JI HO SAN SHI I SHI FU\\
SHI SON BU SA MO KO SA\\
MO KO HO JA HO RO MI\\
\end{JAPANESE}
\end{Prayer}

\newpage
%%%%%%%%%%%%%%%%%%%%%%%%%%%%%%%%%%%%%%%%%%%%%%%%%%%%%%%%%%%%%%%%%%%%%%%%%%%%%%%
%%%% MANTRA BHAISADZJAGURU
\begin{Prayer}{mantra_bhaisadzjaguru}
	{MANTRA BHAISADZJAGURU}{-}
	{Buddhy Uzdrawiania}

	\begin{Verse}\wersaliki
	\stanza{
		NA MO BHA GA WA TE	\\
		BHAI SA DŻJA GU RU	\\
		WAI DU RJA
	}

	\stanza{
		PRA BHA RA DŻA JA	\\
		TA THA GA TA JA		\\
		AR HA TE
	}

	\stanza{
		SAM JAK SAM BUD DA JA	\\
		TA DJA THA OM
	}

	\stanza{
		BHAI SA DŻJE	\\
		BHAI SA DŻJE	\\
		BHAI SA DŻJE	\\
		SAM MUD GA TE	\\
		SWA HA
	}
	\end{Verse}
\end{Prayer}

%%%%%%%%%%%%%%%%%%%%%%%%%%%%%%%%%%%%%%%%%%%%%%%%%%%%%%%%%%%%%%%%%%%%%%%%%%%%%%%
%%%% SAND\=OKAI
\begin{Prayer}{sandokai}
	{SAND\=OKAI}{-}
	{Osiągnięcie Zjednoczenia}

\begin{center}
	\scriptsize
	(język japoński)
\end{center}

\bigskip
\begin{JAPANESE}
CHIKUDO DAI SEN NO SHIN, T\=O ZAI MITSU NI AI FUSU, NIN KON NI RIDON ARI,
D\=O NI NAM BOKU NO SO NASHI, REI GEN MY\=O NI K\=O KETTA RI, SHIHA AN NI
RUCH\=U SU, JI O~SH\=U SURU MO MOTO KORE MAYOI, RI NI KANA UMO MATA SATORI
NI ARAZU, \keisu MON MON ISSAI NO KY\=O, EGO TO FU EGO TO, ESHITE SARANI AI
WATARU, SHIKARA ZAREBA KU RAI NI YOTTE J\=U SU, SHIKI MOTO SHITSU ZO
O~KOTONI SHI SH\=O MOTO RAKKU O~KOTO NI SU, AN WA J\=O CH\=U NO KOTO NI
KANAI, MEI WA SEI DAKU NO KU O~WAKATSU, SHIDAI NO SH\=O ONOZU KARA FUKUSU,
KONO SONO HAHA O~URU GA GOTOSHI, HI WA NESSHI, KAZE WA D\=OY\=O, MIZU WA
URU O\=I CHI WA KEN GO, MANAKO WA IRO, MIMI WA ONJ\=O, HANA WA KA, SHITA WA
KAN SO, SHIKAMO ICHI ICHI NO H\=O NI OITE, NE NI YOTTE HABUNPU SU, HON
MATSU SUBE KARAKU SH\=U NI KISU BESHI, SONPI SONO GO O~MOCHI YU, MEI CH\=U
NI ATATTE AN ARI, AN S\=O O~MOTTE AUKOTO NAKARE, AN CH\=U NI ATATTE MEI
ARI, MEI S\=O O~MOTTE MIRU KOTO NAKARE, MEI AN ONO ONO AI TAI SHITE, HISU
RUNI ZEN GO NO AYUMI NO GOTOSHI, \keisu BAN MOTSU ONOZU KARA K\=O ARI,
MASANI Y\=O TO SHO TO O~IU BESHI, JISON SUREBA KAN GAI GASSHI, RI \=O
ZUREBA SEN P\=O SAS\=O, \keisu KOTO O~UKETE WA SUBE KARAKU SH\=U O~ESU
BESHI, MIZUKARA KIKU O~RIS-SURU KOTO NAKARE, SOKU MOKU D\=O O~ESE ZUNBA,
ASHI O~HAKOBU MO IZU KUN ZO MICHI O~SHIRAN, AYUMI \keisu SUSU MUREBA GON
NON NI ARAZU, MA Y\=OTE SEN GA NO KO O~HEDA TSU, \shokei TSUTSUSHIN DE SAN
GEN NO HITO NI M\=OSU, \shokei K\=O IN MUNA SHIKU WATARU KOTO NAKARE.
\end{JAPANESE}
\end{Prayer}

\newpage
%%%%%%%%%%%%%%%%%%%%%%%%%%%%%%%%%%%%%%%%%%%%%%%%%%%%%%%%%%%%%%%%%%%%%%%%%%%%%%%
%%%% OSIĄGNIĘCIE ZJEDNOCZENIA !
\begin{Prayer}{osiagniecie_zjednoczenia}
	{OSIĄGNIĘCIE ZJEDNOCZENIA}{-}
	{Sand\=okai}

\smallskip
\noindent
Umysł wielkiego mędrca Indii został tajemnie przekazany ze wschodu na
zachód. Ludzkie zdolności są błyskotliwe lub \mbox{tępe}, ale Droga nie ma
patriarchów z~północy czy południa. Duchowe źródło promieniuje z~czystością,
poboczne szkoły płyną w~ciemnościach. Chwytanie rzeczy jest pierwotnie
złudzeniem, zgadzanie się z~zasadą nie jest jeszcze oświeceniem. Każda brama
i~wszystkie pola współdziałają i~nie-współdziałają. Gdy współdziałają
wzajemnie się przenikają. Inaczej pozostają na swoim miejscu. Formy różnią
się kształtem i~substancją, dźwięki są różne miłe i~niemiłe. Ciemność łączy
najwyższe i~średnie słowa, światłość rozróżnia wyrażenia czyste i~skalane.
Cztery wielkie elementy same powracają do swej natury, tak jak dziecko
znajduje swoją matkę. Ogień grzeje, wiatr porusza się, woda moczy, ziemia
jest twarda. Oko rozpoznaje formy i~kolory, ucho dźwięki i~głosy, nos
zapachy, a~język smaki. Dlatego każde zjawisko (dharmy) jest jak liście,
pojawiające się w~zależności od korzenia. Początek i~koniec muszą powrócić
do esencji. Szlachetny i~wulgarny, każdy używa swojej mowy. W~samym środku
światła jest ciemność, ale nie zajmuj się aspektem ciemności, W~środku
ciemności jest światło, ale nie usiłuj widzieć aspektu światła. Światło
i~ciemność są w~opozycji do siebie, podobnie jak przednia i~tylna noga
w~marszu. Miriad rzeczy ma swoje własne zasługi, wyrażone zgodnie z~funkcją
i~miejscem. Jeśli rzeczy istnieją, to pasują do siebie jak pokrywa do pudła,
jeśli są zgodne z~zasadą, są jak spotykające się groty strzał. Słysząc słowa
powinieneś uchwycić esencję. Nie ustanawiaj sam reguł. Jeśli nie rozumiesz
Drogi, kiedy ją widzisz, jak możesz znać ścieżkę, gdy stajesz na niej
nogami? Jeśli czynisz postęp w~marszu, nie jest to sprawą bliska czy daleka.
Jeśli tkwisz w~ułudzie, zostajesz oddzielony przez góry i~rzeki. Dlatego
z~pokorą zwracam się do praktykujących, to co tajemne: nie spędzajcie
nadaremnie dni i~nocy.
\end{Prayer}

%%%%%%%%%%%%%%%%%%%%%%%%%%%%%%%%%%%%%%%%%%%%%%%%%%%%%%%%%%%%%%%%%%%%%%%%%%%%%%%
%%%% H\=OKY\=O ZANMAI
\begin{Prayer}{hokyozanmai}
	{H\=OKY\=O ZANMAI}{-}
	{Klejnotowe Zwierciadło Samadhi}

\begin{center}
	\scriptsize
	(język japoński)
\end{center}

% (!!!) ten tekst zawiera myślniki służące do przeniesienia np. DO ZUREBA
% myślnik jest tylko jeden! /w.

\begin{JAPANESE}
NYOZE NO H\=O, BUSSO MITSU NI FUSU, NANJI IMA KORE O~ETARI, YOROSHIKU YOKU
H\=OGO SUBESHI, \keisu GINWAN NI YUKI O~MORI, MEIGETSU NI RO O~KAKUSU, RUI
SHITE HITOSHI KARAZU, KONZURU TOKINBA TOKORO O~SHIRU, KOKORO KOTO NI ARA
ZAREBA RAIKI MATA OMOMUKU, D\=O-ZUREBA KAKY\=U O~NASHI, TAGAEBA KOCHO NI
OTSU, HAISOKU TOMO NI HI NARI, TAIKAJU NO GOTOSHI, TADA MONSAI NI
ARAWASEBA, SUNAWACHI ZENNA NI ZOKUSU, YAHAN SH\=OMEI, TENGY\=O FURO, MONO
NO TAME NI NORI TO NARU, MOCHIITE SHOKU O~NUKU, UI NI ARAZU TO IEDOMO, KORE
GO NAKI NI ARAZU, H\=OKY\=O NI NOZONDE, GY\=OY\=O AI MIRU GA GOTOSHI, NANJI
KORE KARE NI ARAZU, KARE MASANI KORE NANJI, YO NO Y\=ONI NO GOS\=O GANGU
SURU GA GOTOSHI, FUKO FURAI, FUKI FUJU, BABA WAWA, UKU MUKU, TSUINI MONO
O~EZU, GO IMADA TADASHI KARA ZARU GA YUENI, J\=URI RIKK\=O, HENSH\=O EGO,
TATANDE SAN TO NARI, HENJI TSUKITE GO TO NARU, CHI S\=O NO AJIWAI NO
GOTOKU, KONG\=O NO CHO NO GOTOSHI, SH\=OCH\=U MY\=OKY\=O, K\=OSH\=O NARABI
AGU, SH\=U NI TS\=UJI TO NI TS\=UZU, KY\=OTAI KY\=ORO, SHAKUNEN
NARU TO KINBA KITSU NARI, BONGO SUBEKARAZU, TENSHIN NI SHITE MY\=O NARI,
MEIGO NI ZOKU SEZU, INNEN JISETSU, JAKUNEN TO SHITE SH\=OCHO SU, SAI NIWA,
MUKEN NI IRI, DAI NIWA H\=OJO O~ZESSU, G\=OKOTSU NO TAGAI, RIRRYO
NI \=OZEZU, IMA TONZEN ARI, SH\=USHU O~RISSURU NI YOTTE, SH\=USHU
WAKARU, SUNAWACHI KORE KIKU NARI, SHU TS\=UJI SHU KIWAMARU MO, SHINJ\=O
RUCH\=U, HOKA JAKU NI UCHI\=UGOKU WA, TSUNAGERU KOMA, FUKUSERU NEZUMI,
SENSH\=O KORE O~KANASHINDE, H\=O NO DANDO TO NARU, SONO TEND\=O
NISHITAGATTE, SHI O~MOTTE SO TO NASU, TEND\=O S\=OMETSU SUREBA, K\=OSHIN
MIZUKARA YURUSU, KOTETSU NI KANAWAN TO Y\=OSEBA, K\=O ZENKO O~KANZEYO,
BUTSUD\=O O~JOZURU NI NANNAN TO SHITE, JIKK\=OJU O~KANZU \keisu TORA NO
KAKETARU GA GOTOKU, UMA NO YOME NO GOTOSHI, GERETSU ARU O~MOTTE, H\=OKI
CHINGYO, KY\=OI ARU O~MOTTE, RINU BYAKKO, \keisu GEI WA GY\=ORIKI O~MOTTE,
ITE HYAPPO NI ATSU, SENP\=O AI AU, GY\=ORIKI NANZO AZUKARAN, BOKUJIN MASANI
UTAI, SEKIJO TATTE MAU, J\=OSHIKI NO ITARU NI ARAZU, MUSHIRO SHIRYO O~IREN
YA, SHIN WA KIMI NI BUSHI, KO WA CHICHI NI JUNZU JUNZE ZAREBA K\=O NI
ARAZU, BUSE ZAREBA HO NI ARAZU. SENK\=O MITSUY\=O WA, GU NO GOTOKU RO NO
GOTOSHI, \shokei TADA YOKU S\=OZOKU SURU O, \shokei SHUCH\=U NO SHU TO
NAZUKU.
\end{JAPANESE}
\end{Prayer}

%%%%%%%%%%%%%%%%%%%%%%%%%%%%%%%%%%%%%%%%%%%%%%%%%%%%%%%%%%%%%%%%%%%%%%%%%%%%%%%
%%%% DROGOCENNE ZWIERCIADŁO SAMADHI
\begin{Prayer}{drogocenne_zwierciadlo_samadhi}
	{KLEJNOTOWE ZWIERCIADŁO SAMADHI}{-}
	{H\=oky\=ozanmai}

\begin{Verse}
	Ta Dharma jest tajemnie związana z~Buddhami i~Patriarchami, teraz ty ją osiągnąłeś i~powinieneś ją dobrze chronić.\\
	Wypełniona śniegiem srebrna czara, czapla ukryta w~świetle księżyca.\\
	Choć podobne, nie są takie same, gdy je zmieszać wiadomo, które jest gdzie.\\
	Znaczenie nie tkwi w~słowach, ale gdy przychodzi stosowna chwila pojawia się.\\
	Jeśli pojawi się ruch stanie się pułapką, jeśli różnicujesz zaczniesz rozglądać się i~zatrzymasz się.\\
	Odrzucenie i~lgnięcie są błędem.\\
	Jak wielki płonący ogień, gdy tylko to nabierze kształtu pięknej wypowiedzi zostanie zabarwione i~zabrudzone.\\
	O północy jest to prawdziwie świetliste, o~świcie jest niewidoczne.\\
	Jest podstawą dla wszystkich rzeczy, używane wykorzenia całe cierpienie.\\
	Choć jest nie-działaniem, nie jest pozbawione słów.\\
	Tak jak w~klejnotowym zwierciadle~-- kształt i~jego odbicie postrzegają się nawzajem.\\
	Ty nie jesteś tym, ale to jest prawdziwie tobą.\\
	Podobnie jak niemowlę, które ma pełne pięć organów zmysłów.\\
	Bez przychodzenia i~odchodzenia, bez powstawania i~pozostawania.\\
	Ba, ba, ua, ua~-- zdanie bez zdania,\\
	Ostatecznie nic nie zostało powiedziane, ponieważ język jest jeszcze niewłaściwy.\\
	Podwójny trygram ognia tworzy sześć linii, wewnętrzne i~zewnętrzne linie oddziałują na siebie nawzajem.\\
	Złożone tworzą trzy pary.\\
	Całkowicie przekształcone stają się pięcioma.\\
	Jak smak aromatycznej rośliny, lub ramiona wadżry, gdzie środkowe jest wspaniale podtrzymane pomiędzy innymi.\\
	Jednoczesne bębnienie i~śpiew.\\
	Przeniknięcie esencji jest przeniknięciem drogi.\\
	Obejmij sfery i~bądź w~zgodzie z~drogą.\\
	Współdziałanie jest pomyślne, nie powinieneś się temu sprzeciwiać.\\
	Nieskażone staje się wspaniałe.\\
	Nie przynależy do złudzenia i~oświecenia.\\
	Zgodnie z~przyczyną i~związkiem, czasem i~momentem.\\
	Kiedy jest zupełnie spokojne pojawia się świetlistość.\\
	Kiedy jest małe wkracza w~nie-wymiar.\\
	Kiedy jest wielkie jest niezmierzone.\\
	Oddzielenie na szerokość włosa sprawia\\
	niemożność właściwego zharmonizowania.\\
	Teraz mamy nagłe i~stopniowe, by ustalić podstawową zasadę, gdy staje się ona jasna i~zrozumiała staje się regułą.\\
	Nawet, jeśli zrozumiesz podstawową zasadę, to co prawdziwe i~wieczne, wciąż płynie.\\
	Na zewnątrz spokój, wewnątrz ruch.\\
	Jak spętany koń, czy ukryty szczur.\\
	Mędrcom w~przeszłości było ich żal i~wyzwalali je udzielając nauk Dharmy.\\
	Z powodu swoich złudzeń nazywali czarne białym, ale kiedy błędne wyobrażenia ustają, umysł sam się urzeczywistni.\\
	Jeśli chcesz być w~zgodzie ze starożytną drogą, obserwuj przykład pozostawiony przez starożytnych.\\
	Blisko urzeczywistnienia Drogi Buddhy, jeden medytował przez dziesięć kalp pod drzewem.\\
	Jak kulawy tygrys lub nie podkuty koń.\\
	Gdy jesteś prymitywny szukasz wysadzanych klejnotami tronów i~cennych szat.\\
	Kiedy jesteś tym zaskoczony widzisz, że jesteś jak przebiegły klown czy biała krowa.\\
	Hou-i ze swoimi zdolnościami łucznika potrafił trafić w~cel na odległość stu metrów, ale gdy grot strzały osiągnie cel, jaki pożytek jest z~jego zdolności?\\
	Kiedy kamienny mężczyzna śpiewa, drewniana kobieta wstaje do tańca.\\
	Tego nie można zrozumieć uczuciami i~świadomością, jak więc można w~ogóle o~tym myśleć?!\\
	Minister służy władcy, a~syn podąża za ojcem.\\
	Jeśli jest nieposłuszny nie ma synowskich uczuć.\\
	Jeśli minister nie służy, to nie jest żadną pomocą.\\
	Ukryta praktyka i~tajemne działanie, jak głupek, jak idiota.\\
	Tylko wytrwale kontynuuj, a~będziesz nazywany mistrzem pośród mistrzów.\\
\end{Verse}
\end{Prayer}

\newpage
%%%%%%%%%%%%%%%%%%%%%%%%%%%%%%%%%%%%%%%%%%%%%%%%%%%%%%%%%%%%%%%%%%%%%%%%%%%%%%%
%%%% KANROMON

\newcommand{\kanromonsection}[2]{%
	\begingroup
	\leftskip  0pt
	\parindent 0pt
	\bigskip
	\noindent $\cdots$ \textit{#1} $\cdots$\par
	\vspace{5pt}
	\noindent {\scriptsize #2}\par
	\bigskip
	\endgroup
	\nopagebreak
}

\begin{Prayer}{kanromon}
	{KANROMON}{-}
	{Brama Słodkiego Nektaru Amrity Wejścia w~Nirwanę}

\begin{center}
\textbf{Uwaga}: tekst zaznaczony $\cdots$ \emph{sześcioma kropkami} $\cdots$ oznacza tytuły sekcji i~nie powinien być recytowany
\end{center}

\begin{Verse}
\samepage{
\kanromonsection{Bush\=o sanb\=o}
	{Zapraszanie Trzech Klejnotów\\(język chiński)}

	\wersaliki
	NAMU JIP-P\=O BUTSU\\
	NAMU JIP-P\=O HO\\
	NAMU JIP-P\=O S\=O\\
	NAMU HON SHI SHAKYAMUNI BUTSU\\
	NAMU DAI ZU DAI HI KY\=U KU KANZEON BOS\=A\\
	NAMU KEI KY\=O ANAN SON JA.
}

{
\kanromonsection{Ch\=osh\=o hotsugan}{(język japoński)}

	ZE SHO SHU T\=O {\scriptsize (tylko prowadzący)}

\bigskip{
	\wersaliki
	HOSSHIN SHITE IKKI NO J\=O JIKI O~BUJI SHITE,\\
	AMANEKU JIPP\=O,\\
	G\=U JIN KOK\=U,\\
	SH\=U HEN HOKKAI,\\
	MI JIN SETCH\=U,\\
	SH\=O KOKUDO NO ISSAI NO GAKI NI HODOKOSU,\\
	SEN M\=O KU ON,\\
	SAN SEN CHISHU,\\
	NAI SHIK\=O YA NO SHOKI JIN T\=O,\\
	K\=O KITATTE KOKO NI ATSU MARE,\\
	WARE IMA HI MIN SHITE,\\
	AMANEKU NANJI MI JIKI O~HODO KOSU.\\
	NEGAWAKUWA NANJI KAKKAKU,\\
	WAGA KONO JIKI O~UKETE,\\
	TENJI MOTTE JIN KO K\=U KAI NO SHO BUTSU GY\=U SH\=O,\\
	ISSAI NO UJ\=O NI KU Y\=O SHITE,\\
	NANJI TO UJ\=O TO,\\
	AMANEKU MINA B\=O MAN SEN KOTO O,\\
	MATA NEGAWAKUWA NANJI GAMI,\\
	KONO SHU JIKI NI J\=O JITE,\\
	KUO HANA RETE GE DASSHI,\\
	TEN NI SH\=O JITE RAKU O~UKE,\\
	JIPP\=O NO J\=O DOMO\\
	KOKORO NI SHITA GATTE YU \=OSHI,\\
	BODAI SHIN O~HASSHI,\\
	BODAI D\=O O~GY\=OJI,\\
	T\=O RAI NI SA BUSSHITE,\\
	NAGAKU TAI TEN NAKU,\\
	SAKI NI D\=O O~URU MONO WA,\\
	CHI KATTE AI DO DASSEN KOTO O,\\
	MATA NEGAWAKUWA NANJI RA,\\
	CH\=U YAG\=O J\=O NI,\\
	WARE O~Y\=O GOSHITE,\\
	WAGA SHO GAN O~MAN ZEN KOTO O.
}

\stanza{
	\wersaliki
	NEGAWAKUWA KONO JIKI O~HODO KOSU,\\
	SHO SH\=O NO KUDOKU,\\
	AMANEKU MOTTE HOKKAI NO UJ\=O NI ESE SHITE,\\
	MORO MORO NO UJ\=O TO,\\
	BY\=ODO KU NARAN,\\
	MORO MORO NO UJ\=O TO TOMO NI,\\
	ONAJI KU KONO FUKU O~MOTTE,\\
	KOTO GOTO KU MOTTE SHIN NYO HOKKAI,\\
	MUJ\=O BODAI,\\
	ISSAI CHI CHI NI EK\=O SHITE,\\
	NEGAWAKUWA SUMI YAKA NI J\=O BUSSHITE,\\
	YOKA O~AMANEKU KOTO NAKA RAN.
}

\stanza{
	\wersaliki
	(HOKKAI NO GAN JIKI)\\
	NEGAWAKUWA KONO H\=O NI J\=OJITE,\\
	TOKU J\=O BUSSURU KOTO O~EN.
}
}

\samepage{
\kanromonsection{Unsh\=u kijin ch\=osh\=o darani}{(język chiński)}
% !!! tłumaczenia wstawić tytułów z~Soto Shu Scriptures

	\wersaliki
	N\=O B\=O\\
	BO HO RI\\
	GYA RI TA RI\\
	TA T\=A GYA TA YA.
}

\samepage{
\kanromonsection{Ha jigokumon kai ink\=o darani}{(język chiński)}

	\wersaliki
	ON BO HO TEI RI\\
	GYA TA RI \\
	TA T\=A GYA TA YA.
}

\samepage{
\kanromonsection{Mury\=o itoku jizai k\=omy\=o kaji onjiki darani}{(język chiński)}

	\wersaliki
	NO MAKU \\
	SARA BA \\
	TA T\=A GYA TA \\
	BARO KI TEI\\
	ON\\
	SAN BA R\=A \\
	SAN BA R\=A UN.
}

\samepage{
\kanromonsection{M\=o kanro h\=omi darani}{(język chiński)}

	\wersaliki
	N\=O MAKU\\
	SORO BAYA\\
	TA T\=A GYA TAYA\\
	TA NYA TA \\
	ON\\
	SORO SORO\\
	HARA SORO\\
	HARA SORO\\
	SO WA KA.
}

\samepage{
\kanromonsection{Birushana ichiji shin suirin kan darani}{(język chiński)}

	\wersaliki
	N\=O MAKU\\
	SAN MAN DA \\
	BOTA NAN BAN.
}

\newpage
\kanromonsection{Go nyorai h\=og\=o ch\=osh\=o darani}{(język chiński)}

\stanza{
	\wersaliki
	NAMU TA H\=O NYORAI.\\
	N\=O BO\\
	BA GYA BA TEI \\
	HARA BOTA\\
	ARA TAN N\=O YA\\
	TA T\=A GYA TA YA.\\
	JO KEN TON G\=O FUKU CHI EN MAN.
}

\stanza{
	\wersaliki
	NAMU MY\=O SHIKI SHIN NYORAI.\\
	N\=O BO\\
	BA GYA BA TEI\\
	SORO BAYA \\
	TA T\=A GYA TA YA.\\
	HA SH\=U RO GY\=O EN MAN S\=O K\=O
}

\stanza{
	\wersaliki
	NAMU KAN RO \=O NYORAI.\\
	N\=O BO\\
	BA GYA BA TEI\\
	AMIRI TEI\\
	A RAN JA YA\\
	TA T\=A GYA TA YA.\\
	KAN B\=O SHIN JIN RY\=O JUKE RAKU.
}

\stanza{
	\wersaliki
	NAMU K\=O HAKU SHIN NYORAI.\\
	N\=O BO\\
	BA GYA BA TEI\\
	BI HO RA GYA\\
	TARA YA\\
	TA T\=A GYA TA YA.\\
	IN K\=O K\=O DAI ON JIKI J\=U B\=O.
}

\stanza{
	\wersaliki
	NAMU RI FU I~NYORAI.\\
	N\=O BO \\
	BA GYA BA TEI\\
	ABA EN \\
	GYA RA YA \\
	TA T\=A GYA TA YA \\
	KU FU SHITSU JORI GAKI SHU.
}

\samepage{
\kanromonsection{Hotsu bodaishin darani}{(język chiński)}

	\wersaliki
	ON\\
	B\=O JI SHITTA\\
	BO DA \\
	HA DA YA MI.
}

\samepage{
\kanromonsection{Ju bosatsu sanmayakai darani}{(język chiński)}

	\wersaliki
	ON\\
	SAN MAYA \\
	SA TO BAN.
}

{
\kanromonsection{Daih\=o r\=okaku zenj\=u himitsu darani}{(język chiński)}

	\wersaliki
	N\=O MAKU\\
	SARA BA TA T\=A GYA TA NAN\\
	ON\\
	BI HO RA \\
	GYA RA BEI\\
	MANI HARA BEI\\
	TA TA TA NI TA SHA NI\\
	MANI MANI\\
	SO HA RA BEI\\
	BI MA REI SHA GYA RA\\
	GEN BI REI \\
	UN NUN JIN BARA JIN BARA\\
	BODA \\
	BIRO KI TEI\\
	KU GYA \\
	CHI SHUT-TA\\
	GYA RA BEI\\
	SO WA KA\\
	ON MANI \\
	BAJI REI UN\\
	ON MANI DA REI\\
	UN BATTA.
}

\samepage{
\kanromonsection{Shobutsu k\=omy\=o shingon kanch\=o darani}{(język chiński)}

	\wersaliki
	ON\\
	A BO GYA\\
	BEI ROSHA N\=O\\
	MAKA BODA RA\\
	MANI HAN DOMA\\
	JIN BARA HARA BA RI\\
	TA YA UN.
}

\samepage{
\kanromonsection{Hakken gedatsu darani}{}

	\wersaliki
	ON\\
	BA SARA\\
	BO KISHA B\=O K\=U.
}

\samepage{
\kanromonsection{Ek\=o ge}{(język chiński)}

	\wersaliki
	I SU SHU AN SHU SEN GEN\\
	HO TO BU MO KI RO TE\\
	SON SHA FU RA JU BU KYU\\
	MO SHA RI KU SAN NAN YO\\
	SU IN SAN YU SHI AN SHI\\
	SAN ZU HA NAN KU SHUN SAN\\
	KYU MO KUI KO SEN NAN SU\\
	JIN SHU RIN NUI SAN JIN ZU
}
\end{Verse}
\end{Prayer}

%%%%%%%%%%%%%%%%%%%%%%%%%%%%%%%%%%%%%%%%%%%%%%%%%%%%%%%%%%%%%%%%%%%%%%%%%%%%%%%
%%%% BRAMA SŁODKIEGO NEKTARU AMRITY WEJŚCIA W~NIRWANĘ
\begin{Prayer}{brama_slodkiego_nektaru_amrity_wejscia_w_nirwane}
	{BRAMA SŁODKIEGO NEKTARU AMRITY\\WEJŚCIA W~NIRWANĘ}{BRAMA SŁODKIEGO NEKTARU AMRITY WEJŚCIA W~NIRWANĘ}
	{Kanromon}

\bigskip
\begin{Verse}
	\keisu (przy 3 powtórzeniu)\\
	CHWAŁA BUDDZIE W DZIESIĘCIU KIERUNKACH\\
	CHWAŁA DHARMIE W DZIESIĘCIU KIERUNKACH\\
	CHWAŁA SANDZE W DZIESIĘCIU KIERUNKACH\\
	\keisu (przy 3 powtórzeniu)\\
	CHWAŁA PIERWOTNEMU MISTRZOWI SIAKIAMUNIEMU BUDDZIE\\
	CHWAŁA WIELCE WSPÓŁCZUJĄCEMU\\
	WIELCE MIŁUJĄCEMU\\
	AWALOKITEŚWARZE BODHISATTWIE\\
	\keisuzgaszenie (przy 3 powtórzeniu)\\
	CHWAŁA PRZEKAZUJĄCEMU NAUKĘ - ARAHATOWI ANANDZIE\\
	(całość powtórzyć 3 razy)\\
\end{Verse}

\newpage
\begin{flushleft}
\bf\scriptsize	ZAPROSZENIE I ŚLUBOWANIE BODHICITTY
\end{flushleft}

\begin{Verse}
	O WSZYSTKIE ZGROMADZENIA!\\
	EMANUJĄC BODHICITTĘ\\
	OFIAROWUJEMY NACZYNIE CZYSTEGO POKARMU WSZYSTKIM GŁODNYM DUCHOM\\
	WE WSZYSTKICH KRAINACH I MIEJSCACH,\\
	W DZIESIĘCIU KIERUNKACH NIEZMIERZONEJ PRZESTRZENI I CZASU, I WE WSZYSTKICH CZĄSTECZKACH DHARMADHATU.\\
	DUCHY TYCH, KTÓRZY ZMARLI PRZED NAMI W NIESKOŃCZONEJ PRZESZŁOŚCI ORAZ WŁADCY ZIEMI, RZEK, GÓR ORAZ BOGOWIE - DEMONY DZIKICH USTRONI I WIELKICH PRZESTRZENI, WSZYSCY PRZYBĄDŹCIE I ZBIERZCIE SIĘ TUTAJ, TERAZ ZE WSPÓŁCZUCIA OFIAROWUJEMY WAM WSZYSTKIM JEDZENIE.\\
	NASZYM ŻYCZENIEM JEST, ABYŚCIE DZIĘKI TEJ MANTRZE PRZEMIENIENIA POŻYWIENIA UWOLNILI SIĘ OD CIERPIENIA I OSIĄGNĘLI WYZWOLENIE, ODRODZILI SIĘ W NIEBIE, DOZNALI ULGI I ZGODNIE Z WŁASNYM ŻYCZENIEM MOGLI ODWIEDZAĆ CZYSTE KRAINY BUDDHÓW W DZIESIĘCIU KIERUNKACH. OBYŚCIE ROZWINĘLI BODHICITTĘ - UMYSŁ I PRAGNIENIE OŚWIECENIA, PRAKTYKOWALI DROGĘ OŚWIECENIA I STALI SIĘ W PRZYSZŁOŚCI BUDDHAMI. CI, KTÓRZY WCZEŚNIEJ OSIĄGNĘLI DROGĘ, NIECH Z NIEUSTAJĄCĄ WYTRWAŁOŚCIĄ WYZWALAJĄ INNE ISTOTY DZIĘKI SILE SWOICH ŚLUBOWAŃ. NASZĄ MODLITWĄ I POSTANOWIENIEM JEST, ABYŚCIE CHRONILI NAS NIEUSTANNIE W DZIEŃ I W NOCY, DZIĘKI CZEMU BĘDZIEMY MOGLI WYPEŁNIĆ NASZE ŚLUBOWANIA. MODLIMY SIĘ I ŚLUBUJEMY, ŻE WSZYSTKIE ISTOTY NA CAŁYM ŚWIECIE DHARMADHATU ZOSTANĄ OBDAROWANE ZASŁUGAMI ZRODZONYMI Z OFIAROWANIA TEGO JEDZENIA, TAK BY WSZYSCY DZIELILI ZE SOBĄ DUCHOWY SPOKÓJ I BYŚMY WRAZ ZE WSZYSTKIMI ISTOTAMI OSIĄGNĘLI SZCZĘŚCIE.\\
	WSZYSTKIE ZASŁUGI Z TEGO PŁYNĄCE ZWRACAMY DHARMADHATU PRAWDZIWEJ RZECZYWISTOŚCI, NIEPRZEŚCIGNIONEMU OŚWIECENIU I ABSOLUTNEJ MĄDROŚCI.\\
	OBYŚCIE JAK NAJSZYBCIEJ OSIĄGNĘLI STAN BUDDHY I NIGDY WIĘCEJ NIE POWODOWALI POWSTAWANIA POJĘCIA JAŹNI I INNYCH. OBY WSZYSTKIE ISTOTY DHARMDHATU, UŻYWAJĄC TEJ DHARMY, SZYBKO OSIĄGNĘŁY STAN BUDDHY.\\
\end{Verse}

\begin{flushleft}
\bf\scriptsize	UN SHU KIJIN CHOSHO DHARANI
\end{flushleft}
\begin{Verse}
	\keisu NO BOO BOHO RI GYARI TARI \keisuzgaszenie (przy 3 powtórzeniu) TATAA GYATA YA\\
\end{Verse}

\begin{flushleft}
\bf\scriptsize	HA JIGOKU MONKAI IN KOO DHARANI
\end{flushleft}
\begin{Verse}
	\keisu ON BOHO TEI RI GYATARI \keisuzgaszenie (przy 3 powtórzeniu) TATAA GYATA YA\\
	MURYOO ITOKU JIZAI KOOMYOO\\
\end{Verse}

\begin{flushleft}
\bf\scriptsize	KAJI ONJIKI DHARANI
\end{flushleft}	
\begin{Verse}
	\keisu NOO MAKU SARA BA TATAA GYATA BARO KITEI ON \keisuzgaszenie (przy 3 powtórzeniu) SAN BARAA SAN BARAA UN\\
\end{Verse}

\begin{flushleft}
\bf\scriptsize	MOO KANRO HOO MI DHARANI\\
\end{flushleft}	
\begin{Verse}
	\keisu NOO MAKU SORO BAYA TATAA GYATA YA TANYATA ON SORO SORO HARA SORO \keisuzgaszenie (przy 3 powtórzeniu) HARA SORO SOWA KA\\
\end{Verse}

\begin{flushleft}
\bf\scriptsize	BIRUSHANA ICHIJI SHIN SUI RINKAN DHARANI\\
\end{flushleft}	
\begin{Verse}
	\keisu\keisu NOO MAKU SAN MAN DA \keisuzgaszenie (przy 3 powtórzeniu) BOTA NAN BAN\\
\end{Verse}

\begin{flushleft}
\bf\scriptsize	GO NYORAI HOO GOO CHOO SHOO DHARANI\\
\end{flushleft}	
\begin{Verse}
	\keisu\keisu NAMU TAHOO NYORAI NOO BO BAGYABATEI HARA BOTA ARA TAN NOO YA TATAA GYATA YA JOKEN TON GOO FUKU CHI EN MAN\\
	\keisu NAMU MYOO SHIKI SHIN NYORAI NOO BO BAGYA BA TEI SORO BAYA TATAA GYATA YA HASHUU ROGYOO EN MAN SOO KOO\\
	\keisu\keisu NAMU KAN RO OO NYORAI NOO BO BAGYA BA TEI AMIRI TEI ARAN JA YA TATAA GYATA YA KAN POO SHIN JIN RYOO JUKE RAKU\\
	\keisu NAMU KOO HAKU SHIN NYORAI NOO BO BAGYA BA TEI BI HORA GYA TARA YA TATAA GYATA YA IN KOO KOO DAI ON JIKI JUU BOO\\
	\keisu NAMU RI FUI NYORAI NOO BO BAGYA BA TEI ABA EN GYARA YA TATAA GYATA YA KUFU SHITSU JORI GAKI SHU\\
\end{Verse}

\begin{flushleft}
\bf\scriptsize	HOTSU BODAI SHIN DHARANI\\
\end{flushleft}	
\begin{Verse}
	\keisuzgaszenie ON BOO JI SHITTA BODA HADA YA MI\\
	$[$Doshi powtarza sam mantrę i po nim Daishu całość trzy razy$]$\\
\end{Verse}

\begin{flushleft}
\bf\scriptsize	JU BOSA SAN MAYA KAI DHARANI\\
\end{flushleft}	
\begin{Verse}
	\keisuzgaszenie ON SAN MAYA SATO BAN\\
	$[$Doshi powtarza sam mantrę i po nim Daishu całość trzy razy$]$\\
	DAI HOO ROO KAKU ZEN JUU\\
\end{Verse}

\begin{flushleft}
\bf\scriptsize	HIMITSU KON PON DHARANI\\
\end{flushleft}	
\begin{Verse}
	\keisu NOO MAKU SARA BA TATAA GYA TA NAN ON BIHO RA GYARA BEI MANI HARA BEI TA TA TANI TA SHA NI MANI MANI SO HARA BEI BI MA REI SHAGYA RA GEN BI REI UN NUN JIN BARA JIN BARA BODA BIRO KI TEI KUGYA CHI SHUT TA GYARA BEI SOWA KA ONMANI BAJI REI UN ON MANI DA REI UN BAT TA\\
	SHO BUTSU KOO MYOO SHIN GON\\
\end{Verse}

\begin{flushleft}
\bf\scriptsize	KAN CHOO DHARANI\\
\end{flushleft}	
\begin{Verse}
	\keisu ON ABOGYA BEI ROSHA NOO MAKA BODARA\\
	MANI HAN DOMA JIN BARA HARA BARI TAYA UN!\\
\end{Verse}
\end{Prayer}

\newpage
%%%%%%%%%%%%%%%%%%%%%%%%%%%%%%%%%%%%%%%%%%%%%%%%%%%%%%%%%%%%%%%%%%%%%%%%%%%%%%%
%%%% DAI SEGAKI
\begin{Prayer}{dai_segaki}
	{DAI SEGAKI}{-}
	{Jya jin nyu ry\=o shi}

\begin{center}
\scriptsize Inny tytuł: \textit{Kai Kanromon}
\end{center}

\bigskip
\begin{JAPANESE}
\textbf{JA JIN NY\=U RY\=O SHI.}\\
SAN SH\=I I SHII F\=U.\\
IN KAN HA KAI SHIN.\\
I SHII YUI SHIN Z\=O.\\
NA MU JI H\=O F\=U.\\
NA MU JI H\=O H\=A.\\
NA MU JI H\=O SEN.\\
NA MU HON SU SHI KYA MU NI F\=U.\\
NA MU DAI ZU DAI HI KY\=U KU KAN SHI IN BU S\=A\\
NA MU K\=I K\=O O NAN SON SH\=A.\\
\ \\
NA MU SA B\=O.\\
TO TO GYA T\=O.\\
PO RYO KI CHI. EN. \\
SAN MO R\=A. SAN MO R\=A. KIN. \\
\ \\
NA MU SU RYO BO Y\=A. \\
TO TO GYA TO Y\=A. \\
TO JI T\=O. EN. \\
SU RY\=O SU RY\=O. \\
BO YA SU RY\=O. \\
BO YA SU RY\=O.\\
SO MO KO. \\
\ \\
NA MU SA MAN D\=A. \\
HO TO NAN PAN. \\
\ \\
NA MU H\=O SHIN JI RAI. \\
NA MU TO H\=O JI RAI. \\
NA MU MY\=O SHI SHIN JI RAI. \\
NA MU K\=O HA SHIN JI RAI. \\
NA MU RI FU I JI RAI. \\
NA MU KAN RO Y\=O JI RAI. \\
NA MU O MI TO JI RAI. \\
\ \\
NA MU O MI TO PO Y\=A TO TO GYA TO Y\=A.\\
TO NI YA T\=O. \\
O MI R\=I TSU BO MII. \\
O MI RI T\=O. \\
SHI TA BO M\=I. \\
O MI RI T\=O. \\
BI GYA RA CHII. \\
O MI RI T\=O. \\
BI GYA RA T\=O. \\
GYA MI N\=I. GYA GYA N\=O. \\
SHI TO GYA RI. SO MO KO. \\
\ \\
JIN SH\=U KY\=A J\=I JIN NIN SH\=I. \\
F\=U SH\=I \=O S\=A SH\=U K\=I SHIN.\\
GEN KAI B\=O MON SH\=A KEN SH\=IN. \\
SH\=I D\=O Y\=U M\=I SAN ZEN D\=O. \\
K\=I \=I SAN P\=O H\=A B\=U J\=I. \\
KY\=U KIN T\=E SHIN B\=U J\=O K\=A. \\
KUN T\=E B\=U HEN JIN MI R\=AI.\\
\=I SH\=I SHUN SAN ZUN B\=A SH\=I. \\
\ \\
JI TEN KI JIN SH\=U.\\
GO KIN SU JI KY\=U. \\
SU JI HEN. JI H\=O I SHII KI JIN KY\=U. \\
\ \\
I S\=U SH\=U AN SH\=U SEN G\=EN. \\
H\=O T\=A B\=U M\=O K\=I R\=O T\=E. \\
SON SH\=A F\=U R\=A J\=U B\=U KY\=U. \\
M\=O SH\=A R\=I K\=U SAN NAN NY\=O. \\
S\=U IN SAN NY\=U SH\=I AN SH\=I. \\
SAN Z\=U H\=A NAN K\=U SHUN SAN. \\
KY\=U M\=O KUI K\=O SEN NAN S\=U. \\
JIN SH\=U RIN NUI. SAN JIN Z\=U. \\
\ \\
GEN NI SU KUN T\=E. \\
FU GY\=U \=O I SHII. \\
GO TEN I SHUN SAN. \\
KAI KY\=U JIN BU D\=O. \\

\noindent%
J\=I H\=O SAN SH\=I I SHII SHI BU. \\
SHI SON BU S\=A. MO K\=O S\=A. \\
MO K\=O H\=O J\=A H\=O R\=O MI. \\
\end{JAPANESE}
\end{Prayer}

%%%%%%%%%%%%%%%%%%%%%%%%%%%%%%%%%%%%%%%%%%%%%%%%%%%%%%%%%%%%%%%%%%%%%%%%%%%%%%%
%%%% OFIAROWANIE GŁODNYM DUCHOM
\begin{Prayer}{ofiarowanie_glodnym_duchom}
	{OFIAROWANIE GŁODNYM DUCHOM}{-}
	{-}

\bigskip
\begin{Verse}
	Jeśli chcecie poznać wszystkich Buddhów przeszłości, teraźniejszości i~przyszłości, powinniście kontemplować naturę Dharmadhatu jako twór tylko Umysłu.\\
	Chwała Buddhom w~dziesięciu kierunkach.\\
	Chwała Dharmie w~dziesięciu kierunkach.\\
	Chwała Sandze w~dziesięciu kierunkach.\\
	Chwała Siakiamuniemu Buddzie, Prawdziwemu Mistrzowi.\\
	Chwała Awalokiteśwarze Bodhisattwie, Wielce Współczującemu, Wielce Miłującemu, Wyzwalającemu z~Cierpienia.\\
	Chwała Anandzie, Arhatowi Przekazującemu Naukę.\\
	Namah sarva-tathagatavalokite! Om!\\
	Sambala, sambala! Hum!\\
	Namah surupaja tathagataja!\\
	Tadjatha,\\
	Om suru suru paja suru paja suru svaha!\\
	Namah samantabuddhanam, vam!\\
	Namah Ratnaketu Tathagata\\
	Namah Prabhutaratna Tathagata\\
	Namah Surupakaja Tathagata\\
	Namah Vipulakaja Tathagata\\
	Namah Abhajankara Tathagata\\
	Namah Amritaradża Tathagata\\
	Namah Amitabha Tathagata\\
	Namo Amitabhaja Tathagataja! Tadjatha, amritodbhave, amritasiddhe, amritabhave, amritavikrante, amrita-vikranta-gamine, gaganakirtikare!\\
	Svaha!\\
\end{Verse}

\begin{center}
	(mokugyo)
\end{center}


Dzięki nadprzyrodzonej mocy tej dharani jedzenie i~picie zostały oczyszczone
i~ofiarowane niezliczonym duchom. Modlimy się, aby ich pragnienie i~głód
zostały w~pełni ugaszone i~by porzuciły swoje pożądania, aby porzuciły ciemne
miejsca egzystencji i~odrodziły się w~dobrych krainach. Następnie niech przyjmą
schronienie w~Trzech Klejnotach i~obudzą pragnienie nieprześcignionego
Oświecenia i~w pełni je urzeczywistnią. W~ten sposób osiągnięta zasługa jest
niewyczerpalna i~trwa wiecznie, pozwalając wszystkim istotom mieć taki sam
udział w~pokarmie Dharmy.


O wy, wszystkie duchy, teraz ofiarowujemy wam to jedzenie, które wypełnia
dziesięć kierunków świata, niech wszyscy z~was przyjmą je w~całości.


Praktykując korzeń dobrej zasługi, modlimy się, by odpłacić to, co jesteśmy
winni naszym rodzicom za otrzymaną dobroć i~wszystko, co dla nas uczynili.
Niech ci, którzy jeszcze żyją, cieszą się szczęśliwym, dostatnim i~długim
życiem. Ci, którzy już zmarli, niech zostaną uwolnieni z~cierpienia i~niech
narodzą się w~spokojnym i~dobrym miejscu.


Niech wszystkie istoty w~trzech światach, które otrzymują cztery
błogosławieństwa, oraz te, które cierpią w~trzech złych krainach i~z powodu
ośmiu skalań, wyrażą skruchę z~powodu swoich wszystkich złych czynów
i~oczyszczą się ze swoich zanieczyszczeń, tak by zostały uwolnione z~koła
odradzania się w~świecie cierpienia Saha i~narodziły się w~Czystej Krainie
Wiecznej Szczęśliwości~-- Sukhawati.


Modlimy się, by cnota zasługi tego ofiarowania objęła wszystkie istoty, abyśmy
wraz ze wszystkimi istotami osiągnęli Drogę Buddhy.


Wszyscy Buddhowie,


Wszyscy Szlachetni Bodhisattwowie Mahasattwowie


W dziesięciu kierunkach, trzech światach


Wielka Doskonała Mądrość.
\end{Prayer}

%%%%%%%%%%%%%%%%%%%%%%%%%%%%%%%%%%%%%%%%%%%%%%%%%%%%%%%%%%%%%%%%%%%%%%%%%%%%%%%
%%%% URODZINY BUDDHY (tekst z zeszytu Daibosatsu Zendo)
\begin{Prayer}{urodziny_buddhy}
	{URODZINY BUDDHY}{-}
	{Wiersz nektaru amrity dla Buddhy-dziecka.}

\bigskip
\begin{Verse}\wersaliki
	GO KIN KAM MO SHI JI RAI\\
	JIN SHI SO NEN KUN TE JU\\
	U SHU SHUN SAN RIN RI KU\\
	ZUN SHIN JI RAI JIM PA SHIN\\
\end{Verse}
	% wstawić polskie tłumaczenie (!!!)
\end{Prayer}

\newcommand{\dharanititle}[2]{%
	\bigskip
	\begingroup
	\leftskip  0pt
	\parindent 0pt
	\noindent \textbf{#1} \textit{#2}\par\nopagebreak
	\endgroup
	\medskip
	\vbox}

\newpage
%%%%%%%%%%%%%%%%%%%%%%%%%%%%%%%%%%%%%%%%%%%%%%%%%%%%%%%%%%%%%%%%%%%%%%%%%%%%%%%
%%%% DHARANI
\begin{Prayer}{dharani}
	{DHARANI}{-}
	{-}

\begin{Verse}
\dharanititle{Shobo Kuju}{(oby prawdziwa Dharma trwała)}{

	ON A~BEI DA BI DEI SOWA KA
}


\dharanititle{Dai Ryu O}{(Strażnicy świątyni)}{

	\wersaliki
	NAN DA RYU O\\
	BATSU NAN DA RYU O\\
	SHA KA RA RYU O\\
	WA SHU KITSU RYU O\\
	TOKU SHAKA RYU O\\
	A NAWA SAT TA RYU O\\
	MA NA SHI RYU O\\
	U HA TSU RA RYU O\\
	ZEN NYO RYU O
}

\dharanititle{Ko Myo Dharani}{(Dharani Jaskrawego Światła)}{

	\wersaliki
	ON ABO KYA BEI ROSHA NO MAKA BO DA RA\\
	MANI HAN DO MA JIM BA RA\\
	HARA BARI TA YA UN
}
\end{Verse}
\end{Prayer}

%%%%%%%%%%%%%%%%%%%%%%%%%%%%%%%%%%%%%%%%%%%%%%%%%%%%%%%%%%%%%%%%%%%%%%%%%%%%%%%
%%%% DHARANI DO 13 BUDDÓW
\begin{Prayer}{dharani_do_buddow}
	{DHARANI DO 13 BUDDHÓW}{-}
	{-}

\bigskip
\begin{Verse}
\stanza{
	Fudo myoo (acala vidya raja)~--\\
	/groźna forma Vayrochany Buddhy/\\
	Na maku sa man da ba zara dan kan
}

\stanza{
	Shaka Nyorai (Shakyamuni Tathagata)\\
	Na maku sa man da ba da nan baku
}

\stanza{
	Monju Bosatsu (Manjusri Bodhisattwa)\\
	On a~ra ha sha no
}

\stanza{
	Fugen Bosatsu (Samantabhadra Bodhisattwa)\\
	On san ma ya sa to ban
}

\stanza{
	Jizo Bosatsu (Ksitgarbha)\\
	On ka ka ka bi san ma e sowa ka
}

\stanza{
	Miroku Bosatsu (Maitreya Bodhisattwa)\\
	On bai tare ya sowa ka
}

\stanza{
	Yakushi Nyorai (Bhaisajya Guru Tathagata)\\
	On ko ro ko ro sen da ri ma to gi sowa ka
}

\stanza{
	I Da Ten (Skanda -- Bóstwo Chroniące Świątynię)\\
	On ita tei ta moko tei ta sowa ka
}

\stanza{
	Kanzeon Bosatsu (Avalokiteśvara Bodhisattwa)\\
	On a~ro ri kya sowa ka
}

\stanza{
	Shu Ya Jin (Bóstwo Obserwacji Nocy)\\
	On ba sam ba en tei shuya jin sowa ka
}

\stanza{
	Seishi Bosatsu (Maha Sathamaprapta Bodhisattwa)\\
	On san zan zan saku sowa ka
}

\stanza{
	Amida Nyorai (Amitabha Tathagata)\\
	On amirita tei zei ka ra un
}

\stanza{
	Ashuku Nyorai (Akshobhya Tathaplain)\\
	On a~ki shi yo bi ya un
}

\stanza{
	DaiNichi Nyorai (Mahawairochana Tathagata)\\
	/Diamentowego Królestwa~-- Vajrdhatu/\\
	On ba za ra da to ban
}

\stanza{
	Dainichi Nyorai (Vayrochana Tathagata)\\
	/Królestwa lona~-- Garbhadhatu/\\
	On a~bi ra un ken
}

\stanza{
	Kokuzo Bosatsu (Akashagarbha Bodhisattwa)\\
	On ba zara ara tanno on taraku sowa ka
}
\end{Verse}
\end{Prayer}

\newpage
%%%%%%%%%%%%%%%%%%%%%%%%%%%%%%%%%%%%%%%%%%%%%%%%%%%%%%%%%%%%%%%%%%%%%%%%%%%%%%%
%%%% SHUSH\=OGI
\begin{Prayer}{shushogi}
	{SHUSH\=OGI}{-}
	{Znaczenie praktyki i oświecenia, Eihei D\=ogen Zenji}

% (!!!) ten tekst można ładnie sformatować na podstawie soto shu scriptures!!!

\begin{center}
DAI ISSH\=O, S\=OJO
\end{center}

\begin{JAPANESE}
SH\=O O~AKIRAME SHI O~AKIRAMURU WA BUKKE ICHI-DAI-JI NO IN-NEN NARI,
SH\=OJI NO NAKA NI HOTOKE AREBA SH\=OJI NASHI, TADA SH\=OJI SUNAWACHI NEHAN
TO KOKORO-ETE, SH\=OJI TO SHITE IT\=O BEKI MO NAKU, NEHAN TO SHITE NEG\=O
BEKI MO NASHI, KONO TOKI HAJIMETE SH\=OJI O~HANARU-RU BUN ARI, TADA
ICHI-DAI-JI IN-NEN TO G\=UJIN SUBESHI. \keisu NIN-SHIN URU-KOTO KATASHI,
BUPP\=O \=O KOTO MARE NARI, IMA WARERA SHUKU-ZEN NO TASUKURU NI YORITE,
SUDE NI UKE GATAKI NINSHIN O~UKE TARU NOMI NI ARAZU, AI GATAKI BUPP\=O NI
AI TATEMATSURERI, SH\=OJI NO NAKA NO ZENSH\=O, SAISH\=O NO SH\=O NARU
BESHI, SAISH\=O NO ZENSHIN O~ITAZURA NI SHITE ROMEI O~MUJ\=O NO KAZE NI
MAKASURU KOTO NAKARE. MUJ\=O TANOMI GATASHI, SHIRAZU ROMEI IKANARU MICHI NO
KUSA NIKA OCHIN, MI SUDE NI WATAKUSHI NI ARAZU, INOCHI WA K\=OIN NI
UTSU-SARETE SHIBARAKU MO TODOME GATASHI, K\=O-GAN IZUKU E KA SARI NISHI,
TAZUNEN TO SURU NI SH\=OSEKI NASHI. TSURA-TSURA KANZURU TOKORO NI \=OJI NO
FUTATABI \=O BEKARA ZARU \=OSHI, MUJ\=O TACHIMACHI NI ITARU TOKI WA KOKU\=O
DAIJIN SHINJITSU J\=UBOKU SAISHI CHINH\=O TASUKURU NASHI, TADA HITORI
K\=OSEN NI OMOMUKU NOMI NARI, ONORE NI SHITAGAI-YUKU WA TADA KORE ZEN-AKU
GOTT\=O NOMI NARI. IMA NO YO NI INGA O~SHIRAZU GOPP\=O O~AKIRAMEZU, SANZE
O~SHIRAZU, ZEN-AKU O~WAKIMAE-ZARU JAKEN NO TOMOGARA NIWA GUNSU BEKARAZU,
\=OYOSO INGA NO D\=ORI REKI-NEN TO SHITE WATAKUSHI NASHI, Z\=OAKU NO MONO
WA OCHI SHUZEN NO MONO WA NOBORU, G\=ORI MO TAGAWA ZARU NARI, MOSHI INGA
B\=OJITE MUNASHI-KARAN GA GOTOKI WA, SHO-BUTSU NO SHUSSE ARU BEKARAZU,
SOSHI NO SEI-RAI ARU BEKARAZU. ZEN-AKU NO H\=O NI SANJI ARI, HIT\=OTSU NIWA
JUNGEN-H\=OJU, FUTATSU NIWA JUNJI-SH\=OJU, MITSU NIWA JUNGO-JIJU, KORE
O~SANJI TO IU, BUSSO NO D\=O O~SHUJ\=U SURU NIWA, SONO SAISHO YORI KONO
SANJI NO GOPP\=O NO RI O~NARAI AKIRAMURU NARI. \keisu SHIKA ARAZAREBA \=OKU
AYAMARITE JAKEN NI OTSURU NARI, TADA JAKEN NI OTSURU NOMI NI ARAZU, AKUD\=O
NI OCHITE CH\=OJI NO KU O~UKU. \keisu MASANI SHIRU-BESHI KONJ\=O NO WAGAMI
FUTATSU NASHI, MITSU NASHI, ITAZURA NI JAKEN NI OCHITE MUNASHIKU AKUG\=O
O~KANTOKU-SEN, OSHIKARA-ZARAMEYA, AKU O~TSUKURI NAGARA AKU NI ARAZU TO
OMOI, \shokei AKU NO H\=O ARU-BEKARAZU TO JASHI-YUI SURU NI YORITE \shokei
AKU NO H\=O O~KANTOKU SEZARU NIWA ARAZU.
\end{JAPANESE}


\begin{center}
DAI NISH\=O, SANGEMETSUZAI
\end{center}


\begin{JAPANESE}
\keisu BUSSO AWEREMI NO AMARI K\=ODAI NO JIMON O~HIRAKI OKERI, KORE ISSAI
SHUJ\=O O~SH\=ONY\=U SESHI-MEN GA TAME NARI, NINDEN TAREKA IRA-ZARAN, KANO
SANJI NO AKU-GOPP\=O KANARAZU KANZU-BESHI TO IEDOMO, SANGE SURU GA GOTOKI
WA OMOKI O~TENJITE KY\=OJU SESHIMU, MATA METSU-ZAI SH\=OJ\=O NARASHI-MURU
NARI. \keisu SHIKA-AREBA J\=OSHIN O~MOPPARA NI SHITE ZEN-BUTSU NI SANGE
SUBESHI, INMO SURU TOKI ZEN-BUTSU SANGE NO KUDOKU-RIKI WARE O~SUKUITE
SH\=OJ\=O NARASHIM, KONO KUDOKU YOKU MUGE NO J\=OSHIN SH\=OJIN O~SH\=OCH\=O
SESHIMURU NARI. J\=OSHIN ICHIGEN SURU TOKI, JITA ONAJIKU TEN-ZE RARURU
NARI, SONO RIYAKU AMANEKU J\=O HIJ\=O NI K\=OBURASHIMU. SONO DAISHI WA,
NEGAWAKU WA WARE TATOI KAKO NO AKU-G\=O \=OKU KASANARITE SH\=OD\=O N INNEN
ARI TOMO, BUTSU-D\=O NI YORITE TOKUD\=O SERISHI SHOBUTSU SHOSO WARE
O~AWARE-MITE G\=ORUI O~GEDATSU SESHIME, GAKUD\=O SAWARI NAKARA-SHIME, SONO
KUDOKU H\=OMON AMANEKU \keisu MUJIN HOKKAI NI J\=UMAN MIRIN SERAN, AWAREMI
O~WARE NI BUNPU SUBESHI, BUSSO NO \=OSHAKU WA WARERA NARI, WARERA GO
T\=ORAI WA BUSSO NARAN. \keisu GA SHAKU SHO-Z\=O SHO-AKU-G\=O, KAI Y\=U
MU-SHIN TON-JIN-CHI, J\=U-SHIN KU I~SHI SHO-SH\=O, ISSAI GA KON KAI SAN-GE,
KAKU NO GOTOKU SANGE SUREBA KANARAZU BUSSO NO MY\=OJO ARU NARU, \shokei
SHIN-NEN SHIN-GI HORRO BYAKU-BUTSU SUBESHI, \shokei HORRO NO CHIKARA ZAIKON
O~SHITE SH\=OIN SESHI MURU NARI.
\end{JAPANESE}

% (!!!) końce linii w~oryginale były kiepsko widoczne i~mogą być braki!!!


\begin{center}
DAI SANSH\=O, JUKAINY\=UI\\
\end{center}


\begin{JAPANESE}
\keisu TSUGI NIWA FUKAKU BU-PP\=O-S\=O NO SANB\=O O~UYAMA TATEMATSURU
BESHI, SH\=O O~KAE MI O~KAE TEMO SANB\=O O~KUY\=OSHI UYAMAI TATEMATSU-RAN
KOTO O~NEG\=O BESHI SAI-TEN T\=O-D\=O BUSSO SH\=ODEN SURU TOKORO WA KUGY\=O
BU-PP\=O-S\=O NRAI. \keisu MOSHI HAKU-FUKU SH\=OTOKU NO SHUJ\=O WA SANB\=O
NO MY\=OJI NAO KIKI TATEMATSURA-ZARU NARI, IKANI IWAN YA KIESHI TATEMATSURU
KOTO O~ENYA ITAZURU NI SHOHITSU O~OSORETE SANJIN KIJIN T\=O NI KIESHI, ARUI
WA GED\=O NO SEITA NI KIE SURU KOTO NAKARE, KARE WA SONO KIE NI YORITE
SHUKU O~GEDATSU SURU KOTO NASHI, HAYAKU BU-PP\=O-S\=O NO SANB\=O NI KIE SHI
TATEMATSURU-TE SHUKU O~GEDATSU SURU NOMI NI ARAZU BODAI O~J\=OJ\=U SUBESHI.
SONO KIE SANB\=O TOWA MASANI J\=OSHIN O~MOPPARA NI SHITE ARUIWA NYORAI
GENZAI-SE NIMO ARE, ARUIWA NYORAI METSU-GO NIMO ARE, GASSH\=O SHI TEIZU
SHITE KUCHI NI TONAETE IWAKU, NAMU-KIE-BUTSU, NAMU-KIE-H\=O, NAMU-KIE-S\=O,
HOTOKE WA KORE DAISHI NARUGA YUE NI KIE SU, H\=O WA RY\=OYAKU NARU GA YUE
NI KIE SU. S\=O WA SH\=OYU NARU GA YUE NI KIE SU, BUTSU-DESHI TO NARU KOTO
KANARAZU SANKI O~UKETE SONO NOCHI SHOKAI O~UKURU NARI, SHIKA AREBA
SUNAWACHI SANKI NI YORITE TOKKAI ARU NARI. KONO KIE BU-PP\=O-S\=O NO KUDOK
KANARAZU KANN\=O D\=OK\=O SURU TOKI J\=OJU SURU NARI, TATOI TENJ\=O NINGEN
JIGOKU KICHIKU NARITO IE DOMO, KANN\=O D\=OK\=O SUREBA KANARAZU KIE SHI
TATEMATSURU NARI, SUDE NI KIESHI TATEMATSURU GA GOTOKI WA SHOJO SESE ZAI
ZAI SHOSHO NI Z\=OCH\=O SHI, KANARAZU SHAKKU RUITOKU SHI, ANOKU
TARA-SANMYAKU SANBODAI O~J\=OJ\=U SURU NARI, SHIRU BESHI SANKI NO KUDOKU
SORE SAISON SAIJ\=O JINJIN FUKASHIGI NARI TO IU KOTO, SESON SUDE NI
SH\=OMY\=O SHIMASHI MASU, SHUJ\=O MASANI SHINJU SUBESHI. TSUGI NIWA MASANI
SANJU-J\=OKAI O~UKE TATEMATSURU BESHI, DAI-ICHI SH\=O-RITSUGI-KAI, DAI-NI
SH\=O-ZENB\=O-KAI, DAI-SAN SH\=O-SHUJ\=O-KAI NARI, TSUGI NIWA MASANI
J\=UJ\=U-KINKAI O~UKE TATEMATSURU BESHI, DAI-ICHI FUSESSH\=O-KAI, DAI-NI
FUCH\=UT\=O-KAI, DAI-SAN FUJAIN-KAI, DAI-SHI FUM\=OGO-KAI, DAI-GO
FU KOSHU-KAI, DAI-ROKU FUSEKKA-KAI, DAI-SHICHI FUJISAN-KITA-KAI,
DAI-HACHI FUKEN-H\=OZAI-KAI, DAI-KU FUSHIN-I-KAI, DAI-J\=U
FUB\=O-SANB\=O-KAI NARI.
\end{JAPANESE}


\begin{center}
J\=ORAI SANKI, SANJUJ\=O-KAI, J\=UJ\=UKIN-KAI, KORE SHOBUTSU NO JUJISHI
TAM\=O TOKORO NARI.
\end{center}


\begin{JAPANESE}
JUKAI SURU GA GOTOKI WA, SANZE NO SHOBUTSU NO SHOSH\=O NARU
ANOKU-TARA-SANMYAKU-SANBODAI KONG\=O FUE NO BUKKA O~SH\=O SURU NARI, TARE
NO CHININ KA GONGU SEZARAN, SESON AKIRAKA NI ISSAI SHUJ\=O NO TAME NI
SHIMESHI MASHI MASU, SHUJ\=O BUKKAI O~UKUREBA, SUNAWACHI SHOBUTSU NO KURAI
NI IRU, KURAI DAIGAKU NI ONAJ\=USHI OWARU, MAKOTO NI KORE SHOBUTSU NO MIKO
NARI TO. \keisu SHOBUTSU NO TSUNE NI KONO NAKA NI J\=UJI TARU, KAKUKAKU NO
H\=OMEN NI CHIKAKU O~NOKOSAZU, SUNJ\=O NO TOKO-SHINAE NI KONO NAKA NI
SHIY\=O SURU, KAKU-KAKU NO CHIKAKU NI H\=OMEN ARAWA-REZU, \keisu KONO TOKI
JIPP\=O HOKKAI NO TOCHI S\=OMOKU SH\=OHEKI GARYAKU MINA BUTSUJI O~NASU
O~MOTTE, SONO OKOSU TOKORO NO F\=USUI NO RIYAKU NI AZUKARU TOMO-GARA, MINA
JINMY\=O FUKASHIGI NO BUKKE NI MY\=OSHI SERARETE CHIKAKI SATORI O~ARAWASU,
\shokei KORE O~MUI NO KUDOKU TO SU, KORE O~MUSA NO KUDOKU TO SU, \shokei
KORE HOTSU-BODAI-SHIN NARI.
\end{JAPANESE}


\begin{center}
DAI YONSH\=O, HOTSUGANRISH\=O
\end{center}


\begin{JAPANESE}
\keisu BODAI SHIN O~OKOSU TO IU WA, ONORE IMADA WATARA-ZARU SAKI NI ISSAI
SHUJ\=O O~WATASAN TO HOTSUGAN-SHI ITONAMU NARI, TATOI ZAIKE NIMO ARE, TATOI
SHUKKE NIMO ARE, ARUI WA TENJ\=O NIMO ARE, ARUI WA NINGEN NIMO ARE, KU NI
ARITO IU TOMO RAKU NI ARI TO IU TOMO, HAYAKU JIMI-TOKUDO-SEN-DOTA NO KOKORO
O~OKOSU BESHI. \keisu SONO KATACHI IYASHI TO IU TOMO, KONO KOKORO
O~OKOSEBA, SUDE NI ISSAI SHUJ\=O NO D\=OSHI NARI, TATOI HICHISAI NO
NYORY\=U NARI TOMO SUNAWACHI SHISHU NO D\=OSHI NARI, SHUJ\=O NO JIFU NARI,
NANNYO O~RONZURU KOTO NAKARE, KORE BUTSUD\=O GOKUMY\=O NO H\=OSOKU NARI,
MOSHI BODAI-SHIN O~OKO-SHITE NOCHI, ROKU-SHU SHISH\=O NI RINDEN-SU TO
IEDOMO, SONO RINDEN NO INNEN MINA BODAI NO GY\=OGAN TO NARU NARI, SHIKA
AREBA, J\=URAI NO K\=OIN WA TATOI MUNASHI-KU SUGOSU TO IU TOMO, KONJ\=O NO
IMADA SUGI ZARU AIDA NI ISOGITE HOTSUGAN SUBESHI, TATOI HOTOKE NI NARU-BEKI
KUDOKU JUKU-SHITE ENMAN SUBESHI TO IU TOMO, NAO MEGURA-SHITE SHUJ\=O NO
J\=OBUTSU TOKUD\=O NI EK\=O SURU NARI, ARUIWA MURY\=O-G\=O OKONAITE SHUJ\=O
O~SAKI NI WATA-SHITE MIZUKARA WA TSUI NI HOTOKE NI NARAZU, TADASHI SHUJ\=O
O~WATASHI SHUJ\=O O~RIYAKU SURU MO ARI. SHUJ\=O O~RIYAKUSU TO IU WA SHIMAI
NO HANNYA ARI, HITOTSU NIWA FUSE, FUTATSU NIWA AIGO, MITSU NIWA RIGY\=O,
YOTSU NIWA D\=OJI, KORE SUNAWACHI SATTA NO GY\=OGAN NARAI, SONO FUSE TO IU
WA MUSABORA-ZARU NARI, WAGA MONO NI ARAZARE DOMO FUSE O~SAE ZARU D\=ORI
ARI, SONO MONO NO KAROKI O~KIRA WAZU, SONO K\=O NO JITSU NARU BEKI NARI,
SHIKA AREBA SUNAWACHI IKKU ICHIGE NO H\=O OMO FUSE SUBESHI, SHI-SH\=O
TA-SH\=O NO ZENSHU TO NARU, ISSEN ISS\=O NO TAKARA O~MO FUSE SUBESHI, SHISE
TASE NO ZENGON O~KIZASU, H\=O MO TAKARA NARU BESHI, TAKARA MO H\=O NARU
BESHI, TADA KARE GA H\=OSHA O~MUSABORAZU MIZU, KARA GA CHIKARA O~WAKATSU
NARI, FUNE O~OKI HASHI O~WATASU MO FUSE NO DANDO NARI, CHISH\=O SANGY\=O
MOTO YORI FUSE NI ARA ZARU KOTO NASHI. AIGO TO IU WA, SHUJ\=O O~MIRU NI,
MAZU JIAI NO KOKORO O~OKOSHI, KOAI NO GONGO O~HODOKOSU NARI, JINEN SHUJ\=O
Y\=UNYO SHAKUSHI NO OMOI O~TAKUWAETE GONGO SURU WA AIGO NARI, TOKU ARU WA
HOMU BESHI, TOKU NAKI WA AWAREMU BESHI, ONTEKI O~G\=OBUKUSHI, KUNSHI
O~WABOKU NARASHI-MURU KOTO AIGO O~KONPON TO SURU-NARI, MUKAITE AIGO O~KIKU
WA OMOTE O~YOROKOBA-SHIME, KOKORO O~TANOSHIKUSU, MUKAWAZU-SHITE AIGO O~KIKU
WA KIMO NI MEIJI TAMASHI NI MEIZU, AIGO YOKU KAITEN NO CHIKARA ARU KOTO
O~GAKUSU BEKI NARI, RIGY\=O TO IU WA KISEN NO SHUJ\=O NI OKITE RIYAKU NO
ZENGY\=O O~MEGURASU NARI, KY\=UKI O~MI BY\=OJAKU O~MISHI TOKI, KARE GA
H\=OSHA O~MOTOMEZU, TADA HITOE NI RIGY\=O NI MOY\=O SARURU NARI, GUNIN
OMOIWAKU WA RITA O~SAKI TO SEBA MIZU-KARA GA RI HABU-KARENU BESHI TO, SHIKA
NIWA ARA-ZARU NARI. RIGY\=O WA IPP\=O NARI, AMA-NEKU JIITA O~RISURU NARI.
\keisu D\=OJI TO IU WA FUI NARI, JI NIMO FUI NARI, TA NIMO FUI NARI,
TATOEBA NINGEN NO NYORAI WA NINGEN NI D\=OZERU GA GOTOSHI, \keisu TAO SHITE
JI NI D\=OZE-SHIMETE NOCHI NI JI O~SHITE TA NI D\=OZE-SHIMARU D\=ORI ARU
BESHI, JITA WA TOKI NI SHITAG\=O-TE MUKY\=U NARI, UMI NO MIZU O~JISE ZARU
WA D\=OJI NARI. KONO YUE NI YOKU MIZU ATSUMARI-TE UMI TO NARU NARI. \=OYOSO
BODAI-SHIN NO GY\=OGAN NIWA KAKU NO GOTOKU NO D\=ORI SHIZUKA NI SHIYUI
SUBESHI, SOTSUJI NI SURU KOTO NAKARE, \shokei SAIDO SH\=OJU NI ISSAI
SHUJ\=O MINA KE O~K\=OBURAN \shokei KUDOKU O~RAIHAI KUGY\=O SUBESHI.
\end{JAPANESE}


\begin{center}
DAI GOSH\=O, GY\=OJIH\=OON
\end{center}


\begin{JAPANESE}
\keisu KONO HOTSU-BODAI-SHIN, \=OKU WA NAN-EN-BU NO NINSHIN NI HOSSHIN
SUBEKI NARI, IMA KAKU NO GOTOKU NO INNEN ARI, GANSH\=O-SHI SHABA KOKUDO-SHI
KITA RERI, KEN-SHAKA-MUNI-BUTSU O~YOROKOBA ZARANYA. \keisu SHIZUKA NI
OM\=O-BESHI, SH\=OB\=O YO NI RUFU SEZARAN TOKI WA, SHINMEI O~SH\=OB\=O NO
TAME NI H\=OSHA SEN KOTO O~NEG\=O TOMO \=O BEKARAZU, SH\=OB\=O NI \=O
KONNICHI NO WARERA O~NEG\=O BESHI, MIZUYA, HOTOKE NO NOTAMA-WAKU, MUJ\=O
BODAI O~ENZETSU SURU SHI NI AWAN NIWA, SHUSH\=O O~KANZURU KOTO NAKARE,
Y\=OGAN O~MIRU KOTO NAKARE, HI O~KIR\=O KOTO NAKARE, OKONAI O~KANGA-URU
KOTO NAKARE, TADA HANNYA O~SONJ\=U SURU GA YUE NI, NICHI NICHI SAN JI NI
RAIHAI SHI, KUGY\=O SHITE, SARA NI GENN\=O NO KOKORO O~SH\=OZE SHIMURU KOTO
NAKARE TO. IMA NO KEN-BUTSU MON-P\=O WA BUSSO MENMEN NO GY\=OJI YORI
KITA-RERU JION NARI, BUSSO MOSHI TANDEN SEZU BA, IKA NI SHITE KA KON NICHI
NI ITARAN, IKKU NO ONO NAO H\=OSHA SUBESHI, IPP\=O N\=O ON NAO H\=OSHA
SUBESHI, IWANYA SH\=OB\=OGENZ\=O MUJ\=O-DAIH\=O NO DAION KORE O~H\=OSHA
SEZARAN YA BY\=OJAKU NAO ON O~WASUREZU SANPU NO KAN YOKU H\=OSHA ARI,
KY\=UKI NAO ON O~WASUREZU, YOFU NO IN YOKU H\=OSHA ARI. CHIKU-RUI NAO ON
O~H\=OZU, JINRUI IKADEKA ON O~SHIRA-ZARAN. SONO H\=OSHA WA YOGE NO H\=O WA
ATARU BEKARAZU, TADA MASA NI NICHI NICHI NO GY\=OJI, SONO H\=OSHA NO
SH\=OD\=O NARU-BESHI, IWAYURU NO D\=ORI WA NICHI NICHI NO SEIMEI O~NAOOZARI
NI SEZU, WATAKUSHI NI TSUI-YASAZARAN TO GY\=OJI SURU NARI. K\=OIN WA YA
YORI MO SUMIYAKA NARI, SHINMEI WA TSUYU YORI MO MOROSHI, IZURE NO ZENGY\=O
H\=OBEN ARITE KA SUGI NISHI ICHI-NICHI O~FUTATABI KAESHI ETARU, ITAZURU NI
HYAKU SAI IKERAN WA URAMU BEKI JITSU-GETSU NARI, KANA-SHIMU BEKI KAI-GAI
NARI, TATOI HYAKU SAI NO JITSU GETSU WA SH\=O-SHIKI NO NUBI TO CHIS\=O
SUTOMO, SONO NAKA ICHI-NICHI NO GY\=OJI O~GY\=OSHU SEBA ISSH\=O NO
HYAKU-SAI O~GY\=OSHU SURU NOMI NIARAZU, HYAKU SAI NO TASH\=O O~MO DOSHU
SUBEKI NARI, KONO ICHI-NICHI NO SHINMEI WA, T\=OTOBU BEKI SHINMEI NARI,
T\=OTOBU BEKI KEIGAI NARI, KONO GY\=OJI ARAN SHINJIN MIZU-KARA MO AISU
BESHI, MIZU-KARA MO UYAM\=O BESHI, WARERA GA GY\=OJI NI YORITE SHOBUTSU NO
GY\=OJI GENJ\=O-SHI, SHOBUTSU NO \keisu DAID\=O TS\=UDATSU SURU NARI, SHIKA
AREBA SUNAWACHI ICHI-NICHI NO GY\=OJI KORE SHOBUTSU NO SHUSHI NARI,
SHOBUTSU NO GY\=OJI NARI. \keisu IWAYURU SHOBUTSU TOWA SHAKA-MUNI-BUTSU
NARI, SHAKA-MUNI-BUTSU KORE SOKUSHIN ZEBUTSU NARI, KAKO GENZAI MIRAI NO
SHOBUTSU, TOMO NI HOTOKE TO NARU TOKI WA KANARAZU SHAKA-MUNI-BUTSU TO
NARU-NARI KORE SOKU-SHIN ZEBUTSU NARI, SOKU-SHIN ZEBUTSU TO IU WA \shokei
TARE TO IU ZOTO SHINSAI NI SANKY\=U SUBESHI, \shokei MASA NI BUTSUON O~H\=O
ZURU NITE ARAN.
\end{JAPANESE}
\end{Prayer}

%%%%%%%%%%%%%%%%%%%%%%%%%%%%%%%%%%%%%%%%%%%%%%%%%%%%%%%%%%%%%%%%%%%%%%%%%%%%%%%
%%%% TRAKTAT O~PRAKTYCE I~OŚWIECENIU
\begin{Prayer}{traktat_o_praktyce}
	{ZNACZENIE PRAKTYKI I OŚWIECENIA}{-}
	{Shush\=ogi, Eihei D\=ogen Zenji}

\bigskip
\noindent\textbf{1. OGÓLNY WSTĘP}\par
\medskip

Wyjaśnienie narodzin i wyjaśnienie śmierci jest dla ucznia Buddhy Przyczyną Jednej Wielkiej Sprawy. Jeśli w narodzinach i śmierci jest Buddha, to nie ma narodzin i śmierci. Jedynie zrozum, że narodziny i śmierć są nirwaną. Nie czuj niechęci do narodzin i śmierci, ani nie pragnij nirwany. W tym momencie po raz pierwszy zostaniesz uwolniony od narodzin i śmierci. Tylko całkowicie, do samego dna, powinieneś zbadać tę Przyczynę Jednej Wielkiej Sprawy.


Trudno jest osiągnąć ludzkie ciało, rzadko można napotkać Dharmę Buddhy. Teraz dzięki mocy cnotliwych  uczynków dokonanych w przeszłości otrzymaliśmy nie tylko trudne do osiągnięcia ludzkie ciało, lecz również zostało nam ofiarowane spotkanie z Dharmą Buddhy. Dlatego jest to najlepsze odrodzenie pośród narodzin i śmierci, najwyższy rodzaj życia. Nie wolno marnować na próżno swojego cennego i dobrego ludzkiego ciała, wystawiając je na wiatr przemijalności.


Na przemijalności nie sposób polegać. Czy nie wiesz, że życie jest tak kruche jak spadające źdźbło trawy? To ciało nie jest mną, życie przemija wraz z upływającym czasem i nie można zatrzymać go nawet na chwilę. Gdy raz zniknie świeża twarz młodości, nie znajdziecie nawet jej śladów. Jeśli z uwagą rozważymy przeszłe wydarzenia, to widzimy, że nie możemy spotkać ich ponownie. 


Gdy nagle pojawi się nietrwałość, wówczas ani królowie, ministrowie, krewni, słudzy, uczniowie, żona i dzieci, ani rzadkie klejnoty nie uratują nas. Umrzemy samotnie, a towarzyszyć nam będzie jedynie nasza dobra i zła karma. Teraz, w dzisiejszym świecie, powinniśmy unikać towarzystwa ludzi o fałszywych poglądach, którzy nie są świadomi przyczyny i skutku, nie mają wglądu w zapłatę za popełnione uczynki, nie znają trzech światów i nie odróżniają dobra i zła. 


Ogólnie prawo przyczynowości jest jasne i nie ma w nim żadnego ja. Czyniący zło upadną (do piekła), praktykujący dobro podążą w górę (do nieba). W prawie przyczyny i skutku nie ma najmniejszego uchybienia. Jeśli przyczyna i skutek nie byłyby takie, to wszyscy Buddowie nie mogliby pojawiać się na świecie ani też Bodhidharma nie mógłby przybyć do Chin.


Zapłata za dobro i zło pojawia się w trzech okresach czasu -- zapłata doświadczana w obecnym życiu,  zapłata doświadczana w życiu następującym po obecnym i zapłata doświadczana w późniejszych żywotach. To nazywa się trzema okresami czasu. Prawdy o zapłacie karmicznej w trzech okresach czasu należy się uczyć i zrozumieć na samym początku praktyki drogi Buddhów i Patriarchów.  


W przeciwnym razie wielu popełni błędy i będzie utrzymywać fałszywe poglądy. Nie tylko popadniecie w błędne poglądy, ale wpadniecie w świat zła i doświadczycie długiego okresu cierpienia. 


W tym życiu masz tylko jedno ciało, a nie dwa lub trzy. Czy to nie głupie, gdy utrzymując fałszywe poglądy, bezmyślnie czynisz zło, myśląc, że nie robisz źle, podczas gdy w rzeczywistości tak robisz. Nie możesz uniknąć zapłaty za swoje złe czyny, nawet jeśli uważasz, że skoro nie wierzysz w zapłatę za złe czyny, to nie doświadczysz skutków złych czynów.



\newpage
\par\noindent\textbf{2. SKRUCHA I ZNISZCZENIE ZŁYCH CZYNÓW}\par
\medskip

Buddhowie i Patriarchowie otworzyli szeroko wielkie bramy współczucia, to jest potwierdzone, autentyczne wejście dla wszystkich istot, kto z ludzi i bogów nie mógłby w nie wejść? Choć skutki złych czynów muszą nadejść w jednym z trzech okresów czasu, skrucha pomniejsza ich ciężar, niszcząc złe czyny i przynosząc czystość. 


Dlatego z całą szczerością serca czyńmy skruchę przed Buddą. Moc cnoty zasługi skruchy czynionej przed Buddą ratuje i oczyszcza nas, ta cnota zasługi sprawia, że rozwijamy nieskalaną przeszkodami czystą wiarę, zaufanie i wysiłek oraz przedłuża nasze życie. Kiedy raz pojawia się czysta wiara i zaufanie, zmienia nas i innych, swoim dobrodziejstwem ogarniając  wszystkie rzeczy ożywione i nieożywione. To wielkie znaczenie [zawarte jest w modlitwie]:


\begin{quote}


Modlę się i ślubuję -- nawet jeśli nagromadzenie moich przeszłych złych czynów jest tak wielkie, że tworzy przeszkodę w praktykowaniu Drogi Buddhy, błagam o współczucie wszystkich  Buddów i wielkich mistrzów, aby uwolnili mnie od złych skutków moich czynów, aby usunęli wszystkie przeszkody w praktykowaniu Drogi. Cnota zasługi powszechnej bramy Dharmy wypełnia całkowicie niewyczerpalną Dharmadhatu, oby podzieliła się ze mną swoim współczuciem. Buddhowie i Patriarchowie przeszłości byli nami, my w przyszłości będziemy Buddhami i Patriarchami.

	\begin{verse}
		Wszystkie złe czyny popełnione przeze mnie w przeszłości,\\
		wszystkie mające przyczynę w nie mającej początku\\
		chciwości, gniewie i głupocie,\\
		narodzone z ciała, mowy i umysłu,\\
		wszystkie teraz całkowicie wyznaję ze skruchą.
	\end{verse}
\end{quote}

Czyniąc skruchę w ten sposób, na pewno otrzymamy niewidzialną pomoc Buddów i Patriarchów. Utrzymując tę myśl umyśle i działając w ten sposób, należy wyjawić skruchę Buddzie. Siła wyjawienia skruchy niszczy korzenie zła.

\bigskip
\noindent\noindent\textbf{3. PRZYJĘCIE WSKAZAŃ I WEJŚCIE NA POZIOM [BUDDHY] }\par
\medskip

Następnie powinniście okazać głęboką cześć Buddzie, Dharmie i Sandze -- Trzem Skarbom. Choć życie może się zmienić i ciało może się zmienić, to zawsze powinniśmy modlić się i ślubować, że wobec Trzech Klejnotów będziemy czynić ofiary oraz je czcić i szanować. 


Buddowie i Patriarchowie zarówno w Indiach, jak i we wschodnich krainach prawidłowo przekazywali tę pełną szacunku cześć dla Buddhy, Dharmy i Sanghi. Ludzie pozbawieni cnoty i nie mający szczęścia nie są w stanie nawet usłyszeć imion Trzech Skarbów, jakże mogliby znaleźć w nich schronienie? 


Nie działaj podobnie do tych, którzy ogarnięci strachem na próżno szukają schronienia w górskich bóstwach i duchach lub oddają cześć niebuddyjskim sanktuariom, ponieważ niemożliwe jest osiągnięcie w ten sposób wyzwolenia od cierpienia. Zamiast tego, szybko znajdź schronienie w Buddzie, Dharmie i Sandze, dążąc nie tylko do wyzwolenia od cierpienia, lecz również do całkowitego oświecenia. 


Znajdowanie schronienia w Trzech Skarbach oznacza dokładnie  całkowicie czystą wiarę i zaufanie. Czy to za życia Tathagaty, czy też po, ludzie powinni składać razem ręce i ze spuszczonymi głowami śpiewać co następuje:


\newpage
\begin{verse}
	Znajdujemy schronienie w Buddzie.\\
	Znajdujemy schronienie w Dharmie.\\
	Znajdujemy schronienie w Sandze.

	Znajdujemy schronienie w Buddzie, ponieważ jest naszym wielkim Nauczycielem.\\
	Znajdujemy schronienie w Dharmie, ponieważ jest dobrym lekarstwem.\\
	Znajdujemy schronienie w Sandze, ponieważ złożona jest z doskonałych przyjaciół.

\end{verse}

To tylko dzięki znajdowaniu schronienia w Trzech Klejnotach można stać się uczniem Buddhy i być przysposobionym do przyjęcia wszystkich pozostałych przykazań. 


Zasługa znalezienia schronienia w Buddzie, Dharmie i Sandze zostaje urzeczywistniona, jako duchowa odpowiedź. Ci, którzy doświadczają tej odpowiedzi, znajdują schronienie w Trzech Klejnotach niezależnie od tego, czy istnieją jako niebiańskie czy ludzkie istoty, mieszkańcy piekieł, głodne duchy czy zwierzęta. W wyniku tego nagromadzona zasługa nieuchronnie wzrasta poprzez różne etapy istnienia, prowadząc ostatecznie do najwyższego, niedoścignionego oświecenia -- Anuttara Samiak Sambodhi.


Wiedz, że sam Czczony przez Świat urodził się świadom faktu, że ta zasługa ma niezmierzoną wartość i nieprzeniknioną głębię. Dlatego wszystkie żyjące stworzenia powinny przyjąć to schronienie. 


Następnie powinniśmy przyjąć Trzy Czyste Wskazania:


\begin{Enumerate}
	\item Pierwsze Wskazanie praktykować dyscyplinę.
	\item Drugie Wskazanie praktykować dobre dharmy.
	\item Trzecie Wskazanie pomagać odczuwającym istotom.
\end{Enumerate}


Następnie powinniśmy przyjąć Wskazania Dziesięć Ważnych Zakazów:


\begin{Enumerate}
	\item Pierwsze Wskazanie nie zabijać,
	\item Drugie Wskazanie nie kraść,
	\item Trzecie Wskazanie nie być haniebnie rozwiązłym,
	\item Czwarte Wskazanie nie kłamać,
	\item Piąte Wskazanie nie pić alkoholu,
	\item Szóste Wskazanie nie mówić o~błędach innych,
	\item Siódme Wskazanie nie chwalić się i~nie niszczyć innych,
	\item Ósme Wskazanie nie żałować Dharmy, ani bogactwa,
	\item Dziewiąte Wskazanie nie wpadać w~gniew,
	\item Dziesiąte Wskazanie nie oczerniać Trzech Klejnotów.
\end{Enumerate}


Wszyscy Buddowie otrzymali i przestrzegali Trzech Schronień, Trzech Czystych Wskazań i Wskazania Dziesięciu Ważnych Zakazów. Dzięki otrzymaniu tych wskazań wszyscy Buddhowie trzech światów osiągnęli Owoc Stanu Buddhy, Najwyższą Mądrość Oświecenia -- Anuttara Samiak Sambodhi, Diamentowe, Niezniszczalne Oświecenie. Czy istnieje jakaś mądra osoba, która z chęcią nie poszukiwałaby tego?


Czczony Przez Świat jasno pokazał wszystkim odczuwającym istotom, że gdy otrzymują wskazania Buddhów, to osiągają poziom wszystkich Buddhów i urzeczywistniają to samo Wielkie Oświecenie, naprawdę stając się ich dziećmi. Wszyscy Buddowie przebywają w tym stanie, postrzegają wszystko bez pozostawiania jakichkolwiek śladów. 


Kiedy zwykłe istoty czynią z tego swoje miejsce pobytu, postrzegają wszystko bez rozróżniania. Wówczas wszystko w dziesięciu kierunkach Dharmadatu: ziemia, trawa, drzewa, płoty, dachówki czy też kamienie -- wszystko jest Buddhą i działaniem Buddhy, i kiedy otrzymuje się ukazanie objawienia zwykłych zjawisk w tak niezwykły sposób, wówczas wszystkie istoty jawią się w cudowny sposób jako ciało przemienienia Buddhy, cudownie ochraniane, intymnie ukazując stan Doskonałego Oświecenia. To jest cnota zasługi nie-czynienia, to jest cnota zasługi nie-stwarzania, to jest przebudzenie ku Oświeconemu Umysłowi -- Bodhicitta.


\bigskip
\noindent\textbf{4. ŚLUBOWANIE BODHICITTY I PRZYNOSZENIE POŻYTKU ISTOTOM}\par
\medskip


Przebudzenie Bodhicitty oznacza ślubowanie, że nie doświadczę wyzwolenia, zanim nie wyzwolę wszystkich odczuwających istot.  Czy to człowiek świecki czy mnich, czy istota niebiańska czy ludzka, czy istota cierpiąca czy czująca radość, powinna obudzić w swoim sercu pragnienie i ślubować: ,,Oby inni przede mną osiągnęli wyzwolenie''.


Kto przebudził w sobie taki umysł, to chociaż byłby pokornej powierzchowności, jest już nauczycielem i przewodnikiem wszystkich istot. Nawet mała siedmioletnia dziewczynka nagów może zostać nauczycielem czterech grup uczniów Buddhy i współczującym ojcem wszystkich istot; mężczyźni i kobiety są całkowicie równi -- to jest  subtelne i wspaniałe Prawo Drogi Buddhy. 


Jeśli przebudzi się Bodhicittę, to choć wędruje się poprzez sześć krain istnienia i cztery formy narodzin, to przyczyny tej wędrówki, wszystkie stają się praktyką ślubowania Bodhi. Dlatego, nawet jeśli do tej pory w obecnym życiu, na próżno marnotrawiłeś swój czas, powinieneś szybko uczynić to ślubowanie, gdy jest jeszcze czas. Jeśli nawet zdobyłeś wystarczającą zasługę, aby urzeczywistnić Stan Buddhy, powinieneś ofiarować ją i zwrócić wszystkim istotom, aby mogły urzeczywistnić Oświecenie. 


Są tacy, którzy przez niezliczone kalpy praktykowali, pomagając innym odczuwającym istotom, by pierwsze osiągnęły wyzwolenie, sami nie osiągając Stanu Buddhy, w ten sposób wyzwalając odczuwające istoty i odpowiadając na modlitwy istot. 


Istnieją cztery rodzaje Mądrości-Pradżni, które przynoszą pożytek odczuwającym istotom -- pierwsze: ofiary, drugie: miłujące słowa, trzecie: życzliwość i czwarte: utożsamienie. Wszystkie są praktyką ślubowania Bodhisattwy. 


Składanie ofiar oznacza niepożądanie. Chociaż prawdą jest, że w istocie jaźń-ja nie istnieje, to nie jest to przeszkodą  w praktykowaniu dawania ofiar. Nieznaczność ofiary nie budzi niechęci; to cnotliwy uczynek jest istotą tego. Dlatego powinno się ofiarowywać nawet jedno zdanie lub jeden wiersz Dharmy, ponieważ staje się to nasieniem dobra w obecnym i przyszłym życiu. Podobnie gdy daje się materialny skarb, powinniśmy praktykować ofiarowywanie nawet jeśli to będzie jedna moneta czy źdźbło trawy,  ponieważ jest to wypuszczaniem korzenia dobra w tym świecie i cierpiącym świecie.


Dharma jest również skarbem, skarb jest również Dharmą. Byli tacy, którzy nie szukając odpłaty dawali swoją pomoc i budowali łodzie i mosty do przeprawy jako ofiarowanie, tak jak są nimi zarabianie na życie i wytwarzanie dóbr.


Miłujące słowa mają taki sens, że gdy widzi się wszystkie istoty, w sercu budzi się do nich współczująca miłość i zwraca się do nich serdecznie. To znaczy traktuje się je tak, jakby były naszymi własnymi ukochanymi dziećmi, do których zwracamy się słowami pełnymi miłości. Cnotliwy powinien być chwalony, a nad pozbawionym cnót należy się litować. Miłujące słowa są pierwotnym źródłem przezwyciężenia gorzkiej nienawiści i ustanawiania zgody z innymi. Bezpośrednie słyszenie wypowiadanych słów rozjaśnia oblicze i raduje serce. Natomiast jeszcze głębsze wrażenie na ciele i duszy robi usłyszenie o miłujących słowach wypowiadanych o tobie podczas twej nieobecności. Powinieneś wiedzieć, że miłujące słowa mają moc poruszenia niebios.


Życzliwość oznacza szlachetne pomaganie odczuwającym istotom, wysokim i niskim, bez szczególnej intencji. Ten, kto pomógł bezradnemu żółwiowi lub zranionemu wróblowi -- nie oczekiwał żadnej nagrody za swoją pomoc. Każdy po prostu działał powodowany swoimi uczuciami życzliwości. Głupcy myślą, że jeśli postawią na pierwszym miejscu korzyść innych, to ucierpią na tym ich własne interesy. Jednakże mylą się. Życzliwość jest wszechobejmująca, przynosi w równym stopniu korzyść mnie, jak i innym.


Utożsamienie oznacza nie-rozróżnianie -- nie robienie żadnej różnicy między sobą a innymi. Przykładem jest Tathagata, który jako człowiek prowadził takie samo życie jak my, ludzkie istoty. Inni mogą być utożsamiani z ja, a jaźń-ja z innymi. Z upływem czasu ,,ja'' i ,,inni'' stają się jednym. Utożsamienie jest jak ocean, który nie odpycha żadnej wody, dlatego wszystkie zebrane wody mogą tworzyć ocean.


Powyższe praktyki są naprawdę praktyką ślubowania Bodhicitty, spokojnie zastanów się nad tym. Nie traktuj tego lekko. Czcij i oddaj szacunek ich zasłudze, która jest w stanie uratować wszystkie istoty, umożliwiając im przejście na drugi brzeg.


\bigskip
\noindent\textbf{5. UTRZYMANIE PRAKTYKI I~WDZIĘCZNOŚĆ}\par
\medskip

Przebudzenie się ku Umysłowi Bodhi jest głównie możliwe dla istot ludzkich, żyjących w tym świecie na południowym kontynencie Dżambudwipa. Teraz mamy taki związek przyczynowy. Dzięki ślubowaniu odrodziliśmy się w tym świecie cierpienia -- saha, czyż nie powinniśmy się radować z ujrzenia Siakiamuniego Buddhy?!


Spokojnie rozważ ten fakt, że gdyby to były czasy, gdy prawdziwa Dharma jeszcze nie rozprzestrzeniła się po świecie, zetknięcie z nią byłoby niemożliwe, nawet gdybyśmy chcieli poświęcić nasze życie, aby to uczynić. Dzisiaj powinniśmy modlić się i ślubować, by spotkać prawdziwą Dharmę. Słuchaj, co powiedział Buddha: ,,Kiedy spotkasz mistrza, który objaśnia Nieprześcignione Oświecenie, nie powinieneś patrzeć na jego urodzenie, na jego wygląd, nie powinieneś czuć niechęci do jego wad, ani myśleć o jego zachowaniu. Tylko z szacunku dla mądrości-pradżni z czcią skłaniaj się przed nim trzy razy dziennie i składaj ofiary, nie dając mu żadnego powodu do zmartwień''.


Teraz widzimy Buddhę i słyszymy Dharmę dzięki współczującemu błogosławieństwu, które przyszło do nas z praktyki Buddów i Patriarchów przekazanej twarzą w twarz. Gdyby nie bezpośredni przekaz Buddhów i Patriarchów, jak mogłoby ono dojść do nas dziś? 


Powinniśmy być wdzięczni nawet za jedno zdanie, powinniśmy być wdzięczni nawet za jedną naukę Dharmy, czyż nie powinniśmy być wdzięczni za wielkie błogosławieństwo Nieprześcignionej Wielkiej Dharmy Skarbnicy Oka Prawdziwego Prawa. Zraniony wróbel nie zapomniał okazanej mu życzliwości, nagradzając swojego dobroczyńcę trzema srebrnymi pierścieniami. Nie zapomniał też o tym bezradny żółw, który wynagrodził swojego dobroczyńcę pieczęcią Yüan-pu-t'ing. 


Jeśli nawet zwierzęta okazują swoją wdzięczność za okazaną im dobroć, jak mogą nie zrobić tego ludzkie istoty. Prawdziwą drogą wdzięczności jest codzienna praktyka. To znaczy, że nie powinniście zaniedbywać codziennie swojego życia ani trwonić czasu -- to jest praktyka. Czas pędzi szybciej niż strzała; życie jest bardziej ulotne niż rosa. Bez względu na to, jakie posiadasz dobre upaja, po raz drugi nie możesz przywrócić minionego dnia. Przeżyć sto lat bez celu to dni i miesiące pełne goryczy, to stanie się cierpiącym trupem. Nawet gdybyś przez sto lat był niewolnikiem swoich zmysłów, jeśli oddasz się praktyce nawet przez jeden dzień, nie tylko odzyskasz sto lat życia w tym świecie, ale uratujesz sto lat w następnym życiu. Należy doceniać każdy dzień życia; należy szanować ciało. 


Ta praktyka jest okazaniem miłości wobec ciała i umysłu i jest poszanowaniem siebie samego. W oparciu o naszą praktykę, zostaje ukazana i osiągnięta praktyka wszystkich Buddhów, i osiągamy Wielką Drogę wszystkich Buddhów. Dlatego każdy dzień praktyki jest nasieniem wszystkich Buddhów, praktyką wszystkich Buddhów. 


Wszyscy Buddhowie są Siakiamunim Buddhą. Siakiamuni Buddha znaczy, że ten umysł jest Buddhą, wszystkimi Buddhami przeszłości, teraźniejszości i przyszłości, którzy kiedy stają się Buddą, niezawodnie stają się Siakiamunim Buddhą. Takie jest znaczenie, że ten umysł jest Buddhą. Dokładnie musisz zbadać, o kim mówi to powiedzenie, że ten umysł jest Buddhą, wówczas prawdziwie okażesz wdzięczność za otrzymane błogosławieństwo Buddhy.

\end{Prayer}

%%%%%%%%%%%%%%%%%%%%%%%%%%%%%%%%%%%%%%%%%%%%%%%%%%%%%%%%%%%%%%%%%%%%%%%%%%%%%%%
%%%% FUKANZAZENGI
\begin{Prayer}{fukanzazengi}
	{FUKANZAZENGI}{-}
	{Powszechne Wyjaśnienie Zazen, Eihei D\=ogen Zenji}

\bigskip
\begin{JAPANESE}
TAZUNURU NI SORE D\=O MOTO EN Z\=U IKA DE KA SHUSH\=O O~KARAN, SH\=U J\=O
JI ZAI NAN ZO KU F\=U O~TSUIYA SAN. I~WAN YA ZEN TAI HARU KA NI JIN NAI
O~IZU, TARE KA HOSSHIKI NO SHU DAN O~SHIN ZEN, \=O YOSO T\=O JO O~HANAREZU,
A~NI SHU GY\=O NO KAYAKU T\=O O~MOCHI URU MONO NARAN YA. SHIKA RE DO MO
G\=O RI MO SA AREBA, TEN CHI HARUKA NI HEDATARI, I~JUN WAZU KA NI OKOREBA
FUN ZEN TO SHITE SHIN O~SHISSU. TA TO I~E NI HOKORI GO NI YUTA KA NI SHITE
BECCHI NO CHI TS\=U O~E, D\=O O~E, SHIN O~AKIRA ME TE SH\=O TEN NO SHII KI
O~KO SHI, NITT\=O NO HEN RY\=O NI SH\=O Y\=O SU TO IEDO MO, HOTONDO
SHUSSHIN NO KARRO O~KI KESSU. I~WAN YA KA NO GI ON NO SH\=O CHI TA RU, TAN
ZA ROKU NEN NO SH\=O SEKI MI TSU BE SHI, SH\=O RIN NO, SHIN IN O~TSUTAURU,
MEN PEKI KU SAI NO SEI MEI NAO KI KO YU, KO SH\=O SUDE NI SHIKARI, KON JIN
NAN ZO BEN ZE ZA RU. YU E NI SUBEKARAKU KOTO O~TAZUNE GO O~\=O NO GE GY\=O
O~KY\=U SU BE SHI. SUBEKARAKU E K\=O HEN SH\=O NO TAI HO O~GAKU SU BE SHI.
SHIN JIN JI NEN NI DARRAKU SHITE HON RAI NO MEN MO KU GEN ZEN SEN. IN MO NO
JI O~EN TO HOSSEBA KY\=U NI IN MO NO JI O~TSUTO ME YO. SORE SAN ZEN WA J\=O
SHITSU YORO SHI KU ON JIKI SETSU ARI. SHO EN O~H\=O SHA SHI, BAN JI O~KY\=U
SO KU SHI TE ZEN NAKU O~OMO WA ZU ZE HI O~KAN SURU KOTO NAKARE. SHIN
I~SHIKI NO UN TEN O~YA ME, NEN S\=O KAN NO SHIKI RY\=O O~YA ME TE SA BUTTO
HAKARU KOTO NAKARE, A~NI ZA GA NI KAKA WARAN YA. YONO TSUNE ZA SHO NI WA
ATSU KU ZA MOTTO SHIKI, UE NI FUTON O~MOCHIU, ARUI WA KEKKA FUZA, ARUI WA
HANKA FUZA, IWAKU KEKKA FUZA WA MAZU MIGI NO ASHI O~MOTTE HIDARI NO MOMO NO
UE NI ANJI, HIDARI NO ASHI O~MIGI NO MOMO NO UE NI ANZU. HANKA FUZA WA TADA
HIDARI NO ASHI O~MOTTE MIGI NO MOMOO OSUNARI, YURUKU E TAI O~KAKETE SEI SEI
NA RA SHI MU BE SHI. TSUGI NI MIGI NO TE O~HIDARI NO ASHI NO UE NI ANJI,
HIDARI NO TANAGOKORO O~MIGI NO TANAGOKORO NO UE NI ANJI, RY\=O NO DAI BO
SHI MUKAITE AI SASOU SUNAWACHI SH\=O SHIN TAN ZA SHITE, HIDARI NI SOBADACHI
MIGI NI KATAMUKI, MAE NI KUGUMARI SHIRIE NI AOGU KOTO O~EZARE, MIMI TO KATA
TO TAI SHI HANA TO HOZO TO TAI SE SHI MEN KOTO O~Y\=O SU. SHITA, UE NO
AGITO NI KAKETE SHIN SHI AI TSUKE, ME WA SUBEKARAKU TSUNE NI HIRA KU BE
SHI, BI SOKU KASU KA NI TS\=U JI SHIN S\=O SUDE NI TOTONO E TE KAN KI
ISSOKU SHI, SA Y\=U Y\=O SHIN SHITE GOTSU GOTSU TO SHITE ZA J\=O SHITE KO
NO FU SHI RY\=O TEI O~SHI RY\=O SE YO. FU SHI RY\=O TEI I~KAN GA SHI RY\=O
SEN, HI SHI RY\=O, KORE SUNAWACHI ZA ZEN NO Y\=O JUTSU NARI. IWA YURU ZA
ZEN WA SH\=U ZEN NI WA ARAZU, TADA KORE AN RAKU NO H\=O MON NARI, BO DAI
O~G\=U JIN SURU NO SHU SH\=O NARI, K\=O AN GEN J\=O, RA R\=O IMADA ITARAZU,
MO SHI KO NO I~O E BA RY\=U NO MIZU O~URU GA GOTOKU TORA NO YAMA NI YO RU
NI NITARI, MASA NI SHIRUBESHI SH\=O B\=O ONOZUKARA GEN ZEN SHI, KON SAN MA
ZU BOKU RAKU SURU KOTO O, MO SHI ZA YO RI TA TA BA JO JO TO SHITE MI O~UGO
KA SHI, AN SH\=O TO SHITE TATSUBESHI. SOTSU B\=O NA RU BE KA RA ZU, KATTE
MI RU CH\=O BON OSSH\=O, ZA DATSU RY\=U B\=O MO KO NO CHIKARA NI ICHI NIN
SURU KOTO O. IWAN YA MATA SHI KAN SHIN TSUI O~NEN ZU RU NO TEN KI, HOKKEN
B\=O KATTO KO SURU NO SH\=O KAI MO, IMADA KORE SHI RY\=O FUN BETSU NO YO KU
GE SURU TOKORO NI ARAZU, A~NI JIN Z\=U SHU SH\=O NO YO KU SHI RU TOKORO TO
SEN YA. SH\=O SHIKI NO HOKA NO II GI TA RU BE SHI, NAN ZO CHI KEN NO SAKI
NO KI SOKU NI ARA ZA RU MONO NARAN YA. SHIKA RE BA SUNAWACHI J\=O CHI KA GU
O~RON ZE ZU, RI JIN DON SHA O~ERABU KOTO NAKARE. SEN ITSU NI KU F\=U SE BA
MASA NI KORE BEN D\=O NARI. SHU SH\=O ONOZUKARA ZEN NA SE ZU, SHU K\=O SARA
NI KORE BY\=O J\=O NARU MONO NARI. OYOSO SORE JI KAI TA H\=O, SAI TEN T\=O
CHI, HITO SHI KU BUCCHIN O~JI SHI MOPPARA SH\=U F\=U O~HOSHIIMAMA NI SU,
TADA TA ZA O~TSUTOMETE GOCCHI NI SA E RA RU, MAN BETSU SEN SHA TO IU TO
IEDO MO, SHI KAN NI SAN ZEN BEN D\=O SU BE SHI, NAN ZO JI KE NO ZA J\=O
O~B\=O KYAKU SHITE MIDARI NI TA KOKU NO JIN KY\=O NI KYO RAI SEN. MO SHI
IPPO O~AYAMAREBA T\=O MEN NI SHA KA SU. SUDE NI NIN SHIN NO KI Y\=O
O~ETARI, MUNASHIKU K\=O IN O~WATARU KOTO NAKARE, BUTSU D\=O NO Y\=O KI O~HO
NIN SU. TARE KA MIDARI NI SEKKA O~TANOSHI MAN, SHIKANO MINARAZU, GY\=O
SHITTA S\=O RO NO GOTOKU, UN MEI WA DEN K\=O NI NITARI. SHUKU KOTSU TO
SHITE SUNAWACHI K\=U JI SHU YU NI SUNAWACHI SHISSU. KOINEGAWAKU WA SORE SAN
GAKU NO K\=O RU, HISA SHI KU MO Z\=O NI NA RATTE SHIN RY\=U O~AYA SHI MU
KOTO NAKARE, JIKI SHI TAN TEKI NO D\=O NI SH\=O JIN SHI, ZETSU GAKU MU I~NO
HITO O~SON KI SHI, BUTSU BUTSU NO BO DAI NI GATT\=O SHI SO SO NO ZAN MAI
O~TEKI SHI SE YO. HISA SHI KU IN MO NA RU KOTO O~NA SA BA SUBEKARAKU KORE
IN MO NA RU BE SHI, H\=O Z\=O ONOZUKARA HIRAKETE JU Y\=O NYO I~NARAN.
\end{JAPANESE}
\end{Prayer}

%%%%%%%%%%%%%%%%%%%%%%%%%%%%%%%%%%%%%%%%%%%%%%%%%%%%%%%%%%%%%%%%%%%%%%%%%%%%%%%
%%%% POWSZECHNE WYJAŚNIENIE ZAZEN
\begin{Prayer}{ogolne_zalecenia}
	{POWSZECHNE WYJAŚNIENIE ZAZEN}{-}
	{Fukanzazengi, Eihei D\=ogen Zenji}

\bigskip
U samego źródła Droga jest pierwotnie doskonała i~wszystko ogarniająca, po
co więc tymczasowa praktyka i~oświecenie? Istota pierwotnej prawdy jest
całkowicie wolna, dlaczego więc tracić czas na praktykę?


Co więcej, wszystko jest całkowicie poza kurzem świata, któż mógłby uwierzyć
w~sposoby wymiatania go? Wielka stolica nie jest oddzielona od tego miejsca,
po co więc używać stóp i~głowy praktyki?


A jednak, gdy pomylisz się o~najcieńszy włos, pojawi się przepaść jak między
niebem i~ziemią. Jeśli pojawi się najmniejszy błąd, utracisz umysł, który
wpadnie w~pomieszanie. Możesz być dumny ze swojego rozumienia i~mieć
głębokie urzeczywistnienie lub zdobyć nieprzeciętną mądrość i~osiągnąć
Drogę, rozjaśnić swój umysł i~rozwinąć wielkiego ducha determinacji, jednak
jeśli pozwalasz sobie na kręcenie się wkoło na samym początku, stracisz
ścieżkę wyzwolenia.


Powinieneś podążać za przykładem z~Dżetawana, musisz spojrzeć na sześć lat
siedzenia w~prostej pozycji, a~ten, który przekazał pieczęć umysłu Szaolin,
siedział przez dziewięć lat twarzą do ściany i~wciąż jeszcze jest to słynne.
Jeśli tacy byli starożytni święci mistrzowie, to dlaczego dzisiaj ludzie nie
praktykują?


Z tego powodu zaprzestań prób zrozumienia słów i~wypowiedzi. Naucz się
zwracać i~kierować promień światła do wewnątrz. Twoje ciało i~umysł odpadną
same z~siebie i~ukaże się twoja pierwotna twarz. Jeśli chcesz to osiągnąć,
zastosuj to od razu.


Do zazen potrzebny jest spokojny pokój. Jedź i~pij w~umiarkowany sposób.
Pozostaw wszystkie związki i~daj spokój miriadom rzeczy i~spraw. Nie myśl
ani dobrze, ani źle. Nie wolno ci oceniać niczego jako prawdziwe lub
fałszywe. Zaprzestań sterowania umysłem, intelektem czy świadomością,
zaprzestań badania myślą, wyobrażeniem czy obserwacją. Nie próbuj nawet stać
się buddhą. Jak bycie buddhą mogłoby ograniczać się do siedzenia czy
leżenia?


Zazwyczaj w~miejscu do siedzenia rozkłada się grubą matę, a~na niej kładzie
się futon. Usiądź w~pozycji pełnego lotosu lub pół-lotosu.


W pozycji pełnego lotosu połóż prawą stopę na lewym udzie, a~następnie lewą
stopę na prawym udzie. W~pozycji pół-lotosu połóż lewą stopę na prawym
udzie. Ubranie i~pas zawiąż luźno i~ułóż je w~schludny sposób. Następnie
połóż prawą dłoń na wierzchu lewej stopy, a~lewą dłoń na wierzchu prawej
dłoni, tak aby końce kciuków stykały się delikatnie.


Siedź wyprostowany we właściwej pozycji ciała, nie przechylając się w~prawo
czy w~lewo ani nie zginając się w~przód lub do tyłu. Uszy powinny być równo
z~barkami, a~nos równo z~pępkiem. Oprzyj język na podniebieniu, zęby i~wargi
powinny być zamknięte. Powinieneś trzymać otwarte oczy. Oddychaj spokojnie
przez nos.


Kiedy właściwie ułożysz ciało, zrób jeden oddech ,,ziewnięcia energii''.
Następnie zakołysz ciałem w~lewo i~w prawo. Ze spokojem siedź nieporuszenie,
myśl głębię nie-myśli. Jak myśleć głębię nie-myśli? Nie-myśl. To jest
tajemnica sztuki zazen.


Zazen, o~którym mówię, to nie jest nauka medytacji. Jest to brama Dharmy
spokoju i~radości. Jest to praktyka i~całkowite urzeczywistnienie
oświecenia. To jest zamanifestowany i~urzeczywistniony koan, pułapki i~pęta
nigdy nie mogą tego dosięgnąć. Jeśli to osiągniesz, staniesz się podobny do
smoka w~wodzie lub do tygrysa odpoczywającego na górze. Prawdziwa Dharma
sama manifestuje się przed tobą, a~zaciemnienia i~rozproszenie zostają
odcięte.


Gdy wstajesz z~siedzenia, poruszaj się spokojnie i~powoli i~wstawaj powoli,
nie wolno robić tego nagle i~agresywnie. Patrząc w~przeszłość, widzimy, że
przekroczenie tego, co zwykłe i~święte, oraz umieranie siedząc czy stojąc
zależy od powierzenia się mocy zazen.


Ponadto doprowadzenie do punktu zwrotnego z~użyciem palca, masztu, igły lub
młotka lub spowodowanie oświecenia za pomocą hossu, pięści, kija i~krzyku
nigdy nie może być zrozumiane rozróżniającą myślą. Jak można by w~ogóle
zrozumieć praktykę i~oświecenie za pomocą nadnaturalnych mocy?! Jest to
majestat poza dźwiękiem i~kolorem. Czyż nie jest to tym, co jest przed
pojawieniem się wszelkiej wiedzy i~poglądów?


Nie zajmuj się mądrością, która jest powyżej, ani głupotą, która jest
poniżej. Nie wolno ci porównywać inteligentnych ludzi z~tępymi. Jeśli
będziesz praktykował z~oddaniem, będzie to prawidłową praktyką Drogi.
Praktyka i~oświecenie nie są skalane~-- są codzienną rzeczą.


W tym świecie i~w innych światach, w~Indii i~w Chinach, utrzymywano pieczęć
Buddhy i~każda szkoła ma swoją własną tradycję. Praktykuj tylko siedzenie,
pozostając w~nieporuszonej determinacji. Mimo, iż możesz znaleźć się
w~miriadach różnych okoliczności, z~całym sercem praktykuj sanzen oddając
się Drodze. Po cóż porzucać swoje miejsce do siedzenia we własnym domu
i~włóczyć się bezcelowo w~kurzu odległych krain? Gdy raz zrobisz błędny
krok, potkniesz się o~to, co jest przed tobą i~przejdziesz obok.


Osiągnąłeś cenne ludzkie ciało, nie marnuj mijających nocy i~dni. Chroniąc
esencję Drogi Buddhy, kto mógłby radować się zabawą wywoływania iskier
z~krzemienia? Natura formy jest jak rosa na trawie, szczęście życia jest jak
światło błyskawicy. Jej światło w~jednej chwili staje się puste, w~jednej
chwili znika bez śladu.


Szlachetni praktykujący, modlę się i~proszę~-- przez długi czas uczyliście
się kopii słonia, ale nie wątpcie w~prawdziwego smoka. Bezpośrednio zróbcie
postęp na Drodze, czcijcie człowieka, który jest poza uczeniem się
i~praktykuje nie-czynienie, wejdźcie w~buty oświecenia Buddhów i~stańcie się
spadkobiercami samadhi patriarchów. Jeśli będziecie praktykowali w~ten
sposób cały czas, to z~pewnością staniecie się tacy, drogocenny skarbiec sam
się otworzy i~użyjecie klejnotu spełniającego życzenia.
\end{Prayer}


%%%%%%%%%%%%%%%%%%%%%%%%%%%%%%%%%%%%%%%%%%%%%%%%%%%%%%%%%%%%%%%%%%%%%%%%%%%%%%%
%%%% STROFY WIARY W~UMYSŁ
\begin{Prayer}{strofy_wiary_w_umysl}
	{STROFY WIARY W~UMYSŁ}{-}
	{Shin Jin Mei (chin. Hsin Hsin Ming), Sosan Ganchi Zenji}

\begin{Verse}
\let\swskip\smallskip
\bigskip Ta Wielka Droga wcale nie jest trudna dla ludzi wolnych od swych upodobań.\\
\swskip Kiedy znikają chęci i~niechęci, Droga jest jasna, nic nie jest ukryte.\\
\swskip Lecz najdrobniejsze nawet rozróżnienie ziemię i~niebo oddzieli od siebie.\\
\swskip Jeżeli pragniesz jasno ujrzeć prawdę, porzuć to wszystko, co jest za lub przeciw.\\
\swskip Ciągłe wpadanie w~swe ,,lubię'', ,,nie lubię'' jest chorobliwym nawykiem umysłu.\\
\swskip Niedostrzeganie głębi prawdy Drogi wrodzony spokój zakłóca i~burzy.\\
\swskip Droga jest pełna jak bezmiar przestrzeni, gdzie nie ma braku i~nie ma nadmiaru.\\
\swskip Gdy wybieramy albo odrzucamy, zapominamy o~tej prostej prawdzie.\\
\swskip Oba dążenia~-- na zewnątrz do świata, jak i~dążenie ku wewnętrznej pustce -- na pomieszane skazują nas życie.\\
\swskip W~spokoju ujrzyj, że wszystko jest Jednym, a~błędne myśli znikną same z~siebie.\\
\swskip Powstrzymywanie siebie od działania jest też działaniem, które cię wypełnia.\\
\swskip Jeżeli będziesz wciąż trwać w~rozdwojeniu, nigdy jedności doświadczyć nie zdołasz.\\
\swskip Jedności owej nieuprzytomnienie sprawi, że zabrniesz w~odmęty sprzeczności.\\
\swskip Jeśli przyjmujesz, że istnieją rzeczy, to gubisz prawdę ich rzeczywistości,\\
\swskip lecz zakładanie nieistnienia rzeczy również w~niezgodzie jest z~rzeczywistością.\\
\swskip Im więcej o~tym mówisz i~rozmyślasz, tym od istoty oddalasz się bardziej.\\
\swskip Odetnij słowa i~zbyteczne myśli, a~cały wszechświat będzie twoim domem.\\
\swskip Jeśli powrócisz do korzenia zjawisk, odkryjesz sedno znaczenia wszystkiego.\\
\swskip Jeżeli ścigać zaś będziesz pozory, pierwotne źródło przeoczysz z~pewnością.\\
\swskip Budzić się, znaczy przekraczać natychmiast zarówno pustkę, jak to co jest formą.\\
\swskip Wszelkie przemiany w~naszym pustym świecie rzeczywistymi jawią się z~niewiedzy.\\
\swskip Zamiast szalonej pogoni za prawdą, porzuć poglądy, które tak miłujesz.\\
\swskip Nie pozostawaj dłużej w~rozdwojeniu, powstrzymaj siebie od wszelkiej gonitwy.\\
\swskip Jeśli ślad dobra lub zła jest obecny, Umysł się gubi i~traci swą jasność.\\
\swskip Dwoistość płynie z~Jednego Umysłu, ale i~jego nie próbuj się chwytać.\\
\swskip Gdy Jeden Umysł trwa nieporuszony, nic na tym świecie zranić cię nie może.\\
\swskip A~gdy nic nie jest w~stanie cię urazić, Wszystkie przeszkody po prostu znikają.\\
\swskip Kiedy zaś znikną przedmioty twych myśli, myślący podmiot zginie razem z~nimi.\\
\swskip Rzeczy się jawią za sprawą umysłu, za sprawą rzeczy umysł się pojawia.\\
\swskip Oba te światy, jak widzisz, są względne, i~ten, i~tamten Pustką jest u~źródła.\\
\swskip A~chociaż w~Pustce nie różnią się wcale, w~każdym z~osobna wszelki kształt się mieści.\\
\swskip Gdy nie osądzasz: ,,to niskie'', ,,to wzniosłe'', to czyją wówczas w~sporach bierzesz stronę?\\
\swskip Ta Wielka Droga nie ma żadnych granic, i~przekracza łatwość, przekracza trudności.\\
\swskip Ci, co poglądów wąskich się trzymają, są bojaźliwi, niezdecydowani; własny szalony pośpiech ich hamuje.\\
\swskip Natychmiast porzuć chwytający umysł, a~wszystko stanie się po prostu sobą. W~istocie nic nie ginie ani nie trwa.\\
\swskip Wejrzyj w~prawdziwą naturę wszystkiego, a~z Wielką Drogą będziesz wówczas w~zgodzie, idąc swobodnie i~bez niepokoju.\\
\swskip Lecz jeśli będziesz żyć w~niewoli myśli, wpadniesz w~niepewność, w~zamęt niejasności.\\
\swskip W~dół cię pociągnie to ogromne brzemię, czemu osądzasz: to złe, a~to dobre?\\
\swskip Jeśli najwyższą chcesz podążać Drogą, to nie odrzucaj świata swoich zmysłów,\\
\swskip ponieważ taki~-- pełny, całkowity~-- świat twoich zmysłów też jest oświeceniem.\\
\swskip Mędrzec nie dąży do żadnego celu, bo tylko głupiec sam siebie krępuje.\\
\swskip Ta Jedna Droga nie zna żadnych różnic, a~człowiek zawsze chce się czegoś trzymać.\\
\swskip Gdy myślą szukasz Wielkiego Umysłu, marnujesz siły w~próżnym dociekaniu.\\
\swskip Ten mały umysł rodzi ruch i~bezruch, lecz przebudzony przekracza je oba.\\
\swskip Nasza ułuda źródłem rozdwojenia~-- te sny są niczym, to kwiaty z~powietrza -- czemu tak bardzo starasz się je chwytać?\\
\swskip Więc raz na zawsze pozbądź się złudzenia: zysku i~straty, dobra i~zła także.\\
\swskip Gdy nie śpisz więcej i~twój umysł czuwa, to sny mamiące nikną same z~siebie.\\
\swskip Jeżeli umysł niczego nie dzieli, to w~tej Jedności wszystko jest już sobą.\\
\swskip Powrót do tego tajemnego Źródła czyni cię wolnym od wszelkich uwikłań.\\
\swskip Gdy wszystko widzisz ,,umysłem równości'', powracasz do swej prawdziwej istoty.\\
\swskip Ten Jeden Umysł przekracza od razu wszelki argument, wszelkie porównanie.\\
\swskip Dążysz do ruchu~-- efektem jest bezruch, dążysz do spokoju, a~rodzi się zamęt.\\
\swskip Gdy spokój i~zamęt swój kres osiągają, to znika wówczas nawet Jeden Umysł.\\
\swskip Oto jest prawda ponad wszelkim prawem, nie sposób nawet opisać jej w~słowach.\\
\swskip Gdy umysł w~pełni jednoczy się z~Drogą, egocentryczne dążenia znikają; odchodzą wszelkie zwątpienia i~zamęt -- prawdziwa wiara płonie w~naszym życiu.\\
\swskip Nic do nas nie lgnie, nic nas nie hamuje, nie ma potrzeby niczego odrzucać.\\
\swskip Wszystko jest jasne, ujawnia się samo bez wytężania sił tego umysłu.\\
\swskip Myśl nie dosięgnie tej doniosłej prawdy i~na nic tutaj zdadzą się uczucia.\\
\swskip W~tym najprawdziwszym, czystym świecie Pustki nie ma już ,,siebie'' ani żadnych ,,innych''.\\
\swskip By się zjednoczyć z~tą rzeczywistością, doświadcz natychmiast prawdy tego ,,nie-dwa''.\\
\swskip W~tym właśnie ,,nie-dwa'' wszystko jest Jednością, nie ma podziału i~nic nie jest poza.\\
\swskip We wszystkich miejscach i~o każdej porze mędrcy się budzą ku tej właśnie prawdzie.\\
\swskip Droga jest poza przestrzenią i~czasem, jedna jej chwila równa jest wieczności.\\
\swskip Prawda jest wszędzie przed twymi oczami, nie tylko tutaj, nie tylko gdzieś indziej.\\
\swskip I~rozróżnienia: ,,to duże'', ,,to małe'' w~żaden już sposób zmylić cię nie mogą.\\
\swskip To co największe jest również najmniejsze, bo ograniczeń nie ma tu zupełnie.\\
\swskip To co się jawi, nie istnieje wcale, to czego nie ma, jest zawsze obecne -- jeżeli tego nie pojmujesz jeszcze, daleko jesteś od wewnętrznej prawdy.\\
\swskip Jedno jest wszystkim, a~wszystko jest jednym, w~tym się przejawia doskonałość rzeczy.\\
\swskip A~kiedy wiara i~Umysł są jednym, słowa i~myśli tego nie obejmą, nie ma tu bowiem ani wczoraj, ani jutra, ani dziś.\\
\end{Verse}
\end{Prayer}

%%%%%%%%%%%%%%%%%%%%%%%%%%%%%%%%%%%%%%%%%%%%%%%%%%%%%%%%%%%%%%%%%%%%%%%%%%%%%%%
%%%% HAKUIN ZENJI ZAZEN WASAN
\begin{Prayer}{hakuin_zenji_zazen_wasan}
	{HAKUIN ZENJI ZAZEN WASAN}{-}
	{Mistrza Hakuina Pieśń ku Chwale Zazen}

\begin{Verse}
\wersaliki
\stanza{
	SHUJ\=O HONRAI HOTOKE NARI\\
	MIZU TO K\=ORI NO GOTOKU NITE\\
	MIZU O~HANARETE K\=ORI NAKU\\
	SHUJ\=O NO HOKA NI HOTOKE NASHI
}

\stanza{
	SHUJ\=O CHIKAKI O~SHIRAZU SHITE\\
	T\=OKU MOTOMURU HAKANASA YO\\
	TATOEBA MIZU NO NAKA NI ITE\\
	KATSU O~SAKEBU GA GOTOKU NARI
}

\stanza{
	CH\=OJA NO IE NO KO TO NARITE\\
	HINRI NI MAYOU NI KOTONARAZU\\
	ROKUSHU RINNE NO INNEN WA\\
	ONORE GA GUCHI NO YAMIJI NARI
}

\stanza{
	YAMIJI NI YAMIJI O~FUMISOETE\\
	ITSUKA SH\=OJI O~HANARU BEKI\\
	SORE MAKAEN NO ZENJ\=O WA\\
	SH\=OTAN SURU NI AMARI ARI
}

\stanza{
	FUSE YA JIKAI NO SHOHARAMITSU\\
	NENBUTSU ZANGE SHUGY\=O T\=O\\
	SONO SHINA \=OKI SHOZENGY\=O\\
	MINA KONO UCHI NI KISURU NARI
}

\stanza{
	ICHIZA NO K\=O O~NASU HITO MO\\
	TSUMISHI MURY\=O NO TSUMI HOROBU\\
	AKUSHU IZUKU NI ARINU BEKI\\
	J\=ODO SUNAWACHI T\=OKARAZU
}

\stanza{
	KATAJIKENAKUMO KONO NORI O\\
	HITOTABI MIMI NI FURURU TOKI\\
	SANTAN ZUIKI SURU HITO WA\\
	FUKU O~URU KOTO KAGIRI NASHI
}

\stanza{
	IWANYA MIZUKARA EK\=O SHITE\\
	JIKI NI JISH\=O O~SH\=O SUREBA\\
	JISH\=O SUNAWACHI MUSH\=O NITE\\
	SUDENI KERON O~HANARETARI
}

\stanza{
	INGA ICHINYO NO MON HIRAKE\\
	MUNI MUSAN NO MICHI NAOSHI\\
	MUS\=O NO S\=O O~S\=O TO SHITE\\
	YUKUMO KAERUMO YOSO NARAZU
}

\stanza{
	MUNEN NO NEN O~NEN TO SHITE\\
	UTAU MO MAU MO NORI NO KOE\\
	ZANMAI MUGE NO SORA HIROKU\\
	SHICHI ENMY\=O NO TSUKI SAEN
}

\stanza{
	KONO TOKI NANI O~KA MOTOMU BEKI\\
	JAKUMETSU GENZEN SURU YUE NI\\
	T\=OSHO SUNAWACHI RENGEKOKU\\
	KONO MI SUNAWACHI HOTOKE NARI
}
\end{Verse}
\end{Prayer}

%%%%%%%%%%%%%%%%%%%%%%%%%%%%%%%%%%%%%%%%%%%%%%%%%%%%%%%%%%%%%%%%%%%%%%%%%%%%%%%
%%%% MISTRZA HAKUINA PIEŚŃ KU CHWALE ZAZEN
\begin{Prayer}{mistrza_hakuina_piesn}
	{MISTRZA HAKUINA\\PIEŚŃ KU CHWALE ZAZEN}{MISTRZA HAKUINA PIEŚŃ KU CHWALE ZAZEN}
	{Hakuin Zenji Zazen Wasan}

\begin{Verse}
	Pierwotnie, od samego początku, żyjące istoty są Buddhą.\\
	Podobnie jak woda i~lód,\\
	Bez wody nie ma lodu,\\
	Poza żyjącymi istotami nie ma Buddhy.\\
	Żywe istoty nie wiedzą, że to jest tak blisko\\
	I szukają tego daleko,\\
	Jakby będąc w~wodzie\\
	Krzyczeli z~pragnienia.\\
	Podobni synowi bogacza,\\
	Który błądzi pośród biednych.\\
	Przyczyną odradzania się w~sześciu światach\\
	Jest jaźń na mrocznej drodze głupoty i~niewiedzy.\\
	Wędrując z~ciemnej ścieżki na ciemną ścieżkę,\\
	Kiedy uwolnimy się od narodzin i~śmierci?\\
	Co do Samadhi Mahajany\\
	Nie ma słów na wyrażenie jego pochwały.\\
	Dana, Sila i~wszystkie Paramity,\\
	Praktyka Nenbutsu i~skruchy,\\
	Niezliczone dobre praktyki,\\
	Wszystkie powracają do tego.\\
	Osoba, która raz usiadła [w zazen]\\
	Zniszczy niezliczone złe czyny.\\
	Jak mogą istnieć [wówczas] złe ścieżki?\\
	Czysta Kraina nie jest daleko stąd,\\
	Jeśli, ktoś raz usłyszy\\
	Z wdzięcznością tą Dharmę,\\
	Wychwala ją i~czci,\\
	Taka osoba osiągnie nieograniczone szczęście.\\
	Jeszcze bardziej ktoś, kto zwraca się do wewnątrz,\\
	I sam bezpośrednio potwierdzi Prawdziwą Naturę,\\
	I że własna natura jest Nie-Naturą,\\
	Od razu porzuci głupie dyskusje.\\
	Wówczas otworzy się brama jedności przyczyny i~skutku,\\
	Nie dwie i~nie trzy, prosto biegnie Droga.\\
	Biorąc jako formę, formę nie-formy,\\
	Kiedy odchodzimy i~powracamy nie opuszczamy tego miejsca.\\
	Biorąc jako myśl, myśl nie-myśli,\\
	Śpiew i~taniec są dźwiękiem Dharmy.\\
	Bezkresne jest niebo nieograniczonego Samadhi,\\
	Doskonale jasny jest księżyc Czterech Mądrości.\\
	W tym momencie czegóż możemy szukać?\\
	Nirwana w~jasny sposób pojawia się przed oczyma\\
	I to miejsce jest Krainą Lotosu,\\
	A to ciało jest Buddhą.\\
\end{Verse}
\end{Prayer}

%%%%%%%%%%%%%%%%%%%%%%%%%%%%%%%%%%%%%%%%%%%%%%%%%%%%%%%%%%%%%%%%%%%%%%%%%%%%%%%
%%%% DAI BU CHIN BAN NAN SH\=U REN NEN JIN SH\=U 
%%%% Daikan
\begin{Prayer}{-}
	{DAI BU CHIN BAN NAN SH\=U REN NEN JIN SH\=U\\RYOGON SHU}{DAI BU CHIN BAN NAN SH\=U REN NEN JIN SH\=U}
	{-}

\begin{JAPANESE}
{\bf NAMU REN NEN UI JO.}\\
JI HO ZO.\\
REN NEN UI JO. JI HO ZO.\\
{\scriptsize(3 x)}\\
{\bf NA MU SA TAN DO.}\\


SU GYA T\=O Y\=A. ORA K\=O CH\=I. SAMYAS\=A. FUDO SH\=A. SATA D\=O. FUDO KY\=U
SH\=I SHUNISAN. NAMU S\=A B\=O. FUDO F\=U CH\=I. SATO B\=I BY\=A. NAMU SATO
NAN. SAMYA SAFUD\=O. KYUSHI NAN. SOJA RABO GY\=A. SUGYA NAN. NAMU RY\=O K\=I.
ORA KATO NAN. NAMU S\=U RY\=O TOBONO NAN. NAMU SOGE RIT\=O. GYAMI NAN. NAMU
RY\=O K\=I SAMYA GYATO NAN. SAMYA GYAHO R\=A. CHIBO TONO NAN. NAMU CH\=I B\=O
RISHU NAN. NAMU SHIDO YA. BICHI Y\=A. TORA RISHU NAN. SH\=A H\=O N\=O. KERA
K\=O SOK\=O SORA MOTO NAN. NAMU HORA KOMO N\=I. NAMU IN T\=O R\=A Y\=A. NAMU
BOGYA B\=O CH\=I. RY\=O T\=O R\=A Y\=A. UM\=O H\=O CH\=I. S\=O K\=I Y\=A Y\=A.
NAM BOGYA B\=OCH\=I. NORA YANO Y\=A. HOJA M\=O K\=O. SAMO T\=O R\=A. NAMU SHIGE
RITO Y\=A. NAMU BO GYA B\=O CH\=I. MOKO KYARA Y\=A. CHIRI HORA N\=O KY\=A R\=A.
BIDO R\=A HON\=O KYARA Y\=A. OCHI M\=O CH\=I. SHIMO SHANO N\=I. HOSHI N\=I.
MOTO RIKYA N\=O. NAMU SHIGE RITO Y\=A. NAMBOGYA B\=O CH\=I. TOTO GYAT\=O. KYURA
Y\=A. NAMU HOCHI M\=O KYURA Y\=A. NAMU HOJA R\=A KYURA Y\=A. NAMU M\=O N\=I
KYURA Y\=A. NAMU KYASHA KYURA Y\=A. NAMU BOGYA B\=O CH\=I. CHIRI S\=A SHURA
SHIN\=O. HORA KOR\=A NORA SHAY\=A. TOTO GYATOY\=A. NAMU B\=O GY\=A B\=O CH\=I.
NAMU OMI TOBO Y\=A. TOTO GYATO Y\=A. ORA K\=O CH\=I. SAMYA S\=A FUDO Y\=A. NAMU
B\=O GY\=A B\=O CH\=I. \=O S\=U BIY\=A. TOTO GYATO Y\=A. ORA K\=O CH\=I. SAMYA
S\=A FUDO Y\=A. NAMU B\=O GY\=A B\=O CH\=I. B\=I SH\=A JAYA. KYURY\=O BISHU
RIY\=A. HOR\=A HORA SHAY\=A. TOTO GYATO Y\=A. NAMU B\=O GYA B\=O CH\=I. SAN
BUS\=U BID\=O. SAREN NORARASHA Y\=A. TOTO GYATO Y\=A. ORA K\=O CH\=I. SAMYA
S\=A FUDO Y\=A. NAMU B\=O GY\=A B\=O CH\=I. SHAKI Y\=A MONO \=E. TOTO GYATO
Y\=A ORA K\=O CH\=I. SAMYA S\=A FUDO Y\=A. NAMU B\=O GY\=A B\=O CH\=I. RATA
N\=O KITSU RASHA Y\=A. TOTO GYATO Y\=A. ORA K\=O CH\=I. SAMYA S\=A FUDO Y\=A.
CHIBYA N\=A M\=U SOGE RIT\=O. EI TAN BOGYA B\=O T\=O. SATA D\=O KYATSU SHUNI
SAN. SATA D\=O HODO R\=A. NAMU OHO R\=A SHITAN. HORA. CHIY\=O KIR\=A. SARA B\=O
FUDO KERA K\=O NIKERA K\=O. KEGYA RAKO N\=I. HORA BICHI Y\=A. SHIDO N\=I. OKYA
R\=A. MIRI SH\=U. HORI TORA Y\=A NI KE R\=I. SARA B\=O HODO N\=O. MOSHA N\=I.
SARA B\=O. TOSHU S\=A TOSHI HAN. HONO N\=I HORA N\=I. SH\=A TS\=U R\=A SHICHI
NAN. KERA K\=O SOKO SORA SHAJ\=A. BIDO BEN SANO K\=E R\=I. OSHU S\=A B\=I
SHACHI NAN. NOSHA J\=A TORA SHAJ\=A. HOR\=A SATO NOKE R\=I. OSHU S\=A NAN.
MOK\=O KERA KOSHA J\=A. BIDO BEN SANO K\=E R\=I SABO SH\=A TSURIO N\=I HORA
SHAJ\=A. KORA TOSHI HAN NOSHA NOSHA N\=I. BISHA J\=A SHIDO R\=A. OKI N\=I UT\=O
KYARA SHAJ\=A. OHO R\=A SHIDO KYUR\=A. MOKO HORA SEN SH\=I. MOKO TECH\=O. MOKO
CHISH\=A. MOKO SUI T\=O SHAHO R\=A. MOKO HOR\=A HODO R\=A. HOSHI N\=I. ORI YATO
R\=A. BIRI KY\=U SH\=I  SHIBO BISHA Y\=A. HOJA R\=A MORI CH\=I. BISHA RY\=O
T\=O. HOTO MOKY\=A. HOJA R\=A. SHI I KANO. OSH\=A MORA SHIB\=O. HORA SHID\=O.
HOJA R\=A SEN SH\=I. BISHA RASH\=A. SETO SH\=A. BICHI B\=O FUSHI D\=O. SUMO
RY\=O B\=O. MOKO SUI T\=O. ORI YATO R\=A. MOKO HOR\=A OHO R\=A. HOJA R\=A.
SHAKE RASHI B\=O. HOJA R\=A KYUMO R\=I. KYURA TO R\=I. HOJA R\=A KASA TOSH\=A.
BICHI Y\=A KESHA N\=O. MORI GY\=A. KUSU MOB\=O. KERA TON\=O. BIRU SHAN\=O.
KYURI Y\=A YARA T\=O. SHUNI SAN. BISHA RY\=O. B\=O. MONI SH\=A. HOJA R\=A.
KYAN\=O KYAHO RAB\=O. RYOSHA N\=O. HOJA R\=A TOSHI SH\=A. SUITO SH\=A KYAMO
R\=A. SASHA SH\=I. HORA B\=O. EI CHI ICH\=I. MOTO RAKE N\=O. SOBI RASAN. KIHAN
TSU IN TSUN\=O MO M\=O SH\=A.


{\bf U KI RU SHU KEN NO.}\\
HORA SHASHI D\=O. SATA D\=O KYATSU SHUNI SAN. KUKI TSURYO Y\=O. SEBO N\=O KUKI
TSURYO Y\=O. SHITA HON\=O KUKI TSURYO Y\=O. HORA SHUCHI Y\=A. SABO SH\=A NOKE
R\=A. KUKI TSURYO Y\=O. SABO YASH\=A. KARA SAS\=O. KERA KOSHA J\=A. BIDO BEN
SANO KE R\=A. KUKI TSURYO Y\=O. SHATSU R\=A SHICHI NAN. KERA K\=O SOK\=O SORA
NAN. BIDO BEN SA NO R\=A. KUKI TSURYO Y\=O. RASHA BOGYA BAN. SATA D\=O KYATSU
SHUNI SAN. HORA TEN SHAKI R\=I. MOK\=O SOKO SAR\=A. FUJU SOK\=O SARA. SHIRI
S\=A. KYUSHI SOK\=O SANI CHIR\=I. OBI CH\=I SHIBO RIT\=O. S\=A S\=A AGY\=A.
MOKO HOJA RYOTO R\=A. CHIRI FUBO N\=O. MAN SARA UKIN. SOSHI CHI HOBO TSU. MOMO
IN TSUN\=O MO M\=O SH\=A.


{\bf RA SHA BO YA} SHURA B\=O Y\=A. OKI NIBO Y\=A. UTO KYABO Y\=A. BISHA
BOY\=A. SHASA TORA BOY\=A. HOR\=A SHAKE RABO Y\=A. TOSHI SHABO Y\=A. OSHA NIBO
Y\=A. OKYA R\=A MIRI SHUBO Y\=A. TORA N\=I FUMI KEN. BOGYA BOTO BOY\=A. URAKYA
BOTO BOY\=A. RASH\=A TAN. SHABO Y\=A. NOKY\=A BOY\=A. BISHU TABO Y\=A. SUBO
RANO BOY\=A. YASH\=A KERA K\=O. RASHA S\=U KERA K\=O. BIRI D\=O KERA K\=O.
BISHA J\=A KERA K\=O. FUDO KERA K\=O. KYUHA Z\=A KERA K\=O. FUTA N\=O KERA
K\=O. KYASHA FUTA N\=O KERA K\=O. SHIGE D\=O KERA K\=O. OHA SHIMO R\=A KERA
K\=O. UTA MOTO KERA K\=O. SHAYA KERA K\=O. KIRI HOCHI KERA K\=O. SHATO KORI
NAN. KEBO KORI NAN. RYOCHI R\=A KORI NAN. MOS\=O KORI NAN. MET\=O KORI NAN.
MOSH\=A KORI NAN. SHAT\=O KORI N\=I. SHIBI D\=O KORI NAN. BID\=O KORI NAN.
HOD\=O KORI NAN. OSHU S\=A KORI N\=I. SHID\=O KORI N\=I. CHISA SABI SAN. SAB\=O
KERA KONAN. BIDO YASH\=A SHIDO YAM\=I. KIRA YAM\=I. HORI HOR\=A SHAGY\=A KIRI
TAN. BIDO YASH\=A. SHIDO YAM\=I. KIRA YAM\=I. SAE N\=I KIRI TAN. BIDO YASH\=A
SHIDO YAM\=I KIRA YAM\=I. MOKO HOJU HODO Y\=A. RYOTO R\=A KIRI TAN. BIDO YA
SH\=A. SHIDO YAM\=I KIRA YAM\=I. NORA YAN\=O KIRI TAN. BIDO YASH\=A. SHIDO
YAM\=I KIRA YAM\=I. TOTO GY\=A RYOSA SH\=I KIRI TAN. BIDO YASH\=A SHIDO YAM\=I.
KIRA YAM\=I MOKO KYAR\=A. MOTO RIKYA N\=O. KIRI TAN BIDO YASH\=A SHIDO YAM\=I
KIRA YAM\=I. KYAHO RIGY\=A. KIRI TAN BIDO YASH\=A SHIDO YAM\=I KIRA YAM\=I.
SHAYA KER\=A. MOTO KER\=A. SABO RAT\=O. SOTO N\=O. KIRI TAN BIDO YASH\=A SHIDO
YAM\=I KIRA YAM\=I. SHATSU R\=A. HOKI N\=I. KIRI TAN BIDO YASH\=A SHIDO YAM\=I
KIRA YAM\=I. BIRI Y\=O KIRI SH\=I. NAN T\=O KISA R\=A. KYANO HOCH\=I. SOKI Y\=A
KIRI TAN BI DO YASH\=A SHIDO YAM\=I KIRA YAM\=I. NOKE N\=O SHARA HON\=O. KIRI
TAN BIDO YASH\=A SHIDO YAM\=I KIRA YAM\=I. ORA KAN. KIRI TAN BIDO YASH\=A SHIDO
YAM\=I KIRA YAM\=I. BIDO RAGY\=A. KIRI TAN BIDO YASH\=A SHIDO YAM\=I KIRA
YAM\=I. HOJA RAHO N\=I. KYUKI Y\=A KYUKI Y\=A. KYACHI HOCHI KIRI TAN BIDO YA
SH\=A SHIDO YAM\=I KIRA YAM\=I. RASHA B\=O BOGYA BAN. IN TSUN\=O MOM\=O SHA.


{\bf BO GYA HA SA TAN DO.}\\
HODO R\=A. NAMU SUI TO CHI. OSHI D\=O NORA RAGY\=A. HORA B\=O SHIFU S\=A. BIGYA
SATA D\=O. HOCHI R\=I SHIFU R\=A SHIFU R\=A. TORA TOR\=A. BIDO R\=A BIDO R\=A.
SHIDO SHID\=O. KUKI KUK\=I. HAZA HAZ\=A. HAZA HAZ\=A. HAZA SOK\=O. KI KI HAN.
OMO GYAYA HAN. OHO RACH\=I KOTO HAN. HOR\=A HORA TOHAN. OSU R\=A BIDO R\=A
BOGYA BAN. SABO CH\=I BIBI HAN. SAB\=O NOKYA BIHAN. SABO YASHA BIHAN. SABO KETO
BOBI HAN. SABO FUTA NOBI HAN. KYASHA FUTA NOBI HAN. SABO TORY\=O KICHI BIHAN.
SABO TOSHU BIR\=I KISHU CHIBI HAN. SAB\=O SHIBO RIBI HAN. SABO OHA SHIMO RIBI
HAN. SABO SHAR\=A HONO BIHAN. SABO CHICHI KIBI HAN. SABO TAM\=O TOKIBI HAN.
SABO BIDO YAR\=A SHISHA RIBI HAN SHAYA KER\=A MOTO KER\=A. SABO RAT\=O SOTO
KIBI HAN. BICHI Y\=A SHARI BI HAN. SHATSU R\=A HOKI NIBI HAN. HOJA R\=A KYUMO
R\=I. BIDO YARA SHIBI HAN. MOKO HORA CHIY\=O SHAKI RIBI HAN. HOJA R\=A SH\=O
KERA Y\=A HOR\=A SHAKI RASH\=A E HAN. MO KO KYARA Y\=A. MOKO MOTO RIKYA N\=O.
NAMU SOGE RIT\=O YAHAN. BISHU N\=O. BIE HAN. HORA. K\=O MONIE HAN. OKI NIE HAN.
MOKO KERI EHAN. KERA TOJI EHAN. METO RIE HAN. R\=O TORI EHAN. SHAN BUS\=O EHAN.
KERA RATO RIE HAN. KYAHO RIE HAN. OCHI MOSHI D\=O. KYASHI MOSHA N\=O. HOSU NIE
HAN. ENKISHI. S\=A TOBO SH\=A. MOMO. IN TSUN\=O MO M\=O SH\=A. 


{\bf TO SHU SA SHI DO} OMO TORI SH\=I D\=O. USHA KOR\=A. KYABO KOR\=A. RYOCHI
RAKO R\=A. HOSO KOR\=A. MOSHA KOR\=A. SHATO KOR\=A. SHIBI DOKO R\=A. HORA Y\=A
KOR\=A. KEN TOKO R\=A. FUSU BOKO R\=A. HORA KOR\=A. HOJA KOR\=A. HOBO SHI DO.
TO SHU SA SHI D\=O. R\=O TORA SHID\=O. YASH\=A KERAK\=O. RASHA S\=U KERA K\=O.
BI RI D\=O KE RA K\=O. BISHA J\=A KERA K\=O. FUDO KERA K\=O. KYUHA Z\=A KERA
K\=O. SHIGE D\=O KERA K\=O. UTA MOTO KERA K\=O. SHAYA KERA K\=O. OHA SAMO R\=A
KERA K\=O. SAKI GAS\=A KINI KERA K\=O. RIFU JI KERA K\=O. SHAMI GY\=A KERA
K\=O. SHAKI N\=I KERA K\=O MOTO R\=A. NACHI GY\=A KERA K\=O. ORA B\=O KERA
K\=O. KETO HONI KERA K\=O. SHIFU R\=A. IGY\=A. KIGY\=A. SU I CHI YA GY\=A. TORI
CHI YA GY\=A. SHAT\=O TAGY\=A. NICHI SHIFU R\=A. BISA M\=O. SHIFU R\=A. HOCHI
GY\=A. B\=I CHIGY\=A. SHIRI SHUMI GY\=A. SONI HOCHI GY\=A.  SAB\=O. SHIFU R\=A.
SHIRY\=O KICH\=I. MOTO B\=I. TARYO SHIKEN. OKI  RYOKEN. MOKI RYOKEN. KERI T\=O
RYOKEN. KERA KOKE RAN. KENO SHURAN. TAN TOSHU RAN. KIRI Y\=A SHURAN. MOMO
SHURAN. HORI SHIBO SHURAN. BIRI SHUSA SHURAN. UTORA SHURAN. KESHI SHURAN.
HOSHICHI SHURAN. URY\=O SHURAN. SH\=A GYASHURAN. KASHIDO SHURAN. HODO SHURAN.
SOBO AGY\=A. HORA SHAGYA SHURAN. FUDO BIDO S\=A. SAKI N\=I SHIFU R\=A. TO TO
RYOGY\=A KETO RYOKI SH\=I HORU T\=O B\=I. SAHO RY\=O KORI GY\=A. SHUSA TOR\=A
SONO KER\=A. BISA Y\=U GY\=A. OKI N\=I UTO GY\=A. MORA BIR\=A. KEN TOR\=A. OGYA
R\=A MIRI SH\=U TARE BOGY\=A. CHIRI RAS\=A. BIRI SHUSHI GY\=A. SABO N\=O
KYUR\=A. SU I GY\=A B\=I KERA RIYA SH\=A TORA S\=U. MORA S\=U. BICHI SAN. SOBI
SAN. SHITE D\=O HODO R\=A. MOKO HOJA RY\=O SHUNI SAN. MOKO HOR\=A SHAKI RAN.
YAHO TOD\=O SHAYU SHAN\=O. METO RIN\=O BIDO Y\=A HODO KYARU M\=I. CHISHU HODO
KYARU M\=I. HORA BID\=O HODO KYARU M\=I. TOJI T\=O EN. ONO R\=I BISHA CH\=I.
BIRA HOJA RATO R\=I. HODO HODO N\=I. HOJA R\=A HONI HAN. KUKI TSURYO Y\=O HAN.
S\=O M\=O K\=O.


{\bf MO KO HO JA HO RO MI.}\\
MO KO HOJ\=A H\=O R\=O M\=I.\\
{\scriptsize(3 X)\\
(Na koniec: FUEK\=O)}\\ 
\end{JAPANESE}
\end{Prayer}

\newpage
%%%%%%%%%%%%%%%%%%%%%%%%%%%%%%%%%%%%%%%%%%%%%%%%%%%%%%%%%%%%%%%%%%%%%%%%%%%%%%%
%%%% LINIA PATRIARCHÓW
\begin{Prayer}{-}
	{LINIA PATRIARCHÓW}{-}
	{-} 

\parindent	0pt


{\bf Siedmiu Buddhów Przeszłości}\\[0.5em]

{\wersaliki
BIBASI BUTSU DAIOOSIO\\
SIKI BUTSU DAIOOSIO\\
BISIAFU BUTSU DAIOOSIO\\
KURUSON BUTSU DAIOOSIO\\
KUNAGONMUNI BUTSU DAIOOSIO\\
KASIOO BUTSU DAIOOSIO\\
SIAKAMUNI BUTSU DAIOOSIO\\[1em]
}


{\bf Patriarchowie Indii}\\[0.5em]

{\wersaliki
MAKAKASIOO DAIOOSIO\\
ANANDA DAIOOSIO\\
SIONAŁASIU DAIOOSIO\\
UBAKIKUTA DAIOOSIO\\
DAITAKA DAIOOSIO\\
MISIAKA DAIOOSIO\\
BASIUMITSU DAIOOSIO\\
BUTSUDANANDAI DAIOOSIO\\
FUDAMITTA DAIOOSIO\\
BARISIBA DAIOOSIO\\
FUNAIASIA DAIOOSIO\\
ANABOTEI DAIOOSIO\\
KABIMORA DAIOOSIO\\
NAGIAHARADZIUNA DAIOOSIO\\
KANADAIBA DAIOOSIO\\
RAGORATA DAIOOSIO\\
SOOGIANANDAI DAIOOSIO\\
KAIASIATA DAIOOSIO\\
KUMORATA DAIOOSIO\\
SIAIATA DAIOOSIO\\
BASIUBANDZU DAIOOSIO\\
MANURA DAIOOSIO\\
KAKUROKUNA DAIOOSIO\\
SISIBODAI DAIOOSIO\\
BASIASITA DAIOOSIO\\
FUNIOMITTA DAIOOSIO\\
HANNIATARA DAIOOSIO\\
BODAIDARUMA DAIOOSIO\\[1em]
}


{\bf Patriarchowie Chin}\\[0.5em]

{\wersaliki
TAISO EKA DAIOOSIO\\
KANCI SOOSAN DAIOOSIO\\
DAII DOOSIN DAIOOSIO\\
DAIMAN KOONIN DAIOOSIO\\
DAIKAN ENOO DAIOOSIO\\
SEIGEN GIOOSI DAIOOSIO\\
SEKITOO KISEN DAIOOSIO\\
YAKUSAN IGEN DAIOOSIO\\
UNGAN DONDZIOO DAIOOSIO\\
TOODZAN RIOOKAI DAIOOSIO\\
UNGO DOOIOO DAIOOSIO\\
DOOAN DOOHI DAIOOSIO\\
DOOAN KANSI DAIOOSIO\\
RIOODZAN ENKAN DAIOOSIO\\
DAIIOO KIOOGEN DAIOOSIO\\
TOOSU GISEI DAIOOSIO\\
FUIOO DOOKAI DAIOOSIO\\
TANKA SIDZIUN DAIOOSIO\\
CIOORO SEIRIOO DAIOOSIO\\
TENDOO SOKAKU DAIOOSIO\\
SECCIOO CIKAN DAIOOSIO\\
TENDOO NIODZIOO DAIOOSIO\\[1em]
}


{\bf Patriarchowie Japonii}\\[0.5em]

{\wersaliki
EIHEI DOOGEN DAIOOSIO\\
KOUN EDZIOO DAIOOSIO\\
TETSUU GIKAI DAIOOSIO\\
KEIDZAN DZIOOKIN DAIOOSIO
}
\end{Prayer}

%%%%%%%%%%%%%%%%%%%%%%%%%%%%%%%%%%%%%%%%%%%%%%%%%%%%%%%%%%%%%%%%%%%%%%%%%%%%%%%
%%%% TEIDAI DENPO BUSSO NO MYOGO
\begin{Prayer}{teidai_denpo_busso_no_myogo}
	{TEIDAI DENPO\\BUSSO NO MYOGO}{TEIDAI DENPO BUSSO NO MYOGO}
	{LINIA DHARMY \emph(Imiona Buddhów i Patriarchów linii rinzai zen ---
	Eigenji\emph)} 

\begin{longtable}{lc}
BIBASHI 	& BUTSU \\
SHIKI 		& BUTSU \\
BISHAFU 	& BUTSU \\
KURUSON 	& BUTSU \\
KUNAGON MUNI 	& BUTSU \\
KASHO		& BUTSU \\ [1em]

SHAKYAMUNI 	& BUTSU \\
MAKA KASHO	& SONJA \\
ANAN 		& SONJA \\
SHONA WASHU 	& SONJA \\
UBA KIKUTA 	& SONJA \\
DAI TAKA 	& SONJA \\
MI SHAKA 	& SONJA \\
BASU MITS 	& SONJA \\
Buddha NANDAI 	& SONJA \\
FUKUDA MITTA	& SONJA \\
KYO 		& SONJA \\
FUNA YASHA 	& SONJA \\
MEMYO		& SONJA \\
KABIMORA	& SONJA \\
RYUJU		& SONJA \\
KANA DAIBA	& SONJA \\
RAGORATA	& SONJA \\
SOGYA NANDAI	& SONJA \\
KAYASHATA 	& SONJA \\
KUMORATA	& SONJA \\
SHAYATA 	& SONJA \\
BASU BANZU	& SONJA \\
MANURA 		& SONJA \\
KAKU ROKUNA 	& SONJA \\
SHISHI 		& SONJA \\
BASHA SHITA 	& SONJA \\
FUNYO MITTA 	& SONJA \\
HANNYA TARA 	& SONJA \\ [1em]

BODAI DHARUMA	& DAISHI \\
EKA DAI SO	& ZENJI \\
SO SAN KAN CHI	& ZENJI \\
DO SHIN DAI I	& ZENJI \\
GUNIN DAI MAN	& ZENJI \\
ENO DAI KAN	& ZENJI \\
NAN GAKU EJO 	& ZENJI \\
BASO DO ITSU 	& ZENJI \\
HYAKUJO EKAI 	& ZENJI \\
O BAKU KIUN 	& ZENJI \\
RIN ZAI GIGEN	& ZENJI \\
KO KE SON SHO 	& ZENJI \\
NAN IN EGYO 	& ZENJI \\
FUKETSU EN SHO 	& ZENJI \\
SHUZAN SHO NEN 	& ZENJI \\
FUNNYO ZEN SHO 	& ZENJI \\
SEKISO SOEN 	& ZENJI \\
YO GI HO E 	& ZENJI \\
HAKU UN SHUTAN 	& ZENJI \\
GOSO HO EN	& ZENJI \\
EN GO KOKUGON	& ZENJI \\
KUKYO JO RYU	& ZENJI \\
O AN DON GE	& ZENJI \\
MITTAN KANKETSU	& ZENJI \\
SHOGEN SOGAKU 	& ZENJI \\
UNNAN FUGAN	& ZENJI \\
KIDO CHIGU	& ZENJI \\
NANPO JYO MYO	& ZENJI \\
SHUHO MYO CHO	& ZENJI \\ [1em]

KANZAN EGEN	& ZENJI \\
JUO SO HITSU	& ZENJI \\
MUIN SOIN	& ZENJI \\
NIPPO SOSHUN	& ZENJI \\
GITEN GENSHO	& ZENJI \\
SEKKO SOSHIN	& ZENJI \\
TOYO EICHO	& ZENJI \\
TAIGA ZUI KYO	& ZENJI \\
KOHO GEN KUN	& ZENJI \\
SEN SHO ZUI SHO	& ZENJI \\
IAN CHI SATSU	& ZENJI \\
TO ZEN SO SHIN	& ZENJI \\
YO ZAN KEIYO	& ZENJI \\
GU DO BUNAN	& ZENJI \\
DO KYU EZIU	& ZENJI \\
HAKUIN EKAKU 	& ZENJI \\
GAZAN JITO	& ZENJI \\
TAKUJU KOSEN	& ZENJI \\
SHUN O ZEN ETSU	& ZENJI \\
SUI GAN BUN SHU	& ZENJI \\
KAN SHU GEN SEI	& ZENJI \\
YUZEN GENTATSU	& ZENJI \\
DOKUHO EN ITSU	& ZENJI \\
YUHO SO SHUN	& ZENJI
\end{longtable}

\begin{verse}
	NY\=ANK\=I SHIN ZU FUJI SH\=O K\=AN\\
	JY\=O R\=AI FUN ZU\\
	DAI HI EN MAN MU GE JIN SHU\\
	SU SHI JU HIN B\=UN N\=I\\
	SH\=I S\=U BUJI DAMO ENKA DAISU SAI OSH\=O (BODHIDHARMA)\\
	HA J\=O DAI SHI ZENZU (EKAI)\\
	RIN SHI E SH\=O ZENZU (RINZAI)\\
	SAN KU DEN TEN RI DAI (wszyscy pozostali kapłani tej tradycji)\\
	KAKA JY\=O JY\=U Z\=U \=IN

	J\=I H\=O SAN SHI ISHI SHI BU SHI\\
	SON BU S\=A MOKO S\=A MOKO\\
	HO JYA H\=OR\=O M\=I
\end{verse}
\end{Prayer}

\newpage
%%%%%%%%%%%%%%%%%%%%%%%%%%%%%%%%%%%%%%%%%%%%%%%%%%%%%%%%%%%%%%%%%%%%%%%%%%%%%%%
%%%% BOSATSU GANGY\=O MON
\begin{Prayer}{bosatsu_gangyo_mon}
	{BOSATSU GANGY\=O MON}{-}
	{Ślubowanie Bodhisattwy, Torei Enji}

\bigskip
\begin{JAPANESE}
DESHI SORE GASHI TSUTSUSHINDE SHOH\=O NO JISS\=O O~KANZURU NI, MINA KORE
NYORAI SHINJITSU NO MY\=OS\=O NI SHITE JINJIN SETSUSETSU ICHIICHI FUSHIGI
NO K\=OMY\=O NI ARAZU TO IYU KOTO NASHI. KORE NI YOTTE ITASHI E SENTOKU WA
CH\=ORUI CHIKURUI NI ITARU MADE GASSH\=O RAIHAI NO KOKORO O~MOTTE AIGOSHI
TAMAERI. KARU GA YUE NI JUNIJI CH\=U WARERA GA SHIN MY\=OY\=O GO NO ONJIKI
E FUKU WA MOTO YORI K\=OSO NO DANPI NIKU NI SHITE, GONGEN JIHI NO BUNSHIN
NAREBA, TAREKA AETE KUGY\=O KANSHA SEDZARANYA. MUJ\=O NO KIBUTSU
NAOSHIKARI. IWANYA HITO NI SHITE OROKANARU MONO NI WA HITOSHI ORENMIN
KENNENSHI, TATOE AKUSH\=U ONTEKI TO NATTE WARE O~NONOSHIRI WARE
O~KURUSHIMURU KOTO ARU MO, KORE WA KORE BOSATSU GONGE NO DAIJIHI NI SHITE
MURY\=O G\=ORAI GAKEN HENSH\=U NI YOTTE TSUKURI NASERU WAGAMI NO ZAIG\=O
O~SH\=OMETSU GEDATSU SESHIME TAMAU H\=OBEN NARI TO ISSHIN KIMY\=O GONJI
O~KENJ\=O NI SHITE FUKAKU J\=OSHIN O~OKOSABA, ICHINEN T\=OJ\=O NI RENGE
O~HIRAKI ICHIGE ICHIBUTSU O~GENJI, ZUISHO NI J\=ODO O~SH\=OGON SHI, NYORAI
NO K\=OMY\=O KYAKKA NI GENTETSU SEN. NEGAWAKUWA KONO KOKORO O~MOTTE AMANEKU
ISSAI NI OYOBOSHI. WARERA TO SHUJ\=O TO ONAJIKU SH\=UCHI MADOKA NI SEN KOTO O.
\end{JAPANESE}
\end{Prayer}

%%%%%%%%%%%%%%%%%%%%%%%%%%%%%%%%%%%%%%%%%%%%%%%%%%%%%%%%%%%%%%%%%%%%%%%%%%%%%%%
%%%% ŚLUBOWANIE BODHISATTWY
\begin{Prayer}{slubowanie_bodhisattwy}
	{ŚLUBOWANIE BODHISATTWY}{-}
	{Bosatsu Gangy\=o Mon, Torei Enji}

\bigskip
\noindent
Gdy ja, uczeń Dharmy patrzę na rzeczywisty kształt wszechświata, wszystko
bez wątpienia, jawi się jako tajemnicza prawda Tathagathy. W~każdym
przypadku, w~każdej chwili i~w każdym miejscu nie jest to nic innego niż
cudowne objawienie jego wspaniałego światła. To urzeczywistnienie sprawiło,
że nasi patriarchowie i~pełni cnót mistrzowie zen z~sercem pełnym czci
przejawiali czułą troskę nawet dla takich istot jak zwierzęta i~ptaki. To
urzeczywistnienie uczy nas, że nasze codzienne jedzenie i~picie, ubrania
i~ochrony życia nie są niczym innym niż ciepłym ciałem i~krwią
współczującego wcielenia Buddhy. Jakże więc być niewdzięcznym i~bez szacunku
nawet wobec rzeczy pozbawionych zmysłów, nie mówiąc już o~człowieku? Nawet
gdyby był on głupi, bądźmy wobec niego ciepli i~współczujący. Jeśli
przypadkiem zwróci się on przeciwko nam znęcając się i~prześladując nas,
powinniśmy pokłonić się z~pokornymi słowami w~pełnej czci wierze, że jest
on współczującym wcieleniem Buddhy, który używa tych środków by uwolnić nas
od złej karmy, którą stworzyliśmy i~gromadziliśmy własnymi egoistycznymi
złudzeniami i~przywiązaniami poprzez niezliczone cykle kalpy. Wówczas
w~każdym punkcie naszych myśli będzie kwitł kwiat lotosu, i~każdy kwiat
lotosu będzie ukazywał Buddhę. Ci Buddhowie będą zdobić Subhavati, Czystą
Krainę wszędzie, w~każdej chwili. Oby ten umysł rozciągnął się na cały
wszechświat tak by wszystkie istoty razem osiągnęły doskonałość w~mądrości
Buddhy.
\end{Prayer}

% (!!!) trikaya - słowniczek

% Chwała naukom, które nie znają żadnych organiczeń! \keisu Tak więc: \\
% Om, Khya khya khyani khyani (mów, mów)!
% Hum Hum!
% Jvala jvala prajvala prajvala (blask, blask)!
% Tistha tistha (góra, góra)! \shokei Stri stri (?)!
% Sphata (rozsadzać, rozsadzać)! \shokei Ten, który jest cichy!
% Do wspaniałego, pozdrowienie!

\newpage
%%%%%%%%%%%%%%%%%%%%%%%%%%%%%%%%%%%%%%%%%%%%%%%%%%%%%%%%%%%%%%%%%%%%%%%%%%%%%%%
%%%% EIHEI K\=OSO HOTSUGAMMON
\begin{Prayer}{eihei_koso_hotsugammon}
	{EIHEI K\=OSO HOTSUGAMMON}{-}
	{Wzniecanie Umysłu Oświecenia, Eihei D\=ogen Zenji}

\bigskip
\begin{JAPANESE}
NEGAWAKUWA WARETO ISSAI SHUJ\=O TO KONJ\=O YORI NAISHI SH\=O SH\=O
O~TSUKUSHITE SH\=OB\=O O~KIKU KOTO ARAN, KIKU KOTO ARAN TOKI SH\=OB\=O
O~GIJAKUSEJI, FUSHIN NARU BE KARAZU, MASANI SH\=OB\=O NI AWAN TOKI, SEH\=O
O~SUTETE BUPP\=O JUJISEN, TSUINI DAICHI NO UJ\=O TO TOMONI J\=OD\=O SURU
KOTO O~EN, NEGAWAKUWA WARETATOI KAKO NO AKUG\=O \=O KASANARITE SH\=OD\=O NO
INNEN ARITOMO, BUTSUD\=O NI YORITE TOKUD\=O SERISHI SHOBUTSU SHOSO, WARE
O~AWAREMITE G\=ORUI O~GEDATSU SESHIME, GAKUD\=O SAWARI NAKARASHIME, SONO
KUDOKU H\=OMON AMANEKU MUJIN HOKKAINI J\=U MAN MIRIN SERAN, AWAREMI
O~WARENI BUN PU SUBESHI BUSSO NO \=OSHAKU WA WARERA NARI, WARERA GA T\=ORAI
WA BUSSO NARAN, BUSSO O~G\=OKAN SUREBA ICHI BUSSO NARI, HOSSHIN O~KANS\=O
SURU NI MO ICHI HOSSHIN NARUBESHI, AWAREMI O~SHITTSU HATTA SENNI TOKUBENGI
NARI, RAKUBENGI NARI, KONO YUENI RY\=UGE NO IWAKU, SHAKUJ\=O IMADA
RY\=ODZEZUNBA, IMASUBEKARAKU RY\=OZUBESHI, KONO SHO RUISH\=O NO MI
O~DOSHUSEYO, KOBUTSU IMADA SATORAZAREBA KONJA NI ONAJISHI, SATORI OWAREBA,
KOJIN MO SUNAWACHI KOJIN, SHIZUKANI KONO INNEN O~SANKY\=U SUBESHI, KORE
SH\=OBUTSU NO J\=OT\=O NARI KAKU NO GOTOKU SANGE SUREBA, KANARAZU BUSSO NO
MY\=OJO ARUNARI, SHINNEN SHINGI HORRO BIAKUBUTSU SUBESHI, HORRO NO CHIKARA
ZAIKON O~SHITE SH\=OIN SESHIMURU NARI, KORE ISSHIKI NO SH\=OSHUGY\=O NARI,
SH\=O SHINJIN NARI, SH\=O SHINJIN NARI.
\end{JAPANESE}
\end{Prayer}

%%%%%%%%%%%%%%%%%%%%%%%%%%%%%%%%%%%%%%%%%%%%%%%%%%%%%%%%%%%%%%%%%%%%%%%%%%%%%%%
%%%% WZNIECANIE UMYSŁU OŚWIECENIA, DOGEN ZENJI
\begin{Prayer}{wzniecanie_umyslu_oswiecenia_dogen}
	{ŚLUBOWANIE BODHICITTY,\\ DOGEN ZENJI}{ŚLUBOWANIE BODHICITTY, DOGEN ZENJI}
	{Hotsugammon, Eihei D\=ogen Zenji}

\bigskip
\noindent
Ślubuję, że od obecnego życia aż do wyczerpania wszystkich żywotów razem ze
wszystkimi istotami będę słuchać Prawdziwej Dharmy. Słuchając Prawdziwej
Dharmy nie powstanie we mnie żadne zwątpienie, ani brak wiary. Kiedy spotkam
Prawdziwą Dharmę porzucę światowe dharmy~-- zjawiska, a~przyjmę i~zachowam
Dharmę Buddhy. Ostatecznie razem ze wszystkimi odczuwającymi istotami
Wielkiej Ziemi urzeczywistnię Drogę. Chociaż w~przeszłości nagromadziłem
wiele złej karmy, co stało się przyczyną i~podstawą przeszkód
w~praktykowaniu Drogi, modlę się, aby wszyscy Buddhowie i~Patriarchowie,
którzy osiągnęli Drogę Buddhy, byli współczujący i~uwolnili mnie od skutków
zatrzymującej mnie karmy, pozwalając praktykować Drogę bez przeszkód. Oby
obdarzyli mnie cnotą zasługi bramy Dharmy, która całkowicie wypełnia
niewyczerpalną Dharmadhatu. Starożytni Buddhowie i~Patriarchowie byli nami
i~my w~przyszłości będziemy Buddhami i~Patriarchami. Spoglądając z~czcią na
Buddhów i~Patriarchów, sami jesteśmy Buddhą i~Patriarchą. Kontemplując
przebudzenie Bodhicitty, sami jesteśmy przebudzoną Bodhicittą. Współczucie
sięga wszędzie i~możemy osiągnąć pożytek i~porzucić pożytek, dlatego Ryuge
powiedział: ,,Ci, którzy w~przeszłych żywotach nie osiągnęli Oświecenia,
teraz powinni osiągnąć Oświecenie. W~tym życiu wyzwól ciało będące skutkiem
nagromadzonych żywotów. Jeśli starożytni Buddhowie nie osiągnęliby
Oświecenia, byliby tacy jak ludzie dzisiaj. Jeśli dzisiejsi ludzie osiągną
Oświecenie, staną się jak starożytni''. Spokojnie zbadaj te przyczyny
i~związki, ponieważ to jest dokładny przekaz potwierdzonego Buddhy. Czyniąc
skruchę w~ten sposób, z~pewnością otrzymasz niewidzialną pomoc od Buddhów
i~Patriarchów. Gdy ukazujemy Buddzie myśli umysłu i~czyny ciała, siła tego
wyjawienia wyrywa i~usuwa korzenie wykroczeń. To jest jeden kolor prawdziwej
praktyki, prawdziwego umysłu wiary, prawdziwego ciała wiary.
\end{Prayer}

%%%%%%%%%%%%%%%%%%%%%%%%%%%%%%%%%%%%%%%%%%%%%%%%%%%%%%%%%%%%%%%%%%%%%%%%%%%%%%%
%%%% DAICHI ZENJI HOTSUGANMON
\begin{Prayer}{daichi_zenji_hotsuganmon}
	{DAICHI ZENJI HOTSUGANMON}{-}
	{-}

\bigskip
\begin{JAPANESE}
NEGAWAKUWA WARE KONO BUMOSHOSH\=O NO MIWO MOTTE, SANB\=ONO GANKAINI EK\=O SHI,
ICHID\=O ICHIJ\=O HOSSHIKINI ISEZU, KONJIN YORI BUSSHIN NI ITARUMADE, SONO
CH\=UGEN NI OITE, SH\=OSH\=O SESE, SHUSSH\=O NY\=USHI BUPP\=OWO HANAREZU, ZAIZAI
SHOSHO, HIROKU SHUJ\=O WO DOSHITE HIENWO SH\=OZEZU, ARUIWA KENJUT\=OSEN NO UE,
ARUIWA KAKUT\=OROTANN NO UCHI, TADAKORE SH\=OB\=OGENZ\=O WO MOTTE J\=UTAN TO
NASHITE, ZUISHO NI SHUSAI TO NARAN, FUSHITE NEGAWAKUWA, SANB\=O SH\=OMY\=O,
BUSSO GONEN.
\end{JAPANESE}
\end{Prayer}

%%%%%%%%%%%%%%%%%%%%%%%%%%%%%%%%%%%%%%%%%%%%%%%%%%%%%%%%%%%%%%%%%%%%%%%%%%%%%%%
%%%% DAICHI ZENJI HOTSUGANMON (PL)
\begin{Prayer}{daichi_zenji_hotsuganmon_pl}
	{ŚLUBOWANIE BODHICITTY, \\DAICHI ZENJI}{ŚLUBOWANIE BODHICITTY, DAICHI ZENJI}
	{-}

\bigskip
\noindent
Ślubuję i~ofiarowuję to ciało, zrodzone z~matki i~ojca, oceanowi ślubowań
Trzech Klejnotów, ruch i~bezruch nie różnią się w~metodzie, od tego życia aż
do osiągnięcia Stanu Buddhy i~w~całym tym okresie, przez wszystkie żywoty
i~światy, od narodzin do śmierci nie będę oddzielony od Dharmy Buddhy.
W~każdym kraju i~w~każdym miejscu, wszędzie będę wyzwalał szeroko istoty
i~nie zrodzą się zmęczenie i~znużenie, czy na szczycie mieczy, czy w~gorącej
wodzie, w~piecu i~na gorącym węglu, jedynie w~oparciu o~Skarbnicę Oka
Prawdziwej Dharmy będę tak czynił, wszędzie zachowując uważność. Z~pokorą
i~szacunkiem modlę się o~Oświecenie Trzech Klejnotów i~ochronę Buddhów
i~Patriarchów.
\end{Prayer}

%%%%%%%%%%%%%%%%%%%%%%%%%%%%%%%%%%%%%%%%%%%%%%%%%%%%%%%%%%%%%%%%%%%%%%%%%%%%%%%
%%%% DAIE ZENJI HOTSUGAMMON
% !!! DAI E ZENJI HOTSUGAM MON
\begin{Prayer}{daie_zenji_hotsugammon}
	{DAIE ZENJI HOTSUGAMMON}{-}
	{Wzniecanie Umysłu Oświecenia, Daie Zenji}

\bigskip
\begin{JAPANESE}
TADA NEGAWAKUWA SORE GASHI. D\=OSHIN KENGO NI SHITE. CH\=ON FUTAI. SHITAI
KY\=OAN. SHINJIN YUMY\=O. SHUBY\=O KOTOGOTOKU NOZOKI. KON SAN SUMI YAKANI
SH\=OJI. MUNAN MUSAI. MUMA MUSH\=O. JARO NI MUKAWAZU. JIKI NI SH\=OD\=O NI
ITTE BONN\=O SHOME TSUSHI. CHIE Z\=OCH\=OSHI. TONNI DAIJI O~SATOTTE. HOTOKE
NO EMY\=O O~TSUGI. MORO MORO NO SHUJ\=O O~DOSHITE. BUSSO NO ON O~HOSEN KOTO
O. TSUGI NI KOI NEGAWAKUWA SORE GASHI. RIN MY\=OJU NO TOKI. SH\=OBY\=O
SH\=ON\=O. SHICHINICHI IZENNI. ARAGAJIME SHI NO ITARAN KOTO O~SHITE. ANJ\=U
SH\=ONEN MATSUGO JIZAI NI. KONO MI O~SUTE OWATTE. SUMIYAKANI BUTSUDO NI
SH\=OJI. MANOATARI SHOBUTSU NI MAMIE. SH\=OGAKU NO KI O~UKE. HOKKAI NI
BUNSHIN SHITE. AMANEKU SHUJ\=O O~DOSEN KOTO O~. JIPP\=O SANZE ISSAI NO
SHOBUTSU. SHOSON BOSATSU MAKASATSU. MAKA HANYA HARAMITSU.
\end{JAPANESE}
\end{Prayer}

%%%%%%%%%%%%%%%%%%%%%%%%%%%%%%%%%%%%%%%%%%%%%%%%%%%%%%%%%%%%%%%%%%%%%%%%%%%%%%%
%%%% WZNIECANIE UMYSŁU OŚWIECENIA,\\ DAIE ZENJI
\begin{Prayer}{wzniecanie_umyslu_oswiecenia_daie}
	{WZNIECANIE UMYSŁU OŚWIECENIA,\\ DAIE ZENJI}{WZNIECANIE UMYSŁU OŚWIECENIA, DAIE ZENJI}
	{Hotsugammon, Daie Zenji}

\bigskip
\noindent
Moim jedynym pragnieniem jest być silnie zdeterminowanym w~poszukiwaniu
Prawdy, tak by nie pojawiły się żadne wątpliwości bez względu na to jak
długo przyszło by mi jej szukać; Być lekkim i~swobodnym w~czterech
częściach ciała; Być mocnym i~nieustraszonym w~ciele i~umyśle, być wolnym
od chorób i~odwrócić od siebie zarówno uczucia przygnębienia jak i~euforii;
Być wolnym od klęski, nieszczęścia, szkodliwych wpływów i~przeszkód; Nie
poszukiwać prawdy na zewnątrz siebie tak, abym w~jednej chwili wkroczył na
właściwą drogę; Być nieprzywiązanym do wszelkich myśli tak bym mógł
osiągnąć doskonały, jasny i~czysty umysł pradźni i~natychmiast osiągnąć
oświecenie w~Wielkiej Sprawie. I~tym samym kontynuować duchowe życie Buddhów
i~wyzwolić wszystkie odczuwające istoty, które cierpią w~kole narodzin
i~śmierci. W~ten sposób ofiarowuję swoją wdzięczność za współczucie Buddhów
i~Patriarchów. Moim dalszym pragnieniem jest to bym zanim umrę miał
przeczucie śmierci na 7 dni przed jej nadejściem i~żebym wówczas nie był
nazbyt chorym, ani nazbyt cierpiał, tak abym mógł pełen spokoju przebywać
we właściwym stanie umysłu. I~bym porzucając to ciało i~będąc
nieprzywiązanym do niczego mógł odrodzić się w~krainie Buddhów, ujrzeć ich
twarzą w~twarz i~otrzymać od nich potwierdzenie najwyższego oświecenia
dzieląc się nieskończenie w~Dharmadhathu tak by wyzwolić wszystkie
odczuwające istoty. Ofiarowuję to wszystkim Buddhom
i~Bodhisattwom-Mahasattwom w~przeszłości, teraźniejszości i~przyszłości,
w~10 kierunkach i~Mahapradźni Paramicie.
\end{Prayer}

%%%%%%%%%%%%%%%%%%%%%%%%%%%%%%%%%%%%%%%%%%%%%%%%%%%%%%%%%%%%%%%%%%%%%%%%%%%%%%%
%%%% 
\begin{Prayer}{shichichiyooteishi_yuikai}
	{SHICHICHIY\=OTEISHI YUIKAI}{-}
	{}

\bigskip
\begin{JAPANESE}
WARE NI SANT\=O NO DESHIARI. IWAYURU M\=ORETSU NI SHITE. SHYOEN O H\=OGE SHITE
SENICHI NI KOJI O KIY\=UMEI SURU KORE O JIY\=OT\=O TOSU. SHUGY\=OJIYUN NARAZU 
HAKUSATSU NI SHITE GAKU O KONO MU KORE O CHIY\=UTO\=O TO IU. MIZUKA KOREI NO
K\=OKI O KURAMA SHITE. TADABUTSUSO NO ENDA O TASHIMU KORE O GET\=O TO NAZUKU. 
MUSHI SORE SHIN O GESHIYO NI YOWASHIME. GY\=O O BUNHITSU NI TATSURU MONO. KORE WA
KORETEIZU NO ZOKUNIN NARI. MOTSUTEGET\=O NASUNITA...
\end{JAPANESE}
\end{Prayer}

%%%%%%%%%%%%%%%%%%%%%%%%%%%%%%%%%%%%%%%%%%%%%%%%%%%%%%%%%%%%%%%%%%%%%%%%%%%%%%%
%%%% K\=OZEN DAIT\=O KOKUSHI YUIKAI
\begin{Prayer}{kozen_daito_kokushi_yuikai}
	{K\=OZEN DAIT\=O KOKUSHI YUIKAI}{-}
	{Ostatnie Upomnienie, Dait\=o Kokushi}

\bigskip
\begin{JAPANESE}
NANJIRA SHONIN, KONO SANCH\=U NI KITATTE D\=O NO TAMENI, K\=OBE O~ATSUMU.
EJIKI NO TAME NI SURU KOTO NAKARE. KATA ATTE KIZU TO IU KOTO NAKU, KUCHI
ATTE KURAWAZU TO IU KOTO NASHI, TADA SUBEKARAKU J\=UNI JICH\=U, MURIE NO
TOKORO NI MUKATTE KIWAME KITARI KIWAME SARUBESHI K\=OIN YANO GOTOSHI,
TSUTSUSHINDE Z\=OY\=OJIN SURU KOTO NAKARE KANSHU SEYO, KANSHU SEYO.
R\=OS\=O ANGYA NO NOCHI, ARUI WA JIMON HANK\=O, BUKKAKU KY\=OKAN, KINGIN
CHIRIBAME, TASHU NY\=ONETSU, ARUIWA JIUKY\=O F\=UJ\=U, CH\=OZA FUGA,
ICHIJIKI B\=OSAI ROKUJI GY\=OD\=O, TATOI INMO NI SHISARU TO IE DOMO, BUSSO
FUDEN NO MY\=OD\=O O~MOTTE KY\=OKAN KAZAI SEZUNBA, TACHIMACHI INGA
O~HATSUMUSHI, SHINP\=U CHI NI OTSU. MINA KORE JAMA NO SHUZOKU NARI.
R\=OS\=O YO O~SARI KOTO HISASHIKU TOMO JISON TO SH\=O SURU KOTO O~YURUSAJI.
ARUIWA MOSHI ICHI NIN ARI YAGAI NI MENZETSU SHI, IPPA B\=OTEI SEKKYAKU
SH\=ONAI NI, YASAIKON O~NITE, KISSHITE HI O~SUGOSU TOMO, SENNICHI NI KOJI
O~KY\=UMEI SURU TEIWA, R\=OS\=O TO NICHI NICHI SH\=OKEN, H\=OON TEI NO HITO
NARI, TAREKA AETE KY\=OKOTSU SENYA. BENSEN, BENSEN.
\end{JAPANESE}
\end{Prayer}

%%%%%%%%%%%%%%%%%%%%%%%%%%%%%%%%%%%%%%%%%%%%%%%%%%%%%%%%%%%%%%%%%%%%%%%%%%%%%%%
%%%% OSTATNIE UPOMNIENIE DAIT\=O KOKUSHI
\begin{Prayer}{ostatnie_upomnienie}
	{OSTATNIE UPOMNIENIE\\ DAIT\=O KOKUSHI}{OSTATNIE UPOMNIENIE DAIT\=O KOKUSHI}
	{K\=ozen Dait\=o Kokushi Yuikai}

\bigskip
\noindent
Wy wszyscy, przyszliście do tego klasztoru by studiować Drogę, a~nie dla
zdobycia ubrań i~pożywienia. Dopóki macie ramiona, będziecie mieli co
nosić, dopóki będziecie mieli usta, będziecie mieli co jeść. Bądźcie zawsze
przytomni, przez cały dzień by z~całych sił stawać twarzą w~twarz z~tym co
niepojmowalne. Czas przemija jak strzała, więc pokornie proszę nie
pozwólcie by wasze umysły zakłócały ziemskie sprawy. Przejrzyjcie!
Przejrzyjcie! Gdy już ten stary mnich uda się na swą ostatnią pielgrzymkę,
niektórzy z~was mogą objąć duże i~prosperujące świątynie z~sutrami pisanymi
złotem i~srebrem z~salą Buddhy pełną od tłumu wiernych. Inni mogą śpiewać
sutry i~dharani, długo medytować bez kładzenia się, podążając za regułą
jednego posiłku o~świcie, praktykując drogę poprzez cały dzień, jeśli nawet
pełni poświęcenia spędzicie czas w~ten sposób lecz cudowna, subtelna,
nieprzekazywalna Droga Buddhów i~Patriarchów nie będzie w~waszych sercach to
stracicie karmiczny związek z~poprzednikami i~prawdziwa praktyka zen
upadnie. A~wy wszyscy zredukujecie się do poziomu plemienia złych duchów.
Bez względu na to ile czasu upłynie od śmierci tego starego mnicha nie mają
oni prawa zwać się moimi następcami. Jeśli jednak jeden z~was osiądzie na
pustkowiu spędzając życie w~małej chacie pokrytej cienką strzechą żywiąc
się korzonkami leśnych roślin gotowanymi w~garnku o~połamanych nogach,
jeśli odda się on szczerze i~wyłącznie własnemu duchowemu życiu, będzie on
właśnie tym, który spłaca dług wdzięczności temu staremu mnichowi
spotykając go codziennie twarzą w~twarz. Któż śmiałby pomniejszać
i~lekceważyć takiego kogoś? Wysilajcie się! Wysilajcie się!
\end{Prayer}

\newpage
%%%%%%%%%%%%%%%%%%%%%%%%%%%%%%%%%%%%%%%%%%%%%%%%%%%%%%%%%%%%%%%%%%%%%%%%%%%%%%%
%%%% EN TSUU DAI OO KOKU SHI YUI KAI, (Koonji, 73 - 75)
%\begin{Prayer}{en_tsuu_dai_oo_koku_shi_yui_kai}
%	{EN TSUU DAI OO KOKU SHI YUI KAI}{-}
%	{-}
%
%\bigskip
%\begin{JAPANESE}
%	!!!!!!!!!!!!!!!!!!!!!!!!!!!!!!!!!
%\end{JAPANESE}
%\end{Prayer}

%%%%%%%%%%%%%%%%%%%%%%%%%%%%%%%%%%%%%%%%%%%%%%%%%%%%%%%%%%%%%%%%%%%%%%%%%%%%%%%
%%%% CHUHO OSHO ZAYU NO MEI
%%%% 2025: Poprawki na podstawie "Rinzai Zen Rezitazionen"
\begin{Prayer}{chuho_osho_zayu_no_mei}
	{CH\=UH\=O OSH\=O ZAY\=U NO MEI}{-}
	{}

\bigskip
\begin{JAPANESE}
MASSE NO BIKU, KATACHI SHAMON NI NITE KOKORO NI ZANGI NAKU, MI NI HOUE O TSUKETE, OMOI ZOKUJIN NI SOMU. KUCHI NI KYOUTEN O JUSHITE KOKORO NI TONYOKU O OMOI, HIRU WA MYOURI NI FUKERI YO WA AIJAKU NI YOU. HOKA JIKAI O HYOU SHITE, UCHI MIPPON O NASU. TSUNENI SERO O ITONANDE NAGAKU SHUTSURI O BOUZU. HITOE NI MOUZOU O SHUUSHI, SUDE NI SHOUCHI O NAGEUTSU. \\

HITOTSU NI WA DOUSHIN KENGO NI SHITE SUBEKARAKU KENSHOU O YOUSUBESHI.\\
FUTATSU NI WA WATOU O GICHAKU SHITE SANTETSU O KAMU GA GOTOKU SEYO.\\
MITSU NI WA CHOU ZABUTON WAKI SEKI NI TSUKURU KOTO NAKARE.      \\
YOTSU NI WA BUSSO NO GO O MITE TSUNENI MIZUKARA ZANGI SEYO.\\
ITSUTSU NI WA KAITAI SHOUJOU NI SHITE SHINJIN O KEGASU KOTO NAKARE.\\
MUTTSU NI WA IGI JAKUJOU NI SHITE BOURAN O HOSHII MAMA NI SURU KOTO NAKARE.\\
NANATSU NI WA SHOUGO TEISEI KESHOU O KONOMU KOTO NAKARE.\\
YATSU NI WA HITO NO SHINZURU NASHI TO IEDOMO HITO NO SOSHIRI O UKURU KOTO NAKARE.\\
KOKONOTSU NI WA TSUNE NI JOUSHUU O TAZUSAE TE DOUSHA NO CHIRI O HARAE. \\
TOO NI WA DOUGYOU UMU KOTO NOUSHITE AKUMADE ONJIKI SURU KOTO NAKARE. \\

SHOUJI JIDAI, KOUIN OSHIMU BESHI. MUJOU JINSOKU, TOKI HITO O MATAZU. JINSHIN UKEGATASHI. IMA SUDENI UKU. BUPPOU KIKI GATASHI, IMA SUDE NI KIKU. KONOMI KONJOU NI MUKATTE DOSE ZUNBA SARA NI IZURE NO TOKORO NI MUKATTE KA KONOMI O DOSEN.
\end{JAPANESE}
\end{Prayer}

%%%%%%%%%%%%%%%%%%%%%%%%%%%%%%%%%%%%%%%%%%%%%%%%%%%%%%%%%%%%%%%%%%%%%%%%%%%%%%%
%%%% Życiowe zasady Ch\=uh\=o Osh\=o
\begin{Prayer}{zyciowe_zasady_chuho_osho}
	{ŻYCIOWE ZASADY CH\=UH\=O OSH\=O}{-}
	{Ch\=uh\=o Osh\=o Zay\=u No Mei\footnote{Pełny tekst z przypisami: \url{https://mahajana.net/teksty/chuho-osho-zayu-no-mei.pdf}}}

\bigskip
\noindent%
W czasach upadku Dharmy, mnisi zewnętrznie przypominają praktykujących, noszą szaty Dharmy, lecz nie mają w sercu wstydu.
Choć odziani w szaty buddyjskie, ich myśli skalane są kurzem świata.
Ustami recytują sutry, ale w sercu pielęgnują chciwość.
W dzień gonią za sławą i zyskiem, w nocy upajają się przywiązaniami.
Na zewnątrz pokazują przestrzeganie wskazań, lecz w ukryciu je łamią.
Cały czas zajęci są sprawami świeckimi, zapominając całkowicie o wyzwoleniu.
Kurczowo trzymają się iluzji, już dawno porzuciwszy prawdziwą mądrość.\\

\noindent%
Po pierwsze, utrzymuj silną wolę przebudzenia i dąż wytrwale do poznania swojej prawdziwej natury.\\
Po drugie, trzymaj się koanu z niewzruszoną uwagą, jakbyś gryzł kawałek surowego żelaza.\\
Po trzecie, nie opieraj się na boku poduszki, nie rozsiadaj się wygodnie – siedź prosto i czujnie.\\
Po czwarte, nieustannie rozważaj słowa Buddhów i patriarchów i zawsze odczuwaj wobec nich skruchę.\\
Po piąte, zachowuj czystość wskazań, nie kalaj ciała i umysłu.\\
Po szóste, postępuj z godnością i spokojem, unikaj hałaśliwości i nieokiełznania.\\
Po siódme, mów mało i cicho, nie ulegaj pustemu śmiechowi ani żartom.\\
Po ósme, nie szukaj aprobaty innych, nie przejmuj się krytyką czy obmową.\\
Po dziewiąte, zawsze miej przy sobie miotłę, by usuwać kurz z sali medytacyjnej i otoczenia.\\
Po dziesiąte, nigdy nie ustawaj w praktyce, nie objadaj się ani nie opijaj do syta.\\

\noindent%
Narodziny i śmierć to sprawy najwyższej wagi. Nie marnuj ani chwili.\\
Nietrwałość działa szybko, czas nie czeka na nikogo.\\
Uzyskanie ludzkiego ciała jest rzadkością – a jednak je otrzymałeś.\\
Usłyszenie Dharmę Buddhy jest trudne – a jednak ją słyszysz.\\
Jeśli nie przekroczysz na drugi brzeg w tym właśnie życiu, to kiedy i gdzie wyzwolisz to ciało?
\end{Prayer}

\newcommand{\InGathaTitle}[1]{%
	\bigskip
	{
	\leftskip 0pt
	\parindent 0pt
	\noindent \textsl{#1}\par\nopagebreak
	}
}

\newcommand{\InGathaBoldTitle}[1]{%
	\bigskip
	{
	\leftskip 0pt
	\parindent 0pt
	\noindent \textbf{#1}\par\nopagebreak
	}
}

	%%%%%%%%%%%%%%%%%%%%%%%%%%%%%%%%%%%%%%%%%%%%%%%%%%%%%%%%%%%%%%%%%%%%%%%%%%%%%%%
%%%% JAKUSHITSU ZENJI (SHŌTŌ KOKUSHI) YUIKAI
\begin{Prayer}{jakushitsu_zenji_yuikai}
	{JAKUSHITSU ZENJI (SHŌTŌ KOKUSHI) YUIKAI}{-}
	{}

\bigskip
\begin{JAPANESE}
RŌSETSU. IMA SEEN MASANI TSUKINAN TOSU. YOTTE MOROMORO NO HŌZOKURA NI
KOMEISU. YO GA KŌZEN NO NOCHI WO MATTE. YOROSHIKU SUBEKARAKU RINKA NI ATO
WO KURAMASHI. KASHU TŌKŌ SHITE. ISSHŌ WO OENAN KOTO WO HAKARUBESHI. KAIKYŌ
NI IWAKU. MASANI KAINYŌ WO HANARETE. SANKAN KŪTAKU NI DOKUSHO KANKYO
SUBESHI TO UN NEN.\\

KORE SUNAWACHI WAGA HOTOKE SAIGO NO JIKUN NARI. NANZO JUNPŌ SEZARU BESHI.
NANJIRA KAKKAKU SHŌGON GONSHU SEYO. KOINEGAWAKUWA. KESA NO MOTO NI MUKATTE.
SHINJIN WO SHIKKYAKU SEZARAN KOTO WO. KORE YO GA FUKAKU NANJI GA TOMOGARA
NI NOZOMU TOKORO NARI. NANJIRA YO GA KI NO TAYURU WO MIREBA. KYŪNI
SUBEKARAKU SHŪHEN SUBESHI. SETSUNI YUIGAI WO TODOMETE MOTTE. HITO WO SHITE
KORE WO MISESHIMURU KOTO NAKARE. TSUCHI WO ŌI. ISHI WO TATAMU KOTO SUDENI
OWARABA. DŌSHI NI SUSUMETE. TADA SHURYOŌGON JINSHŪ IPPEN WO FUZEN NOMI.
SHIKASHITE NOCHI YUGEN WO TOTTE TAISHU NI KAESHI. BŌAN WO MOTTE KŌYA NO
FURŌRA NI FUYOSHITE KAKUZU NI SANJISARE. FURŌ MOSHI MATA KOJI NO I ARABA.
NANJIRA MOROMORO NO DŌYŪ TO AIGISHITE. ICHIRŌSEI NO SHUKUNŌ WO SHŌSHITE
MOTTE ANJU NI ATETE. TA NO SAISUI NO BENDŌ WO TAZUNURU TEI NO UNSUI HINDEI
NO TAMENI. ICHIGE ITŌ ANZEN BENDŌ NO SHOZAI TO NAKISAN MO MATA KA NARI. YO
WA MATA IUBEKI KOTO NASHI. YUIZOKU YUIZOKU.\\
\end{JAPANESE}
\end{Prayer}

%%%%%%%%%%%%%%%%%%%%%%%%%%%%%%%%%%%%%%%%%%%%%%%%%%%%%%%%%%%%%%%%%%%%%%%%%%%%%%%
%%%% Ostatnie Pouczenia Mistrza Jakushitsu
\begin{Prayer}{ostatnie_pouczenia_mistrza_jakushitsu}
	{OSTATNIE POUCZENIA MISTRZA JAKUSHITSU}{-}
	{Jakushitsu Zenji Yuikai\footnote{Pełny tekst z przypisami: \url{https://mahajana.net/teksty/jakushitsu-zenji-yuikai.pdf}}}

\bigskip
\noindent%
Jużem stary i przyszło mi wyczerpać związki [karmiczne] łączące mnie ze światem. Przekażę zatem wam, mym współbraciom w dharmie, ostatnią swą wolę. Po tym jak odejdę, żyjcie na polanach, uprawiając ziemię i skrywając swoje ślady pośród kęp drzew. W Sutrze zostało powiedziane: „Stroniąc od zgiełku [tego świata], żyjcie samotnie w cichym i spokojnym miejscu w górach”. Takie jest oto ostatnie łaskawe pouczenie Buddy. Czemuż więc nie przestrzegać tego z szacunkiem? Niechaj każdy z was czyni wysiłki w praktyce i proszę was byście nie zaprzepaścili tego ciała schowanego pod szatą. To moje głębokie pragnienie. Zauważywszy, żem wyzionął ducha, niezwłocznie pogrzebcie me zwłoki. Pod żadnym pozorem nie wolno wam pokazywać ich ludziom. A po tym, jak je pochowacie, usypcie kopiec z ziemi, połóżcie nań kamienie i poproście tych, którzy są wam mili, by wraz z wami wyrecytowali raz jeden zaklęcie śūraṃgama. Następnie oddajcie cały majątek i ziemię patronowi, świątynię ofiarujcie starszyźnie ze wsi Takano, a sami rozejdźcie się w swoje strony. Jeśli jednak starszyzna stanowczo odmówi, naradźcie się wspólnie i zaproście doświadczonego w praktyce, by objął pieczę nad świątynią. Sami zatroszczcie się o drewno na opał i wodę, by wspólnie ze zgromadzeniem tych, co przemierzają świat niczym chmury, niczym płynąca woda, móc wytężać się w praktyce medytacji zarówno latem, jak i zimą. Poza tym nie mam wam nic do powiedzenia. Taka ma ostatnia prośba.
        
\end{Prayer}

%%%%%%%%%%%%%%%%%%%%%%%%%%%%%%%%%%%%%%%%%%%%%%%%%%%%%%%%%%%%%%%%%%%%%%%%%%%%%%%
%%%% IDA TEN FU GIN
\begin{Prayer}{idaten_fugin}
	{IDATEN FUGIN}{-}
	{Dharani dla Idaten (skr. Skanda), strażnika świątyni}

\begin{Verse}
	(7x każdy wers)\\
\stanza{
	ON IDA TE-TA MOKO TE-TA SOWAKA\\
	ON MOKO KYARAYA SOWAKA\\
	ON KENBAYA KENBAYA SOWAKA\\
	ONBA SANBA ENTEI SHUYAJIN SOWAKA\\
}
\end{Verse}
\end{Prayer}

%%%%%%%%%%%%%%%%%%%%%%%%%%%%%%%%%%%%%%%%%%%%%%%%%%%%%%%%%%%%%%%%%%%%%%%%%%%%%%%
%%%% NY\=UYOKU NO GE
\begin{Prayer}{nyuyoku_no_ge}
	{NY\=UYOKU NO GE}{-}
	{Strofa kąpieli}


\InGathaTitle{Przed kąpielą:}

\begin{Verse}
\stanza{
	MOKU YOKU SHIN TAI	\\
	T\=O GAN SHU J\=O	\\
	SHIN JIN MU KU		\\
	NAI GE K\=O KETSU.
}
\end{Verse}

\medskip
\noindent
Zamierzamy wziąć kąpiel dla dobra wszystkich żyjących istot;
Oby nasze ciała i~umysły zostały oczyszczone wewnętrznie i~zewnętrznie.
\end{Prayer}

%%%%%%%%%%%%%%%%%%%%%%%%%%%%%%%%%%%%%%%%%%%%%%%%%%%%%%%%%%%%%%%%%%%%%%%%%%%%%%%
%%%% SENMEN NO GE
\begin{Prayer}{senmen_no_ge}
	{SENMEN NO GE}{-}
	{Strofa mycia twarzy}

\InGathaTitle{Biorąc szczoteczkę do zębów:}
\begin{Verse}
\stanza{
	SHU J\=U Y\=O JI	\\
	T\=O GAN SHU J\=O	\\
	SHIN TOKU SH\=O B\=O	\\
	JIN NEN SH\=O J\=O.
}
\end{Verse}

\medskip
\noindent
Bierzemy szczoteczkę do zębów dla dobra wszystkich żyjących istot;
Obyśmy wszyscy szybko zrozumieli prawdę i~stali się naturalnie czyści.

\InGathaTitle{Myjąc zęby:}
\begin{Verse}
\stanza{
	SHIN SHAKU Y\=O JI	\\
	T\=O GAN SHU J\=O	\\
	TOKU CH\=O BUKU GE	\\
	ZEI SHO BON N\=O.
}
\end{Verse}

\medskip
\noindent
Czyścimy nasze zęby tego ranka dla dobra wszystkich żyjących istot;
Ponieważ są pod wpływem korzenia iluzji, obyśmy rozgnietli iluzję, tak jak
ta szczoteczka rozgniata się na zębach.

\InGathaTitle{Płucząc jamę ustną:}
\begin{Verse}
\stanza{
	SO SO KU SHI		\\
	T\=O GAN SHU J\=O	\\
	KO J\=O H\=O MON	\\
	KU GY\=O GE DATSU.
}
\end{Verse}

\medskip
\noindent
Płuczemy usta po szczotkowaniu dla dobra wszystkich żyjących istot;
Obyśmy zbliżyli się do najwyższych nauk dla naszego wyzwolenia.

\InGathaTitle{Myjąc twarz:}
\begin{Verse}
\stanza{
	I~SUI SEN MEN		\\
	T\=O GAN SHU J\=O	\\
	TOKU J\=O H\=O MON	\\
	Y\=O MU KU ZEN.
}
\end{Verse}

\medskip
\noindent
Myjemy nasze twarze czystą wodą dla dobra wszystkich żyjących istot;
Obyśmy urzeczywistnili najwyższe nauki i~byli na zawsze wolni od przywiązań.
\end{Prayer}

\newpage
%%%%%%%%%%%%%%%%%%%%%%%%%%%%%%%%%%%%%%%%%%%%%%%%%%%%%%%%%%%%%%%%%%%%%%%%%%%%%%%
%%%% HONZONJ\=OGU EK\=O
\begin{Prayer}{honzonjogu_eko}
	{HONZONJ\=OGU EK\=O}{-}
	{-}

\begin{Verse}
\stanza{
	J\=ORAI, MAKAHANNYA HARAMITTA SHINGY\=O O~FUJU SURU KUDOKU WA, \shokei\\
	DAION KY\=OSHU HONSHI SHAKAMUNIBUTSU,\\
	K\=OSO J\=OY\=O DAISHI, TAISO J\=OSAI DAISHI\\
	NI KUYO SHI TATEMATSURI,\\
	MUJ\=O BUK-KA BODAI O~SH\=OGON SU. \shokei\\
	FU SHITE NEGAWAKU WA, SHION SUBETE H\=OJI,\\
	SANNU HITOSHIKU TASUKE, HOK-KAI NO UJ\=O TO,\\
	ONAJIKU SHUCHI O~MADOKA NI SEN KOTO O.
}

\stanza{
	\rightskip 0pt
	\parindent 0pt
	\leftskip  0pt
	Ofiarowujemy zasługi tej recytacji Sutry Serca Wielkiej Doskonałej
	Mądrości Siakjamuniemu Buddzie, K\=oso J\=oy\=o Daishi i~Taiso
	J\=osai Daishi.  Modlimy się, abyśmy mogli okazać naszą wdzięczność
	czterem dobroczyńcom, ocalić wszystkie istoty w~trzech światach
	i~osiągnąć doskonałość czterech mądrości wraz ze wszystkimi
	istotami. Modlimy się o~dobro wszystkich istot i~ustanie
	nieszczęść.
}
\end{Verse}
\end{Prayer}

%%%%%%%%%%%%%%%%%%%%%%%%%%%%%%%%%%%%%%%%%%%%%%%%%%%%%%%%%%%%%%%%%%%%%%%%%%%%%%%
%%%% HONZON EK\=O
\begin{Prayer}{honzon_eko}
	{HONZON EK\=O}{-}
	{Ofiarowanie}

\begin{Verse}\wersaliki
\stanza{
	JI HO SAN SHI I~SHI FU\\
	SHI SON BU SA MO KO SA\\
	MO KO HO JA HO RO MI
}
\end{Verse}

\begin{Verse}
\stanza{
	\keisu Wszyscy Buddhowie, Wszyscy Szlachetni Bodhisattwowie,
	\keisu\footnotemark\ Mahasattwowie,\\
	
	W dziesięciu kierunkach, trzech światach,\\

	\keisu Wielka Doskonała Mądrość.
}\footnotetext{Uderzenie w~keisu tylko przy trzecim ofiarowaniu.}
\end{Verse}
\end{Prayer}

\newpage
%%%%%%%%%%%%%%%%%%%%%%%%%%%%%%%%%%%%%%%%%%%%%%%%%%%%%%%%%%%%%%%%%%%%%%%%%%%%%%%
%%%% FUEK\=O
\begin{Prayer}{fueko}
	{FUEK\=O}{-}
	{-}

\begin{center}
	\scriptsize
	(język japoński)
\end{center}

\begin{Verse}
\stanza{\wersaliki
	NEGAWAKU-WA KONO-KUDOKU-O MOTTE\\
	AMANEKU ISSAI-NI OYOBOSHI\\
	WARERA-TO SHUJ\=O-TO-MINATOMONI\\
	BUTSUD\=O-O J\=OZEN-KOTO-O.
}

\stanza{\wersaliki
	JI HO SAN SHI I~SHI FU\\
	SHI SON BU SA MO KO SA\\
	MO KO HO JA HO RO MI
}

\stanza{
	Modlimy się aby ta nagromadzona cnota zasługi przeniknęła wszędzie.\\
	Obyśmy wszyscy razem z~odczuwającymi istotami osiągnęli Drogę Buddhy.
}

\stanza{
	Wszyscy Buddhowie, Wszyscy Szlachetni Bodhisattwowie, Mahasattwowie,\\
	W dziesięciu kierunkach, trzech światach,\\
	Wielka Doskonała Mądrość.
}
\end{Verse}
\end{Prayer}

\newpage
%%%%%%%%%%%%%%%%%%%%%%%%%%%%%%%%%%%%%%%%%%%%%%%%%%%%%%%%%%%%%%%%%%%%%%%%%%%%%%%
%%%% ZWROT ZASŁUG !
\begin{Prayer}{zwrot_zaslug}
	{ZWROT ZASŁUG}{-}
	{-}

\bigskip
\begin{Verse}
	W OPARCIU O KORZEŃ DOBRA PRAKTYKI OBECNEGO ZGROMADZENIA\\
	ODPOWIADAMY NA CNOTĘ ZASŁUGI WYSIŁKU NASZYCH OJCÓW I MATEK,\\
	BY CI, KTÓRZY JESZCZE ŻYJĄ, OSIĄGNĘLI NIEZMIERZONE SZCZĘŚCIE I ŻYCIE,\\
	OBY CI, KTÓRZY ZMARLI, ODDZIELILI SIĘ OD CIERPIENIA I OSIĄGNĘLI DOBRE ODRODZENIE.\\
	NIECH WSZYSTKIE ISTOTY W TRZECH ŚWIATACH, KTÓRE OTRZYMUJĄ CZTERY BŁOGOSŁAWIEŃSTWA,\\
	I TE, KTÓRE CIERPIĄ W TRZECH ZŁYCH KRAINACH ORAZ CIERPIĄCE Z POWODU OŚMIU SKALAŃ,\\
	WYRAŻĄ SKRUCHĘ Z POWODU SWOICH WSZYSTKICH ZŁYCH CZYNÓW I OCZYSZCZĄ SIĘ ZE SWOICH ZANIECZYSZCZEŃ,\\
	TAK BY ZOSTAŁY UWOLNIONE Z KOŁA ODRADZANIA SIĘ W ŚWIECIE CIERPIENIA SAHA I NARODZIŁY SIĘ W CZYSTEJ KRAINIE
\end{Verse}
\end{Prayer}

\newpage
%%%%%%%%%%%%%%%%%%%%%%%%%%%%%%%%%%%%%%%%%%%%%%%%%%%%%%%%%%%%%%%%%%%%%%%%%%%%%%%
%%%% Dwie modlitwy recytowane szczególnie ...

\begin{center}
Dwie modlitwy recytowane szczególnie w Dzień Narodzin Siakiamuniego Buddhy
\end{center}

%%%%%%%%%%%%%%%%%%%%%%%%%%%%%%%%%%%%%%%%%%%%%%%%%%%%%%%%%%%%%%%%%%%%%%%%%%%%%%%
%%%% MODLITWY PRZY PRAKTYCE MISEK BUDDHY W CZASIE POSIŁKU
\begin{Prayer}{-}
	{MODLITWY PRZY PRAKTYCE MISEK BUDDHY W CZASIE POSIŁKU}{-}
	{-}

\begin{flushleft}
Wszyscy:
\end{flushleft}

\begin{Verse}
	Buddha urodził się w Kapilawastu,\\
	Osiągnął Drogę w Magadha,\\
	Wyjaśniał Dharmę w Waranasi,\\
	Wszedł w Nirwanę w Kushinagara.\\
	Teraz rozwijam miski Tathagaty,\\
	Ślubuję razem z całym zgromadzeniem\\
	Urzeczywistnić pustkę i nirwanę trzech kół.
\end{Verse}

\begin{flushleft}
Prowadzący:
\end{flushleft}

\begin{Verse}
	Z czcią myśląc o Trzech Klejnotach,\\
	Które pieczętują zrozumienie,\\
	Polecając się Szlachetnemu Zgromadzeniu przywołujmy w umyśle:
\end{Verse}

\begin{flushleft}
Wszyscy:
\end{flushleft}

\begin{center}
	DZIESIĘĆ IMION BUDDHÓW
\end{center}
\begin{Verse}
	Całkowicie Czysta Dharmakaya Wairoczana Buddha,\\
	Doskonale Pełne Ciało Odpowiedzi Sambhogakaya Wairoczana Buddha,\\
	Dziesięć Tysięcy Bilionów Ciał Przemienienia Nirmanakaya Siakiamuni Buddha,\\
	Mający narodzić się w przyszłości Maitreja Buddha,\\
	Wszyscy Buddhowie w dziesięciu kierunkach trzech światach,\\
	Mahajana Saddharma Pundarika Sutra,\\
	Wielki Święty Mańdziuśri Bodhisattwa,\\
	Wielki Pojazd Mahajana Samantabhadra Bodhisattwa,\\
	Wielki Współczujący Mahakaruna Awalokiteśwara Bodhisattwa,\\
	Wszyscy Szlachetni Bodhisattwowie Mahasattwowie,\\
	Wielka Doskonała Mądrość.
\end{Verse}

\begin{flushleft}
Prowadzący:
\end{flushleft}

\begin{center}
	Śniadanie
\end{center}
\begin{Verse}
	Poranny posiłek dziesięciu dobrodziejstw\\
	Żywi praktykujących.\\
	Jego owoce są niezmierzone,\\
	Wypełniają nas wieczną radością.
\end{Verse}

\begin{center}
	Obiad
\end{center}
\begin{Verse}
	Trzy cnoty i sześć smaków\\
	Ofiarowujemy Buddzie i Sandze.\\
	Wszystkie istoty Dharamdhatu\\
	Tak samo czynią ofiarę.
\end{Verse}

\begin{flushleft}
Wszyscy:
\end{flushleft}

\begin{center}
	MODLITWA PIĘCIU KONTEMPLACJI
\end{center}
\begin{Verse}
	Po pierwsze rozważamy, jak dzięki wielkim i małym zasługom ten pokarm tu się pojawił.\\
	Po drugie rozważamy naszą zasługę i praktykę i to,\\
	czy zasługujemy na ten posiłek.\\
	Po trzecie rozważamy pożądanie jako przeszkodę w umyśle.\\
	Po czwarte rozważamy, że to jedzenie jest prawdziwym\\
	dobrym lekarstwem, podtrzymującym nasze życie.\\
	Po piąte rozważamy, że przyjmujemy\\
	ten posiłek po to, by osiągnąć Drogę.
\end{Verse}

\begin{center}
	Ofiarowanie duchom
\end{center}
\begin{Verse}
	Wy, wszystkie zgromadzenia duchów, demonów i bóstw,\\
	Teraz czynimy dla was ofiarę.\\
	To jedzenie wypełnia dziesięć kierunków\\
	I jest ofiarowane wszystkim demonom i bóstwom.
\end{Verse}

\begin{center}
	Modlitwa ofiarowania miski
\end{center}
\begin{Verse}
	Najwyższym Trzem Klejnotom,\\
	Średnim Czterem Błogosławieństwom,\\
	Najniższym sześciu ścieżkom istnienia,\\
	Wszystkim tak samo czynimy ofiarowanie.\\
	Po pierwsze, jemy, by odciąć całe zło,\\
	Po drugie, jemy, by praktykować całe dobro,\\
	Po trzecie, jemy, by wyzwolić odczuwające istoty,\\
	Tak, byśmy razem ze wszystkimi istotami osiągnęli Drogę Buddhy.
\end{Verse}

\begin{center}
	Modlitwa mycia misek
\end{center}
\begin{Verse}
	Woda, którą myliśmy miski,\\
	Jest jak smak niebiańskiej amrity.\\
	Ofiarowujemy ją wszystkim zgromadzeniom duchów, demonów i bóstw,\\
	Tak, by osiągnęły pełną satysfakcję.\\
	OM MAKURA SAI SOWA KA!
\end{Verse}

\begin{flushleft}
	Prowadzący:
\end{flushleft}


\begin{center}
	MODLITWA CZYSTOŚCI MIMO POZOSTAWANIA W TYM ŚWIECIE.	
\end{center}
\begin{Verse}
	Przebywanie w tym świecie, podobnym do pustej przestrzeni,\\
	Jest podobne do lotosu w błotnistej wodzie,\\
	Całkowicie czysty Umysł przekracza świat.\\
	Czynimy pełny pokłon przed nieprześcignionym Bhagawatem.
\end{Verse}
\end{Prayer}

%%%%%%%%%%%%%%%%%%%%%%%%%%%%%%%%%%%%%%%%%%%%%%%%%%%%%%%%%%%%%%%%%%%%%%%%%%%%%%%
%%%% OFIAROWANIE PANU BUDDZIE
\begin{Prayer}{ofiarowanie_panu_buddzie}
	{OFIAROWANIE PANU BUDDZIE}{-}
	{-}

\bigskip
\begin{Verse}
\stanza{
	Zebraną cnotę zasługi recytacji \ldots\ zwracamy:\\
	\shokei {\scriptsize (ukłon)} Wielkiemu Błogosławionemu Mistrzowi Religii Pierwotnemu Mistrzowi Siakjamuniemu Buddzie,\\
	Mistrzowi Tej Świątyni Tathagacie,\\
	Wysokiemu Patriarsze, Wielkiemu Mistrzowi Dziojo,\\
	Wielkiemu Patrairsze, Wielkiemu Mistrzowi Dziosai,\\
	czcząc Oświecenie~-- nieprześcigniony owoc Stanu Buddhy. \shokei
	{\scriptsize (koniec ukłonu)}\\
}

\stanza{
	Ofiarowujemy błogosławieństwo recytacji wszystkim niebiańskim obrońcom Nauki Buddhy,\\
	Świętym Mędrcom broniącym Naukę Buddhy,\\
	boskim obrońcom tej świątyni {\scriptsize (lub odpowiednio: tego miejsca praktyki)} i~jej {\scriptsize (jego)} ziemi oraz Daigensiuri Bodhisattwie,\\
	prawdziwym zarządcom świątyni.\\
	Modlimy się o~pokój naszego kraju, pokój i~harmonię wszystkich krajów,\\
	długie i~szczęśliwe życie wyznawców w~dziesięciu kierunkach,\\
	spokój tej świątyni {\scriptsize (tego miejsca praktyki)}, bezpieczeństwo i~szczęście oceanu istot.\\
	Obyśmy wszyscy wraz z~żyjącymi istotami w~świecie zjawisk osiągnęli doskonałą mądrość.
}
\end{Verse}
\end{Prayer}

%%%%%%%%%%%%%%%%%%%%%%%%%%%%%%%%%%%%%%%%%%%%%%%%%%%%%%%%%%%%%%%%%%%%%%%%%%%%%%%
%%%% OFIAROWANIE ARHATOM
\begin{Prayer}{ofiarowanie_arhatom}
	{OFIAROWANIE ARHATOM}{-}
	{-}

\bigskip
\begin{Verse}
	Modlimy się przed zwierciadłem oświecenia, czyniąc pokłon czcimy błogosławieństwo mocy i~odpowiedzi na modlitwy spływające od Buddhów i~Bodhisattwów.\\
	Zasługę recytacji \ldots\ zwracamy wiecznym Trzem Klejnotom w~dziesięciu kierunkach,\\
	Świętym Mędrcom niezliczonych oceanów urzeczywistnienia,\\
	Szesnastu Wielkim Arhatom, wszystkim, którzy odpowiadają na nasze ofiarowania.
	Modlimy się o~Trzy Mądrości i~Sześć Mocy, aby okres upadku Nauki Buddhy zmienił się w~okres Prawdziwej Dharmy,\\
	Aby Pięć Mocy i~Osiem Wglądów ukazało niezliczonym istotom stan nie--narodzenia, aby w~tej świątyni {\scriptsize (w~tym miejscu praktyki)} obracało się Koło Dharmy i~Koło Wsparcia dla mnichów, uczniów Buddhy,\\
	Aby kataklizmy wody, ognia i~wiatru nigdy nie spotkały tego kraju.
\end{Verse}
\end{Prayer}

%%%%%%%%%%%%%%%%%%%%%%%%%%%%%%%%%%%%%%%%%%%%%%%%%%%%%%%%%%%%%%%%%%%%%%%%%%%%%%%
%%%% OFIAROWANIE PATRIARCHOM
\begin{Prayer}{ofiarowanie_patriarchom}
	{OFIAROWANIE PATRIARCHOM}{-}
	{-}

\bigskip
\begin{Verse}
\stanza{
	Z~czcią modlimy się do prawdziwie współczujących\\
	czynimy pokłon czcząc oświecone zwierciadło.\\
	Zebrane zasługi recytacji \ldots\ ofiarowujemy wszystkim pokoleniom patriarchów przekazującym Lampę Dharmy:

	\medskip
	$\cdots$ {\scriptsize (linia Przekazu Dharmy)} $\cdots$

	\medskip
	Obyśmy zwrócili naszą wdzięczność za otrzymane najwyższe błogosławieństwo współczucia.
}
\end{Verse}
\end{Prayer}

\newpage
%%%%%%%%%%%%%%%%%%%%%%%%%%%%%%%%%%%%%%%%%%%%%%%%%%%%%%%%%%%%%%%%%%%%%%%%%%%%%%%
%%%% OFIAROWANIE ŚWIĄTYNI
\begin{Prayer}{ofiarowanie_swiatyni}
	{OFIAROWANIE ŚWIĄTYNI}{-}
	{-}

\bigskip
\begin{Verse}
	Modlimy się do Trzech Klejnotów. Czyniąc pokłon oddajemy cześć Zwierciadłu Oświecenia. Cnotę zasługi płynącą z~recytacji \ldots\ zwracamy wszystkim zmarłym mnichom i~mniszkom (z~tej świątyni, zmarłym mnichom i~mniszkom) na całym świecie (dharmadhatu) oraz wszystkim opiekunom tego miejsca praktyki i~opiekunom Dharmy.\\
	Wszystkim duchom zmarłych w~nagłych wypadkach, śmiercią męczeńską, śmiercią samobójczą, z~powodu morderstw i~wojen we wszystkich krajach.\\
	Wszystkim rodzinom i~osobom przyczynowo związanym z~tą świątynią {\scriptsize (tego miejsca praktyki)}, Sześciu Rodzinom i~Siedmiu Przeszłym Pokoleniom matek i~ojców czystego zgromadzenia (na tym sesshin / mnichów i~mniszek z~tej świątyni) oraz wszystkim żyjącym istotom. Obyśmy wszyscy razem osiągnęli Doskonałe Oświecenie.
\end{Verse}
\end{Prayer}

%%%%%%%%%%%%%%%%%%%%%%%%%%%%%%%%%%%%%%%%%%%%%%%%%%%%%%%%%%%%%%%%%%%%%%%%%%%%%%%
%%%% DWA OFIAROWANIA
\begin{Prayer}{dwa_ofiarowania}
	{DWA OFIAROWANIA}{-}
	{-}

\InGathaBoldTitle{Ofiarowanie pierwsze}

\begin{Verse}
\stanza{
	Pokornie zwracamy się do światła Trzech Klejnotów.\\
	Zasługi \ldots\ ofiarowujemy wszystkim zmarłym mnichom w~światach Dharmy,\\
	wszystkim poległym na wojnach,\\
	wszystkim zmarłym członkom rodzin uczestników zazen,\\
	wszystkim odczuwającym istotom na całym świecie.\\
	Obyśmy wszyscy osiągnęli doskonałe oświecenie.
}
\end{Verse}

\newpage
\InGathaBoldTitle{Ofiarowanie drugie}

\bigskip
\begin{Verse}
	Z czcią ofiarowujemy zasługi śpiewów \ldots\\
	Wielkiemu Mistrzowi i~Założycielowi Siakjamuniemu Buddzie\footnote{Jeśli na butsudanie zamiast Buddhy Siakjamuniego jest inny Buddha lub Bodhisattwa to wymienia się jego imię.},\\
	Wielkiemu Mistrzowi Bodhidharmie Daishi Daiosho,\\
	Wielkiemu Mistrzowi Sokei Eno Zenji Daiosho,\\
	Wielkiemu Mistrzowi Rinzai Gigen Zenji Daiosho,\\
	Wielkiemu Mistrzowi T\=ozan Ryokai Zenji Daiosho,\\
	Wielkiemu Mistrzowi Myoan Eisai Zenji Daiosho,\\
	Wielkiemu Mistrzowi Eihei Dogen Zenji Daiosho,\\
	Wielkiemu Mistrzowi Keizan Jokin Zenji Daiosho,\\
	Wielkiemu Mistrzowi Hakuin Ekaku Zenji Daiosho.\\
	i~wszystkim pokoleniom Buddhów i~Patriarchów przekazującym światło Dharmy.
\end{Verse}

\bigskip
\begin{Verse}
	Ofiarowujemy zasługi tych śpiewów wiecznym Trzem Klejnotom w~dziesięciu kierunkach.\\
	Obyśmy mogli spłacić ich współczujące błogosławieństwo.\\
	Opiekuńczym bóstwom tej świątyni {\scriptsize (lub odpowiednio: tego miejsca praktyki)}, chroniącym Dharmę Dewom i~wszystkim bóstwom.\\
	Oby Prawdziwa Dharma zawsze rozkwitała. Oby pokój i~harmonia istniała między narodami i~w tej świątyni {\scriptsize (lub: w~tym miejscu praktyki)}.\\
	Oby wszyscy osiągnęli szczęście i~pomyślność.
\end{Verse}
\end{Prayer}

\newpage
%%%%%%%%%%%%%%%%%%%%%%%%%%%%%%%%%%%%%%%%%%%%%%%%%%%%%%%%%%%%%%%%%%%%%%%%%%%%%%%
%%%% OFIAROWANIE GŁÓWNEMU BUDDZIE NA OŁATARZU
\noindent
To jest ofiarowanie, które robi się codziennie przed rodzinnym ołtarzem w
domu. Można zwiększyć ilość recytacji tekstów. Tradycyjnie recytuje się
Sutrę Serca, ale można recytować dowolne Sutry i Dharani.

\begin{Prayer}{-}
	{OFIAROWANIE GŁÓWNEMU BUDDZIE NA OŁTARZU}{-}
	{-}

\bigskip
\begin{Verse}
Zgromadzoną zasługę recytacji:\\
{\it Sutry Serca Wielkiej Doskonałej Mądrości}\\
z czcią ofiarowujemy:\\
Wielkiemu Błogosławionemu Panu Doktryny\\
Pierwotnemu Mistrzowi Siakiamuniemu Buddzie\\
{\scriptsize({\it w tym miejscu powinno się wyrecytować imię Buddhy lub Bodhisattwy, który jest czczony na ołtarzu, przed którym się modlimy})}\\
Głównemu Buddzie/Bodhisattwie \dotfill\\
Wysokiemu Patriarsze Wielkiemu Mistrzowi Dziojo,\\
Wielkiemu Patriarsze Wielkiemu Mistrzowi Dziosai,\\
ozdabiając Oświecenie najwyższy owoc Stanu Buddhy.\\
Modlimy się i ślubujemy z szacunkiem, odpłacić cztery błogosławieństwa, uratować istoty w trzech światach, i wraz ze  wszystkimi istotami Dharmadhatu jednakowo i całkowicie urzeczywistnić wszystkie mądrości.\\
Jednocześnie modlimy się o pomyślność Rodziny
{\scriptsize({\it należy wyrecytować nazwisko Rodziny, o pomyślność której się modlimy})}\\
\dotfill\\
długie życie jej potomków, wolność od nieszczęść i~o~wszystkie przyczyny szczęścia.\\

\medskip
\samepage{
	WSZYSCY BUDDHOWIE\\
	WSZYSCY SZLACHETNI\\
	BODHISATTWOWIE MAHASATTWOWIE\\
	W DZIESIĘCIU KIERUNKACH TRZECH ŚWIATACH\\
	WIELKA DOSKONAŁA MĄDROŚĆ
}
\end{Verse}
\end{Prayer}

\newpage
%%%%%%%%%%%%%%%%%%%%%%%%%%%%%%%%%%%%%%%%%%%%%%%%%%%%%%%%%%%%%%%%%%%%%%%%%%%%%%%
%%%% OFIAROWANIE DLA ZMARŁYCH PRZODKÓW
\noindent
To jest ofiarowanie dla naszych rodziców, dziadków, krewnych czy
przyjaciół, które robimy w rocznicę śmierci, lub co miesiąc w dzień kiedy
osoba zmarła.

\begin{Prayer}{-}
	{OFIAROWANIE DLA ZMARŁYCH PRZODKÓW}{-}
	{-}

\parskip 0pt
Odświeżający księżyc Bodhisattwy, doświadcza w rzeczywistości pustkę, odczuwające istoty oczyszczając wodę umysłu, pojawiają się w świetle Oświecenia -- Bodhi.\\
Z czcią modlimy się do Trzech Klejnotów, czyniąc pokłon czcimy oświecające zwierciadło.\\
Zebraną zasługę recytacji\\
{\scriptsize({\it należy wyrecytować tytuł Sutry, Dharani, którą recytowaliśmy dla zmarłego})}\\
\null\dotfill\null\\
zwracamy\\
{\scriptsize({\it teraz należy podać imię Dharmy zmarłej osoby lub jeśli nie ma imienia Dharmy to imię i nazwisko})}\\
\null\dotfill\null\\
oraz czystym duchom pokoleń zmarłych Rodziny\\
{\scriptsize({\it podać nazwisko Rodziny})}\\
\null\dotfill\null\\
siedmiu pokoleniom matek i ojców oraz ich sześciu rodzajom krewnych, duchom w trzech światach przyczynowo związanym i niezwiązanym z ofiarodawcami.\\
Modlimy się o to, by zakończyły się dla niego/niej kalpy ciemności i niewiedzy, a osiągnął/a i objawił/a cudowną mądrość prawdziwej pustki, oraz szybko doświadczył/a stan nie-narodzin, potwierdzając owoc Stanu Buddhy.\\

\begin{Verse}
\samepage{
	WSZYSCY BUDDHOWIE\\
	WSZYSCY SZLACHETNI\\
	BODHISATTWOWIE MAHASATTWOWIE\\
	W DZIESIĘCIU KIERUNKACH TRZECH ŚWIATACH\\
	WIELKA DOSKONAŁA MĄDROŚĆ
}
\end{Verse}
\end{Prayer}

%%%%%%%%%%%%%%%%%%%%%%%%%%%%%%%%%%%%%%%%%%%%%%%%%%%%%%%%%%%%%%%%%%%%%%%%%%%%%%%
%%%% OFIAROWANIE ZASŁUGI DLA CHOREJ OSOBY
\begin{Prayer}{-}
	{OFIAROWANIE ZASŁUGI DLA CHOREJ OSOBY}{-}
	{-}

\bigskip
\noindent
{\scriptsize\it Zalecane Sutry i Dharani: Sutra Serca, Dai Hi Sin Dharani, Sio Sai Mio Kicidzio Dharani, Przedłużająca Życie Sutra Kannon w Dziesięciu Wersach.}


\smallskip
\noindent
Zasługi recytacji\\
\null\dotfill\null\\
oraz kwiaty, światło świec i kadzidła\\
{\scriptsize({\it jeśli przygotowano dodatkowo inne ofiary jak owoce, herbatę i inne to należy je w tym miejscu wymienić})}\\
\null\dotfill\null\\
ofiarowujemy Wiecznym Trzem Klejnotom w dziesięciu kierunkach,\\
wszystkim Buddhom i Bodhisattwom Trzech Okresów Czasu\\
oraz wszystkim Świętym Obrońcom Dharmy.\\
Modlimy się za szczęście Rodziny\\
{\scriptsize({\it nazwisko Rodziny chorej osoby})}\\
\null\dotfill\null\\
i za wszystkich jej przodków.\\
Szczególnie modlimy się o zdrowie\\
{\scriptsize({\it imię i nazwisko chorej osoby})}\\
\null\dotfill\null\\
Oby jego/jej ciało odzyskało harmonię Czterech Wielkich Elementów, oraz siłę i zdrowie.\\
Oby osiągnął szczęście i długie życie oraz wszystkie przyczyny szczęścia i długiego życia.

\par\bigskip
\begin{Verse}
\samepage{
	WSZYSCY BUDDHOWIE\\
	WSZYSCY SZLACHETNI\\
	BODHISATTWOWIE MAHASATTWOWIE\\
	W DZIESIĘCIU KIERUNKACH TRZECH ŚWIATACH\\
	WIELKA DOSKONAŁA MĄDROŚĆ
}
\end{Verse}
\end{Prayer}

%%%%%%%%%%%%%%%%%%%%%%%%%%%%%%%%%%%%%%%%%%%%%%%%%%%%%%%%%%%%%%%%%%%%%%%%%%%%%%%
%%%% OFIAROWANIE DLA ZMARŁEJ OSOBY W PIERWSZYCH TYGODNIACH PO ŚMIERCI
\begin{Prayer}{-}
	{OFIAROWANIE DLA ZMARŁEJ OSOBY\\W~PIERWSZYCH TYGODNIACH PO ŚMIERCI}{-}
	{-}

Ofiarowane tutaj kadzidło, kwiaty, światło świec, czystą wodę oraz recytację\\
{\scriptsize({\it wymienić tytuły Sutr i Dharani, które były recytowane})}\\
	\null\dotfill\null\\
ofiarowujemy duchowi zmarłego/zmarłej\\
{\scriptsize{\it wymienić imię Dharmy lub imię i nazwisko osoby zmarłej})}\\
	\null\dotfill\null\\
Modlimy się o to, by po przyczynowym rozpadzie Czterech Wielkich Elementów, zostały ozdobione Stany Zasługi.\\

\par\bigskip
\begin{Verse}
\samepage{
	WSZYSCY BUDDHOWIE\\
	WSZYSCY SZLACHETNI\\
	BODHISATTWOWIE MAHASATTWOWIE\\
	W DZIESIĘCIU KIERUNKACH TRZECH ŚWIATACH\\
	WIELKA DOSKONAŁA MĄDROŚĆ
}
\end{Verse}
\end{Prayer}

%%%%%%%%%%%%%%%%%%%%%%%%%%%%%%%%%%%%%%%%%%%%%%%%%%%%%%%%%%%%%%%%%%%%%%%%%%%%%%%
%%%% OGÓLNE OFIAROWANIE
\begin{Prayer}{-}
	{OGÓLNE OFIAROWANIE}{-}
	{-}

Z czcią modlimy się aby zebrana cnota zasługi całkowicie przeniknęła wszędzie i ogarnęła wszystkich, obyśmy razem ze wszystkimi istotami osiągnęli Drogę Buddhy

\samepage{
	WSZYSCY BUDDHOWIE\\
	WSZYSCY SZLACHETNI\\
	BODHISATTWOWIE MAHASATTWOWIE\\
	W DZIESIĘCIU KIERUNKACH TRZECH ŚWIATACH\\
	WIELKA DOSKONAŁA MĄDROŚĆ
}
\end{Prayer}

% #1 -- źródło cytatu
\newenvironment{Cite}[1]{%
	\bigskip
	\def\source{#1}
	\begingroup
	\raggedright
	\leftskip 1cm
	\itshape
}
{\par\endgroup\medskip\hspace*{0.4\textwidth}\source}

%%%%%%%%%%%%%%%%%%%%%%%%%%%%%%%%%%%%%%%%%%%%%%%%%%%%%%%%%%%%%%%%%%%%%%%%%%%%%%%
%%%% KILKA CYTATÓW
\begin{Prayer}{cytaty}
	{KILKA CYTATÓW}{-}
	{-}

\bigskip
\samepage{
\begin{Cite}{Sutra Diamentowa}
	Wszystkie złożone rzeczy są jak sen,\\
	Złudzenie, bańka, cień ---\\
	Są jak kropla rosy i~błysk światła;\\
	Tak powinny być rozpatrywane.
\end{Cite}
}

\samepage{
\begin{Cite}{Sutra Lankavatara}
	Rzeczy nie są takie jakimi się zdają być\\
	Ani też inne.
\end{Cite}
}

\samepage{
\begin{Cite}{Hyakujo}
	Dzień bez pracy jest dniem bez jedzenia.
\end{Cite}
}

\samepage{
\begin{Cite}{Rinzai Roku}
	Jeśli staniesz się swoim własnym mistrzem,\\
	Gdziekolwiek staniesz jest to Prawdziwe miejsce.\\
	Chociaż złoty pył jest drogocenny,\\
	Kiedy dostanie się do oczu\\
	Zaciemnia widzenie.
\end{Cite}
}

\samepage{
\begin{Cite}{Ummon}
	Jakkolwiek cudowna nie jest rzecz, być może\\
	Lepiej jest wcale jej nie mieć.
\end{Cite}
}

\samepage{
\begin{Cite}{Ryokan}
	Skąd jest moje życie?\\
	Dokąd prowadzi?\\
	Siedzę sam w~swojej chacie,\\
	Medytuję cicho, jednak gorliwie;\\
	Z całym moim myśleniem nie wiem skąd\\
	Ani czy zmierzam do jakiegoś dokąd;\\
	I tak to jest z~moją teraźniejszością,\\
	Nieustannie zmieniając się --- wszystko w~pustce!\\
	W tej pustce jest na chwilę ego,\\
	Ze swoimi tak i~nie.\\
	Wiem nie tylko gdzie je ustawić.\\
	Podążam za moją karmą tak jak się porusza,\\
	Z doskonałym zadowoleniem.
\end{Cite}
}

\samepage{
\begin{Cite}{}
	Wiosną setki kwiatów,\\
	Latem rześki wiatr,\\
	Jesienią księżyc nad polem,\\
	A zimą płatki śniegu\\
	---~towarzyszą ci.\\
	Jeśli bezużyteczne sprawy nie zaśmiecają twojego umysłu\\
	Każda pora, to cudowna pora.
\end{Cite}
}

\samepage{
\begin{Cite}{Mumon-kan}
	Błękitne niebo, jasny dzień~---\\
	Przestań wszędzie szukać!\\
	Jeśli ciągle pytasz, ,,Co to jest Buddha?''\\
	Chowasz w~kieszeni skradzione rzeczy i~udajesz niewinnego.
\end{Cite}
}

\samepage{
\begin{Cite}{}
	Wielka jest sprawa Narodzin i~Śmierci.\\
	Życie przemijające jest i~nietrwałe\\
	Przebudźcie się!\\
	Nie marnujcie ani chwili.
\end{Cite}
}
\end{Prayer}

%\newpage
%%%%%%%%%%%%%%%%%%%%%%%%%%%%%%%%%%%%%%%%%%%%%%%%%%%%%%%%%%%%%%%%%%%%%%%%%%%%%%%
%%%% UWAGI DO TRANSKRYPCJI JĘZYKA JAPOŃSKIEGO
\begin{Prayer}{uwagi_do_transkrypcji}
	{UWAGI DO TRANSKRYPCJI\\JĘZYKA JAPOŃSKIEGO}{UWAGI DO TRANSKRYPCJI JĘZYKA JAPOŃSKIEGO}
	{-}

\begin{tabular}{l l}
\emph{ch}	& czytamy jak polskie \emph{ć}\\
\emph{j}	& czytamy jak polskie \emph{dź}\\
\emph{sh}	& czytamy jak polskie \emph{ś}\\
\emph{w}	& czytamy jak polskie \emph{ł}\\
\emph{y}	& czytamy jak polskie \emph{j}\\
\emph{z}	& czytamy jak polskie \emph{dz}\\
\end{tabular}

\parskip   1em
\parindent 0pt

	\par Ponadto kreseczka nad samogłoską (np. \textit{\=o, \=a, \=u}) oznacza jej wydłużenie. Alternatywnym zapisem wydłużenia może być podwójna samogłoska (np. \textit{oo, aa, uu}). Dla wydłużonego \textit{o} stosuje się też zapis \textit{ou}, co oznacza też nieco zmienioną wymowę (słyszalne, słabe \textit{u}). 
\par Symbole \keisu oraz \shokei oznaczają odpowiednio uderzenie w~keisu i~shokei.

\bigskip
Inne uwagi:
\begin{itemize}
 \item Teksty recytowane przy posiłkach zostały wydane osobno.
 \item Niektóre tłumaczenia na język polski zostały podane jedynie dla ogólnej orientacji, nie nadają się do recytacji.
\end{itemize}
\end{Prayer}

%% alfabetyczny spis treści %%
\newpage

\noindent{\large \textbf{Alfabetyczny spis treści}}
\bigskip

{
\parskip 0.3em
\parfillskip 0pt
% To jest wersja pliku ze starszej wersji zeszytu - nie wiem czy jest aktualna
\par\noindent ATTA DIPA \dotfill \makebox[1.0cm][r]{\pageref{atta_dipa}}
\par\noindent BOSATSU GANGY\=O MON \dotfill \makebox[1.0cm][r]{\pageref{bosatsu_gangyo_mon}}
\par\noindent BRAMA SŁODKIEGO NEKTARU AMRITY WEJŚCIA W~NIRWANĘ \dotfill \makebox[1.0cm][r]{\pageref{brama_slodkiego_nektaru_amrity_wejscia_w_nirwane}}
\par\noindent BU CHIN SON SHIN DARANI \dotfill \makebox[1.0cm][r]{\pageref{bu_chin_son_shin_dharani}}
\par\noindent CH\=UH\=O OSH\=O ZAY\=U NO MEI \dotfill \makebox[1.0cm][r]{\pageref{chuho_osho_zayu_no_mei}}
\par\noindent CZTERY ŚLUBOWANIA \dotfill \makebox[1.0cm][r]{\pageref{cztery_slubowania}}
\par\noindent DAICHI ZENJI HOTSUGANMON \dotfill \makebox[1.0cm][r]{\pageref{daichi_zenji_hotsuganmon}}
\par\noindent DAICHI ZENJI HOTSUGANMON (j. polski) \dotfill \makebox[1.0cm][r]{\pageref{daichi_zenji_hotsuganmon_pl}}
\par\noindent DAIE ZENJI HOTSUGAMMON \dotfill \makebox[1.0cm][r]{\pageref{daie_zenji_hotsugammon}}
\par\noindent DAIHI SHIN DARANI \dotfill \makebox[1.0cm][r]{\pageref{dai_hi_sin_dharani}}
\par\noindent DAI SEGAKI \dotfill \makebox[1.0cm][r]{\pageref{dai_segaki}}
\par\noindent DHARANI DO 13 BUDDÓW \dotfill \makebox[1.0cm][r]{\pageref{dharani_do_buddow}}
\par\noindent DHARANI \dotfill \makebox[1.0cm][r]{\pageref{dharani}}
\par\noindent DHARANI WIELCE WSPÓŁCZUJĄCEGO \dotfill \makebox[1.0cm][r]{\pageref{dharani_wielce_wspolczujacego}}
\par\noindent DROGOCENNE ZWIERCIADŁO SAMADHI \dotfill \makebox[1.0cm][r]{\pageref{drogocenne_zwierciadlo_samadhi}}
\par\noindent DWA OFIAROWANIA \dotfill \makebox[1.0cm][r]{\pageref{dwa_ofiarowania}}
\par\noindent DWA RODZAJE EK\=O (DO DOMU) \dotfill \makebox[1.0cm][r]{\pageref{dwa_rodzaje_eko}}
\par\noindent DZIESIĘĆ BUDDYJSKICH WSKAZAŃ \dotfill \makebox[1.0cm][r]{\pageref{dziesiec_buddyjskich_wskazan}}
\par\noindent DZIESIĘĆ IMION BUDDY \dotfill \makebox[1.0cm][r]{\pageref{dziesiec_imion_buddy}}
\par\noindent EIHEI K\=OSO HOTSUGAMMON \dotfill \makebox[1.0cm][r]{\pageref{eihei_koso_hotsugammon}}
\par\noindent EMMEI JIKKU KANNON GY\=O \dotfill \makebox[1.0cm][r]{\pageref{emmei_jikku}}
\par\noindent FUEK\=O \dotfill \makebox[1.0cm][r]{\pageref{fueko}}
\par\noindent FUKANZAZENGI \dotfill \makebox[1.0cm][r]{\pageref{fukanzazengi}}
\par\noindent GATHA OTWARCIA SUTR \dotfill \makebox[1.0cm][r]{\pageref{gatha_otwarcia_sutr}}
\par\noindent GATHA SIEDMIU BUDDÓW \dotfill \makebox[1.0cm][r]{\pageref{gatha_siedmiu_buddow}}
\par\noindent GATHA SKRUCHY \dotfill \makebox[1.0cm][r]{\pageref{gatha_skruchy}}
\par\noindent GYAKU ON JIN SHU \dotfill \makebox[1.0cm][r]{\pageref{gyaku_on_jin_shu}}
\par\noindent HAKUIN ZENJI ZAZEN WASAN \dotfill \makebox[1.0cm][r]{\pageref{hakuin_zenji_zazen_wasan}}
\par\noindent HAKU SHIN DHARANI \dotfill \makebox[1.0cm][r]{\pageref{haku_shin_dharani}}
\par\noindent H\=OKY\=O ZANMAI \dotfill \makebox[1.0cm][r]{\pageref{hokyozanmai}}
\par\noindent HONZON EK\=O \dotfill \makebox[1.0cm][r]{\pageref{honzon_eko}}
\par\noindent HONZONJ\=OGU EK\=O \dotfill \makebox[1.0cm][r]{\pageref{honzonjogu_eko}}
\par\noindent JAKUSHITSU ZENJI YUIKAI\dotfill \makebox[1.0cm][r]{\pageref{jakushitsu_zenji_yuikai}}
\par\noindent JYU BUTSU MYO \dotfill \makebox[1.0cm][r]{\pageref{jyu_butsu_myo}}
\par\noindent KAIKYOGE \dotfill \makebox[1.0cm][r]{\pageref{kaikyoge}}
\par\noindent KAN NIN FU MON PIN KIN \dotfill \makebox[1.0cm][r]{\pageref{kai_nin_fu_mon_pin_kin}}
\par\noindent KAI KANROMON \dotfill \makebox[1.0cm][r]{\pageref{dai_segaki}}
\par\noindent KANROMON \dotfill \makebox[1.0cm][r]{\pageref{kanromon}}
\par\noindent KILKA CYTATÓW \dotfill \makebox[1.0cm][r]{\pageref{cytaty}}
\par\noindent KON GO HAN NYA HA RA MI KYO \dotfill \makebox[1.0cm][r]{\pageref{kon_go_han_nya_ha_ra_mi_kyo}}
\par\noindent K\=OZEN DAIT\=O KOKUSHI YUIKAI \dotfill \makebox[1.0cm][r]{\pageref{kozen_daito_kokushi_yuikai}}
\par\noindent MAKA HANNYA HARAMITTA SHINGY\=O \dotfill \makebox[1.0cm][r]{\pageref{hannya_shin_gyo}}
\par\noindent MANTRA BHAISADZJAGURU \dotfill \makebox[1.0cm][r]{\pageref{mantra_bhaisadzjaguru}}
\par\noindent MISTRZA HAKUINA PIEŚŃ KU CHWALE ZAZEN \dotfill \makebox[1.0cm][r]{\pageref{mistrza_hakuina_piesn}}
\par\noindent MODLITWA POKŁONÓW\\ PRZED TRZEMA KLEJNOTAMI \dotfill \makebox[1.0cm][r]{\pageref{trzy_skarby}}
\par\noindent MODLITWA PRZED RELIKWIAMI BUDDY \dotfill \makebox[1.0cm][r]{\pageref{przed_relikwiami}}
\par\noindent MY\=OH\=ORENGEKY\=O KANZEON\\ BOSATSU FUMONBONGE \dotfill \makebox[1.0cm][r]{\pageref{myohorengekyo_kanzeonbosatsu_fumonbonge}}
\par\noindent MY\=OH\=ORENGEKY\=O NYORAI JURY\=OHONGE \dotfill \makebox[1.0cm][r]{\pageref{myohorengekyo_nyorai_juryohonge}}
\par\noindent NY\=UYOKU NO GE \dotfill \makebox[1.0cm][r]{\pageref{nyuyoku_no_ge}}
\par\noindent OFIAROWANIE ARHATOM \dotfill \makebox[1.0cm][r]{\pageref{ofiarowanie_arhatom}}
\par\noindent OFIAROWANIE GŁODNYM DUCHOM \dotfill \makebox[1.0cm][r]{\pageref{ofiarowanie_glodnym_duchom}}
\par\noindent OFIAROWANIE PANU BUDDZIE \dotfill \makebox[1.0cm][r]{\pageref{ofiarowanie_panu_buddzie}}
\par\noindent OFIAROWANIE PATRIARCHOM \dotfill \makebox[1.0cm][r]{\pageref{ofiarowanie_patriarchom}}
\par\noindent OFIAROWANIE ŚWIĄTYNI \dotfill \makebox[1.0cm][r]{\pageref{ofiarowanie_swiatyni}}
\par\noindent OSIĄGNIĘCIE ZJEDNOCZENIA \dotfill \makebox[1.0cm][r]{\pageref{osiagniecie_zjednoczenia}}
\par\noindent OSTATNIE POUCZENIA MISTRZA JAKUSHITSU \dotfill \makebox[1.0cm][r]{\pageref{ostatnie_pouczenia_mistrza_jakushitsu}}
\par\noindent OSTATNIE UPOMNIENIE DAIT\=O KOKUSHI \dotfill \makebox[1.0cm][r]{\pageref{ostatnie_upomnienie}}
\par\noindent PIEŚŃ KLEJNOTU BUDDY \dotfill \makebox[1.0cm][r]{\pageref{piesn_klejnotu_buddy}}
\par\noindent PIEŚŃ OBMYWANIA BUDDY \dotfill \makebox[1.0cm][r]{\pageref{piesn_obmywania_buddy}}
\par\noindent POWSZECHNE WYJAŚNIENIE ZAZEN \dotfill \makebox[1.0cm][r]{\pageref{ogolne_zalecenia}}
\par\noindent PRZEDŁUŻAJĄCA ŻYCIE SUTRA KANNON W~DZIESIĘCIU WERSACH \dotfill \makebox[1.0cm][r]{\pageref{sutra_kannon}}
\par\noindent RAIHAI GE \dotfill \makebox[1.0cm][r]{\pageref{raihaige}}
\par\noindent SAND\=OKAI \dotfill \makebox[1.0cm][r]{\pageref{sandokai}}
\par\noindent SANGEMON \dotfill \makebox[1.0cm][r]{\pageref{sangemon}}
\par\noindent SANKIE MON / SANKIKAI \dotfill \makebox[1.0cm][r]{\pageref{sankiemon}}
\par\noindent SANKIRAIMON \dotfill \makebox[1.0cm][r]{\pageref{sankiraimon}}
\par\noindent SANZON RAIMON \dotfill \makebox[1.0cm][r]{\pageref{sanzonraimon}}
\par\noindent SENMEN NO GE \dotfill \makebox[1.0cm][r]{\pageref{senmen_no_ge}}
\par\noindent SHARIRAIMON \dotfill \makebox[1.0cm][r]{\pageref{shariraimon}}
\par\noindent SHICHI BUTSU TS\=UKAIGE \dotfill \makebox[1.0cm][r]{\pageref{shichi_butsu_tsukaige}}
\par\noindent SHIGUSEIGANMON \dotfill \makebox[1.0cm][r]{\pageref{shiguseiganmon}}
\par\noindent SH\=OSAI MY\=OKICHIJ\=O DARANI \dotfill \makebox[1.0cm][r]{\pageref{sio_saj_mio}}
\par\noindent SHO SAI MYO KI J\=O JIN SH\=U \dotfill \makebox[1.0cm][r]{\pageref{sho_sai_myo_ki_jo}}
\par\noindent SHUSH\=OGI \dotfill \makebox[1.0cm][r]{\pageref{shushogi}}
\par\noindent STROFA POKŁONÓW \dotfill \makebox[1.0cm][r]{\pageref{strofa_poklonow}}
\par\noindent STROFA SZACUNKU\\ DLA TRZECH CZCIGODNYCH \dotfill \makebox[1.0cm][r]{\pageref{strofa_szacunku}}
\par\noindent STROFY WIARY W~UMYSŁ \dotfill \makebox[1.0cm][r]{\pageref{strofy_wiary_w_umysl}}
\par\noindent SUTRA LOTOSU WSPANIAŁEGO PRAWA\\ROZDZIAŁ: POWSZECHNA BRAMA AWALOKITEŚWARY, OBSERWUJĄCEGO DŹWIĘKI ŚWIATA BODHISATTWY \dotfill \makebox[1.0cm][r]{\pageref{sutralotosu_powszechna_brama}}
\par\noindent SUTRA LOTOSU WSPANIAŁEGO PRAWA ROZDZIAŁ: WIECZNE ŻYCIE TATHAGATY \dotfill \makebox[1.0cm][r]{\pageref{sutralotosu_wieczne_zycie}}
\par\noindent SUTRA SERCA WIELKIEJ DOSKONAŁEJ MĄDROŚCI \dotfill \makebox[1.0cm][r]{\pageref{serce_doskonalej_madrosci}}
\par\noindent ŚLUBOWANIE BODHISATTWY \dotfill \makebox[1.0cm][r]{\pageref{slubowanie_bodhisattwy}}
\par\noindent TAKKESAGE \dotfill \makebox[1.0cm][r]{\pageref{tekkesage}}
\par\noindent TEIDAI DENPO BUSSO NO MYOGO \dotfill \makebox[1.0cm][r]{\pageref{teidai_denpo_busso_no_myogo}}
\par\noindent TI-SARANA \dotfill \makebox[1.0cm][r]{\pageref{ti-sarana}}
\par\noindent TRAKTAT O~PRAKTYCE I~OŚWIECENIU \dotfill \makebox[1.0cm][r]{\pageref{traktat_o_praktyce}}
\par\noindent TRZY OGÓLNE POSTANOWIENIA \dotfill \makebox[1.0cm][r]{\pageref{trzy_ogolne_postanowienia}}
\par\noindent TRZY SCHRONIENIA \dotfill \makebox[1.0cm][r]{\pageref{trzy_schronienia}}
\par\noindent UWAGI DO TRANSKRYPCJI JĘZYKA JAPOŃSKIEGO \dotfill \makebox[1.0cm][r]{\pageref{uwagi_do_transkrypcji}}
\par\noindent VANDANA \dotfill \makebox[1.0cm][r]{\pageref{vandana}}
\par\noindent WIERSZ KASIAJA \dotfill \makebox[1.0cm][r]{\pageref{wiersz_kesy}}
\par\noindent WZNIECANIE UMYSŁU OŚWIECENIA, DAIE ZENJI \dotfill \makebox[1.0cm][r]{\pageref{wzniecanie_umyslu_oswiecenia_daie}}
\par\noindent WZNIECANIE UMYSŁU OŚWIECENIA, DOGEN ZENJI \dotfill \makebox[1.0cm][r]{\pageref{wzniecanie_umyslu_oswiecenia_dogen}}
\par\noindent YAKUSHI SON SHO DHARANI \dotfill \makebox[1.0cm][r]{\pageref{yakushi_son_sho_dharani}}
\par\noindent ZWROT ZASŁUG \dotfill \makebox[1.0cm][r]{\pageref{zwrot_zaslug}}
\par\noindent ŻYCIOWE ZASADY CH\=UH\=O OSH\=O \dotfill \makebox[1.0cm][r]{\pageref{zyciowe_zasady_chuho_osho}}

}

\newcommand{\gitversion}{d132c47 2025-07-30}
 

%%-- stopka --%%
\vfill
\rule{\textwidth}{1pt}
\begin{center}
	{\large \tt mahajana.net}

	\medskip
	{\small Skład: \LaTeXe, VI Improved, GNU/Linux}\\
	{\footnotesize\texttt{git version: \gitversion}}
\end{center}

\end{document}
%%-- eof --%%
